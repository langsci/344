\chapter{Information structure}\is{information structure|(}
\label{ch:inf}
\label{sec:topfoc}
In this chapter, I describe grammatical markers relating to information structure. Kalamang has three elements that deal with structuring information: topic marker \textit{me}, and focus markers \textit{=a} and \textit{=ba}, introduced in §\ref{sec:leftd}. The topic marker follows the postpositional phrase (PP). It does not co-occur with object marker \textit{=at} and follows the other postpositions. Alternatively, it follows the predicate. The focus markers attach to the PP.\is{postpositional phrase}

Although the definition and study of both topic and focus has been problematised \parencite{matic2013,ozerov2018}, the analytic appeal of these labels is that they are broad categories that can subsume many interactional and discourse-structuring aspects of communication \parencite{ozerov2018}. %NB Needed? I don't think so unless you address the problematization further.
It is as such that I use them: broad labels for a word and two enclitics that deal with the management of information in discourse. Precisely because \textit{me} and \textit{=a} (and to a lesser extent \textit{=ba}) are so versatile, I refrain from attempting a more precise analysis here. For that, a better understanding of other parts of the Kalamang grammar, as well as a bigger corpus, are needed.

In this chapter, only the topic and focus markers are treated. Constituents may also be topicalised through fronting, see §\ref{sec:leftd}. The intonation of focused constituents is discussed in §\ref{sec:focint}.


\section{Topic marker \textit{me}}\is{topic|(}%\is{clause delimiter|(}
\label{sec:discme}
\label{sec:clauseinme}
\textit{Me} is a topic marker that follows either the NP/PP or the predicate.\is{postpositional phrase} A topic marker is an ``entity that the speaker identifies, about which information (...) is then given'' \parencite[][27]{krifka2012}. It is the starting point of an utterance with presupposed information that is then commented on \parencite{foley2007}. When \textit{me} follows a NP or PP, the NP or PP is topicalised. When \textit{me} follows the predicate, the whole clause is topicalised, for example by making it conditional. Anti-topics (specified or reactivated topics) are not marked morphologically, but by placing the anti-topic at the right edge of the clause, as described in §\ref{sec:leftd}. In the examples in this section, the topicalised constituents are given within square brackets.

\textit{Me} can be used after a NP or PP to mark it as the topic in clauses with a verbal (see \ref{exe:metrans}, \ref{exe:mepron} and~\ref{exe:mestat}), nominal (see \ref{exe:menom}), \is{quantifier clause}quantifier (see \ref{exe:mequant}) or locative (see \ref{exe:meloc}) predicate, but is not obligatory in any of these contexts.

\begin{exe}
	\ex \gll [enem beladar-pas] \textbf{me} buok=at kuru bara pier=ki bes to\\
	woman Netherlands-woman {\glme} betel=\textsc{obj} bring descend \textsc{1du.in=ben} good right\\
	\glt `The Dutchwoman brought down betel for us, good, right?' (Context: subject is given but not recently mentioned.) \jambox*{\href{http://hdl.handle.net/10050/00-0000-0000-0004-1BBD-5}{[conv12\_7:02]}}
	\label{exe:metrans}
	\ex \gll [ka] \textbf{me} an se kat percai=nin\\
	\textsc{2sg} {\glme} \textsc{1sg} {\glse} \textsc{2sg.obj} believe=\textsc{neg}\\
	\glt `As for you, I don't believe you any more.' \jambox*{\href{http://hdl.handle.net/10050/00-0000-0000-0004-1BBA-8}{[stim2\_8:47]}}
	\label{exe:mepron}
	\ex \gll ki Tanggor=ka=ta rar [Tanggor] \textbf{me} mambon\\
	\textsc{2pl} Tanggor=\textsc{lat}={\glta} go.\textsc{pl.imp} Tanggor {\glme} \textsc{exist}\\
	\glt `You go to Tanggor, at Tanggor there are [fish].' (Context: Tanggor is given, in this very example.) \jambox*{\href{http://hdl.handle.net/10050/00-0000-0000-0004-1BCD-C}{[conv3\_1:33]}}
	\label{exe:mestat}
	\ex \gll [pas me] \textbf{me} mungkin berupa jim\\
	woman \textsc{dist} \textsc{me} maybe be\_a\_form\_of jinn\\
	\glt `That woman is maybe a kind of \textit{jinn}.' (Context: description of a video. The woman has been mentioned; this is a comment about her identity.) \jambox*{\href{http://hdl.handle.net/10050/00-0000-0000-0004-1C9C-7}{[stim24\_0:37]}}
	\label{exe:menom}	
	\ex \gll [wa] \textbf{me} eir eh\\
	\textsc{prox} {\glme} two \textsc{tag}\\
	\glt `These are two, right?' (Context: learning to weave a basket, pointing at strips of material in the teacher's hand.) \jambox*{\href{http://hdl.handle.net/10050/00-0000-0000-0004-1BB5-B}{[conv17\_15:39]}}
	\label{exe:mequant}	
	\ex \gll [tumtum eir] \textbf{me} kewe-un=ko\\
	children two \textsc{me} house-\textsc{3poss=loc}\\
	\glt `Two children are in their house.' (Context: frog story. Children have been mentioned; description of a new page in the book.) \jambox*{\href{http://hdl.handle.net/10050/00-0000-0000-0004-1BAF-5}{[stim20\_0:54]}}
	\label{exe:meloc}
\end{exe}

The topic marker follows postpositions, as in~(\ref{exe:finbon}) and~(\ref{exe:komeneko}). Topicalised objects lack object postposition \textit{=at} (example~\ref{exe:bobmun}). The topic marker is typically used following NPs or PPs that refer to given referents. It can also topicalise a time adverbial (example~\ref{exe:ambun}). Another way to mark topics, which may be combined with topic marker \textit{me}, is fronting of the topicalised constituent (§\ref{sec:leftd}).

\begin{exe}
	\ex \gll [Arifin=bon Mei=bon] \textbf{me} nat=nin\\
	Arifin=\textsc{com} Mei=\textsc{com} {\glme} consum=\textsc{neg}\\
	\glt `Arifin and Mei don't eat.' \jambox*{\href{http://hdl.handle.net/10050/00-0000-0000-0004-1B9F-F}{[conv9\_6:12]}}
	\label{exe:finbon}
	\ex \gll [watko] \textbf{me} lempuang-tumun wilak=ko me neko\\
	\textsc{prox.loc} {\glme} island-child sea=\textsc{loc} {\glme} inside\\
	\glt `Here, the small island inside the sea.' \jambox*{\href{http://hdl.handle.net/10050/00-0000-0000-0004-1BE8-0}{[stim43\_1:33]}}
	\label{exe:komeneko}
	\ex \gll [tumtum taukon]\textsubscript{\upshape Obj} \textbf{me} Bobi emun=a kona\\
	children few {\glme} Bobi mother=\textsc{foc} see\\
	\glt `A few children, Bobi's mother saw [them].' \jambox*{\href{http://hdl.handle.net/10050/00-0000-0000-0004-1BCE-D}{[conv4\_5:09]}}
	\label{exe:bobmun}
	\ex \gll kier Luis=bon [\textbf{kasur}] \textbf{me} bo sor-sanggara\\
	\textsc{2du} Luis=\textsc{com} tomorrow {\glme} go fish-search\\
	\glt `(As for) tomorrow, you and Luis go fishing.' \jambox*{\href{http://hdl.handle.net/10050/00-0000-0000-0004-1BD7-2}{[narr3\_2:26]}}
	\label{exe:ambun}
\end{exe}

More seldomly, new participants are introduced with \textit{me}. This is limited to unimportant participants. In~(\ref{exe:sikank}), the cat is introduced; it is not mentioned thereafter, and is only used in the story as a prop to create a slightly threatening atmosphere. In examples of the type in~(\ref{exe:wameta}), where properties of referents are described out of context, \textit{me} is also used, often combined with a demonstrative.

\begin{exe}
	\ex \gll [sikan kon] \textbf{me} in=at tiri pareir\\
	cat one {\glme} \textsc{1pl.excl=obj} run follow\\
	\glt `A cat ran after us.' (Context: a story about going to someone's house, first mention of the cat.) \jambox*{\href{http://hdl.handle.net/10050/00-0000-0000-0004-1B9F-F}{[conv9\_25:05]}}
	\label{exe:sikank}
	\ex \gll [ror wa] \textbf{me} tabusik\\
	tree \textsc{prox} {\glme} short\\
	\glt `This tree is short.' (Context: no context, elicitation, transl. of Malay \textit{pohon ini pendek} `this tree is short'.) \jambox*{\href{http://hdl.handle.net/10050/00-0000-0000-0004-1C60-A}{[elic\_adj\_4]}}
	\label{exe:wameta}
\end{exe}

In a few instances, \textit{me} marks two referents in the same clause. In~(\ref{exe:anmed}), the second instance of \textit{me} could alternately be analysed as a distal demonstrative (but would still be topicalised because it is fronted), but in~(\ref{exe:lekop}) the analysis seems untlikely to fit, as there is already a demonstrative modifying each noun. Analysing \textit{me} as a definite marker also does not fit, as it does not appear consistently on anaphoric definites (referents that have been mentioned in the text). 

\begin{exe}
	\ex \gll [an] \textbf{me} [don] \textbf{me} ka an ∅=nin\\
	\textsc{1sg} {\glme} thing {\glme} \textsc{2sg} \textsc{1sg} give=\textsc{neg}\\
	\glt `As for me, and as for that thing, you didn't give [it] to me.' \jambox*{\href{http://hdl.handle.net/10050/00-0000-0000-0004-1BBD-5}{[conv12\_17:57]}}
	\label{exe:anmed}
	\ex	\gll [lek opa] \textbf{me} [afokat opa] \textbf{me} kona\\
	goat {\glopa} {\glme} avocado {\glopa} {\glme} see\\
	\glt `That goat sees those avocados.' \jambox*{\href{http://hdl.handle.net/10050/00-0000-0000-0004-1BD1-D}{[stim31\_0:51]}}
	\label{exe:lekop}
\end{exe}	

\textit{Me} is also very common after the predicate, topicalising the entire clause. In most cases, this predicate carries irrealis \textit{=et} (§\ref{sec:et}) or non-final \textit{=ta} (§\ref{sec:ta}). 
% \textit{-ta} and \textit{me} (ca 130 hits); \textit{=et} and \textit{me} (about 250 hits in the corpus, i.e. very common)

\begin{exe}
	\ex \gll Mustafa esun toni [ka ruot=et] \textbf{me} kamun naun=saet=in\\
	Mustafa father.\textsc{3poss} say \textsc{2sg} dig={\glet} {\glme} \textsc{2sg-proh} earth=only=\textsc{proh}\\
	\glt `Mustafa's father said if you go digging, don't [bring] earth only.' \jambox*{\href{http://hdl.handle.net/10050/00-0000-0000-0004-1BA3-3}{[conv10\_4:55]}}
	\label{exe:ruretme}
	\ex \gll pulor \textbf{opa} [ka tu=ta] \textbf{me} tama-ba-kadok os-kadok\\
	betel\_leaf {\glopa} \textsc{2sg} pound={\glta} {\glme} where-\textsc{foc}-side beach-side\\
	\glt `That betel leaf [that] you were pounding, on which side is it, on the beach-side?' \jambox*{\href{http://hdl.handle.net/10050/00-0000-0000-0004-1BBD-5}{[conv12\_8:51]}}
	\label{exe:pulor}
	\ex \gll newer=i koyet, [mindi=ta] \textbf{me} mier se marua\\
	pay={\gli} finish like\_that={\glta} {\glme} \textsc{3du} {\glse} move\_seawards\\
	\glt `After paying, they went seawards.' \jambox*{\href{http://hdl.handle.net/10050/00-0000-0000-0004-1BC1-0}{[narr19\_8:12]}}
	\label{exe:mindita}
\end{exe} 

The combination of irrealis \textit{=et} followed by \textit{me} is also seen in pre-posed topic constructions as in~(\ref{exe:eiretme}). This is an unusual way to present a topic: it is much more common to do so without \textit{=et}.

\begin{exe}
	\ex \gll [siada eir=et] \textbf{me} ma se taraouk\\
	kind\_of\_fish two={\glet} {\glme} \textsc{3sg} {\glse} store\\
	\glt `As for those two siadas, he had already stored [them].' \jambox*{\href{http://hdl.handle.net/10050/00-0000-0000-0004-1BA3-3}{[conv10\_10:55]}}
	\label{exe:eiretme}	
\end{exe}	

Topic marker \textit{me} is homophonous with distal \is{demonstrative!distal}demonstrative \textit{me}, and has likely developed from it. Demonstratives are known to develop into topic markers \parencite{devries1995}.\footnote{As well as into copulas and definite markers \parencite{diessel1999}.} They have also been shown to have clause-combining \parencite{breun2020} and discourse-structuring functions \parencite{francois2005,naess2011}. The Kalamang process of grammaticalisation is likely ongoing, judging by the various uses of \textit{me} sketched in this section, and moving away from a demonstrative use, judging by its few occurrences as an ordinary exophoric demonstrative.

There are also a considerable number of occurrences of \textit{me} at the beginning of a clause, whose function cannot be determined based on the currently available data. There are various possible analyses: it can be short for \textit{mera} `then, so', \textit{mena} `later, otherwise', it can be a filler,\footnote{\textcite{hayashi2010} mention demonstratives as sources for fillers to deal with word-formation trouble, but I have only observed \textit{me} functioning as a filler between clauses (see also~§\ref{sec:ph} for placeholder \textit{neba}).} or it can be a specific use of topic marker or distal demonstrative \textit{me}. The following four examples show two subsequent utterances each, where the second utterance starts with \textit{me}.

\begin{exe}
	\ex 
	\begin{xlist}
		\ex \gll mu toni kasur yuwa=ba seng paku=et\\
		\textsc{3pl} want tomorrow \textsc{prox}=\textsc{foc} roof nail={\glet}\\
		\glt `They want to nail the roof tomorow.'
		\ex \gll \textbf{me} sobas=ta me di=saran\\
		\textsc{me} dawn={\glta} {\glme} \textsc{caus}=ascend\\	
		\glt `? in the morning, put [carry it] up.' \jambox*{\href{http://hdl.handle.net/10050/00-0000-0000-0004-1BD7-2}{[narr3\_11:20]}}
	\end{xlist}
	\ex 
	\begin{xlist}
		\ex \gll mier se bara sabua nerunggo\\
		\textsc{3du} {\glse} descend tent inside\\
		\glt `They go down into the tent.'
		\ex \gll \textbf{me} mu mu=at bon taluk\\
		\textsc{me} \textsc{3pl} \textsc{3pl=obj} bring exit\\
		\glt `? they bring them out.' \jambox*{\href{http://hdl.handle.net/10050/00-0000-0000-0004-1B85-F}{[narr5\_4:24]}}
	\end{xlist}
	\ex 
	\begin{xlist}
		\ex \gll mu kor opa se duk\\
		\textsc{3pl} leg {\glopa} {\glse} bump\_into\\
		\glt `They already bumped that leg into something.'
		\ex \gll \textbf{me} mera kasian Rani emun toni {\ob}...{\cb}\\
		\textsc{me} then anyway Rani mother.\textsc{3poss} say\\
		\glt `? then anyway, Rani's mother said...' \jambox*{\href{http://hdl.handle.net/10050/00-0000-0000-0004-1BC3-B}{[conv7\_7:46]}}
	\end{xlist}
	\ex 
	\begin{xlist}
		\ex \gll an se mara ki perusahan=ka bot=kin=ta me\\
		\textsc{1sg} {\glse} move\_seawards \textsc{2pl} company=\textsc{lat} go={\glkin}={\glta} {\glme}\\
		\glt `I went to sea: ``Are you going to the mainland [lit. to the company]?'''
		\ex \gll \textbf{me} an se mara me eh suol-ca se bes o bes mera\\
		\textsc{me} \textsc{1sg} {\glse} move\_seawards {\glme} \textsc{int.quot} back-\textsc{2sg.poss} {\glse} good \textsc{emph} good \textsc{int}\\
		\glt `? I went to sea like: ``How is your back?'' ``Oh fine, of course.''' \jambox*{\href{http://hdl.handle.net/10050/00-0000-0000-0004-1BA3-3}{[conv10\_0:22]}}
	\end{xlist}
\end{exe}	

%mayb e filler:
%me an an urap metko keluatluat an kelua toni mamaun toni eh
%me tiritiri ma toni mas
%me: kasian wis me nyong emun se toni eh

%
%me me: (technically not possbile that first is topic, so exclude)
%me me he manga he me
%me me mier melelu me me mena mu wolnelebor (itu artinya, that is)
%
%me, mu toni he
%me ewai koyet se
%me sontum se tebonggan barani koyet
%me se muawi koyera me se sontum se bubar se selesai

Much remains to be investigated regarding the exact functions of Kalamang \textit{me}. For example, a more thorough analysis is needed to understand what guides the use vs. non-use of \textit{me} with new and given referents, with non-verbal predicates, and when combining clauses. Although \textit{me} is one of the most frequent words in the corpus, which means there is a relative wealth of data available, the mechanisms that regulate its use seem quite refined, and a more sophisticated analysis is therefore outside the scope of this grammar.\is{topic|)}

\section{Focus}\is{focus|(}
\label{sec:a}
Focused constituents are those which the speaker tries to introduce into the discourse \parencite{foley2007}, or ``essential piece[s] of new information'' \parencite[][63]{comrie1989}. Kalamang has two focus markers: the enclitics \textit{=a} and \textit{=ba}.

The enclitic \textit{=a} is a focus marker that typically attaches to the NP or PP.\is{postpositional phrase} Question words often carry this enclitic and are naturally focused. See~(\ref{exe:namana}) and~(\ref{exe:tamatkoa}).

\begin{exe}
	\ex
	{\gll naman=\textbf{a} kiem-an yuwa=at kuet\\
		who=\textsc{foc} basket-\textsc{1sg.poss} \textsc{prox=obj} bring\\
		\glt `Who took my basket?' \jambox*{\href{http://hdl.handle.net/10050/00-0000-0000-0004-1BD1-D}{[stim31\_3:08]}}
	}
	\label{exe:namana}
	\ex \gll mu tamatko=\textbf{a} kajie\\
	\textsc{3pl} where=\textsc{foc} pick\\
	\glt `Where did they pick [chestnuts]?' \jambox*{\href{http://hdl.handle.net/10050/00-0000-0000-0004-1BA2-F}{[conv11\_2:49]}}
	\label{exe:tamatkoa}
\end{exe}

It is also common for \textit{=a} to occur on the object in a question-answer pair like~(\ref{exe:terques}), where obviously the piece of new information is that which the interlocutor is asking about, \textit{kapurui}, the name of a dish.

\begin{exe}
	\ex \glll ``Ki nebara paruotkin?'' Mu he toni mu kapurui mera paruotkin.\\
	ki neba=at=\textbf{a} paruot=kin mu se toni mu kapurui met=a paruot=kin\\
	\textsc{2pl} what=\textsc{obj=foc} make={\glkin} \textsc{3pl} {\glse} say \textsc{3pl} kapurui \textsc{dist.obj=foc} make={\glkin}\\
	\glt `{``}What do you want to make?'' They said they wanted to make that kapurui.' \jambox*{\href{http://hdl.handle.net/10050/00-0000-0000-0004-1BA4-1}{[conv16\_14:11]}}
	\label{exe:terques}
\end{exe}

We also encounter \textit{=a} on manner adverbial \is{demonstrative!manner}demonstratives (Chapter~\ref{ch:dems}), where there is a natural focus on the part translating as `like this'. Consider~(\ref{exe:wandia}), where the demonstrative is also an answer to a question.
%AH is the pattern with =a  not just that it is on: question words and belonging answer
\begin{exe}
	\ex
	\label{exe:wandia}
	{\glll ``Tamandi nina?'' ``Wandi, wandia paru!''\\
		tamandi nina wandi wandi=\textbf{a} paru\\
		how grandmother like\_this like\_this=\textsc{foc} do.\textsc{imp}\\
		\glt `{``}How, grandmother?'' ``Like this, do like this.''' \jambox*{\href{http://hdl.handle.net/10050/00-0000-0000-0004-1BB7-9}{[conv19\_2:38]}}
	}
\end{exe}

(\ref{exe:kiafoc}) and~(\ref{exe:bonafoc}), together with~(\ref{exe:terques}) above, show that focus marker \textit{=a} comes after postpositions\is{postposition}. %also after poss, but before kin?

\begin{exe}
	\ex \gll emun se sap=ki=\textbf{a} mara∼mara\\
	mother.\textsc{3poss}  {\glse} stick=\textsc{ins=foc} move\_landwards∼\textsc{prog}\\
	\glt `His mother came walking with a stick.' \jambox*{\href{http://hdl.handle.net/10050/00-0000-0000-0004-1BA3-3}{[conv10\_22:07]}}
	\label{exe:kiafoc}
	\ex \gll mier pas kahen kon=bon=\textbf{a} min\\
	\textsc{3du} woman tall one=\textsc{com=foc} sleep\\
	\glt `She and a tall woman are sleeping.' \jambox*{\href{http://hdl.handle.net/10050/00-0000-0000-0004-1BBC-4}{[narr24\_5:13]}}
	\label{exe:bonafoc}
\end{exe}

The clitic \textit{=ba} is a secondary focus marker found only on demonstratives (example~\ref{exe:yuwaba}), proper names\is{names!personal names} (example~\ref{exe:tomib}), question word \textit{naman} `who' and question word root \textit{tama} (example~\ref{exe:puloro}). It is much rarer than \textit{=a}. \textit{=ba} is often, but not exclusively, used with questions. 

\begin{exe}
	\ex \gll mungkin dodon-un yuwa=\textbf{ba} neba=ko pue to semen=ko\\
	maybe clothes-\textsc{3poss} \textsc{prox=foc} \textsc{ph=loc} hit right concrete=\textsc{loc}\\
	\glt `Maybe these clothes of his hit the whatsit, right, the concrete.' \jambox*{\href{http://hdl.handle.net/10050/00-0000-0000-0004-1BC3-B}{[conv7\_3:16]}}
	\label{exe:yuwaba}
	\ex \gll wat na=ten yuwa naman=a yuwa {\ob}...{\cb} Tomi=\textbf{ba} ye\\
	coconut consume=\textsc{at} \textsc{prox} who=\textsc{foc} \textsc{prox} {} Tomi=\textsc{foc} or\\
	\glt `[Someone] who is drinking coconut here, who is this, Tomi?' \jambox*{\href{http://hdl.handle.net/10050/00-0000-0000-0004-1BE7-5}{[stim42\_14:22]}}
	\label{exe:tomib}
	\ex	\gll pulor opa ka tu=ta me tama=\textbf{ba}-kadok os-kadok\\
	betel\_leaf \textsc{dem} \textsc{2sg} pound-\textsc{ta} {\glme} where=\textsc{foc}-side beach-side\\
	\glt `That betel leaf that you were pounding, on which side is it, on the beach-side?' \jambox*{\href{http://hdl.handle.net/10050/00-0000-0000-0004-1BBD-5}{[conv12\_8:51]}}
	\label{exe:puloro}
\end{exe}

The meaning of \textit{=ba} as opposed to \textit{=a} remains unclear. One also finds mixing of the two focus markers, as in~(\ref{exe:mantanba}).

\begin{exe}
	\ex \glll ``Namana somin?'' ``Ge o, Tete Mantanba, Tete Loklomin.''\\
	naman=\textbf{a} somin ge o tete Mantan=\textbf{ba} tete Loklomin\\
	who=\textsc{foc} die no \textsc{emph} grandfather Mantan=\textsc{foc} grandfather Loklomin\\
	\glt `{``}Who died?'' ``Oh, grandfather Mantan, grandfather Loklomin.''' \jambox*{\href{http://hdl.handle.net/10050/00-0000-0000-0004-1BC3-B}{[conv7\_0:47]}}
	\label{exe:mantanba}
\end{exe}	 

\textit{=ba} is also used in a filler construction consisting of proximal demonstrative \textit{wa}, \textit{=ba} and progressive \textit{=teba}.

\begin{exe}
	\ex \gll ma tok \textbf{wa=ba=teba}\\
	\textsc{3sg} still \textsc{prox=foc=prog}\\
	\glt `He still, eh...' \jambox*{\href{http://hdl.handle.net/10050/00-0000-0000-0004-1BB3-0}{[narr7\_9:33]}}
	\ex \gll neba kaman-un, kaman \textbf{wa=ba=teba}\\
	what grass-\textsc{3poss} grass \textsc{prox=foc=prog}\\
	\glt `What kind of grass, grass eh...' \jambox*{\href{http://hdl.handle.net/10050/00-0000-0000-0004-1BCA-4}{[conv20\_32:44]}}
\end{exe}
\is{focus|)}\is{information structure|)}
