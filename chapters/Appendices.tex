\chapter{Text}
This is a story\footnote{Archived at \url{http://hdl.handle.net/10050/00-0000-0000-0004-1BC6-C}. This text serves as an example. Five other glossed and translated texts are published as \textcite{visser2021mace}. ELAN and XML files with glosses and translations are available for all texts in the corpus at \url{http://hdl.handle.net/10050/00-0000-0000-0004-1B9D-6}.} told by Malik Yarkuran (M) on the 1st of April 2019. It tells about a time when he spotted a small vessel, went towards it and discovered there was a naked tourist on board. The story was recorded in the kitchen of Sebi Yarkuran (S), who also was present during the recording. The storyteller had told this story before, and retold it on tape at the researcher's (E) request.

\section*{Free translation}
Malik: `I went fishing. I looked like this, ``Hey, a ship at the shore up there!'' Then I sailed landwards. I sailed landwards, oh, it was a tourist.' Eline: `What kind of ship?' Malik: `A tourist ship, it was at Tanggor.' Sebi: `A small ship.' Malik: `He came from Pulo Pisang, I asked, from Pulo Pisang. Then we watched. He went to throw the anchor. Then, a man. Is he wearing trousers or not? We were curious, right. So we went. Mas said: ``Let's sail that way to look.'' ``Yes,'' we sailed that way. We sailed that way until we stranded. Stranded, we looked, Mas said: ``Hey, he isn't wearing trousers!'' He wasn't wearing trousers, Mas said: ``Hey, put on trousers!'' He said: ``Yes, yes, yes!'' After getting a towel, he threw it over his legs. His penis dangled.' Sebi: `His bottom was very white.' Malik: `His penis dangled, then we sat chatting. Chatting, he said: ``Do you want to drink?'' We said: ``No.'' He said: ``I just give that to you guys, okay.'' He gave us two bottles. Then he said: ``If it's possible, can I exchange the alcohol for lobster?'' ``Oh, yes, yes, yes.'' Then we sailed back, got two lobsters, brought them back and gave them to him.'

\section*{Glossed text}
\begin{exe}
	\ex 
	\begin{xlist}
		\exi{M:}
		\glll An bo war.\\
		an bo war\\
		\textsc{1sg} go fish\\ 
		\glt `I went fishing.'
	\end{xlist}
	
	
	\ex 
	\begin{xlist}
		\exi{M:}
		\glll An wandi komera: ``Eh kapal kona kabisko osa.''\\
		an wandi komet=ta eh kapal kon=a kibis=ko osa\\
		\textsc{1sg} like\_this look={\glta} \textsc{quot} ship one=\textsc{foc} shore=\textsc{loc} \textsc{up}\\ 
		\glt `I looked like this, ``Hey, a ship at the shore up there!'''
	\end{xlist}
	
	\ex 
	\begin{xlist}
		\exi{M:}
		\glll Terus an se tiri mara.\\
		terus an se tiri mara\\
		then \textsc{1sg} {\glse} sail move\_landwards\\ 
		\glt `Then I sailed landwards.'
	\end{xlist}
	
	\ex 
	\begin{xlist}
		\exi{M:}
		\glll An tiri mara o padahal turisontum.\\
		an tiri mara o padahal turis-sontum\\
		\textsc{1sg} sail move\_landwards \textsc{emph} however tourist-person\\ 
		\glt `I sailed landwards, oh, it was a tourist.'
	\end{xlist}
	
	\ex 
	\begin{xlist}
		\exi{E:}
		\glll Nebakapal?\\
		neba-kapal\\
		what-ship\\ 
		\glt `What kind of ship?'
	\end{xlist}
	
	\ex 
	\begin{xlist}
		\exi{M:}
		\glll Kapal turis, ma Tanggorko.\\
		kapal turis ma Tanggor=ko\\
		ship tourist \textsc{3sg} Tanggor=\textsc{loc}\\ 
		\glt `A tourist ship, it was at Tanggor.'
	\end{xlist}
	
	\ex 
	\begin{xlist}
		\exi{S:}
		\glll Kapal cicauna kon.\\
		kapal cicaun=a kon\\
		ship small=\textsc{foc} one\\ 
		\glt `A small ship.'
	\end{xlist}
	
	\ex
	\begin{xlist}
		\exi{M:}
		\glll Ma Pulo Pisanggata, an gerket, Pulo Pisanggata.\\
		ma Pulo Pisang=ka=ta an gerket Pulo Pisang=ka=ta\\
		\textsc{3sg} Pulo Pisang=\textsc{lat}={\glta} \textsc{1sg} ask Pulo Pisang=\textsc{lat}={\glta}\\ 
		\glt `He came from Pulo Pisang, I asked, from Pulo Pisang.'
	\end{xlist}
	Pulo Pisang (Pulau Pisang, Banana Island) is an island close to Fakfak.
	
	\ex 
	\begin{xlist}
		\exi{M:}
		\glll Terus in se kometkomet.\\
		terus in se komet∼komet\\
		then \textsc{1pl.excl} {\glse} look∼\textsc{prog}\\ 
		\glt `Then we were watching.'
	\end{xlist}
	The speaker is together with a Javanese man he refers to as Mas.
	
	\ex 
	\begin{xlist}
		\exi{M:}
		\glll Ma he ra saorat paruak.\\
		ma se ra saor=at paruak\\
		\textsc{3sg} {\glse} anchor=\textsc{obj} throw\\ 
		\glt `He went to throw the anchor.'
	\end{xlist}
	
	\ex 
	\begin{xlist}
		\exi{M:}
		\glll Terus esnema kon.\\
		terus esnem=a kon\\
		then man=\textsc{foc} one\\ 
		\glt `Then, a man.'
	\end{xlist}
	
	\ex 
	\begin{xlist}
		\exi{M:}
		\glll Ma sungsungat napaki ye ge?\\
		ma sungsung=at napaki ye ge\\
		\textsc{3sg} trousers=\textsc{obj} wear or not\\ 
		\glt `Is he wearing trousers or not?'
	\end{xlist}
	
	\ex 
	\begin{xlist}
		\exi{M:}
		\glll In penasaran to.\\
		in penasaran to\\
		\textsc{1pl.excl} curious right\\ 
		\glt `We were curious, right.'
	\end{xlist}
	
	\ex 
	\begin{xlist}
		\exi{M:}
		\glll Me in se ra.\\
		me in se ra\\
		\textsc{top} \textsc{1pl.excl} {\glse} move\_path\\ 
		\glt `So we went.'
	\end{xlist}
	
	\ex 
	\begin{xlist}
		\exi{M:}
		\glll Mas toni: ``Eh, pi tiri ra komeret.''\\
		Mas toni eh pi tiri ra komet=et\\
		Mas say \textsc{quot} sail move\_path look={\glet}\\
		\glt `Mas said: ``Let's sail that way to look.'''
	\end{xlist}
	
	\ex 
	\begin{xlist}
		\exi{M:}
		\glll ``Yo,'' in se tiri ra.\\
		yo in se tiri ra\\
		yes \textsc{1pl.excl} {\glse} sail move\_path\\ 
		\glt `{``}Yes,'' we sailed that way.'
	\end{xlist}
	
	\ex 
	\begin{xlist}
		\exi{M:}
		\glll Tiri ra sampe nasandar.\\
		tiri ra sampe nasandar\\
		sail move\_path until strand\\ 
		\glt `[We] sailed that way until we stranded.'
	\end{xlist}
	
	
	\ex 
	\begin{xlist}
		\exi{M:}
		\glll Nasandarte me, in komera me, ma toni: ``Eh ma sungsung napakinin!''\\
		nasandar=te me in kome=ta me ma toni eh ma sungsung napaki=nin\\
		strand={\glte} {\glme} \textsc{1pl.excl} look={\glta} {\glme} \textsc{3sg} say \textsc{quot} \textsc{3sg} trousers wear=\textsc{neg}\\ 
		\glt `Stranded, we looked, he [Mas] said: ``Hey, he isn't wearing trousers!'''
	\end{xlist}
	
	\ex 
	\begin{xlist}
		\exi{M:}
		\glll Sungsung napakinin, ma toni: ``Eh sungsunga napakire!''\\
		sungsung napaki=nin ma toni eh sungsung=a napaki=re\\
		trousers wear=\textsc{neg} \textsc{3sg} say \textsc{quot} trousers=\textsc{foc} wear=\textsc{imp}\\ 
		\glt `[He] wasn't wearing trousers, he [Mas] said: ``Hey, put on trousers!'''
	\end{xlist}
	
	\ex 
	\begin{xlist}
		\exi{M:}
		\glll Ma toni: ``Yo, yo, yo!''\\
		ma toni yo yo yo\\
		\textsc{3sg} say yes yes yes\\ 
		\glt `He said: ``Yes, yes, yes!'''
	\end{xlist}
	
	\ex 
	\begin{xlist}
		\exi{M:}
		\glll Ma handuat jieni koyet paruai kor kerunggo.\\
		ma handuk=at jien=i koyet paruak=i kor keit-un=ko\\
		\textsc{3sg} towel=\textsc{obj} get={\gli} finish throw={\gli} leg top-\textsc{3poss}=\textsc{loc}\\
		\glt `After getting a towel, he threw it over his legs.'
	\end{xlist}
	
	\ex 
	\begin{xlist}
		\exi{M:}
		\glll Us naunggang.\\
		us nau=gang\\
		penis \textsc{rec}=hang\\ 
		\glt `[His] penis dangled.'
	\end{xlist}
	\ex 
	\begin{xlist}
		\exi{S:}
		\glll Kasamanun mindi bo irisaet.\\
		kasaman-un mindi bo iris=saet\\
		bottom-\textsc{3poss} like\_that go white=very\\ 
		\glt `His bottom was very white.'
	\end{xlist}
	\ex 
	\begin{xlist}
		\exi{M:}
		\glll Us naunggang terus in se melelu garung.\\
		us nau=gang terus in se melelu garung\\
		penis \textsc{rec}=hang then \textsc{1pl.excl} {\glse} sit chat\\ 
		\glt `[His] penis dangled, then we sat chatting.'
	\end{xlist}
	
	\ex 
	\begin{xlist}
		\exi{M:}
		\glll Garung, ma toni: ``Ki minumkin?''\\
		garung ma toni ki minum=kin\\
		chat \textsc{3sg} say \textsc{2pl} drink={\glkin}\\ 
		\glt `Chatting, he said: ``Do you want to drink?'''
	\end{xlist}
	By `drink' is meant `alcoholic drink'.
	
	\ex 
	\begin{xlist}
		\exi{M:}
		\glll In toni: ``Ge.''\\
		in toni ge\\
		\textsc{1pl.excl} say no\\ 
		\glt `We said: ``No.'''
	\end{xlist}
	
	\ex 
	\begin{xlist}
		\exi{M:}
		\glll Ma toni: ``Met me diki rebaet eh.''\\
		ma toni met me di=ki ∅=teba=et eh\\
		\textsc{3sg} say \textsc{dist.obj} {\glme} \textsc{caus}=\textsc{2pl} give={\glteba}={\glet} \textsc{tag}\\
		\glt `He said: ``[I] just give that to you guys, okay.'''
	\end{xlist}
	
	\ex 
	\begin{xlist}
		\exi{M:}
		\glll Ma botal eiri din.\\
		ma botal eir-i di=in ∅\\
		\textsc{3sg} bottle two-\textsc{objqnt} \textsc{caus}=\textsc{1pl.excl} give\\ 
		\glt `He gave us two bottles.'
	\end{xlist}
	
	\ex 
	\begin{xlist}
		\exi{M:}
		\glll Terus ma toni: ``Kalo bisaet bisa natukar udangbon?''\\
		terus ma toni kalo bisa=et bisa natukar udang=bon\\
		then \textsc{3sg} say if can={\glet} can exchange lobster=\textsc{com}\\ 
		\glt `Then he said: ``If it's possible, can [I] exchange [the alcohol] for lobster?''
	\end{xlist}
	
	\ex 
	\begin{xlist}
		\exi{M:}
		\glll ``O yo yo yo.''\\
		o yo yo yo\\
		oh yes yes yes\\ 
		\glt `{''}Oh, yes, yes, yes.'''
	\end{xlist}
	
	\ex 
	\begin{xlist}
		\exi{M:}
		\glll In se koi tiri ran udangat eiri jie kuru mia ma.\\
		in se koi tiri ran udang=at eir-i jie kuru mia ma ∅\\
		\textsc{1pl.excl} {\glse} then sail move\_path lobster=\textsc{obj} two-\textsc{objqnt} get bring come \textsc{3sg} give\\ 
		\glt `Then we sailed back, got two lobsters, brought [them] back and gave [them] to him.'
	\end{xlist}
	The speaker and his friend sailed to the live-fish storage place where Mas worked.
\end{exe}
	

\chapter{List of bound morphemes}\label{sec:boundmorphs}
Morphophonological rules applying to these bound morphemes (affixes and clitics) include lenition, voicing, velarisation and elision. See §\ref{sec:morphphon} for description and exemplification of all rules, and Chapter~\ref{ch:word} for morphological processes.

\begin{longtable}{*4{l}}
	\caption{Affixes\label{tab:boundmorphsa}}\\
        \lsptoprule
        form & allomorphs & function & reference\\ \midrule\endfirsthead
        \midrule form & allomorphs & function & reference\\\midrule\endhead%
        \endfoot\lspbottomrule\endlastfoot
        \textit{-ahutak} & \textendash & restricting pronoun & §\ref{sec:alonepron}\\
        \textit{-an} & \textendash & \textsc{1sg.poss} & Ch~\ref{ch:poss}\\
        \textit{-bes} & \textendash & quantity demonstrative & §\ref{sec:mqqd}\\
        \textit{-ca} & \textit{-ja}, \textit{-ya} & \textsc{2sg.poss} & Ch~\ref{ch:poss}\\
        \textit{-ce} & \textit{-je}, \textit{-ye} & \textsc{2pl.poss} & Ch~\ref{ch:poss}\\
        \textit{-et} & \textendash & agentive nominaliser & §\ref{sec:nounorig}\\
        \textit{-gan} & \textit{-nggan} & `all' & §§\ref{sec:allprons}, \ref{sec:quantinfl}\\
        \textit{-i} & \textendash & \textsc{objqnt} & §\ref{sec:numobj}\\
        \textit{-mahap} & \textendash & `all' & §\ref{sec:quantall}\\
        \textit{-mun} & \textendash & \textsc{proh} & §\ref{sec:proh}\\
        \textit{-mur} & \textendash & \textsc{kin.pl} & §\ref{sec:kinterms}\\
        \textit{-naninggan} & \textendash & encompassing pronoun & §\ref{sec:allprons}\\
        \textit{-ndi} & \textendash & `like' & §\ref{sec:mqqd}, \ref{sec:wcq}\\
        \textit{-pe} & \textit{-we}, \textit{-be} & \textsc{1pl.incl.poss} & Ch~\ref{ch:poss}\\
        \textit{-re} & \textendash & apprehensive & §\ref{sec:apprmood}\\
        \textit{-rip} & \textendash & degree demonstrative & §\ref{sec:mqqd}\\
        \textit{-sen} & \textendash & quantity demonstrative & §\ref{sec:mqqd}\\
        \textit{-te} & \textit{-re}, \textit{-de} & distributive & §\ref{sec:quantinfl}\\
        \textit{-un} & \textendash & \textsc{3poss}, \textsc{1pl.excl.poss} & Ch~\ref{ch:poss}\\
        \textit{-un} & \textendash & \textsc{nmlz} & §\ref{sec:der}\\ \midrule 
        \textit{al-} & \textendash & \textsc{clf\_strip} & §\ref{sec:clf}\\
        \textit{ar-} & \textendash & \textsc{clf\_stem} & §\ref{sec:clf}\\
        \textit{ep-} & \textit{ew-} & \textsc{clf\_group} &  §\ref{sec:clf}\\
        \textit{et-} & \textit{er-} & \textsc{clf\_an} &  §\ref{sec:clf}\\
        \textit{kis-}& \textendash & \textsc{clf\_long} &  §\ref{sec:clf}\\
        \textit{mir-}& \textendash & \textsc{clf\_canoe} &  §\ref{sec:clf}\\
        \textit{nak-}& \textit{na-} & \textsc{clf\_fruit1} &  §\ref{sec:clf}\\
        \textit{nar-}& \textendash & \textsc{clf\_round} &  §\ref{sec:clf}\\
        \textit{pel-}& \textendash &\textsc{clf\_comb} &  §\ref{sec:clf}\\
        \textit{poup-}& \textit{pouw-} & \textsc{clf\_bundle} &  §\ref{sec:clf}\\
        \textit{pur-}& \textendash & \textsc{clf\_piece} &  §\ref{sec:clf}\\
        \textit{rur-}& \textendash &\textsc{clf\_skewer} &  §\ref{sec:clf}\\
        \textit{tabak-} & \textit{taba-}& \textsc{clf\_half} & §\ref{sec:clf}\\
        \textit{tak-}& \textit{ta-}& \textsc{clf\_leaf} &  §\ref{sec:clf}\\
        \textit{tang-}& \textendash & \textsc{clf\_seed} &  §\ref{sec:clf}\\
        \textit{tep-}& \textit{tew-}& \textsc{clf\_fruit2} &  §\ref{sec:clf}\\
\end{longtable}

\begin{longtable}{*4{l}}
	\caption{Clitics\label{tab:boundmorphs}}\\
        \lsptoprule form & allomorphs & function & reference\\\midrule\endfirsthead
        \midrule form & allomorphs & function & reference\\\midrule\endhead
        \endfoot\lspbottomrule\endlastfoot
        \textit{=a} & \textendash & focus & §\ref{sec:a}\\
        \textit{=at} & \textendash & object & §\ref{sec:at}\\
        \textit{=bon} & \textendash & comitative & §\ref{sec:comi}\\
        \textit{=ero} & \textendash &conditional &§\ref{sec:condmood}\\
        \textit{=et} & \textendash &irrealis&§\ref{sec:et}\\
        \textit{=i} & \textendash &predicate linker&§\ref{sec:mvci}\\
        \textit{=in} & \textendash &prohibitive&§\ref{sec:proh}\\
        \textit{=ka} & \textit{=ngga}&lative&§\ref{sec:lat}\\
        \textit{=kap}& \textit{=nggap} &similative&§\ref{sec:simcase}\\
        \textit{=ki} & \textit{=nggi}&benefactive&§\ref{sec:ben}\\
        \textit{=ki} & \textit{=nggi}&instrumental&§\ref{sec:ins}\\
        \textit{=kin} & \textit{=in}&volitional&§\ref{sec:kin}\\		
        \textit{=ko} & \textit{=o}, \textit{=nggo}&locative &§\ref{sec:loc}\\
        \textit{=kongga} & \textit{=ongga}, \textit{=nggongga}&animate lative&§\ref{sec:animloclat}\\
        \textit{=konggo} & \textit{=onggo}, \textit{=nggonggo}&animate locative&§\ref{sec:animloclat}\\		
        \textit{=nan} & \textendash & `also' &§\ref{sec:too}\\	
        \textit{=nin} & \textendash &negation&§\ref{sec:clauneg}\\
        \textit{=saet} & \textendash &`exclusively'&§\ref{sec:tenunal}\\			
        \textit{=sawe(t)} & \textendash &excessive&§\ref{sec:degradv}\\	
        \textit{=ta} & \textit{=ra}, \textit{=da}&nonfinal&§\ref{sec:nfin}\\	
        \textit{=taero} & \textit{=raero}, \textit{=daero}& `even if' &§\ref{sec:condmood}\\
        \textit{=taet} & \textit{=raet}, \textit{=daet}&`more; again'&§\ref{sec:taet}\\
        \textit{=tak} & \textit{=rak}, \textit{=dak}&`just; only'&§§\ref{sec:alonepron}, \ref{sec:quantall}, \ref{sec:distmann}, \ref{sec:mvcuntil}\\
        \textit{=tar} & \textit{=rar}, \textit{=dar}&plural imperative&§\ref{sec:imp}\\
        \textit{=te} & \textit{=re}, \textit{=de}&imperative&§\ref{sec:imp}\\
        \textit{=te} & \textit{=re}, \textit{=de}&nonfinal&§\ref{sec:nfin}\\	
        \textit{=teba}& \textit{=reba}, \textit{=deba}&progressive&§\ref{sec:teba}\\						
        \textit{=ten}& \textit{=ren}, \textit{=den}&attributive&§\ref{sec:attr}\\			
        \textit{=tenden}&\textit{=renden}, \textit{=denden} &`so'&§\ref{sec:consconj}\\
        \textit{=tun}& \textendash &intensifier&§§\ref{sec:extrnoun}, \ref{sec:quantinfl}, \ref{sec:verbred}, \ref{sec:degradv}\\ \midrule 
        \textit{di=} & \textendash & causative & §\ref{sec:di}\\
        \textit{ko=} & \textendash & applicative & §\ref{sec:appl}\\
        \textit{nak=} & \textendash & `just'& dictionary\\
        \textit{nau=} & \textendash & reciprocal &§\ref{sec:recp}\\
        \textit{ma=} & \textendash & causative & §\ref{sec:caus}\\
\end{longtable}



\chapter{Corpus}\is{corpus}
\label{sec:corpusapp}
This section provides an overview of the transcribed recordings and elicited data in the corpus, with their corpus tags, titles, length and number of words.

{
	\footnotesize
	\begin{longtable}{p{1cm}p{7cm}EE}
		\caption{Naturalistic recordings} \label{tab:vid}
		\endfirsthead 
		\midrule
		\endhead 
		\lsptoprule 
		&  &mm:ss&words\\
		\midrule\multicolumn{4}{c}{Stimulus-based recordings}\\\midrule
		stim 1 & Jackal and crow narrated by Binkur mama & 1:15&84\\
		stim 2 & Jackal and crow narrated by Mohtar pu bapak & 4:35&247\\
		stim 3 & Jackal and crow retold by Mohtar pu bapak & 2:58&341\\
		stim4 &	Family problems part 1: Erna pu bapak and Bilal pu bapak & 5:41&382\\
		%	stim5 &	Family problems part 1: Mohtar pu bapak and nenek & 14:29\\ not transcr
		stim6 &	Family problems part 2: Erna pu bapak and Bilal pu bapak & 19:32&788\\
		stim7 &	Family problems part 2: Mohtar pu bapak and Om Nos & 10:40&1468\\
		%			stim8 &	Family problems part 2: Mohtar pu bapak and nenek
		%			stim9 &	Family problems part 3: Erna pu bapak and Bilal pu bapak & \\
		%			stim10 &	Family problems part 3: nenek
		%			stim11 &	Family problems part 3: Mohtar pu bapak
		stim12	& Family problems part 3: Mohtar pu bapak and Om Nos & 7:29&706\\
		stim13	& Farm animals: Erna pu bapak and Bilal pu bapak & 3:48&283\\
		stim14	& Farm animals: Binkur pu mama and Mohtar pu bapak & 2:55&113\\
		stim15	& Discussing fishing gear 1 & 5:45&272\\
		stim16	& Discussing fishing gear 2 & 6:24&193\\
		%	stim17	& Frog, where are you narrated by Om Nos
		%	stim18	& Frog, where are you narrated by Om Nos
		%	stim19	& Frog, where are you narrated by Bilal pu bapak
		stim20	& Frog, where are you narrated by Mohtar's father& 6:46&394\\
		stim21	& Frog, where are you narrated by Om Nos for Yeni & 4:30&502\\
		%			stim22	& Japanese mute video narration
		%			stim23	& Japanese mute video narration
		stim24	& Japanese mute video narration by Mohtar's father & 3:37&294\\
		stim25	& Man and tree picture matching 1& 9:41&293\\
		stim26	& Man and tree picture matching 2& 9:08&522\\
		stim27	& Man and tree picture matching 3& 17:28&454\\
		%			stim28	Pear movie narrated by Erna's father
		stim29	& Pear movie narrated by Om Nos and Bilal's father & 2:30&296\\
		stim30	& Pear movie narrated by Djusman & 2:25&241\\
		stim31	& Pear movie narrated by Binkur's mother & 3:24&338\\
		%			stim32	Pear move narrated by a grandmother
		stim33	& Pear movie narrated by Mohtar's father & 2:00&160\\
		stim34	& Pear movie narrated by Ruslan's grandmother & 1:24&105\\
		stim35 & Route description from Arepner to the harbour & 2:25&118\\
		stim36	& Route description from the school to Arepner & 1:49&106\\
		stim37	& Route description from Tat to the school & 1:28&108\\
		stim38	& Space games & 12:22&547\\
		stim39	& Tinker toy picture matching 1 & 2:05&170\\
		stim40	& Tinker toy picture matching 2 & 2:55 &84\\
		%		stim41	Village pictures picture matching
		stim42	& Village pictures picture matching 2 & 17:15&1248\\
		stim43	& Village pictures picture matching 3 & 23:32&862\\
		stim44	& Wooden man picture matching & 2:40&190\\
		stim45	& Wooden man picture matching & 2:02&89\\ 
		subtotal && 3:22:28&11998\\
		\midrule\multicolumn{4}{c}{Narratives}\\\midrule
		narr1 &	Remembering the dead & 8:24&489\\ 
		narr2 &	Marriage negotiations & 14:09&1296\\
		narr3 &	The rituals for putting the roof on a house & 13:55&1338\\
		narr4 &	A wedding & 8:05&712\\
		narr5 &	Wedding rituals & 5:04&449\\
		narr6 &	How to build a bamboo house & 6:38&319\\
		narr7 &	How to build a house &  14:48&1357\\
		narr8 &	When me and my husband went fishing & 7:20&600\\
		narr9 &	Making fried cookies part 1 & 7:01&98\\
		narr10 &	Making fried cookies part 2 & 5:02&30\\
		narr11 &	How to weave mats and baskets & 3:20&238\\
		narr12 &	Nutmegs& 9:20&716\\
		narr13 &	Offerings in the nutmeg plantations&5:42&541\\
		narr14 &	How to make a wooden canoe&6:22&725\\
		%	narr15 &	Kuawi&8:53\\
		narr16 &	Malik's funny story about cigarettes&5:53&467\\ 
		narr17 &	Malik's funny story about the naked tourist&2:40&273\\
		narr18 &	Makuteli: birds on a boat&21:04&1180\\
		narr19 &	Linglong: Monkey and Cuscus sell firewood&16:59&1502\\
		narr20 &	Cassowary and Dog&4:24&363\\
		narr21 &	Crab& 5:34&527\\
		narr22 &	Kuawi&8:53&943\\
		narr23 &	The woman who turned into a lime&6:53&755\\
		narr24 &	Kelengkeleng woman&6:23&535\\
		narr25 &	The money-defecating cow&10:23&1181\\
		narr26 &	Married to a mermaid& 22:10&1787\\
		narr27 &	The providing tree&4:38&542\\
		narr28 &	Suagibaba&13:14&1378\\
		narr29 &	Finding water at Sui&8:00&820\\
		narr30 &	The talking coconut&2:08&228\\
		narr31 &	Traditional medicines that grow on Tat&6:39&505\\
		narr32 &	When I went to cure someone&2:28&158\\
		narr33 &	Leaf medicines part 1&6:16&253\\
		narr34 &	Leaf medicines part 2&5:30&367\\
		narr35 &	Leaf medicines part 3&6:23&426\\
		narr36 &	Leaf medicines part 4&2:22&102\\
		narr37 &	Route description from Mas to Antalisa&5:15&462\\
		narr38 &	Route description within Mas&2:08&144\\
		narr39 &	Why the crow is black&4:35&330\\
		narr40 &	Japanese bombings in WWII&22:45&1529\\
		narr41 &	What I did yesterday&2:42&230\\
		narr42 &	The last two canoes Om Nos built&16:00&1704\\
		narr43 &	When I was young&4:50&288\\
		narr44 &	Fishing and lobster diving with Keica&28:10&3909\\
		narr45 &	When Mayor went to Fakfak to get loans&4:37&417\\
		narr46 & How to make the frame of a house &5:20&509\\
		subtotal && 6:16:02&32422\\
        \midrule\multicolumn{4}{c}{Conversations}\\\midrule
		conv1 &	Netfishing 1 & 7:30&439\\
		conv2 &	Netfishing 2 & 7:30&346\\
		conv3 &	Netfishing 3 & 7:30&425\\
		conv4 &	Netfishing 4 & 7:30&629\\
		conv5 &	Netfishing 5 & 7:30&477\\
		%		conv6 &	Netfishing 6
		conv7 &	The Funeral of Tete Loklomin & 15:08&2109\\
		conv8 &	Tenggelele ritual& 5:36&511\\
		conv9 &	Binkur mama and Bakri mama talk current affairs & 32:45&3537\\
		conv10 &	A conversation about fish & 23:02&3345\\
		conv11 & A conversation about chestnuts&6:41&1074\\
		conv12 &	A kitchen conversation between two grandmothers&22:20&1901\\
		conv13 &	A conversation about rice&13:08&1898\\
		conv14 &	Two grandfathers talk current affairs&9:16&648\\
		conv15 &	A conversation about cooking aubergine and papaya&6:27&588\\
		conv16 &	A conversation about cooking vegetables&15:50&1445\\
		conv17 &	How to weave a wallet part 1&45:00&1014\\
		conv18 &	How to weave a wallet part 2&12:30&69\\
		conv19 &	How to weave a wallet part 3&24:00&292\\
		conv20 &	Mohtar's father and Lamani's father discuss root medicines&49:07&4084\\
		conv21 &	Boat trip around Karas Island 1& 07:30&190\\
		conv22 &	Boat trip around Karas Island 10&1:29&41\\
		conv23 &	Boat trip around Karas Island 12&0:39&14\\
		conv24 &	Boat trip around Karas Island 13&0:44&14\\
		conv25 &	Boat trip around Karas Island 14&4:51&47\\
		conv26 &	Boat trip around Karas Island 6&0:41&16\\
		conv27 &	Boat trip around Karas Island 7&7:30 &33\\
		conv28 &	Boat trip around Karas Island 8&7:30&100\\
		subtotal && 5:49:14&25286\\ \midrule
		total && 15:27:44&69706\\ \lspbottomrule
	\end{longtable}



\begin{table}
	\caption{Used stimuli and questionnaires, original source, corpus tag} \label{tab:stim}
	\footnotesize
		\begin{tabularx}{\textwidth}{llQ}
			\lsptoprule 
			\multicolumn{3}{c}{Questionnaires}\\
			\midrule
			Binominals & \textcite{pepper2017}& \href{http://hdl.handle.net/10050/00-0000-0000-0004-1C4D-3}{bin}\\
			Demonstratives & \textcite{wilkins1999}& \href{http://hdl.handle.net/10050/00-0000-0000-0004-1C52-A}{thi}\\
			Iamitives and nondums & \textcite{veselinova2017}& \href{http://hdl.handle.net/10050/00-0000-0000-0004-1C55-7}{iam}\\
			Idematives & \textcite{berg2016}& \href{http://hdl.handle.net/10050/00-0000-0000-0004-1C60-A}{idem}\\
			Naming & Handschuh p.c. & \href{http://hdl.handle.net/10050/00-0000-0000-0004-1C60-A}{nam}\\
			Negation & Veselinova and Miestamo& \href{http://hdl.handle.net/10050/00-0000-0000-0004-1C60-A}{neg, neg19}\\
			Relative clauses & \textcite{downing2010} & \href{http://hdl.handle.net/10050/00-0000-0000-0004-1C60-A}{rel}\\
			TMA & \textcite{dahl1985}& \href{http://hdl.handle.net/10050/00-0000-0000-0004-1C62-D}{tam}\\
			Valency & \textcite{valpal}& \href{http://hdl.handle.net/10050/00-0000-0000-0004-1C65-5}{val}\\\midrule 
			\multicolumn{3}{c}{Picture-matching tasks}\\\midrule                            
			Man and tree \& Space games & \textcite{levinson1992} & \href{http://hdl.handle.net/10050/00-0000-0000-0004-1BAC-8}{stim13}, \href{http://hdl.handle.net/10050/00-0000-0000-0004-1BAB-6}{stim14} \href{http://hdl.handle.net/10050/00-0000-0000-0004-1B9D-6}{stim25-27}, \href{http://hdl.handle.net/10050/00-0000-0000-0004-1B9D-6}{stim38-40}, \href{http://hdl.handle.net/10050/00-0000-0000-0004-1BE7-5}{stim42}\\\midrule 
			%			Village pictures& self-made&\\ %corpus link?
			\multicolumn{3}{c}{Picture stimuli}\\\midrule
			Family problems & \textcite{carroll2009}& \href{http://hdl.handle.net/10050/00-0000-0000-0004-1BA8-B}{stim4}, \href{http://hdl.handle.net/10050/00-0000-0000-0004-1BA9-9}{stim6}, \href{http://hdl.handle.net/10050/00-0000-0000-0004-1BAA-C}{stim7}, \href{http://hdl.handle.net/10050/00-0000-0000-0004-1BB0-D}{stim12}\\
			Focus & via Arthur Holmer & \href{http://hdl.handle.net/10050/00-0000-0000-0004-1C60-A}{foc}\\
			Frog story& \textcite{mayer1969}& \href{http://hdl.handle.net/10050/00-0000-0000-0004-1BAF-5}{stim20}, \href{http://hdl.handle.net/10050/00-0000-0000-0004-1B94-F}{stim21}\\
			Jackal and crow& \textcite{kelly2011}& \href{http://hdl.handle.net/10050/00-0000-0000-0004-1B9D-6}{stim1-3}\\
			Topological relations & \textcite{bowerman1992}& \href{http://hdl.handle.net/10050/00-0000-0000-0004-1C63-9}{top}\\\midrule 
			\multicolumn{3}{c}{Video stimuli}\\\midrule
			Cut and break &\textcite{cutbreak2001}& \href{http://hdl.handle.net/10050/00-0000-0000-0004-1C50-E}{cut}\\
			Ditransitives &\textcite{skopeteas2007}& \href{http://hdl.handle.net/10050/00-0000-0000-0004-1B66-5}{notebook 3, page 345}\\
			Japanese story& via Arthur Holmer& \href{http://hdl.handle.net/10050/00-0000-0000-0004-1C9C-7}{stim24}\\
			Motion verbs & \textcite{moverb2001}& \href{http://hdl.handle.net/10050/00-0000-0000-0004-1C56-3}{mot}\\
			Pear movie& \textcite{chafe1975}& \href{http://hdl.handle.net/10050/00-0000-0000-0004-1B9D-6}{stim29-34}\\
			Put &\textcite{bowerman2004}& \href{http://hdl.handle.net/10050/00-0000-0000-0004-1C5E-6}{put1}, \href{http://hdl.handle.net/10050/00-0000-0000-0004-1C5F-7}{put2}\\
			Reciprocal constructions &\textcite{evans2004}& \href{http://hdl.handle.net/10050/00-0000-0000-0004-1C61-7}{rec}\\
			Staged events &\textcite{stageve2001}& \href{http://hdl.handle.net/10050/00-0000-0000-0004-1C64-3}{stag}\\
			\lspbottomrule
		\end{tabularx}
	
\end{table}
%topologies I've contributed to: Naming (Handschuh), incorporation (Van Lier), negation (Veselinova/Miestamo), iamitives etc (Veselinova), numerals (Chan), compounds (Pepper), LexiRumah (Kaiping et al), loose stuff for Antoinette, Phoible, kinship terms (Barth), maybe some apprehensive stuff. + dicitonaria, IDS.
}
\chapter{Wordlist}
\label{ch:wordlist}
Below is a Kalamang-English \isi{wordlist}\is{glossary} with grammatical categories and a reversal entries index. A more elaborate version, with English and Malay translational equivalents, example sentences, pronunciation, variants, cross-references, pictures, scientific names, semantic domains and notes is published as \textcite{dictionaria}. The raw data are archived in \textit{The Kalamang collection} \parencite{vissercorpus}\footnote{At \url{http://hdl.handle.net/10050/00-0000-0000-0004-1BFE-F}.} and the Paradisec archive.\footnote{At \url{http://catalog.paradisec.org.au/repository/EV1}.}

\section*{Wordlist abbreviations}
\begin{tabbing}
	\scshape gram\hspace*{.5ex}\=\hspace*{8cm}\= \kill
	\scshape adv \> adverbial modifier\\
	\scshape clf \> classifier\\
	\scshape cnj \> conjunction\\
	\scshape dem \> demonstrative\\
	\scshape gram  \> grammatical marker\\
	\scshape int \> interjection\\
	\scshape n \> noun\\
	\scshape part \> particle\\
	\scshape phrs \> phrase\\
	\scshape pro \> pronoun\\
	\scshape q \> question word\\
	\scshape qnt \> quantifier\\
	\scshape v \>  regular verb\\
%	\scshape v -n/-t \> irregular verb\\
	\scshape vi \> intransitive verb\\
	\scshape vt \> transitive verb\\
\end{tabbing}

\lohead{}
\rohead{}
\ohead{}
\ihead{}
\chead{\rightmark\hfill\leftmark}
\pagebreak% skip 690b7d7f-f931-4734-98df-760cd1be6945 ([...]-), no pos
% skip 63cf8f5e-09da-460f-b12f-dd7092e15759 (*tem), no pos
% skip 8ebb8f45-0ae0-49c3-9524-d79c8273fabc (kelelen), no pos
% skip 8148c5f7-6340-46c6-864d-70279e20cf2a (kalar), no pos
% skip 937eb9cd-afd2-4450-b6d6-96375efb1b1a (-[...]), no pos
% skip 59b7297b-859a-4021-b7d1-2f336342950c (opina), no pos
% skip 1427b2b3-9763-4d55-93a3-49ffb7ccbe93 ((...)), no pos
% skip 01aa3dd4-2a85-421b-afe9-71b5ef35a053 (tingting), no pos
% skip 95c47d96-14ff-4fa8-a255-4f6466598b86 (serentak), no pos
% skip 59eb8259-860f-4035-8901-91d90d4f312a (kanggup), no pos
% skip a3ad9937-c965-44c4-b4dc-f14944a0e3a7 (dumun), no pos
% skip 1f0620ef-7e54-4347-a5cd-a15a48ffe99a (-kon), no pos
% skip 0bbf16dc-70b1-4cd0-890e-147e5c0f42a8 (kadiri), no pos
% skip e6ff7405-c5d0-456a-97e3-4f0c1b9b540d (-talin-), no pos
\bgroup
% \restorebottom %should take care of uneven vertical starts of the columns. Untested. SN
\section*{Wordlist Kalamang-English}
\raggedright 

\begin{letter}{a}
%------------------------------
\newentry
\headword{=a}%
\pos{gramm}%
\glosses{focus marker}%
%------------------------------
\newentry
\headword{a}%
\pos{int}%
\glosses{interjection}%
%------------------------------
\newentry
\headword{a}%
\pos{int}%
\glosses{filler}%
%------------------------------
%------------------------------
\newentry
\headword{a'a}%
\pos{int}%
\glosses{yes}%
%------------------------------
%------------------------------
%------------------------------
\newentry
\headword{adat}%
\pos{n}%
\glosses{tradition}%
%------------------------------
\newentry
\headword{ade}%
\pos{int}%
\glosses{pejorative interjection}%
%------------------------------
\newentry
\headword{adi}%
\pos{int}%
\glosses{interjection of pain}%
%------------------------------
%------------------------------
%------------------------------
%------------------------------
\newentry
\headword{adu}%
\pos{int}%
\glosses{interjection of surprise or pain}%
%------------------------------
%------------------------------
%------------------------------
\newentry
\headword{afukat}%
\pos{n}%
\glosses{avocado}%
%------------------------------
%------------------------------
%------------------------------
\newentry
\headword{ahat}%
\pos{n}%
\glosses{sunday}%
%------------------------------
%------------------------------
\newentry
\headword{-ahutak}%
\pos{gramm}%
\glosses{alone}%
%------------------------------
%------------------------------
\newentry
\headword{ajar}%
\sensenr{}%
\pos{v}%
\glosses{teach}%
\sensenr{}%
\pos{v}%
\glosses{continue}%
%------------------------------
\newentry
\headword{ak}%
\pos{n}%
\glosses{sea-side}%
%------------------------------
\newentry
\headword{akal}%
\pos{n}%
\glosses{sense}%
%------------------------------
\newentry
\headword{aknar}%
\pos{n}%
\glosses{chest}%
%------------------------------
\newentry
\headword{aknar kangun}%
\pos{n}%
\glosses{collar bone}%
%------------------------------
\newentry
\headword{akpis}%
\pos{n}%
\glosses{convex side}%
%------------------------------
\newentry
\headword{*al}%
\pos{clf}%
\glosses{classifier for strips}%
%------------------------------
\newentry
\headword{*al}%
\pos{n}%
\glosses{string}%
%------------------------------
%------------------------------
\newentry
\headword{alangan}%
\pos{v}%
\glosses{unable to do}%
%------------------------------
\newentry
\headword{alanganrep}%
\pos{v}%
\glosses{look for trouble}%
%------------------------------
\newentry
\headword{alar}%
\pos{n}%
\glosses{fish}%
%------------------------------
%------------------------------
%------------------------------
\newentry
\headword{alkon}%
\pos{qnt}%
\glosses{one string}%
%------------------------------
\newentry
\headword{Almahera}%
\pos{n}%
\glosses{Halmahera}%
%------------------------------
%------------------------------
%------------------------------
\newentry
\headword{am}%
\pos{n}%
\glosses{breast}%
%------------------------------
\newentry
\headword{am belun}%
\pos{n}%
\glosses{nipple}%
%------------------------------
\newentry
\headword{am perun}%
\pos{n}%
\glosses{breast milk}%
%------------------------------
%------------------------------
\newentry
\headword{amdir}%
\pos{n}%
\glosses{garden}%
%------------------------------
\newentry
\headword{amdir komaruk}%
\pos{v}%
\glosses{clear land}%
%------------------------------
%------------------------------
\newentry
\headword{amkeit}%
\pos{v}%
\glosses{give birth}%
%------------------------------
%------------------------------
\newentry
\headword{an}%
\pos{pro}%
\glosses{I}%
%------------------------------
\newentry
\headword{-an}%
\pos{gram}%
\glosses{my}%
%------------------------------
\newentry
\headword{anahutak}%
\pos{pro}%
\glosses{I alone}%
%------------------------------
%------------------------------
%------------------------------
\newentry
\headword{andain}%
\pos{pro}%
\glosses{I alone}%
%------------------------------
\newentry
\headword{Andan}%
\pos{n}%
\glosses{Banda Islands}%
%------------------------------
\newentry
\headword{ang}%
\pos{n}%
\glosses{turban shell}%
%------------------------------
\newentry
\headword{anggas}%
\pos{n}%
\glosses{door}%
%------------------------------
\newentry
\headword{anggas padenun}%
\pos{n}%
\glosses{doorpost}%
%------------------------------
\newentry
\headword{anggon}%
\pos{pro}%
\glosses{my}%
%------------------------------
%------------------------------
%------------------------------
%------------------------------
\newentry
\headword{anka}%
\pos{n}%
\glosses{number}%
%------------------------------
\newentry
\headword{anti}%
\pos{n}%
\glosses{antidote; resistant}%
%------------------------------
\newentry
\headword{anting}%
\pos{n}%
\glosses{earrings}%
%------------------------------
\newentry
\headword{ao}%
\pos{int}%
\glosses{interjection}%
%------------------------------
\newentry
\headword{ap}%
\pos{qnt}%
\glosses{five}%
%------------------------------
%------------------------------
\newentry
\headword{*ar}%
\pos{n}%
\glosses{stem}%
%------------------------------
\newentry
\headword{ar}%
\pos{v}%
\glosses{expel}%
%------------------------------
\newentry
\headword{ar}%
\pos{v}%
\glosses{dive}%
%------------------------------
\newentry
\headword{ar}%
\pos{n}%
\glosses{sound}%
%------------------------------
\newentry
\headword{ar}%
\pos{v}%
\glosses{make a sound}%
%------------------------------
\newentry
\headword{*ar}%
\pos{clf}%
\glosses{classifier for stems}%
%------------------------------
\newentry
\headword{Arabir}%
\pos{n}%
\glosses{Arabir}%
%------------------------------
\newentry
\headword{aragadi}%
\pos{n}%
\glosses{saw}%
%------------------------------
%------------------------------
\newentry
\headword{arat}%
\pos{n}%
\glosses{seam}%
%------------------------------
%------------------------------
%------------------------------
%------------------------------
\newentry
\headword{arekmang}%
\pos{vi}%
\glosses{scream}%
%------------------------------
%------------------------------
\newentry
\headword{aremun}%
\pos{vi}%
\glosses{big}%
%------------------------------
\newentry
\headword{arep}%
\pos{n}%
\glosses{pond; bay}%
%------------------------------
%------------------------------
\newentry
\headword{arepner}%
\pos{n}%
\glosses{Arepner}%
%------------------------------
\newentry
\headword{arerara}%
\pos{int}%
\glosses{interjection for anger}%
%------------------------------
%------------------------------
\newentry
\headword{ariemun}%
\pos{n}%
\glosses{Friday}%
%------------------------------
%------------------------------
%------------------------------
%------------------------------
\newentry
\headword{arun}%
\pos{n}%
\glosses{stem}%
%------------------------------
\newentry
\headword{arwa}%
\pos{n}%
\glosses{spirit}%
%------------------------------
\newentry
\headword{*as}%
\pos{n}%
\glosses{edge}%
%------------------------------
\newentry
\headword{asal}%
\pos{cnj}%
\glosses{as long as}%
%------------------------------
\newentry
\headword{asar}%
\pos{n}%
\glosses{afternoon}%
%------------------------------
\newentry
\headword{asaskon}%
\pos{vi}%
\glosses{loose}%
%------------------------------
%------------------------------
\newentry
\headword{asokmang}%
\pos{v}%
\glosses{short of breath}%
%------------------------------
%------------------------------
\newentry
\headword{asun}%
\pos{n}%
\glosses{edge}%
%------------------------------
\newentry
\headword{=at}%
\pos{gramm}%
\glosses{object marker}%
%------------------------------
\newentry
\headword{atau}%
\pos{cnj}%
\glosses{or}%
%------------------------------
\newentry
\headword{ator}%
\pos{v}%
\glosses{arrange}%
%------------------------------
\newentry
\headword{au}%
\pos{n}%
\glosses{infant}%
%------------------------------
%------------------------------
\end{letter}
\begin{letter}{b}
\newentry
\headword{=ba}%
\pos{gramm}%
\glosses{focus marker}%
%------------------------------
\newentry
\headword{ba}%
\sensenr{}%
\pos{cnj}%
\glosses{but}%
\sensenr{}%
\pos{cnj}%
\glosses{numeral linker}%
%------------------------------
%------------------------------
%------------------------------
%------------------------------
%------------------------------
\newentry
\headword{bak}%
\pos{n}%
\glosses{container}%
%------------------------------
\newentry
\headword{Baki Tanggiun}%
\pos{n}%
\glosses{Baki Tanggiun}%
%------------------------------
%------------------------------
%------------------------------
%------------------------------
\newentry
\headword{bal}%
\pos{n}%
\glosses{dog}%
%------------------------------
%------------------------------
\newentry
\headword{balak}%
\pos{n}%
\glosses{beam}%
%------------------------------
\newentry
\headword{balama}%
\pos{v}%
\glosses{heat.in.fire}%
%------------------------------
%------------------------------
\newentry
\headword{balaok}%
\pos{v}%
\glosses{show}%
%------------------------------
%------------------------------
%------------------------------
\newentry
\headword{balikawuok}%
\pos{n}%
\glosses{green bean}%
%------------------------------
\newentry
\headword{Baliwawa Anggasun}%
\pos{n}%
\glosses{Baliwawa Anggasun}%
%------------------------------
\newentry
\headword{balkawuok}%
\pos{n}%
\glosses{plant}%
%------------------------------
\newentry
\headword{baluku}%
\pos{n}%
\glosses{eel}%
%------------------------------
%------------------------------
\newentry
\headword{banku}%
\pos{n}%
\glosses{bench}%
%------------------------------
%------------------------------
%------------------------------
\newentry
\headword{bara}%
\pos{v}%
\glosses{descend}%
%------------------------------
\newentry
\headword{barahala}%
\pos{v}%
\glosses{lazy}%
%------------------------------
\newentry
\headword{=barak}%
\pos{adv}%
\glosses{too; any; even}%
%------------------------------
\newentry
\headword{barala}%
\pos{n}%
\glosses{illness}%
%------------------------------
\newentry
\headword{barang}%
\pos{n}%
\glosses{turmeric}%
%------------------------------
\newentry
\headword{baranggap}%
\pos{vi}%
\glosses{yellow}%
%------------------------------
%------------------------------
%------------------------------
\newentry
\headword{bareireimun}%
\pos{qnt}%
\glosses{very much}%
%------------------------------
%------------------------------
\newentry
\headword{barotma}%
\pos{vt}%
\glosses{turn}%
%------------------------------
%------------------------------
\newentry
\headword{baru}%
\pos{cnj}%
\glosses{then}%
%------------------------------
%------------------------------
\newentry
\headword{bataku}%
\pos{n}%
\glosses{brick}%
%------------------------------
%------------------------------
\newentry
\headword{bawang iriskapten}%
\pos{n}%
\glosses{garlic}%
%------------------------------
\newentry
\headword{bawang kerkapten}%
\pos{n}%
\glosses{red onion}%
%------------------------------
\newentry
\headword{bayam}%
\pos{n}%
\glosses{spinach}%
%------------------------------
\newentry
\headword{bayas}%
\pos{n}%
\glosses{sea sand}%
%------------------------------
\newentry
\headword{bebak}%
\pos{n}%
\glosses{duck}%
%------------------------------
%------------------------------
\newentry
\headword{bekiem}%
\pos{n}%
\glosses{shoulder}%
%------------------------------
\newentry
\headword{bekiemkang}%
\pos{n}%
\glosses{shoulder blade}%
%------------------------------
\newentry
\headword{Beladar}%
\pos{n}%
\glosses{The Netherlands}%
%------------------------------
\newentry
\headword{belajar}%
\pos{v}%
\glosses{learn}%
%------------------------------
%------------------------------
\newentry
\headword{belbel}%
\pos{vi}%
\glosses{sharp}%
%------------------------------
\newentry
\headword{belek}%
\pos{n}%
\glosses{can}%
%------------------------------
\newentry
\headword{belen}%
\pos{n}%
\glosses{tongue}%
%------------------------------
%------------------------------
%------------------------------
%------------------------------
%------------------------------
%------------------------------
%------------------------------
%------------------------------
%------------------------------
%------------------------------
\newentry
\headword{bes}%
\pos{vi}%
\glosses{good}%
%------------------------------
\newentry
\headword{-bes}%
\pos{gram}%
\glosses{this/that much/many}%
%------------------------------
%------------------------------
\newentry
\headword{bet}%
\pos{n}%
\glosses{goal}%
%------------------------------
%------------------------------
\newentry
\headword{biar}%
\pos{cnj}%
\glosses{even if}%
%------------------------------
%------------------------------
\newentry
\headword{biasa}%
\pos{v}%
\glosses{normal}%
%------------------------------
\newentry
\headword{biawas}%
\pos{n}%
\glosses{plant}%
%------------------------------
%------------------------------
%------------------------------
%------------------------------
\newentry
\headword{bintang}%
\pos{n}%
\glosses{washtub}%
%------------------------------
\newentry
\headword{bintulak}%
\pos{n}%
\glosses{tilefish}%
%------------------------------
\newentry
\headword{bir}%
\pos{n}%
\glosses{beer}%
%------------------------------
\newentry
\headword{bira}%
\pos{n}%
\glosses{bira}%
%------------------------------
\newentry
\headword{birbir}%
\pos{n}%
\glosses{fish}%
%------------------------------
\newentry
\headword{bisa}%
\pos{v}%
\glosses{can}%
%------------------------------
\newentry
\headword{bitko}%
\pos{v}%
\glosses{carry on back}%
%------------------------------
\newentry
\headword{bo}%
\sensenr{}%
\pos{v}%
\glosses{go}%
\sensenr{}%
\pos{v}%
\glosses{until}%
%------------------------------
\newentry
\headword{boda}%
\pos{vi}%
\glosses{stupid}%
%------------------------------
\newentry
\headword{bol}%
\pos{n}%
\glosses{mouth; rim}%
%------------------------------
\newentry
\headword{bola}%
\pos{n}%
\glosses{ball}%
%------------------------------
%------------------------------
%------------------------------
%------------------------------
\newentry
\headword{bolkoyal}%
\pos{v}%
\glosses{eat}%
%------------------------------
\newentry
\headword{bolkul}%
\pos{n}%
\glosses{lip}%
%------------------------------
%------------------------------
\newentry
\headword{bolodak}%
\pos{adv}%
\glosses{just a little}%
%------------------------------
\newentry
\headword{bolon}%
\pos{qnt}%
\glosses{a little}%
%------------------------------
%------------------------------
\newentry
\headword{bon}%
\pos{v}%
\glosses{bring}%
%------------------------------
\newentry
\headword{=bon}%
\pos{gramm}%
\glosses{comitative}%
%------------------------------
\newentry
\headword{bonaras}%
\pos{v}%
\glosses{be angry}%
%------------------------------
\newentry
\headword{bonasau}%
\pos{v}%
\glosses{do; try}%
%------------------------------
\newentry
\headword{boncis}%
\pos{n}%
\glosses{green beans}%
%------------------------------
\newentry
\headword{bor}%
\pos{n}%
\glosses{drill}%
%------------------------------
\newentry
\headword{bor}%
\pos{v}%
\glosses{drill}%
%------------------------------
%------------------------------
\newentry
\headword{borara}%
\pos{v}%
\glosses{front}%
%------------------------------
%------------------------------
\newentry
\headword{borma}%
\pos{vt}%
\glosses{open limbs}%
%------------------------------
\newentry
\headword{borun}%
\pos{n}%
\glosses{road}%
%------------------------------
\newentry
\headword{bot}%
\pos{n}%
\glosses{journey}%
%------------------------------
\newentry
\headword{botal}%
\pos{n}%
\glosses{bottle}%
%------------------------------
%------------------------------
\newentry
\headword{boubou}%
\pos{v}%
\glosses{bathe}%
%------------------------------
\newentry
\headword{boukbouk}%
\pos{v}%
\glosses{bark}%
%------------------------------
\newentry
\headword{bubir}%
\pos{n}%
\glosses{porridge}%
%------------------------------
\newentry
\headword{bugar}%
\pos{n}%
\glosses{fish}%
%------------------------------
\newentry
\headword{buk}%
\pos{n}%
\glosses{book}%
%------------------------------
%------------------------------
%------------------------------
\newentry
\headword{bula}%
\pos{n}%
\glosses{k.o. fish}%
%------------------------------
\newentry
\headword{bunga}%
\pos{n}%
\glosses{flower}%
%------------------------------
\newentry
\headword{bunga arun}%
\pos{n}%
\glosses{blossom}%
%------------------------------
\newentry
\headword{bunga kupukupu}%
\pos{n}%
\glosses{bastard valerian}%
%------------------------------
\newentry
\headword{bunga rampi}%
\pos{n}%
\glosses{pandanus leaf}%
%------------------------------
\newentry
\headword{bungbung}%
\pos{vi}%
\glosses{big heap}%
%------------------------------
\newentry
\headword{buok}%
\pos{n}%
\glosses{betel; betel nut}%
%------------------------------
\newentry
\headword{buok teun}%
\pos{n}%
\glosses{betel nut}%
%------------------------------
\newentry
\headword{buokbuok}%
\pos{v}%
\glosses{chew betel}%
%------------------------------
\newentry
\headword{buoksarun}%
\pos{n}%
\glosses{offering}%
%------------------------------
%------------------------------
\newentry
\headword{burbur}%
\pos{v}%
\glosses{hit}%
%------------------------------
\newentry
\headword{Burewun}%
\pos{n}%
\glosses{Burewun}%
%------------------------------
\newentry
\headword{busbus}%
\pos{n}%
\glosses{parrot}%
%------------------------------
\newentry
\headword{bustang}%
\pos{n}%
\glosses{nose}%
%------------------------------
\newentry
\headword{bustang posun}%
\pos{n}%
\glosses{nostril}%
%------------------------------
\newentry
\headword{but}%
\pos{n}%
\glosses{stairs}%
%------------------------------
%------------------------------
\end{letter}
\begin{letter}{c}
\newentry
\headword{-ca}%
\pos{gram}%
\glosses{your (\textsc{sg})}%
%------------------------------
\newentry
\headword{-ca}%
\pos{n}%
\glosses{man}%
%------------------------------
\newentry
\headword{cam}%
\pos{v}%
\glosses{take care of}%
%------------------------------
\newentry
\headword{cam}%
\pos{n}%
\glosses{tree}%
%------------------------------
\newentry
\headword{-cam}%
\pos{n}%
\glosses{man}%
%------------------------------
\newentry
\headword{campur}%
\pos{v}%
\glosses{mix}%
%------------------------------
\newentry
\headword{canam}%
\pos{n}%
\glosses{man; male}%
%------------------------------
%------------------------------
\newentry
\headword{cangkir}%
\pos{n}%
\glosses{cup}%
%------------------------------
%------------------------------
%------------------------------
%------------------------------
\newentry
\headword{cat}%
\pos{v}%
\glosses{paint}%
%------------------------------
\newentry
\headword{caun}%
\pos{n}%
\glosses{small}%
%------------------------------
\newentry
\headword{-ce}%
\pos{n}%
\glosses{your (\textsc{pl})}%
%------------------------------
\newentry
\headword{cek}%
\pos{v}%
\glosses{check}%
%------------------------------
\newentry
\headword{cengki}%
\pos{n}%
\glosses{clove tree}%
%------------------------------
%------------------------------
%------------------------------
\newentry
\headword{cerita}%
\pos{v}%
\glosses{tell}%
%------------------------------
\newentry
\headword{cerita}%
\pos{n}%
\glosses{story}%
%------------------------------
%------------------------------
%------------------------------
\newentry
\headword{cicaun}%
\pos{n}%
\glosses{small one}%
%------------------------------
\newentry
\headword{cicaun}%
\pos{vi}%
\glosses{small}%
%------------------------------
\newentry
\headword{cici}%
\pos{n}%
\glosses{drop}%
%------------------------------
\newentry
\headword{cigi}%
\pos{n}%
\glosses{k.o. fish hook}%
%------------------------------
%------------------------------
\newentry
\headword{coba}%
\pos{v}%
\glosses{try}%
%------------------------------
%------------------------------
\newentry
\headword{cok}%
\pos{n}%
\glosses{sugar palm}%
%------------------------------
%------------------------------
%------------------------------
%------------------------------
%------------------------------
%------------------------------
%------------------------------
\end{letter}
\begin{letter}{d}
\newentry
\headword{dadir}%
\pos{n}%
\glosses{k.o. fish}%
%------------------------------
%------------------------------
\newentry
\headword{dagim}%
\pos{n}%
\glosses{meat}%
%------------------------------
\newentry
\headword{dakdak}%
\pos{v}%
\glosses{chop}%
%------------------------------
%------------------------------
\newentry
\headword{daladala}%
\pos{n}%
\glosses{shell}%
%------------------------------
%------------------------------
\newentry
\headword{dalang}%
\pos{v}%
\glosses{jump}%
%------------------------------
\newentry
\headword{dalangdalang}%
\pos{v}%
\glosses{bounce}%
%------------------------------
%------------------------------
\newentry
\headword{daluang}%
\pos{n}%
\glosses{bamboo}%
%------------------------------
%------------------------------
\newentry
\headword{damar lelak}%
\pos{n}%
\glosses{tree}%
%------------------------------
\newentry
\headword{Damartimtim}%
\pos{n}%
\glosses{Damartimtim}%
%------------------------------
%------------------------------
\newentry
\headword{dan}%
\pos{v}%
\glosses{bury}%
%------------------------------
\newentry
\headword{dandang}%
\pos{n}%
\glosses{boiler}%
%------------------------------
\newentry
\headword{dare}%
\pos{v}%
\glosses{sink}%
%------------------------------
\newentry
\headword{dareok}%
\pos{v}%
\glosses{swallow}%
%------------------------------
%------------------------------
\newentry
\headword{dari}%
\pos{n}%
\glosses{net}%
%------------------------------
\newentry
\headword{daria}%
\pos{n}%
\glosses{k.o. shell}%
%------------------------------
\newentry
\headword{daru}%
\pos{n}%
\glosses{west}%
%------------------------------
\newentry
\headword{darua}%
\pos{v}%
\glosses{pull out}%
%------------------------------
%------------------------------
\newentry
\headword{daruon}%
\pos{adv}%
\glosses{midday}%
%------------------------------
\newentry
\headword{dauk}%
\pos{n}%
\glosses{sibling-in-law}%
%------------------------------
\newentry
\headword{daun salam}%
\pos{n}%
\glosses{spice}%
%------------------------------
%------------------------------
%------------------------------
\newentry
\headword{dedesi}%
\pos{n}%
\glosses{noose}%
%------------------------------
%------------------------------
%------------------------------
\newentry
\headword{deir}%
\pos{v}%
\glosses{push; bring}%
%------------------------------
\newentry
\headword{dek}%
\pos{v}%
\glosses{dangle}%
%------------------------------
%------------------------------
\newentry
\headword{delepdelep}%
\pos{v}%
\glosses{blink}%
%------------------------------
%------------------------------
\newentry
\headword{desil}%
\pos{n}%
\glosses{planing tool}%
%------------------------------
\newentry
\headword{desil}%
\pos{v}%
\glosses{plane}%
%------------------------------
%------------------------------
\newentry
\headword{di=}%
\pos{gramm}%
\glosses{causative}%
%------------------------------
\newentry
\headword{didir}%
\pos{n}%
\glosses{fireplace}%
%------------------------------
\newentry
\headword{didiras}%
\pos{n}%
\glosses{kitchen}%
%------------------------------
\newentry
\headword{diguar}%
\pos{n}%
\glosses{smoke}%
%------------------------------
%------------------------------
\newentry
\headword{dikolko}%
\pos{v}%
\glosses{move away}%
%------------------------------
\newentry
\headword{dilurpak}%
\pos{n}%
\glosses{month}%
%------------------------------
%------------------------------
\newentry
\headword{din}%
\pos{n}%
\glosses{fire}%
%------------------------------
\newentry
\headword{din paras}%
\pos{n}%
\glosses{flames}%
%------------------------------
\newentry
\headword{din songsong}%
\pos{n}%
\glosses{embers}%
%------------------------------
\newentry
\headword{dinan}%
\pos{v}%
\glosses{burn}%
%------------------------------
%------------------------------
\newentry
\headword{Distrik}%
\pos{n}%
\glosses{Malakuli}%
%------------------------------
\newentry
\headword{diwadiwal}%
\pos{n}%
\glosses{k.o. pandanus}%
%------------------------------
\newentry
\headword{doa}%
\pos{n}%
\glosses{prayer}%
%------------------------------
\newentry
\headword{Dobu}%
\pos{n}%
\glosses{Aru islands}%
%------------------------------
%------------------------------
\newentry
\headword{dodon}%
\pos{n}%
\glosses{things; clothes}%
%------------------------------
%------------------------------
\newentry
\headword{-dok}%
\pos{n}%
\glosses{side}%
%------------------------------
\newentry
\headword{doka}%
\pos{v}%
\glosses{sit and do nothing}%
%------------------------------
\newentry
\headword{doka}%
\pos{n}%
\glosses{heron}%
%------------------------------
%------------------------------
\newentry
\headword{dokadoka}%
\pos{n}%
\glosses{shell}%
%------------------------------
%------------------------------
\newentry
\headword{Dolok}%
\pos{n}%
\glosses{Dolok}%
%------------------------------
\newentry
\headword{don}%
\pos{n}%
\glosses{thing}%
%------------------------------
\newentry
\headword{don iriskap}%
\pos{n}%
\glosses{sugar; white cloth}%
%------------------------------
\newentry
\headword{don konkon}%
\pos{n}%
\glosses{anything}%
%------------------------------
\newentry
\headword{don konkonin}%
\pos{phrs}%
\glosses{it doesn't matter}%
%------------------------------
\newentry
\headword{don penpen}%
\pos{n}%
\glosses{sugar}%
%------------------------------
\newentry
\headword{don pernanan}%
\pos{n}%
\glosses{glass}%
%------------------------------
\newentry
\headword{don yuolyuol}%
\pos{n}%
\glosses{lamp}%
%------------------------------
\newentry
\headword{donenet}%
\pos{n}%
\glosses{black ant}%
%------------------------------
\newentry
\headword{dong}%
\pos{vi}%
\glosses{chewy; tense}%
%------------------------------
\newentry
\headword{dong}%
\pos{vi}%
\glosses{group}%
%------------------------------
\newentry
\headword{donselet}%
\pos{n}%
\glosses{cloth}%
%------------------------------
\newentry
\headword{dorcie}%
\pos{vi}%
\glosses{pulled out}%
%------------------------------
\newentry
\headword{dorma}%
\pos{vt}%
\glosses{pull out}%
%------------------------------
%------------------------------
%------------------------------
\newentry
\headword{dorom}%
\pos{n}%
\glosses{barrel}%
%------------------------------
\newentry
\headword{dowi}%
\pos{n}%
\glosses{seed}%
%------------------------------
%------------------------------
\newentry
\headword{Duan}%
\pos{n}%
\glosses{Duan}%
%------------------------------
\newentry
\headword{dudan}%
\pos{n}%
\glosses{cousin}%
%------------------------------
%------------------------------
\newentry
\headword{dudin}%
\pos{n}%
\glosses{cockroach}%
%------------------------------
%------------------------------
\newentry
\headword{duk}%
\pos{v}%
\glosses{hit}%
%------------------------------
\newentry
\headword{duk}%
\pos{n}%
\glosses{edge}%
%------------------------------
%------------------------------
\newentry
\headword{dumang}%
\pos{v}%
\glosses{explode}%
%------------------------------
%------------------------------
\newentry
\headword{*dun}%
\pos{n}%
\glosses{sibling}%
%------------------------------
\newentry
\headword{duran}%
\pos{n}%
\glosses{durian}%
%------------------------------
\newentry
\headword{duran walanda}%
\pos{n}%
\glosses{soursop}%
%------------------------------
\newentry
\headword{durcie}%
\pos{vi}%
\glosses{with a hole in it}%
%------------------------------
\newentry
\headword{durma}%
\pos{vt}%
\glosses{make a hole}%
%------------------------------
%------------------------------
\end{letter}
\begin{letter}{e}
%------------------------------
\newentry
\headword{e}%
\pos{int}%
\glosses{interjection}%
%------------------------------
\newentry
\headword{e}%
\pos{int}%
\glosses{filler}%
%------------------------------
\newentry
\headword{eba}%
\sensenr{}%
\pos{cnj}%
\glosses{then}%
\sensenr{}%
\pos{cnj}%
\glosses{so that}%
%------------------------------
\newentry
\headword{eba metko}%
\pos{cnj}%
\glosses{and then}%
%------------------------------
%------------------------------
\newentry
\headword{ecua}%
\pos{v}%
\glosses{cry}%
%------------------------------
%------------------------------
\newentry
\headword{eh}%
\pos{int}%
\glosses{quotative}%
%------------------------------
%------------------------------
%------------------------------
\newentry
\headword{-ei}%
\pos{v}%
\glosses{imperative}%
%------------------------------
%------------------------------
%------------------------------
\newentry
\headword{eir}%
\pos{qnt}%
\glosses{two}%
%------------------------------
%------------------------------
\newentry
\headword{eiruk}%
\pos{v}%
\glosses{bend down; kneel}%
%------------------------------
\newentry
\headword{eis}%
\pos{v}%
\glosses{expel}%
%------------------------------
\newentry
\headword{eksuet}%
\pos{n}%
\glosses{thief}%
%------------------------------
\newentry
\headword{eksuet}%
\pos{v}%
\glosses{steal}%
%------------------------------
\newentry
\headword{el}%
\pos{n}%
\glosses{k.o. coarse woven mat}%
%------------------------------
\newentry
\headword{*elak}%
\pos{n}%
\glosses{bottom}%
%------------------------------
\newentry
\headword{elam}%
\pos{n}%
\glosses{firefly}%
%------------------------------
\newentry
\headword{elao}%
\pos{n}%
\glosses{under}%
%------------------------------
%------------------------------
\newentry
\headword{elaun}%
\pos{n}%
\glosses{bottom}%
%------------------------------
\newentry
\headword{elkin}%
\pos{n}%
\glosses{sack}%
%------------------------------
\newentry
\headword{elkin narun}%
\pos{n}%
\glosses{testicles}%
%------------------------------
\newentry
\headword{Elkorom}%
\pos{n}%
\glosses{Elkorom}%
%------------------------------
%------------------------------
%------------------------------
\newentry
\headword{ema}%
\pos{int}%
\glosses{interjection of surprise}%
%------------------------------
\newentry
\headword{ema}%
\pos{n}%
\glosses{mother; aunt; adult woman}%
%------------------------------
\newentry
\headword{ema caun}%
\pos{n}%
\glosses{aunt}%
%------------------------------
\newentry
\headword{ema temun}%
\pos{n}%
\glosses{aunt}%
%------------------------------
%------------------------------
%------------------------------
\newentry
\headword{emgokuk}%
\pos{n}%
\glosses{k.o. bird}%
%------------------------------
%------------------------------
\newentry
\headword{emguk}%
\pos{v}%
\glosses{vomit}%
%------------------------------
\newentry
\headword{emguk}%
\pos{n}%
\glosses{vomit}%
%------------------------------
\newentry
\headword{emnem}%
\pos{n}%
\glosses{old woman}%
%------------------------------
%------------------------------
%------------------------------
\newentry
\headword{emsan}%
\pos{vi}%
\glosses{half-dry}%
%------------------------------
\newentry
\headword{emumur}%
\pos{n}%
\glosses{women}%
%------------------------------
\newentry
\headword{emun}%
\pos{n}%
\glosses{sap}%
%------------------------------
\newentry
\headword{emun}%
\sensenr{}%
\pos{n}%
\glosses{mother.\textsc{3poss}}%
\sensenr{}%
\pos{n}%
\glosses{big}%
%------------------------------
%------------------------------
\newentry
\headword{enem}%
\pos{n}%
\glosses{older or respected woman}%
%------------------------------
\newentry
\headword{enemtumun}%
\pos{n}%
\glosses{female infant}%
%------------------------------
\newentry
\headword{*ep}%
\pos{clf}%
\glosses{classifier for groups}%
%------------------------------
\newentry
\headword{*ep}%
\pos{n}%
\glosses{back}%
%------------------------------
\newentry
\headword{epkadok}%
\pos{n}%
\glosses{backside}%
%------------------------------
\newentry
\headword{epko}%
\pos{n}%
\glosses{behind}%
%------------------------------
\newentry
\headword{epkon}%
\pos{qnt}%
\glosses{one}%
%------------------------------
\newentry
\headword{era}%
\pos{v}%
\glosses{move up}%
%------------------------------
\newentry
\headword{eranun}%
\pos{v}%
\glosses{cannot}%
%------------------------------
%------------------------------
\newentry
\headword{eren}%
\pos{n}%
\glosses{body}%
%------------------------------
%------------------------------
\newentry
\headword{erteng}%
\pos{n}%
\glosses{k.o. small tree snake}%
%------------------------------
\newentry
\headword{eruap}%
\pos{v}%
\glosses{cry}%
%------------------------------
\newentry
\headword{es}%
\pos{n}%
\glosses{ice}%
%------------------------------
\newentry
\headword{esa}%
\pos{n}%
\glosses{father; adult man; uncle}%
%------------------------------
\newentry
\headword{esa caun}%
\pos{n}%
\glosses{uncle}%
%------------------------------
\newentry
\headword{Esa Tanggiun}%
\pos{n}%
\glosses{Esa Tanggiun}%
%------------------------------
\newentry
\headword{esa temun}%
\pos{n}%
\glosses{uncle}%
%------------------------------
\newentry
\headword{esie}%
\pos{int}%
\glosses{yes}%
%------------------------------
\newentry
\headword{eskop}%
\pos{n}%
\glosses{shovel}%
%------------------------------
\newentry
\headword{esmumur}%
\pos{n}%
\glosses{men}%
%------------------------------
\newentry
\headword{esnem}%
\pos{n}%
\glosses{grandfather; man}%
%------------------------------
\newentry
\headword{esnemtumun}%
\pos{n}%
\glosses{male infant}%
%------------------------------
%------------------------------
\newentry
\headword{*et}%
\pos{clf}%
\glosses{classifier for animates}%
%------------------------------
\newentry
\headword{-et}%
\pos{n}%
\glosses{person}%
%------------------------------
\newentry
\headword{et}%
\pos{n}%
\glosses{canoe}%
%------------------------------
\newentry
\headword{=et}%
\pos{gramm}%
\glosses{irrealis}%
%------------------------------
\newentry
\headword{etaman}%
\pos{qnt}%
\glosses{few}%
%------------------------------
%------------------------------
\newentry
\headword{etkon}%
\pos{qnt}%
\glosses{one}%
%------------------------------
%------------------------------
%------------------------------
\newentry
\headword{eun}%
\pos{n}%
\glosses{nest}%
%------------------------------
\newentry
\headword{ewa}%
\pos{v}%
\glosses{speak}%
%------------------------------
\newentry
\headword{ewarom}%
\pos{n}%
\glosses{drool}%
%------------------------------
\newentry
\headword{Ewarong}%
\pos{n}%
\glosses{Ewarong}%
%------------------------------
\newentry
\headword{ewawa}%
\pos{v}%
\glosses{speak}%
%------------------------------
\newentry
\headword{ewun}%
\pos{n}%
\glosses{tree trunk; base}%
%------------------------------
\end{letter}
\begin{letter}{f}
%------------------------------
%------------------------------
\newentry
\headword{fakurat}%
\pos{v}%
\glosses{much}%
%------------------------------
\newentry
\headword{fakurat}%
\pos{v}%
\glosses{destroy}%
%------------------------------
\newentry
\headword{fam}%
\pos{n}%
\glosses{family name}%
%------------------------------
\newentry
\headword{farlak}%
\pos{n}%
\glosses{tarpaulin}%
%------------------------------
\newentry
\headword{farlu}%
\pos{n}%
\glosses{need}%
%------------------------------
%------------------------------
%------------------------------
\newentry
\headword{fer}%
\pos{n}%
\glosses{fish trap}%
%------------------------------
\newentry
\headword{fiber}%
\pos{n}%
\glosses{fibre boat}%
%------------------------------
%------------------------------
\newentry
\headword{fikfika}%
\pos{n}%
\glosses{palm cockatoo}%
%------------------------------
%------------------------------
\newentry
\headword{filoit}%
\pos{v}%
\glosses{whistle}%
%------------------------------
%------------------------------
\newentry
\headword{foto}%
\pos{v}%
\glosses{film; photo}%
%------------------------------
%------------------------------
\end{letter}
\begin{letter}{g}
\newentry
\headword{gading}%
\pos{n}%
\glosses{plank in boat}%
%------------------------------
\newentry
\headword{gaim}%
\pos{v}%
\glosses{sew leaves}%
%------------------------------
\newentry
\headword{gain}%
\pos{n}%
\glosses{mangosteen}%
%------------------------------
\newentry
\headword{gala}%
\pos{n}%
\glosses{spear}%
%------------------------------
%------------------------------
%------------------------------
\newentry
\headword{galip}%
\pos{n}%
\glosses{bud}%
%------------------------------
\newentry
\headword{gambar}%
\pos{n}%
\glosses{picture}%
%------------------------------
%------------------------------
\newentry
\headword{gampang}%
\pos{n}%
\glosses{happy}%
%------------------------------
\newentry
\headword{-gan}%
\pos{qnt}%
\glosses{all}%
%------------------------------
\newentry
\headword{gang}%
\pos{v}%
\glosses{hang}%
%------------------------------
\newentry
\headword{ganggang}%
\pos{v}%
\glosses{hang}%
%------------------------------
\newentry
\headword{ganggie}%
\pos{v}%
\glosses{lift}%
%------------------------------
%------------------------------
\newentry
\headword{gantor}%
\pos{n}%
\glosses{office}%
%------------------------------
%------------------------------
\newentry
\headword{garawi}%
\pos{n}%
\glosses{k.o. coconut}%
%------------------------------
\newentry
\headword{gare}%
\pos{v}%
\glosses{crawl; slither}%
%------------------------------
%------------------------------
\newentry
\headword{gareor}%
\pos{v}%
\glosses{pour; dump; spill}%
%------------------------------
%------------------------------
%------------------------------
\newentry
\headword{garos}%
\pos{vi}%
\glosses{low}%
%------------------------------
\newentry
\headword{garumbang}%
\pos{n}%
\glosses{tent}%
%------------------------------
\newentry
\headword{garung}%
\pos{v}%
\glosses{talk together}%
%------------------------------
\newentry
\headword{gaus}%
\pos{n}%
\glosses{bamboo}%
%------------------------------
\newentry
\headword{gawar}%
\pos{n}%
\glosses{smell}%
%------------------------------
\newentry
\headword{gawar}%
\pos{v}%
\glosses{smell}%
%------------------------------
\newentry
\headword{gawar}%
\sensenr{}%
\pos{n}%
\glosses{lungs}%
\sensenr{}%
\pos{n}%
\glosses{fish trap}%
%------------------------------
\newentry
\headword{gawawi}%
\pos{n}%
\glosses{chiton}%
%------------------------------
\newentry
\headword{gayam}%
\pos{n}%
\glosses{chestnut}%
%------------------------------
\newentry
\headword{ge}%
\pos{adv}%
\glosses{not; no}%
%------------------------------
\newentry
\headword{ge mera}%
\pos{phrs}%
\glosses{nothing}%
%------------------------------
\newentry
\headword{gedung}%
\pos{n}%
\glosses{village building}%
%------------------------------
\newentry
\headword{geigar}%
\pos{v}%
\glosses{hey-ho}%
%------------------------------
%------------------------------
%------------------------------
\newentry
\headword{gelas}%
\pos{n}%
\glosses{glass}%
%------------------------------
\newentry
\headword{gelas}%
\pos{v}%
\glosses{clear}%
%------------------------------
\newentry
\headword{gelem}%
\pos{v}%
\glosses{yawn}%
%------------------------------
\newentry
\headword{gelembung}%
\pos{n}%
\glosses{bubble}%
%------------------------------
\newentry
\headword{gelemun}%
\pos{n}%
\glosses{tusk}%
%------------------------------
%------------------------------
\newentry
\headword{gen}%
\pos{int}%
\glosses{maybe}%
%------------------------------
\newentry
\headword{genggalong}%
\pos{v}%
\glosses{make noise}%
%------------------------------
%------------------------------
\newentry
\headword{genggueng}%
\pos{v}%
\glosses{scream}%
%------------------------------
%------------------------------
%------------------------------
%------------------------------
\newentry
\headword{geries emun}%
\pos{n}%
\glosses{scrubfowl}%
%------------------------------
\newentry
\headword{gerket}%
\pos{v}%
\glosses{ask}%
%------------------------------
\newentry
\headword{get}%
\pos{cnj}%
\glosses{(if) not}%
%------------------------------
\newentry
\headword{get me}%
\pos{cnj}%
\glosses{if not}%
%------------------------------
%------------------------------
\newentry
\headword{giar}%
\pos{vi}%
\glosses{new}%
%------------------------------
\newentry
\headword{giarun}%
\pos{adv}%
\glosses{first}%
%------------------------------
\newentry
\headword{gier}%
\pos{n}%
\glosses{teeth}%
%------------------------------
\newentry
\headword{gierkawer}%
\pos{n}%
\glosses{gums}%
%------------------------------
\newentry
\headword{gigiwang}%
\pos{n}%
\glosses{earrings}%
%------------------------------
\newentry
\headword{ginana}%
\pos{n}%
\glosses{glass}%
%------------------------------
\newentry
\headword{ginggir}%
\pos{n}%
\glosses{afternoon}%
%------------------------------
%------------------------------
\newentry
\headword{girawar}%
\pos{n}%
\glosses{tree}%
%------------------------------
\newentry
\headword{girgir}%
\pos{v}%
\glosses{run away with woman}%
%------------------------------
%------------------------------
\newentry
\headword{giriaun}%
\pos{vi}%
\glosses{aged}%
%------------------------------
\newentry
\headword{giringgining}%
\pos{n}%
\glosses{bee-eater}%
%------------------------------
\newentry
\headword{git}%
\pos{n}%
\glosses{shadow}%
%------------------------------
%------------------------------
\newentry
\headword{go}%
\sensenr{}%
\pos{n}%
\glosses{place}%
\sensenr{}%
\pos{n}%
\glosses{condition}%
%------------------------------
\newentry
\headword{go dung}%
\pos{phrs}%
\glosses{early morning}%
%------------------------------
\newentry
\headword{go ginggir}%
\pos{phrs}%
\glosses{afternoon}%
%------------------------------
%------------------------------
\newentry
\headword{go git}%
\pos{phrs}%
\glosses{it's cloudy}%
%------------------------------
\newentry
\headword{go kerkap}%
\pos{phrs}%
\glosses{dusk}%
%------------------------------
\newentry
\headword{go saerak}%
\pos{phrs}%
\glosses{empty place}%
%------------------------------
\newentry
\headword{go saun}%
\pos{phrs}%
\glosses{(at) night}%
%------------------------------
\newentry
\headword{go sir}%
\pos{phrs}%
\glosses{clear}%
%------------------------------
\newentry
\headword{go yuol}%
\pos{phrs}%
\glosses{day}%
%------------------------------
\newentry
\headword{gobukbuk}%
\pos{vt}%
\glosses{bark}%
%------------------------------
\newentry
\headword{gocie}%
\pos{v}%
\glosses{stay}%
%------------------------------
\newentry
\headword{godarung}%
\pos{n}%
\glosses{thunder}%
%------------------------------
\newentry
\headword{godelep}%
\pos{n}%
\glosses{lightning}%
%------------------------------
%------------------------------
%------------------------------
\newentry
\headword{gogit}%
\pos{n}%
\glosses{k.o. plant}%
%------------------------------
\newentry
\headword{gokabara}%
\pos{vi}%
\glosses{sweep}%
%------------------------------
\newentry
\headword{gol}%
\pos{n}%
\glosses{ball}%
%------------------------------
\newentry
\headword{golip}%
\pos{n}%
\glosses{fish}%
%------------------------------
\newentry
\headword{golma}%
\pos{vt}%
\glosses{wring}%
%------------------------------
%------------------------------
\newentry
\headword{gonggin}%
\pos{v}%
\glosses{know}%
%------------------------------
%------------------------------
%------------------------------
\newentry
\headword{gonggong}%
\pos{n}%
\glosses{jew's harp}%
%------------------------------
\newentry
\headword{gonggung}%
\pos{v}%
\glosses{call; call out}%
%------------------------------
%------------------------------
\newentry
\headword{goni}%
\pos{n}%
\glosses{sack}%
%------------------------------
\newentry
\headword{goparar}%
\pos{n}%
\glosses{wall}%
%------------------------------
\newentry
\headword{*gor}%
\pos{n}%
\glosses{stalk}%
%------------------------------
%------------------------------
\newentry
\headword{goraruo}%
\pos{v}%
\glosses{in the light}%
%------------------------------
\newentry
\headword{goras}%
\pos{n}%
\glosses{crow}%
%------------------------------
\newentry
\headword{goras}%
\pos{n}%
\glosses{periwinkle shell}%
%------------------------------
%------------------------------
\newentry
\headword{Goras Panuan}%
\pos{n}%
\glosses{Goras Panuan}%
%------------------------------
\newentry
\headword{gorip}%
\pos{n}%
\glosses{fish}%
%------------------------------
\newentry
\headword{gorun}%
\pos{n}%
\glosses{stalk}%
%------------------------------
\newentry
\headword{gos}%
\pos{n}%
\glosses{k.o. plant}%
%------------------------------
\newentry
\headword{Gos Ketkein}%
\pos{n}%
\glosses{Gos Ketkein}%
%------------------------------
%------------------------------
%------------------------------
\newentry
\headword{gosomin}%
\pos{v}%
\glosses{disappeared}%
%------------------------------
%------------------------------
\newentry
\headword{gous}%
\pos{n}%
\glosses{k.o. bamboo}%
%------------------------------
%------------------------------
%------------------------------
\newentry
\headword{Gowien}%
\pos{n}%
\glosses{Tana Besar}%
%------------------------------
\newentry
\headword{gowienkier}%
\pos{n}%
\glosses{wasp nest; beehive}%
%------------------------------
%------------------------------
\newentry
\headword{goyas}%
\pos{n}%
\glosses{fish}%
%------------------------------
%------------------------------
\newentry
\headword{guadang}%
\pos{v}%
\glosses{crawl}%
%------------------------------
\newentry
\headword{gual}%
\pos{n}%
\glosses{fish}%
%------------------------------
\newentry
\headword{guanggarien}%
\pos{v}%
\glosses{look around}%
%------------------------------
\newentry
\headword{guap}%
\pos{n}%
\glosses{sea cucumber}%
%------------------------------
\newentry
\headword{guarten}%
\pos{n}%
\glosses{white person}%
%------------------------------
%------------------------------
%------------------------------
\newentry
\headword{gulas}%
\pos{n}%
\glosses{eel}%
%------------------------------
\newentry
\headword{gulasi}%
\pos{n}%
\glosses{ginger-like root}%
%------------------------------
\newentry
\headword{gunting}%
\pos{n}%
\glosses{rafter}%
%------------------------------
\newentry
\headword{guru}%
\pos{n}%
\glosses{teacher}%
%------------------------------
\newentry
\headword{gusi}%
\pos{n}%
\glosses{vase}%
%------------------------------
\end{letter}
\begin{letter}{h}
\newentry
\headword{ha}%
\pos{int}%
\glosses{what}%
%------------------------------
\newentry
\headword{ha}%
\pos{int}%
\glosses{hesitation marker}%
%------------------------------
\newentry
\headword{habis}%
\pos{cnj}%
\glosses{because; after all}%
%------------------------------
\newentry
\headword{habis}%
\pos{v}%
\glosses{finished}%
%------------------------------
%------------------------------
\newentry
\headword{haidak}%
\pos{adv}%
\glosses{true}%
%------------------------------
\newentry
\headword{hajiwak}%
\pos{n}%
\glosses{name of a month}%
%------------------------------
%------------------------------
%------------------------------
%------------------------------
%------------------------------
%------------------------------
\newentry
\headword{halar}%
\pos{v}%
\glosses{get married}%
%------------------------------
\newentry
\headword{halar}%
\pos{n}%
\glosses{marriage}%
%------------------------------
\newentry
\headword{halus}%
\pos{v}%
\glosses{soft; fine}%
%------------------------------
%------------------------------
%------------------------------
\newentry
\headword{handuk}%
\pos{n}%
\glosses{towel}%
%------------------------------
\newentry
\headword{hanya}%
\pos{adv}%
\glosses{only}%
%------------------------------
%------------------------------
%------------------------------
\newentry
\headword{hari}%
\pos{n}%
\glosses{day}%
%------------------------------
\newentry
\headword{hari minggu}%
\pos{n}%
\glosses{Sunday}%
%------------------------------
%------------------------------
\newentry
\headword{harus}%
\pos{adv}%
\glosses{must}%
%------------------------------
%------------------------------
%------------------------------
%------------------------------
%------------------------------
%------------------------------
%------------------------------
%------------------------------
\newentry
\headword{hi}%
\pos{int}%
\glosses{interjection of enjoyment}%
%------------------------------
%------------------------------
\newentry
\headword{hidup}%
\pos{v}%
\glosses{live}%
%------------------------------
\newentry
\headword{hidup}%
\pos{n}%
\glosses{life}%
%------------------------------
%------------------------------
\newentry
\headword{holang}%
\pos{n}%
\glosses{dish}%
%------------------------------
%------------------------------
\newentry
\headword{hukat}%
\pos{n}%
\glosses{fish net}%
%------------------------------
\newentry
\headword{hukat narun}%
\pos{n}%
\glosses{drivers}%
%------------------------------
%------------------------------
%------------------------------
\end{letter}
\begin{letter}{i}
\newentry
\headword{i-}%
\pos{dem}%
\glosses{demonstrative prefix}%
%------------------------------
\newentry
\headword{=i}%
\pos{gramm}%
\glosses{predicate linker}%
%------------------------------
\newentry
\headword{-i}%
\pos{gramm}%
\glosses{quantifier object marker}%
%------------------------------
%------------------------------
\newentry
\headword{iar}%
\pos{v}%
\glosses{hold}%
%------------------------------
\newentry
\headword{iar}%
\pos{v}%
\glosses{move}%
%------------------------------
\newentry
\headword{iar}%
\pos{n}%
\glosses{cave}%
%------------------------------
\newentry
\headword{iban}%
\pos{n}%
\glosses{worm}%
%------------------------------
%------------------------------
\newentry
\headword{iem}%
\pos{n}%
\glosses{gallbladder}%
%------------------------------
\newentry
\headword{-ier}%
\pos{pro}%
\glosses{two}%
%------------------------------
%------------------------------
\newentry
\headword{ih}%
\pos{int}%
\glosses{tag}%
%------------------------------
\newentry
\headword{ikon}%
\pos{qnt}%
\glosses{some}%
%------------------------------
\newentry
\headword{im}%
\pos{n}%
\glosses{banana}%
%------------------------------
\newentry
\headword{im pawan}%
\pos{n}%
\glosses{k.o. banana}%
%------------------------------
\newentry
\headword{im polun}%
\pos{n}%
\glosses{banana sap}%
%------------------------------
\newentry
\headword{im sarawuar}%
\pos{n}%
\glosses{k.o. banana}%
%------------------------------
\newentry
\headword{im selen}%
\pos{n}%
\glosses{k.o. banana}%
%------------------------------
\newentry
\headword{im sepatu}%
\pos{n}%
\glosses{k.o. banana}%
%------------------------------
\newentry
\headword{im sontum}%
\pos{n}%
\glosses{k.o. banana}%
%------------------------------
\newentry
\headword{im yuol putkansuor}%
\pos{n}%
\glosses{k.o. banana}%
%------------------------------
\newentry
\headword{imanana}%
\pos{n}%
\glosses{giant trevally}%
%------------------------------
\newentry
\headword{imbuang}%
\pos{vi}%
\glosses{many}%
%------------------------------
\newentry
\headword{ime}%
\pos{dem}%
\glosses{distal}%
%------------------------------
\newentry
\headword{imene}%
\pos{dem}%
\glosses{distal}%
%------------------------------
%------------------------------
\newentry
\headword{imol}%
\pos{n}%
\glosses{banana leaf}%
%------------------------------
%------------------------------
%------------------------------
\newentry
\headword{in}%
\pos{n}%
\glosses{name}%
%------------------------------
\newentry
\headword{in}%
\pos{pro}%
\glosses{we (\textsc{ex})}%
%------------------------------
\newentry
\headword{=in}%
\pos{gramm}%
\glosses{prohibitive}%
%------------------------------
\newentry
\headword{inamurin}%
\pos{n}%
\glosses{k.o. illness}%
%------------------------------
\newentry
\headword{inaninggan}%
\pos{pro}%
\glosses{we all (\textsc{ex})}%
%------------------------------
\newentry
\headword{indain}%
\pos{pro}%
\glosses{we alone (\textsc{ex})}%
%------------------------------
\newentry
\headword{ingatan}%
\pos{n}%
\glosses{memory}%
%------------------------------
\newentry
\headword{inggon}%
\pos{pro}%
\glosses{our (\textsc{ex})}%
%------------------------------
\newentry
\headword{Inggrismang}%
\pos{n}%
\glosses{English}%
%------------------------------
%------------------------------
\newentry
\headword{inhutak}%
\pos{pro}%
\glosses{we alone (\textsc{ex})}%
%------------------------------
\newentry
\headword{inier}%
\pos{pro}%
\glosses{we two (\textsc{ex})}%
%------------------------------
%------------------------------
%------------------------------
\newentry
\headword{inye}%
\pos{int}%
\glosses{pejorative interjection}%
%------------------------------
%------------------------------
%------------------------------
\newentry
\headword{irar}%
\pos{n}%
\glosses{mat}%
%------------------------------
\newentry
\headword{irausi}%
\pos{n}%
\glosses{golden trevally}%
%------------------------------
\newentry
\headword{iren}%
\pos{vi}%
\glosses{ripe}%
%------------------------------
\newentry
\headword{irie}%
\pos{qnt}%
\glosses{eight}%
%------------------------------
%------------------------------
\newentry
\headword{iriskap}%
\pos{vi}%
\glosses{white}%
%------------------------------
\newentry
\headword{iriskapkap}%
\pos{vi}%
\glosses{very white}%
%------------------------------
\newentry
\headword{irong}%
\pos{n}%
\glosses{lizard}%
%------------------------------
%------------------------------
%------------------------------
\newentry
\headword{is}%
\pos{v}%
\glosses{rotten}%
%------------------------------
\newentry
\headword{isa}%
\pos{n}%
\glosses{8 PM}%
%------------------------------
\newentry
\headword{isak}%
\pos{n}%
\glosses{eastern koel}%
%------------------------------
%------------------------------
%------------------------------
%------------------------------
%------------------------------
%------------------------------
\newentry
\headword{isis}%
\pos{n}%
\glosses{Kuhl's stingray}%
%------------------------------
%------------------------------
\newentry
\headword{iskap}%
\pos{v}%
\glosses{plane}%
%------------------------------
%------------------------------
\newentry
\headword{iskor}%
\pos{n}%
\glosses{k.o. tree}%
%------------------------------
%------------------------------
%------------------------------
%------------------------------
%------------------------------
\newentry
\headword{istirahat}%
\pos{v}%
\glosses{rest}%
%------------------------------
\newentry
\headword{istop}%
\pos{v}%
\glosses{stop}%
%------------------------------
%------------------------------
\newentry
\headword{istrat}%
\pos{n}%
\glosses{street}%
%------------------------------
\newentry
\headword{istrep}%
\pos{n}%
\glosses{stripe}%
%------------------------------
\newentry
\headword{istup}%
\pos{n}%
\glosses{terrace}%
%------------------------------
\newentry
\headword{iun}%
\pos{n}%
\glosses{seedling}%
%------------------------------
\newentry
\headword{iwala}%
\pos{n}%
\glosses{tree fern}%
%------------------------------
\newentry
\headword{iwang}%
\pos{v}%
\glosses{round}%
%------------------------------
\newentry
\headword{iwora}%
\pos{n}%
\glosses{monitor lizard}%
%------------------------------
%------------------------------
\end{letter}
\begin{letter}{j}
\newentry
\headword{jabul}%
\pos{vi}%
\glosses{be lazy}%
%------------------------------
\newentry
\headword{jadi}%
\pos{cnj}%
\glosses{so}%
%------------------------------
\newentry
\headword{jadi}%
\pos{v}%
\glosses{become}%
%------------------------------
\newentry
\headword{jaga}%
\pos{v}%
\glosses{keep watch}%
%------------------------------
%------------------------------
%------------------------------
%------------------------------
%------------------------------
\newentry
\headword{jam}%
\pos{n}%
\glosses{hour; o'clock}%
%------------------------------
%------------------------------
%------------------------------
\newentry
\headword{jangkut}%
\pos{n}%
\glosses{beard}%
%------------------------------
\newentry
\headword{janji}%
\pos{v}%
\glosses{promise}%
%------------------------------
%------------------------------
\newentry
\headword{jarutu}%
\pos{v}%
\glosses{fishing}%
%------------------------------
%------------------------------
\newentry
\headword{Jawa}%
\pos{n}%
\glosses{Java; Javanese}%
%------------------------------
%------------------------------
\newentry
\headword{jawab}%
\pos{v}%
\glosses{answer}%
%------------------------------
%------------------------------
\newentry
\headword{jendela}%
\pos{n}%
\glosses{window}%
%------------------------------
%------------------------------
%------------------------------
\newentry
\headword{jie}%
\pos{v}%
\glosses{get; buy}%
%------------------------------
%------------------------------
\newentry
\headword{jim}%
\pos{n}%
\glosses{jinn}%
%------------------------------
\newentry
\headword{jojon}%
\pos{n}%
\glosses{k.o. tree}%
%------------------------------
%------------------------------
\newentry
\headword{jonsong}%
\pos{n}%
\glosses{motor boat}%
%------------------------------
%------------------------------
%------------------------------
%------------------------------
\end{letter}
\begin{letter}{k}
\newentry
\headword{ka}%
\pos{pro}%
\glosses{you (\textsc{sg})}%
%------------------------------
\newentry
\headword{=ka}%
\pos{gramm}%
\glosses{lative}%
%------------------------------
\newentry
\headword{kababa}%
\pos{n}%
\glosses{palala}%
%------------------------------
\newentry
\headword{kababur}%
\pos{n}%
\glosses{fruit set}%
%------------------------------
%------------------------------
\newentry
\headword{kabai}%
\pos{n}%
\glosses{shirt}%
%------------------------------
\newentry
\headword{kabara}%
\pos{vt}%
\glosses{sweep}%
%------------------------------
\newentry
\headword{kabarua}%
\pos{v}%
\glosses{watch}%
%------------------------------
\newentry
\headword{kabaruap}%
\pos{n}%
\glosses{grouper}%
%------------------------------
\newentry
\headword{kabaruap kerkapkap}%
\pos{n}%
\glosses{grouper}%
%------------------------------
\newentry
\headword{kabaruap kotamtam}%
\pos{n}%
\glosses{roving coral grouper}%
%------------------------------
\newentry
\headword{kabaruap kuskap}%
\pos{n}%
\glosses{grouper}%
%------------------------------
\newentry
\headword{kabas}%
\pos{n}%
\glosses{another}%
%------------------------------
\newentry
\headword{kabas}%
\pos{vi}%
\glosses{other}%
%------------------------------
\newentry
\headword{kabasar}%
\pos{n}%
\glosses{pufferfish}%
%------------------------------
\newentry
\headword{kabiep}%
\pos{n}%
\glosses{club}%
%------------------------------
\newentry
\headword{kabor}%
\pos{v}%
\glosses{be full}%
%------------------------------
\newentry
\headword{kabor}%
\pos{n}%
\glosses{stomach}%
%------------------------------
\newentry
\headword{kabor elaun}%
\pos{n}%
\glosses{rough side leaf}%
%------------------------------
\newentry
\headword{kabor lalang}%
\pos{phrs}%
\glosses{hungry}%
%------------------------------
\newentry
\headword{kaborko}%
\pos{vi}%
\glosses{pregnant}%
%------------------------------
\newentry
\headword{kabornar}%
\pos{n}%
\glosses{stomach illness}%
%------------------------------
%------------------------------
\newentry
\headword{Kabuep}%
\pos{n}%
\glosses{Kabuep}%
%------------------------------
\newentry
\headword{kabun}%
\pos{n}%
\glosses{k.o. tree}%
%------------------------------
\newentry
\headword{kabun}%
\pos{n}%
\glosses{intestines}%
%------------------------------
\newentry
\headword{kaburun}%
\pos{n}%
\glosses{small unripe fruit}%
%------------------------------
%------------------------------
\newentry
\headword{kacok}%
\pos{vi}%
\glosses{angry}%
%------------------------------
%------------------------------
\newentry
\headword{kadam}%
\pos{n}%
\glosses{striped eel catfish}%
%------------------------------
%------------------------------
%------------------------------
\newentry
\headword{kademor}%
\pos{v}%
\glosses{angry}%
%------------------------------
\newentry
\headword{kaden}%
\pos{n}%
\glosses{body}%
%------------------------------
\newentry
\headword{kaden kies}%
\pos{n}%
\glosses{vein}%
%------------------------------
\newentry
\headword{kaden kieskies}%
\pos{n}%
\glosses{veins}%
%------------------------------
%------------------------------
\newentry
\headword{kaden lalang}%
\pos{vi}%
\glosses{sick}%
%------------------------------
\newentry
\headword{kadenenen}%
\pos{n}%
\glosses{body hair}%
%------------------------------
\newentry
\headword{kadera}%
\pos{n}%
\glosses{chair}%
%------------------------------
%------------------------------
\newentry
\headword{kadok}%
\pos{n}%
\glosses{sarong}%
%------------------------------
\newentry
\headword{-kadok}%
\pos{n}%
\glosses{side; part}%
%------------------------------
\newentry
\headword{kafan}%
\pos{v}%
\glosses{wrap in cloth}%
%------------------------------
\newentry
\headword{kahahen}%
\pos{vi}%
\glosses{very far}%
%------------------------------
%------------------------------
\newentry
\headword{kahalongkahalong}%
\pos{n}%
\glosses{k.o. plant}%
%------------------------------
\newentry
\headword{kahaman}%
\pos{n}%
\glosses{bottom}%
%------------------------------
\newentry
\headword{kahamanpos}%
\pos{n}%
\glosses{anus}%
%------------------------------
%------------------------------
\newentry
\headword{kahaminpat}%
\pos{n}%
\glosses{planks roof}%
%------------------------------
\newentry
\headword{kahen}%
\pos{vi}%
\glosses{far; tall; long}%
%------------------------------
\newentry
\headword{kahetma}%
\pos{vt}%
\glosses{open}%
%------------------------------
%------------------------------
\newentry
\headword{kahutak}%
\pos{pro}%
\glosses{you alone (\textsc{sg})}%
%------------------------------
\newentry
\headword{kai}%
\pos{n}%
\glosses{firewood; medicine}%
%------------------------------
\newentry
\headword{kai kala}%
\pos{n}%
\glosses{k.o. plant}%
%------------------------------
%------------------------------
\newentry
\headword{kai kawas}%
\pos{n}%
\glosses{cotton}%
%------------------------------
\newentry
\headword{kai manis}%
\pos{n}%
\glosses{spice}%
%------------------------------
\newentry
\headword{kai modar}%
\pos{n}%
\glosses{marungga}%
%------------------------------
\newentry
\headword{kai taul}%
\pos{n}%
\glosses{cannonball tree}%
%------------------------------
\newentry
\headword{kain}%
\pos{pro}%
\glosses{your (\textsc{sg})}%
%------------------------------
\newentry
\headword{kainasu}%
\pos{n}%
\glosses{pineapple}%
%------------------------------
\newentry
\headword{kaipur}%
\pos{n}%
\glosses{firewood}%
%------------------------------
\newentry
\headword{kaituki}%
\pos{n}%
\glosses{umbilical ovula}%
%------------------------------
%------------------------------
\newentry
\headword{kajie}%
\pos{v}%
\glosses{pick; weave}%
%------------------------------
%------------------------------
%------------------------------
\newentry
\headword{kalabet}%
\pos{n}%
\glosses{earthworm}%
%------------------------------
%------------------------------
\newentry
\headword{Kalamang}%
\pos{n}%
\glosses{Karas inhabitant; Karas Island}%
%------------------------------
\newentry
\headword{Kalamang lempuang}%
\pos{n}%
\glosses{Karas island}%
%------------------------------
\newentry
\headword{Kalamangmang}%
\pos{n}%
\glosses{Kalamang}%
%------------------------------
\newentry
\headword{Kalamangmang}%
\pos{v}%
\glosses{speak Kalamang}%
%------------------------------
%------------------------------
%------------------------------
%------------------------------
\newentry
\headword{kalaor}%
\pos{n}%
\glosses{front}%
%------------------------------
\newentry
\headword{kalap}%
\pos{n}%
\glosses{wild sugarcane}%
%------------------------------
\newentry
\headword{kalar}%
\sensenr{}%
\pos{vi}%
\glosses{ready}%
\sensenr{}%
\pos{vi}%
\glosses{clear}%
%------------------------------
\newentry
\headword{kalau}%
\pos{cnj}%
\glosses{if}%
%------------------------------
\newentry
\headword{kalawen}%
\pos{vi}%
\glosses{soft}%
%------------------------------
\newentry
\headword{kale}%
\pos{n}%
\glosses{kidneys}%
%------------------------------
\newentry
\headword{kalifan}%
\pos{n}%
\glosses{mat}%
%------------------------------
\newentry
\headword{kaling}%
\pos{n}%
\glosses{fishing hook}%
%------------------------------
\newentry
\headword{kaling}%
\pos{vi}%
\glosses{at an angle}%
%------------------------------
\newentry
\headword{kaling}%
\pos{n}%
\glosses{frying pan}%
%------------------------------
\newentry
\headword{kalip}%
\pos{n}%
\glosses{root vegetable}%
%------------------------------
\newentry
\headword{kalis}%
\pos{n}%
\glosses{rain}%
%------------------------------
\newentry
\headword{kalis}%
\pos{v}%
\glosses{rain}%
%------------------------------
%------------------------------
\newentry
\headword{kalis sasarawe}%
\pos{phrs}%
\glosses{drizzle, light rain}%
%------------------------------
\newentry
\headword{kalis tanggir}%
\pos{n}%
\glosses{rainbow}%
%------------------------------
\newentry
\headword{kalkalet}%
\pos{n}%
\glosses{mosquito}%
%------------------------------
%------------------------------
%------------------------------
\newentry
\headword{kalolang}%
\pos{v}%
\glosses{use plumb rule}%
%------------------------------
\newentry
\headword{kalolang}%
\pos{n}%
\glosses{plumb rule}%
%------------------------------
\newentry
\headword{kalomlomun}%
\pos{vi}%
\glosses{very young}%
%------------------------------
\newentry
\headword{kalomun}%
\pos{vi}%
\glosses{unripe}%
%------------------------------
%------------------------------
%------------------------------
\newentry
\headword{kalot}%
\pos{n}%
\glosses{room}%
%------------------------------
\newentry
\headword{kaloum}%
\pos{vi}%
\glosses{weak as a result of not eating}%
%------------------------------
\newentry
\headword{kalour}%
\pos{n}%
\glosses{custom}%
%------------------------------
\newentry
\headword{kalsum}%
\pos{n}%
\glosses{shell}%
%------------------------------
%------------------------------
%------------------------------
\newentry
\headword{kalung}%
\pos{n}%
\glosses{necklace}%
%------------------------------
\newentry
\headword{kama}%
\pos{dv}%
\glosses{send}%
%------------------------------
%------------------------------
\newentry
\headword{kaman}%
\pos{n}%
\glosses{grass}%
%------------------------------
\newentry
\headword{kaman}%
\pos{n}%
\glosses{disease}%
%------------------------------
\newentry
\headword{kamandi}%
\pos{n}%
\glosses{k.o. fish}%
%------------------------------
\newentry
\headword{kamang}%
\pos{v}%
\glosses{treat}%
%------------------------------
\newentry
\headword{kamanget}%
\pos{n}%
\glosses{medicine man}%
%------------------------------
\newentry
\headword{kamaser}%
\pos{n}%
\glosses{orchid}%
%------------------------------
\newentry
\headword{Kambala}%
\pos{n}%
\glosses{Kambala}%
%------------------------------
%------------------------------
\newentry
\headword{kambanau}%
\pos{n}%
\glosses{waspfish}%
%------------------------------
\newentry
\headword{Kambera}%
\pos{n}%
\glosses{Kambera}%
%------------------------------
\newentry
\headword{Kambur}%
\pos{n}%
\glosses{Kambur}%
%------------------------------
\newentry
\headword{kamel}%
\pos{n}%
\glosses{stingray}%
%------------------------------
\newentry
\headword{kamel kir}%
\pos{n}%
\glosses{black-spotted stingray}%
%------------------------------
\newentry
\headword{kamel muradik}%
\pos{n}%
\glosses{(manta) ray}%
%------------------------------
\newentry
\headword{kamen}%
\pos{vi}%
\glosses{wet}%
%------------------------------
%------------------------------
\newentry
\headword{kamera}%
\pos{v}%
\glosses{record}%
%------------------------------
\newentry
\headword{kamfor}%
\pos{n}%
\glosses{stove}%
%------------------------------
\newentry
\headword{kamis}%
\pos{n}%
\glosses{Thursday}%
%------------------------------
%------------------------------
\newentry
\headword{kamual}%
\pos{n}%
\glosses{k.o. pandanus}%
%------------------------------
\newentry
\headword{kamuamual}%
\pos{n}%
\glosses{needlefish}%
%------------------------------
%------------------------------
\newentry
\headword{kamun}%
\pos{n}%
\glosses{widower}%
%------------------------------
\newentry
\headword{kamung}%
\pos{n}%
\glosses{iron}%
%------------------------------
\newentry
\headword{kan}%
\pos{int}%
\glosses{you.know}%
%------------------------------
\newentry
\headword{kanai}%
\pos{n}%
\glosses{pili nut}%
%------------------------------
\newentry
\headword{kanaisasen}%
\pos{n}%
\glosses{a k.o. fish}%
%------------------------------
\newentry
\headword{kanaisasen}%
\pos{n}%
\glosses{inside of pili nut}%
%------------------------------
%------------------------------
%------------------------------
%------------------------------
\newentry
\headword{kanas}%
\pos{n}%
\glosses{fish}%
%------------------------------
\newentry
\headword{kanas kolkol}%
\pos{n}%
\glosses{fish}%
%------------------------------
\newentry
\headword{Kanastangan}%
\pos{n}%
\glosses{Kanastangan}%
%------------------------------
%------------------------------
%------------------------------
\newentry
\headword{Kandarer}%
\pos{n}%
\glosses{Kandarer}%
%------------------------------
\newentry
\headword{kang}%
\pos{vi}%
\glosses{sharp}%
%------------------------------
\newentry
\headword{kang}%
\pos{v}%
\glosses{fold}%
%------------------------------
\newentry
\headword{kang}%
\pos{n}%
\glosses{bone}%
%------------------------------
\newentry
\headword{*kang}%
\pos{n}%
\glosses{thorn}%
%------------------------------
\newentry
\headword{kanggaran}%
\pos{n}%
\glosses{bamboo floor}%
%------------------------------
\newentry
\headword{kanggarom}%
\pos{vi}%
\glosses{slimy}%
%------------------------------
\newentry
\headword{kanggei}%
\pos{v}%
\glosses{play}%
%------------------------------
\newentry
\headword{kanggeirun}%
\pos{n}%
\glosses{game}%
%------------------------------
\newentry
\headword{kanggeit}%
\pos{n}%
\glosses{game}%
%------------------------------
%------------------------------
%------------------------------
\newentry
\headword{kanggin}%
\pos{n}%
\glosses{vegetable}%
%------------------------------
\newentry
\headword{kangginwele}%
\pos{n}%
\glosses{vegetable}%
%------------------------------
\newentry
\headword{kanggir}%
\pos{n}%
\glosses{eye}%
%------------------------------
%------------------------------
\newentry
\headword{kanggir nenen}%
\pos{n}%
\glosses{eyelashes}%
%------------------------------
\newentry
\headword{kanggir pop}%
\pos{v}%
\glosses{be tired}%
%------------------------------
\newentry
\headword{kanggir pulun}%
\pos{n}%
\glosses{eyelid}%
%------------------------------
\newentry
\headword{kanggir saun}%
\pos{vi}%
\glosses{blind}%
%------------------------------
\newentry
\headword{kanggirar}%
\pos{n}%
\glosses{face}%
%------------------------------
\newentry
\headword{kanggirar}%
\pos{v}%
\glosses{face}%
%------------------------------
\newentry
\headword{kanggirnar}%
\pos{n}%
\glosses{pupil; eyeball}%
%------------------------------
\newentry
\headword{kanggisawuo}%
\pos{v}%
\glosses{wash face}%
%------------------------------
\newentry
\headword{kangguar}%
\pos{vi}%
\glosses{unripe}%
%------------------------------
%------------------------------
\newentry
\headword{kanggur}%
\pos{n}%
\glosses{mouth}%
%------------------------------
\newentry
\headword{kanggursau}%
\pos{v}%
\glosses{rinse mouth}%
%------------------------------
%------------------------------
\newentry
\headword{kanggurun}%
\pos{n}%
\glosses{inside canoe}%
%------------------------------
\newentry
\headword{kanggus}%
\pos{n}%
\glosses{jaw; chin}%
%------------------------------
%------------------------------
\newentry
\headword{kangjie}%
\pos{v}%
\glosses{make rim}%
%------------------------------
\newentry
\headword{kangkanggarek}%
\pos{n}%
\glosses{string}%
%------------------------------
\newentry
\headword{kangkangun}%
\pos{n}%
\glosses{thorns}%
%------------------------------
\newentry
\headword{kangun}%
\pos{n}%
\glosses{thorn}%
%------------------------------
\newentry
\headword{kangun nerunggo}%
\pos{n}%
\glosses{marrow}%
%------------------------------
%------------------------------
\newentry
\headword{kanie}%
\pos{v}%
\glosses{tie}%
%------------------------------
\newentry
\headword{-kaning}%
\pos{vi}%
\glosses{very}%
%------------------------------
\newentry
\headword{kaninggonie}%
\pos{qnt}%
\glosses{nine}%
%------------------------------
%------------------------------
\newentry
\headword{kansuor}%
\pos{qnt}%
\glosses{four}%
%------------------------------
%------------------------------
%------------------------------
%------------------------------
\newentry
\headword{kanung}%
\pos{n}%
\glosses{tortoise}%
%------------------------------
\newentry
\headword{kanus}%
\pos{n}%
\glosses{greatgrandparent}%
%------------------------------
%------------------------------
%------------------------------
\newentry
\headword{kanyuot}%
\pos{n}%
\glosses{clam}%
%------------------------------
\newentry
\headword{=kap}%
\pos{gramm}%
\glosses{similative}%
%------------------------------
\newentry
\headword{kap}%
\pos{vi}%
\glosses{rotten}%
%------------------------------
\newentry
\headword{kapal}%
\pos{n}%
\glosses{ship}%
%------------------------------
\newentry
\headword{kapapet}%
\pos{n}%
\glosses{glassy sweeper}%
%------------------------------
\newentry
\headword{kapas}%
\pos{n}%
\glosses{cotton}%
%------------------------------
\newentry
\headword{kapis}%
\pos{n}%
\glosses{scabies}%
%------------------------------
\newentry
\headword{kapuk}%
\pos{n}%
\glosses{kapok tree}%
%------------------------------
\newentry
\headword{kar}%
\pos{n}%
\glosses{vagina}%
%------------------------------
\newentry
\headword{Karabar Lempuang}%
\pos{n}%
\glosses{Karabar Lempuang}%
%------------------------------
\newentry
\headword{karabubu}%
\pos{n}%
\glosses{bubble}%
%------------------------------
\newentry
\headword{karain}%
\pos{pro}%
\glosses{you alone (\textsc{sg})}%
%------------------------------
\newentry
\headword{karajang}%
\pos{v}%
\glosses{work}%
%------------------------------
\newentry
\headword{karajang}%
\pos{n}%
\glosses{work}%
%------------------------------
\newentry
\headword{karam}%
\pos{n}%
\glosses{cooking utensil}%
%------------------------------
\newentry
\headword{karamba}%
\pos{n}%
\glosses{fish cage}%
%------------------------------
\newentry
\headword{karames}%
\pos{v}%
\glosses{be shy}%
%------------------------------
\newentry
\headword{karamun}%
\pos{n}%
\glosses{k.o. tree}%
%------------------------------
\newentry
\headword{karan}%
\sensenr{}%
\pos{vi}%
\glosses{hard}%
\sensenr{}%
\pos{vi}%
\glosses{deep}%
%------------------------------
%------------------------------
%------------------------------
\newentry
\headword{karanjang}%
\pos{n}%
\glosses{basket}%
%------------------------------
\newentry
\headword{karaok}%
\pos{vi}%
\glosses{crushed}%
%------------------------------
\newentry
\headword{karaonggis}%
\pos{vi}%
\glosses{skinny}%
%------------------------------
\newentry
\headword{karaonggis}%
\pos{vi}%
\glosses{blunt}%
%------------------------------
%------------------------------
\newentry
\headword{karap}%
\pos{v}%
\glosses{wrap}%
%------------------------------
\newentry
\headword{kararak}%
\pos{vi}%
\glosses{dry}%
%------------------------------
\newentry
\headword{kararcie}%
\pos{vi}%
\glosses{broken}%
%------------------------------
\newentry
\headword{karariem}%
\pos{n}%
\glosses{surgeonfish}%
%------------------------------
\newentry
\headword{kararma}%
\pos{vt}%
\glosses{hit}%
%------------------------------
\newentry
\headword{kararu}%
\pos{v}%
\glosses{run}%
%------------------------------
%------------------------------
\newentry
\headword{karebar}%
\pos{n}%
\glosses{red ant}%
%------------------------------
\newentry
\headword{karek}%
\pos{n}%
\glosses{rope}%
%------------------------------
\newentry
\headword{karek ewun saerak}%
\pos{n}%
\glosses{plant}%
%------------------------------
%------------------------------
\newentry
\headword{karena}%
\pos{cnj}%
\glosses{because}%
%------------------------------
\newentry
\headword{kareng}%
\pos{n}%
\glosses{frog}%
%------------------------------
\newentry
\headword{karep}%
\pos{n}%
\glosses{lake}%
%------------------------------
\newentry
\headword{karer}%
\pos{n}%
\glosses{k.o. fish}%
%------------------------------
%------------------------------
\newentry
\headword{kariakibi}%
\pos{n}%
\glosses{sea cucumber}%
%------------------------------
\newentry
\headword{kariemun}%
\pos{n}%
\glosses{cape}%
%------------------------------
%------------------------------
\newentry
\headword{Karing}%
\pos{n}%
\glosses{Karing}%
%------------------------------
\newentry
\headword{Karinggoris}%
\pos{n}%
\glosses{Karinggoris}%
%------------------------------
%------------------------------
\newentry
\headword{karok}%
\pos{n}%
\glosses{branch}%
%------------------------------
\newentry
\headword{karok}%
\pos{v}%
\glosses{cut}%
%------------------------------
\newentry
\headword{karop}%
\pos{n}%
\glosses{arrow}%
%------------------------------
\newentry
\headword{karop}%
\pos{n}%
\glosses{firefly}%
%------------------------------
\newentry
\headword{karop}%
\pos{v}%
\glosses{shoot}%
%------------------------------
\newentry
\headword{karopkarop}%
\pos{n}%
\glosses{bee-eater}%
%------------------------------
\newentry
\headword{karor}%
\pos{n}%
\glosses{hermit crab}%
%------------------------------
\newentry
\headword{Karotkarot}%
\pos{n}%
\glosses{Karotkarot}%
%------------------------------
%------------------------------
\newentry
\headword{karuan}%
\pos{n}%
\glosses{bamboo}%
%------------------------------
\newentry
\headword{karuar}%
\pos{v}%
\glosses{dry}%
%------------------------------
\newentry
\headword{karuar}%
\pos{n}%
\glosses{drying rack}%
%------------------------------
%------------------------------
\newentry
\headword{Karumar}%
\pos{n}%
\glosses{Karumar}%
%------------------------------
\newentry
\headword{karuok}%
\pos{qnt}%
\glosses{three}%
%------------------------------
\newentry
\headword{karyak}%
\pos{n}%
\glosses{blood}%
%------------------------------
\newentry
\headword{kas}%
\pos{n}%
\glosses{fish}%
%------------------------------
\newentry
\headword{kasabiti}%
\pos{n}%
\glosses{squash}%
%------------------------------
%------------------------------
\newentry
\headword{kasalong}%
\pos{n}%
\glosses{two-pointed spear}%
%------------------------------
%------------------------------
\newentry
\headword{Kasambil}%
\pos{n}%
\glosses{Kasambil}%
%------------------------------
\newentry
\headword{kasamin}%
\pos{n}%
\glosses{bird}%
%------------------------------
\newentry
\headword{kasamin naun getgetkadok}%
\pos{n}%
\glosses{bird}%
%------------------------------
\newentry
\headword{kasar}%
\pos{v}%
\glosses{rough}%
%------------------------------
%------------------------------
\newentry
\headword{kasawari}%
\pos{n}%
\glosses{cassowary}%
%------------------------------
%------------------------------
\newentry
\headword{kasawircie}%
\pos{vi}%
\glosses{opened}%
%------------------------------
\newentry
\headword{kasawirma}%
\pos{vt}%
\glosses{pull}%
%------------------------------
%------------------------------
\newentry
\headword{kasep}%
\pos{v}%
\glosses{draw a line; mark}%
%------------------------------
%------------------------------
%------------------------------
\newentry
\headword{kasian}%
\pos{int}%
\glosses{anyway; whatever; poor}%
%------------------------------
\newentry
\headword{Kasiang}%
\pos{n}%
\glosses{Kasiang}%
%------------------------------
%------------------------------
%------------------------------
\newentry
\headword{*kasir}%
\pos{n}%
\glosses{joint}%
%------------------------------
\newentry
\headword{kaskas}%
\pos{n}%
\glosses{sea bird}%
%------------------------------
\newentry
\headword{kasko}%
\pos{v}%
\glosses{clean behind}%
%------------------------------
\newentry
\headword{kasom}%
\pos{n}%
\glosses{tusk shell}%
%------------------------------
\newentry
\headword{kasor}%
\pos{n}%
\glosses{tree}%
%------------------------------
\newentry
\headword{kasorma}%
\pos{vt}%
\glosses{grab}%
%------------------------------
\newentry
\headword{kasotma}%
\pos{vt}%
\glosses{peel}%
%------------------------------
%------------------------------
\newentry
\headword{kastupi}%
\pos{n}%
\glosses{parrot}%
%------------------------------
%------------------------------
%------------------------------
\newentry
\headword{kasuo}%
\pos{v}%
\glosses{intercourse}%
%------------------------------
\newentry
\headword{kasuop}%
\pos{n}%
\glosses{k.o. fish}%
%------------------------------
\newentry
\headword{kasur}%
\pos{adv}%
\glosses{tomorrow}%
%------------------------------
%------------------------------
\newentry
\headword{kasut}%
\pos{n}%
\glosses{bamboo}%
%------------------------------
\newentry
\headword{kasut}%
\pos{v}%
\glosses{kawin}%
%------------------------------
\newentry
\headword{kat}%
\pos{n}%
\glosses{river; lake}%
%------------------------------
\newentry
\headword{kataperor}%
\pos{n}%
\glosses{catapult}%
%------------------------------
\newentry
\headword{katawengga}%
\pos{n}%
\glosses{wild/forest breadfruit}%
%------------------------------
\newentry
\headword{katem}%
\pos{vi}%
\glosses{blurry}%
%------------------------------
%------------------------------
%------------------------------
\newentry
\headword{katuk}%
\pos{n}%
\glosses{k.o. plant}%
%------------------------------
\newentry
\headword{kaul}%
\pos{n}%
\glosses{betel}%
%------------------------------
\newentry
\headword{kawalawalan}%
\pos{n}%
\glosses{k.o. tree}%
%------------------------------
\newentry
\headword{kawar}%
\pos{v}%
\glosses{break}%
%------------------------------
%------------------------------
%------------------------------
\newentry
\headword{kawaram}%
\pos{n}%
\glosses{triggerfish}%
%------------------------------
\newentry
\headword{kawaram boldinggap}%
\pos{n}%
\glosses{k.o. triggerfish}%
%------------------------------
\newentry
\headword{kawaramleit}%
\pos{n}%
\glosses{clown triggerfish}%
%------------------------------
\newentry
\headword{kawarcie}%
\pos{v}%
\glosses{snapped}%
%------------------------------
\newentry
\headword{kaware}%
\pos{v}%
\glosses{scratch}%
%------------------------------
\newentry
\headword{kawaret}%
\pos{n}%
\glosses{drum}%
%------------------------------
\newentry
\headword{kawarma}%
\pos{vt}%
\glosses{break; fold}%
%------------------------------
%------------------------------
\newentry
\headword{kawarsuop}%
\pos{n}%
\glosses{eel}%
%------------------------------
\newentry
\headword{kawaruan}%
\pos{v}%
\glosses{peel}%
%------------------------------
\newentry
\headword{kawas}%
\pos{n}%
\glosses{thread}%
%------------------------------
%------------------------------
\newentry
\headword{kawat}%
\pos{n}%
\glosses{branch; stem}%
%------------------------------
%------------------------------
%------------------------------
\newentry
\headword{kawes}%
\pos{vi}%
\glosses{feel cold}%
%------------------------------
\newentry
\headword{kawetkawet}%
\pos{v}%
\glosses{fold}%
%------------------------------
\newentry
\headword{kawiawi}%
\pos{v}%
\glosses{fan}%
%------------------------------
\newentry
\headword{kawien}%
\pos{n}%
\glosses{mushroom}%
%------------------------------
\newentry
\headword{kawier}%
\pos{n}%
\glosses{cap}%
%------------------------------
\newentry
\headword{kawir}%
\pos{n}%
\glosses{christian}%
%------------------------------
%------------------------------
\newentry
\headword{kawotma}%
\pos{v}%
\glosses{peel}%
%------------------------------
\newentry
\headword{kawuok}%
\pos{n}%
\glosses{bean}%
%------------------------------
\newentry
\headword{kawuok kahahen}%
\pos{n}%
\glosses{asparagus bean}%
%------------------------------
%------------------------------
\newentry
\headword{kayakat}%
\pos{n}%
\glosses{bamboo wall}%
%------------------------------
\newentry
\headword{kayu nani}%
\pos{n}%
\glosses{New Guinea Rosewood}%
%------------------------------
\newentry
\headword{ke}%
\pos{n}%
\glosses{slave}%
%------------------------------
%------------------------------
%------------------------------
%------------------------------
%------------------------------
%------------------------------
\newentry
\headword{kecap}%
\pos{n}%
\glosses{ketjap}%
%------------------------------
%------------------------------
\newentry
\headword{kedederet}%
\pos{n}%
\glosses{bird}%
%------------------------------
\newentry
\headword{kedua}%
\pos{n}%
\glosses{darter}%
%------------------------------
%------------------------------
%------------------------------
%------------------------------
\newentry
\headword{keibar}%
\pos{n}%
\glosses{part of outrigger}%
%------------------------------
\newentry
\headword{kein}%
\pos{n}%
\glosses{handle}%
%------------------------------
\newentry
\headword{keir}%
\pos{n}%
\glosses{parrot}%
%------------------------------
\newentry
\headword{keirkeir}%
\pos{n}%
\glosses{lorikeet}%
%------------------------------
\newentry
\headword{keirko}%
\pos{adv}%
\glosses{day after tomorrow}%
%------------------------------
\newentry
\headword{keirun}%
\pos{n}%
\glosses{top}%
%------------------------------
%------------------------------
\newentry
\headword{keit}%
\pos{v}%
\glosses{maintain}%
%------------------------------
\newentry
\headword{keit}%
\pos{n}%
\glosses{land-side}%
%------------------------------
\newentry
\headword{*keit}%
\pos{n}%
\glosses{top}%
%------------------------------
\newentry
\headword{keitar}%
\pos{adv}%
\glosses{day before yesterday}%
%------------------------------
\newentry
\headword{keitkeit}%
\pos{v}%
\glosses{adopted}%
%------------------------------
\newentry
\headword{keitko}%
\pos{n}%
\glosses{top}%
%------------------------------
\newentry
\headword{keitko}%
\pos{n}%
\glosses{above}%
%------------------------------
\newentry
\headword{keitpis}%
\pos{n}%
\glosses{concave side}%
%------------------------------
%------------------------------
%------------------------------
%------------------------------
%------------------------------
%------------------------------
\newentry
\headword{kelelet}%
\pos{n}%
\glosses{prayer}%
%------------------------------
\newentry
\headword{keleng}%
\pos{n}%
\glosses{armpit}%
%------------------------------
%------------------------------
\newentry
\headword{Kelengkeleng}%
\pos{n}%
\glosses{Kelengkeleng}%
%------------------------------
\newentry
\headword{kelikeli}%
\pos{n}%
\glosses{fish}%
%------------------------------
\newentry
\headword{kelkam}%
\pos{n}%
\glosses{naughty}%
%------------------------------
\newentry
\headword{kelkam}%
\pos{n}%
\glosses{ear}%
%------------------------------
\newentry
\headword{kelkam elaun}%
\pos{n}%
\glosses{earlobe}%
%------------------------------
\newentry
\headword{kelkam taun}%
\pos{n}%
\glosses{concha}%
%------------------------------
\newentry
\headword{kelkam toktok}%
\pos{vi}%
\glosses{deaf}%
%------------------------------
\newentry
\headword{kelkampos}%
\pos{n}%
\glosses{ear opening}%
%------------------------------
%------------------------------
\newentry
\headword{kelua}%
\pos{v}%
\glosses{hear; listen}%
%------------------------------
\newentry
\headword{keluar}%
\pos{v}%
\glosses{go out}%
%------------------------------
\newentry
\headword{keluer}%
\pos{n}%
\glosses{crab}%
%------------------------------
%------------------------------
\newentry
\headword{Kemana}%
\pos{n}%
\glosses{Kaimana}%
%------------------------------
\newentry
\headword{kemanur}%
\pos{n}%
\glosses{west}%
%------------------------------
\newentry
\headword{kemanurep}%
\pos{n}%
\glosses{west}%
%------------------------------
\newentry
\headword{kemanurpak}%
\pos{n}%
\glosses{west-wind season}%
%------------------------------
\newentry
\headword{kememe}%
\pos{vi}%
\glosses{weak}%
%------------------------------
%------------------------------
%------------------------------
%------------------------------
\newentry
\headword{kene}%
\pos{n}%
\glosses{k.o. tree}%
%------------------------------
%------------------------------
\newentry
\headword{keraira}%
\pos{n}%
\glosses{prohibition}%
%------------------------------
\newentry
\headword{kerap}%
\pos{n}%
\glosses{border}%
%------------------------------
\newentry
\headword{kerar}%
\pos{n}%
\glosses{turtle}%
%------------------------------
%------------------------------
%------------------------------
\newentry
\headword{kerkap}%
\pos{vi}%
\glosses{red}%
%------------------------------
\newentry
\headword{kerker}%
\pos{n}%
\glosses{shell}%
%------------------------------
\newentry
\headword{kerker}%
\pos{n}%
\glosses{medicinal plant}%
%------------------------------
%------------------------------
\newentry
\headword{kerunggo}%
\pos{n}%
\glosses{on top of; above}%
%------------------------------
%------------------------------
%------------------------------
%------------------------------
%------------------------------
%------------------------------
\newentry
\headword{ketan}%
\pos{n}%
\glosses{parent in law; child in law}%
%------------------------------
%------------------------------
%------------------------------
\newentry
\headword{keteles}%
\pos{n}%
\glosses{maize}%
%------------------------------
\newentry
\headword{ketemu}%
\pos{v}%
\glosses{meet}%
%------------------------------
%------------------------------
%------------------------------
\newentry
\headword{kewa}%
\pos{n}%
\glosses{small loin cloth}%
%------------------------------
\newentry
\headword{kewe}%
\pos{n}%
\glosses{house}%
%------------------------------
\newentry
\headword{kewe padenun}%
\pos{n}%
\glosses{house post}%
%------------------------------
%------------------------------
\newentry
\headword{=ki}%
\sensenr{}%
\pos{gramm}%
\glosses{instrumental}%
\sensenr{}%
\pos{gramm}%
\glosses{benefactive}%
%------------------------------
\newentry
\headword{ki}%
\pos{pro}%
\glosses{you (\textsc{pl})}%
%------------------------------
\newentry
\headword{*kia}%
\pos{n}%
\glosses{sibling}%
%------------------------------
%------------------------------
\newentry
\headword{Kiaba}%
\pos{n}%
\glosses{Kiaba}%
%------------------------------
%------------------------------
\newentry
\headword{kian}%
\pos{n}%
\glosses{my wife}%
%------------------------------
\newentry
\headword{*kiar}%
\pos{n}%
\glosses{wife}%
%------------------------------
%------------------------------
\newentry
\headword{kibi}%
\pos{n}%
\glosses{sea cucumber}%
%------------------------------
\newentry
\headword{kibi karek}%
\pos{n}%
\glosses{sea cucumber}%
%------------------------------
%------------------------------
\newentry
\headword{kibibal}%
\pos{n}%
\glosses{weight watcher}%
%------------------------------
\newentry
\headword{kibis}%
\pos{n}%
\glosses{shore, land, inland}%
%------------------------------
\newentry
\headword{kidi}%
\pos{v}%
\glosses{lie; joke}%
%------------------------------
\newentry
\headword{kiek}%
\pos{n}%
\glosses{shade}%
%------------------------------
\newentry
\headword{kiekter}%
\pos{n}%
\glosses{shadow}%
%------------------------------
\newentry
\headword{kiel}%
\pos{n}%
\glosses{k.o. prawn}%
%------------------------------
\newentry
\headword{*kiel}%
\pos{n}%
\glosses{root}%
%------------------------------
\newentry
\headword{kiel kierun}%
\pos{n}%
\glosses{sand mound}%
%------------------------------
\newentry
\headword{kielar}%
\pos{n}%
\glosses{k.o. pandanus}%
%------------------------------
%------------------------------
\newentry
\headword{kielun}%
\pos{n}%
\glosses{root}%
%------------------------------
\newentry
\headword{kiem}%
\pos{n}%
\glosses{basket}%
%------------------------------
\newentry
\headword{kiem}%
\pos{v}%
\glosses{flee; run}%
%------------------------------
%------------------------------
%------------------------------
\newentry
\headword{kiempanait}%
\pos{dv}%
\glosses{send}%
%------------------------------
\newentry
\headword{kiemsunsun}%
\pos{n}%
\glosses{sea cucumber}%
%------------------------------
\newentry
\headword{kier}%
\pos{pro}%
\glosses{you two (\textsc{pl})}%
%------------------------------
\newentry
\headword{kier}%
\pos{n}%
\glosses{sail}%
%------------------------------
\newentry
\headword{kier}%
\pos{n}%
\glosses{wasp}%
%------------------------------
%------------------------------
\newentry
\headword{kierun}%
\pos{n}%
\glosses{cloud}%
%------------------------------
\newentry
\headword{kies}%
\pos{v}%
\glosses{wrap}%
%------------------------------
\newentry
\headword{*kies}%
\pos{clf}%
\glosses{classifier for long things}%
%------------------------------
\newentry
\headword{kies}%
\pos{n}%
\glosses{wrap}%
%------------------------------
\newentry
\headword{*kies}%
\pos{v}%
\glosses{block}%
%------------------------------
\newentry
\headword{kies}%
\pos{v}%
\glosses{carve}%
%------------------------------
\newentry
\headword{kies koladok}%
\pos{n}%
\glosses{plant}%
%------------------------------
\newentry
\headword{kieskon}%
\pos{qnt}%
\glosses{one}%
%------------------------------
\newentry
\headword{kiet}%
\pos{v}%
\glosses{defecate}%
%------------------------------
\newentry
\headword{kiet}%
\pos{n}%
\glosses{faeces}%
%------------------------------
\newentry
\headword{kietkiet}%
\pos{v}%
\glosses{defecate}%
%------------------------------
\newentry
\headword{kietpak}%
\pos{n}%
\glosses{large intestines}%
%------------------------------
\newentry
\headword{kietpo}%
\pos{n}%
\glosses{fart}%
%------------------------------
\newentry
\headword{kieun}%
\pos{n}%
\glosses{his wife}%
%------------------------------
\newentry
\headword{kieun caun}%
\pos{n}%
\glosses{second wife}%
%------------------------------
\newentry
\headword{kihutak}%
\pos{pro}%
\glosses{you alone (\textsc{pl})}%
%------------------------------
%------------------------------
\newentry
\headword{kilibobang}%
\pos{n}%
\glosses{butterflyfish}%
%------------------------------
%------------------------------
%------------------------------
\newentry
\headword{=kin}%
\pos{gramm}%
\glosses{volitional}%
%------------------------------
\newentry
\headword{kin}%
\pos{pro}%
\glosses{your (\textsc{pl})}%
%------------------------------
\newentry
\headword{=kin}%
\sensenr{}%
\pos{gramm}%
\glosses{possessive}%
\sensenr{}%
\pos{gramm}%
\glosses{part-whole}%
%------------------------------
\newentry
\headword{kinaninggan}%
\pos{pro}%
\glosses{you all (\textsc{pl})}%
%------------------------------
\newentry
\headword{Kindius}%
\pos{n}%
\glosses{Kindius}%
%------------------------------
\newentry
\headword{kinggir}%
\pos{v}%
\glosses{sail}%
%------------------------------
%------------------------------
\newentry
\headword{kinkin}%
\pos{vi}%
\glosses{small}%
%------------------------------
\newentry
\headword{kinkin}%
\pos{v}%
\glosses{hold}%
%------------------------------
\newentry
\headword{kinkinun}%
\pos{vi}%
\glosses{small}%
%------------------------------
\newentry
\headword{kinkinun}%
\pos{n}%
\glosses{the small one(s)}%
%------------------------------
\newentry
\headword{kion}%
\pos{vi}%
\glosses{married}%
%------------------------------
\newentry
\headword{kion}%
\pos{v}%
\glosses{marry}%
%------------------------------
\newentry
\headword{kip}%
\pos{n}%
\glosses{snake}%
%------------------------------
\newentry
\headword{kipkip}%
\pos{n}%
\glosses{larvae}%
%------------------------------
\newentry
\headword{kir}%
\pos{v}%
\glosses{grate}%
%------------------------------
\newentry
\headword{kir}%
\pos{n}%
\glosses{side; kidneys}%
%------------------------------
\newentry
\headword{kir}%
\pos{vi}%
\glosses{greedy}%
%------------------------------
\newentry
\headword{kir}%
\pos{n}%
\glosses{fish}%
%------------------------------
%------------------------------
\newentry
\headword{kirain}%
\pos{pro}%
\glosses{you alone (\textsc{pl})}%
%------------------------------
%------------------------------
%------------------------------
\newentry
\headword{kirarun}%
\pos{n}%
\glosses{side}%
%------------------------------
\newentry
\headword{kirawat}%
\pos{n}%
\glosses{bow planks}%
%------------------------------
%------------------------------
\newentry
\headword{kirkangkang}%
\pos{n}%
\glosses{ribs}%
%------------------------------
%------------------------------
\newentry
\headword{kirun}%
\pos{n}%
\glosses{flank}%
%------------------------------
%------------------------------
%------------------------------
\newentry
\headword{kisileng}%
\pos{n}%
\glosses{sky}%
%------------------------------
%------------------------------
%------------------------------
%------------------------------
\newentry
\headword{ko}%
\pos{n}%
\glosses{shell}%
%------------------------------
\newentry
\headword{ko=}%
\pos{gramm}%
\glosses{applicative}%
%------------------------------
\newentry
\headword{=ko}%
\pos{gramm}%
\glosses{locative}%
%------------------------------
\newentry
\headword{ko=}%
\pos{gramm}%
\glosses{quite}%
%------------------------------
%------------------------------
\newentry
\headword{koalom}%
\pos{v}%
\glosses{spit at}%
%------------------------------
%------------------------------
%------------------------------
\newentry
\headword{kobelen}%
\pos{v}%
\glosses{lick}%
%------------------------------
\newentry
\headword{kobes}%
\pos{v}%
\glosses{reach}%
%------------------------------
\newentry
\headword{kodaet}%
\pos{adv}%
\glosses{again}%
%------------------------------
\newentry
\headword{kodaet}%
\pos{qnt}%
\glosses{one more}%
%------------------------------
\newentry
\headword{kodak}%
\pos{qnt}%
\glosses{one}%
%------------------------------
%------------------------------
\newentry
\headword{koder}%
\pos{v}%
\glosses{add}%
%------------------------------
\newentry
\headword{kodi}%
\pos{v}%
\glosses{whistle-call; message}%
%------------------------------
\newentry
\headword{koecuan}%
\pos{vt}%
\glosses{cry for}%
%------------------------------
\newentry
\headword{koep}%
\pos{n}%
\glosses{ashes}%
%------------------------------
%------------------------------
\newentry
\headword{koewa}%
\pos{v}%
\glosses{angry}%
%------------------------------
\newentry
\headword{kofir}%
\pos{n}%
\glosses{coffee}%
%------------------------------
\newentry
\headword{koi}%
\pos{adv}%
\glosses{again}%
%------------------------------
\newentry
\headword{koi}%
\pos{cnj}%
\glosses{then}%
%------------------------------
\newentry
\headword{kokada}%
\pos{n}%
\glosses{shrimp}%
%------------------------------
\newentry
\headword{kokarap}%
\pos{v}%
\glosses{circle}%
%------------------------------
\newentry
\headword{Kokas}%
\pos{n}%
\glosses{Kokas}%
%------------------------------
\newentry
\headword{kokiem}%
\pos{vt}%
\glosses{run away from}%
%------------------------------
\newentry
\headword{kokies}%
\pos{vt}%
\glosses{wrap}%
%------------------------------
\newentry
\headword{kokir}%
\pos{vi}%
\glosses{near}%
%------------------------------
%------------------------------
\newentry
\headword{kokoak}%
\pos{n}%
\glosses{helmeted friarbird}%
%------------------------------
\newentry
\headword{kokok}%
\pos{n}%
\glosses{chicken}%
%------------------------------
\newentry
\headword{kokok ladok}%
\pos{n}%
\glosses{quail}%
%------------------------------
\newentry
\headword{kokok narun}%
\pos{n}%
\glosses{chicken egg}%
%------------------------------
\newentry
\headword{kokour}%
\pos{v}%
\glosses{not reach; not enough}%
%------------------------------
\newentry
\headword{*kol}%
\pos{n}%
\glosses{outside}%
%------------------------------
\newentry
\headword{kol}%
\pos{v}%
\glosses{out}%
%------------------------------
\newentry
\headword{kolak}%
\pos{n}%
\glosses{mountain; mainland}%
%------------------------------
\newentry
\headword{kolambu}%
\pos{n}%
\glosses{mosquito net}%
%------------------------------
%------------------------------
\newentry
\headword{kolet}%
\pos{n}%
\glosses{stranger}%
%------------------------------
\newentry
\headword{kolga}%
\pos{n}%
\glosses{from outside}%
%------------------------------
\newentry
\headword{koliep}%
\pos{n}%
\glosses{cheek}%
%------------------------------
%------------------------------
%------------------------------
\newentry
\headword{kolkemkem}%
\pos{n}%
\glosses{sejenis siput gerai}%
%------------------------------
\newentry
\headword{kolkiem}%
\pos{n}%
\glosses{thigh}%
%------------------------------
\newentry
\headword{kolkiet}%
\pos{n}%
\glosses{earwax}%
%------------------------------
\newentry
\headword{kolko}%
\pos{n}%
\glosses{outside}%
%------------------------------
\newentry
\headword{kolkol}%
\pos{vi}%
\glosses{entangled}%
%------------------------------
%------------------------------
\newentry
\headword{kolkom}%
\pos{n}%
\glosses{footprint}%
%------------------------------
\newentry
\headword{kolo}%
\pos{v}%
\glosses{take out}%
%------------------------------
\newentry
\headword{kolpanggat}%
\pos{n}%
\glosses{k.o. fish}%
%------------------------------
\newentry
\headword{kolpis}%
\pos{n}%
\glosses{another place}%
%------------------------------
\newentry
\headword{koltengteng}%
\pos{n}%
\glosses{k.o. small fish}%
%------------------------------
\newentry
\headword{kolu welek}%
\pos{n}%
\glosses{bird}%
%------------------------------
\newentry
\headword{koluk}%
\pos{v}%
\glosses{find; meet}%
%------------------------------
%------------------------------
\newentry
\headword{kom}%
\pos{n}%
\glosses{cane}%
%------------------------------
\newentry
\headword{komahal}%
\pos{v}%
\glosses{not know}%
%------------------------------
\newentry
\headword{komain}%
\pos{v}%
\glosses{skewer; stab; fit}%
%------------------------------
%------------------------------
\newentry
\headword{komamun}%
\pos{v}%
\glosses{drop}%
%------------------------------
%------------------------------
\newentry
\headword{komang}%
\pos{n}%
\glosses{throat and neck}%
%------------------------------
%------------------------------
\newentry
\headword{komanggangguop}%
\pos{v}%
\glosses{close roof}%
%------------------------------
\newentry
\headword{komanggasir}%
\pos{n}%
\glosses{neckbone}%
%------------------------------
%------------------------------
\newentry
\headword{komaruk}%
\pos{v}%
\glosses{burn}%
%------------------------------
\newentry
\headword{komasabur}%
\pos{v}%
\glosses{cover; dress}%
%------------------------------
%------------------------------
\newentry
\headword{komasasuk}%
\pos{v}%
\glosses{close}%
%------------------------------
%------------------------------
\newentry
\headword{komayeki}%
\pos{v}%
\glosses{laugh at}%
%------------------------------
\newentry
\headword{kome}%
\pos{vt}%
\glosses{see}%
%------------------------------
\newentry
\headword{kome}%
\pos{v}%
\glosses{look}%
%------------------------------
\newentry
\headword{komelek}%
\pos{v}%
\glosses{burn}%
%------------------------------
\newentry
\headword{komeri}%
\pos{n}%
\glosses{candlenut}%
%------------------------------
\newentry
\headword{komister}%
\pos{v}%
\glosses{bother}%
%------------------------------
\newentry
\headword{komisternin}%
\pos{vt}%
\glosses{untroubled}%
%------------------------------
\newentry
\headword{komurkomur}%
\pos{v}%
\glosses{rinse}%
%------------------------------
\newentry
\headword{komurkomur}%
\pos{n}%
\glosses{cucumber}%
%------------------------------
\newentry
\headword{kon}%
\pos{qnt}%
\glosses{one}%
%------------------------------
%------------------------------
\newentry
\headword{kon tama}%
\pos{q}%
\glosses{which one}%
%------------------------------
\newentry
\headword{kona}%
\pos{v}%
\glosses{see; look}%
%------------------------------
\newentry
\headword{kona}%
\pos{vi}%
\glosses{think}%
%------------------------------
\newentry
\headword{konamin}%
\pos{v}%
\glosses{stab}%
%------------------------------
\newentry
\headword{konasur}%
\pos{v}%
\glosses{face}%
%------------------------------
%------------------------------
%------------------------------
\newentry
\headword{konawaruo}%
\pos{v}%
\glosses{forget}%
%------------------------------
\newentry
\headword{konawol}%
\pos{v}%
\glosses{stick onto}%
%------------------------------
\newentry
\headword{kondisi}%
\pos{n}%
\glosses{condition}%
%------------------------------
\newentry
\headword{konenen}%
\pos{n}%
\glosses{facial hair}%
%------------------------------
\newentry
\headword{konenen}%
\pos{v}%
\glosses{remember}%
%------------------------------
\newentry
\headword{=kongga}%
\pos{gramm}%
\glosses{animate lative}%
%------------------------------
\newentry
\headword{-konggap}%
\pos{qnt}%
\glosses{approximately}%
%------------------------------
\newentry
\headword{konggareor}%
\pos{v}%
\glosses{pour onto}%
%------------------------------
%------------------------------
\newentry
\headword{konggelem}%
\pos{v}%
\glosses{grab}%
%------------------------------
\newentry
\headword{=konggo}%
\pos{gramm}%
\glosses{animate locative}%
%------------------------------
\newentry
\headword{konyak}%
\pos{n}%
\glosses{squid}%
%------------------------------
\newentry
\headword{kopol}%
\pos{v}%
\glosses{sticky}%
%------------------------------
\newentry
\headword{kor}%
\pos{n}%
\glosses{leg}%
%------------------------------
\newentry
\headword{kor kawar mat kalot}%
\pos{n}%
\glosses{back of knee}%
%------------------------------
\newentry
\headword{korabir}%
\pos{v}%
\glosses{jump over}%
%------------------------------
\newentry
\headword{koramtolma}%
\pos{v}%
\glosses{ritual}%
%------------------------------
\newentry
\headword{korap}%
\pos{n}%
\glosses{cross-cousin}%
%------------------------------
%------------------------------
%------------------------------
%------------------------------
%------------------------------
\newentry
\headword{koraruo}%
\pos{v}%
\glosses{bite}%
%------------------------------
%------------------------------
\newentry
\headword{korek}%
\pos{vi}%
\glosses{dead}%
%------------------------------
\newentry
\headword{korel}%
\pos{n}%
\glosses{footsole}%
%------------------------------
\newentry
\headword{korgi marmar}%
\pos{v}%
\glosses{walk}%
%------------------------------
\newentry
\headword{korkancing}%
\pos{n}%
\glosses{ankle bone}%
%------------------------------
\newentry
\headword{korkasir}%
\pos{n}%
\glosses{ankle}%
%------------------------------
\newentry
\headword{korkies}%
\pos{n}%
\glosses{Achilles heel}%
%------------------------------
\newentry
\headword{korko}%
\pos{v}%
\glosses{wear}%
%------------------------------
\newentry
\headword{korkor}%
\pos{v}%
\glosses{cut}%
%------------------------------
\newentry
\headword{korlaus}%
\pos{n}%
\glosses{under side foot}%
%------------------------------
\newentry
\headword{kormul}%
\pos{n}%
\glosses{calf of leg}%
%------------------------------
\newentry
\headword{kornambi}%
\pos{n}%
\glosses{shearwater}%
%------------------------------
\newentry
\headword{korot}%
\pos{v}%
\glosses{slice}%
%------------------------------
%------------------------------
\newentry
\headword{korpak}%
\pos{n}%
\glosses{knee}%
%------------------------------
%------------------------------
\newentry
\headword{korparokparok}%
\pos{n}%
\glosses{toes}%
%------------------------------
\newentry
\headword{kortanggalip}%
\pos{n}%
\glosses{toenails}%
%------------------------------
\newentry
\headword{kortaptap}%
\pos{v}%
\glosses{cut out}%
%------------------------------
\newentry
\headword{korus}%
\pos{n}%
\glosses{shin}%
%------------------------------
\newentry
\headword{kos}%
\pos{v}%
\glosses{grow}%
%------------------------------
\newentry
\headword{kosa}%
\pos{v}%
\glosses{fish in low water}%
%------------------------------
\newentry
\headword{kosalir}%
\pos{vt}%
\glosses{change}%
%------------------------------
\newentry
\headword{kosansan}%
\pos{v}%
\glosses{give packed food}%
%------------------------------
\newentry
\headword{kosara}%
\pos{v}%
\glosses{hit; touch}%
%------------------------------
\newentry
\headword{kosarun}%
\pos{n}%
\glosses{aril; mace}%
%------------------------------
%------------------------------
%------------------------------
\newentry
\headword{koser}%
\pos{n}%
\glosses{key}%
%------------------------------
\newentry
\headword{koser}%
\pos{v}%
\glosses{harvest fruit}%
%------------------------------
\newentry
\headword{koser}%
\pos{v}%
\glosses{lock}%
%------------------------------
\newentry
\headword{kosiaur}%
\pos{v}%
\glosses{hand}%
%------------------------------
\newentry
\headword{kosilep}%
\pos{v}%
\glosses{turn back to}%
%------------------------------
\newentry
\headword{kosin}%
\pos{n}%
\glosses{window frame}%
%------------------------------
\newentry
\headword{kosom}%
\pos{v}%
\glosses{suck; smoke}%
%------------------------------
\newentry
\headword{kosun}%
\pos{n}%
\glosses{growth}%
%------------------------------
\newentry
\headword{Kota Laut}%
\pos{n}%
\glosses{Kota Laut}%
%------------------------------
\newentry
\headword{kotam}%
\pos{v}%
\glosses{skewer}%
%------------------------------
\newentry
\headword{kotarakmang}%
\pos{vt}%
\glosses{startle}%
%------------------------------
\newentry
\headword{kotipol}%
\pos{v}%
\glosses{miss}%
%------------------------------
\newentry
\headword{kotipol}%
\pos{n}%
\glosses{black butcherbird}%
%------------------------------
\newentry
\headword{kotur}%
\pos{n}%
\glosses{dirt}%
%------------------------------
\newentry
\headword{kotur}%
\pos{vi}%
\glosses{dirty}%
%------------------------------
\newentry
\headword{kou}%
\pos{vi}%
\glosses{narrow}%
%------------------------------
\newentry
\headword{kou}%
\pos{v}%
\glosses{blow}%
%------------------------------
\newentry
\headword{koup}%
\pos{v}%
\glosses{hug}%
%------------------------------
\newentry
\headword{kouran}%
\pos{vi}%
\glosses{angry}%
%------------------------------
\newentry
\headword{kous}%
\pos{n}%
\glosses{t-shirt}%
%------------------------------
\newentry
\headword{kous}%
\pos{n}%
\glosses{fish}%
%------------------------------
\newentry
\headword{kout}%
\pos{vt}%
\glosses{curse}%
%------------------------------
\newentry
\headword{koutpol}%
\pos{n}%
\glosses{fish}%
%------------------------------
%------------------------------
%------------------------------
\newentry
\headword{kowam}%
\pos{n}%
\glosses{whale}%
%------------------------------
\newentry
\headword{kowam}%
\pos{v}%
\glosses{weave upwards}%
%------------------------------
\newentry
\headword{kowar}%
\pos{n}%
\glosses{rice package}%
%------------------------------
%------------------------------
%------------------------------
\newentry
\headword{kowaram}%
\pos{v}%
\glosses{clamp}%
%------------------------------
\newentry
\headword{kowaram}%
\pos{n}%
\glosses{tongs}%
%------------------------------
\newentry
\headword{kowarara}%
\pos{v}%
\glosses{make floor}%
%------------------------------
%------------------------------
%------------------------------
%------------------------------
\newentry
\headword{kowarwak}%
\pos{v}%
\glosses{sprinkle}%
%------------------------------
\newentry
\headword{kowat}%
\pos{v}%
\glosses{change}%
%------------------------------
%------------------------------
%------------------------------
%------------------------------
\newentry
\headword{kowewep}%
\pos{vi}%
\glosses{brown; grey}%
%------------------------------
\newentry
\headword{koya}%
\pos{v}%
\glosses{plant}%
%------------------------------
\newentry
\headword{koyak}%
\pos{v}%
\glosses{hit with tool}%
%------------------------------
\newentry
\headword{koyal}%
\pos{v}%
\glosses{scrape; feel itchy}%
%------------------------------
\newentry
\headword{koyal}%
\pos{v}%
\glosses{disturb; mix}%
%------------------------------
%------------------------------
\newentry
\headword{koyelcie}%
\pos{vi}%
\glosses{flipped}%
%------------------------------
\newentry
\headword{koyen}%
\pos{v}%
\glosses{close}%
%------------------------------
\newentry
\headword{koyet}%
\pos{vi}%
\glosses{be finished}%
%------------------------------
%------------------------------
%------------------------------
\newentry
\headword{koyos}%
\pos{v}%
\glosses{climb}%
%------------------------------
%------------------------------
%------------------------------
\newentry
\headword{kualitek}%
\pos{n}%
\glosses{quality}%
%------------------------------
\newentry
\headword{kuang}%
\pos{v}%
\glosses{pass}%
%------------------------------
\newentry
\headword{kuar}%
\pos{v}%
\glosses{cook}%
%------------------------------
%------------------------------
\newentry
\headword{kuat}%
\pos{vi}%
\glosses{strong}%
%------------------------------
\newentry
\headword{kuawi}%
\pos{n}%
\glosses{north}%
%------------------------------
\newentry
\headword{kubalbal}%
\pos{n}%
\glosses{angelfish}%
%------------------------------
\newentry
\headword{kubir}%
\pos{n}%
\glosses{grave}%
%------------------------------
\newentry
\headword{kubirar}%
\pos{n}%
\glosses{graveyard}%
%------------------------------
\newentry
\headword{kucai}%
\pos{n}%
\glosses{grass}%
%------------------------------
\newentry
\headword{kuda}%
\pos{n}%
\glosses{horse}%
%------------------------------
\newentry
\headword{kudakuda}%
\pos{n}%
\glosses{back of boat}%
%------------------------------
\newentry
\headword{kue}%
\pos{n}%
\glosses{pastry}%
%------------------------------
\newentry
\headword{Kueimang}%
\pos{n}%
\glosses{Kilimala language}%
%------------------------------
\newentry
\headword{kuek}%
\pos{n}%
\glosses{fruit bat}%
%------------------------------
\newentry
\headword{kuek}%
\pos{v}%
\glosses{steal}%
%------------------------------
\newentry
\headword{Kuek}%
\pos{n}%
\glosses{Kuek}%
%------------------------------
\newentry
\headword{kuet}%
\pos{v}%
\glosses{bring}%
%------------------------------
\newentry
\headword{Kui}%
\pos{n}%
\glosses{name of people}%
%------------------------------
\newentry
\headword{kuk}%
\pos{n}%
\glosses{fruit-dove}%
%------------------------------
\newentry
\headword{kukis}%
\pos{n}%
\glosses{pastry}%
%------------------------------
%------------------------------
\newentry
\headword{kul}%
\pos{n}%
\glosses{k.o. fish}%
%------------------------------
\newentry
\headword{kul}%
\pos{n}%
\glosses{fig}%
%------------------------------
%------------------------------
%------------------------------
%------------------------------
\newentry
\headword{kulikuli}%
\pos{n}%
\glosses{gong}%
%------------------------------
\newentry
\headword{kulkabok}%
\pos{n}%
\glosses{medicinal plant}%
%------------------------------
\newentry
\headword{kulpanggat}%
\pos{n}%
\glosses{orange-lined triggerfish}%
%------------------------------
%------------------------------
%------------------------------
\newentry
\headword{kulun}%
\pos{n}%
\glosses{skin}%
%------------------------------
\newentry
\headword{kumbai}%
\pos{n}%
\glosses{owl}%
%------------------------------
%------------------------------
\newentry
\headword{kumkum}%
\pos{vi}%
\glosses{male tree}%
%------------------------------
\newentry
\headword{*kun}%
\pos{n}%
\glosses{inside of a tree}%
%------------------------------
%------------------------------
%------------------------------
\newentry
\headword{kunun}%
\pos{n}%
\glosses{inside of a tree}%
%------------------------------
\newentry
\headword{kuotpol}%
\pos{n}%
\glosses{parrotfish}%
%------------------------------
\newentry
\headword{kupkup}%
\pos{n}%
\glosses{husks}%
%------------------------------
\newentry
\headword{kurang}%
\pos{qnt}%
\glosses{less}%
%------------------------------
%------------------------------
\newentry
\headword{kurap}%
\pos{n}%
\glosses{dolphin}%
%------------------------------
\newentry
\headword{kurera}%
\pos{n}%
\glosses{octopus}%
%------------------------------
\newentry
\headword{kurera}%
\pos{n}%
\glosses{grasshopper}%
%------------------------------
\newentry
\headword{kurera}%
\pos{v}%
\glosses{sieve}%
%------------------------------
\newentry
\headword{kurera}%
\pos{n}%
\glosses{basket; sieve}%
%------------------------------
\newentry
\headword{kuru}%
\pos{v}%
\glosses{bring}%
%------------------------------
\newentry
\headword{kurua}%
\pos{n}%
\glosses{ibis}%
%------------------------------
\newentry
\headword{kus}%
\pos{n}%
\glosses{charcoal}%
%------------------------------
%------------------------------
\newentry
\headword{kuskap}%
\pos{vi}%
\glosses{black}%
%------------------------------
%------------------------------
%------------------------------
\newentry
\headword{kusukusu}%
\pos{n}%
\glosses{wild sugarcane}%
%------------------------------
\end{letter}
\begin{letter}{l}
%------------------------------
\newentry
\headword{labis}%
\pos{n}%
\glosses{k.o. fish}%
%------------------------------
\newentry
\headword{labor}%
\pos{n}%
\glosses{fish}%
%------------------------------
\newentry
\headword{labu siam}%
\pos{n}%
\glosses{chayote}%
%------------------------------
\newentry
\headword{ladan}%
\pos{n}%
\glosses{blouze; shirt}%
%------------------------------
%------------------------------
%------------------------------
\newentry
\headword{lajarang}%
\pos{n}%
\glosses{horse}%
%------------------------------
\newentry
\headword{laksasa}%
\pos{n}%
\glosses{giant}%
%------------------------------
\newentry
\headword{lalang}%
\pos{vi}%
\glosses{hot}%
%------------------------------
\newentry
\headword{lalat}%
\pos{v}%
\glosses{die}%
%------------------------------
%------------------------------
\newentry
\headword{laluon}%
\pos{v}%
\glosses{grub}%
%------------------------------
\newentry
\headword{lam}%
\pos{n}%
\glosses{soft coral}%
%------------------------------
\newentry
\headword{lameli}%
\pos{n}%
\glosses{grouper}%
%------------------------------
\newentry
\headword{lamora kasamin}%
\pos{n}%
\glosses{cormorant}%
%------------------------------
%------------------------------
\newentry
\headword{lampur}%
\pos{n}%
\glosses{lamp}%
%------------------------------
\newentry
\headword{lamut}%
\pos{n}%
\glosses{algae}%
%------------------------------
\newentry
\headword{lamut}%
\pos{n}%
\glosses{skin dirt}%
%------------------------------
%------------------------------
\newentry
\headword{langgan}%
\pos{n}%
\glosses{wood}%
%------------------------------
\newentry
\headword{langgar}%
\pos{n}%
\glosses{k.o. tree}%
%------------------------------
%------------------------------
\newentry
\headword{langgour}%
\pos{vi}%
\glosses{too tight}%
%------------------------------
\newentry
\headword{langgulanggur}%
\pos{n}%
\glosses{k.o. illness}%
%------------------------------
%------------------------------
\newentry
\headword{langjut}%
\pos{v}%
\glosses{continue}%
%------------------------------
\newentry
\headword{langka}%
\pos{v}%
\glosses{fight}%
%------------------------------
\newentry
\headword{langsa}%
\pos{n}%
\glosses{plant}%
%------------------------------
\newentry
\headword{langsung}%
\pos{adv}%
\glosses{directly}%
%------------------------------
%------------------------------
\newentry
\headword{Lapangan}%
\pos{n}%
\glosses{Lapangan}%
%------------------------------
\newentry
\headword{lapas}%
\pos{v}%
\glosses{drop}%
%------------------------------
%------------------------------
\newentry
\headword{lapor}%
\pos{v}%
\glosses{report}%
%------------------------------
\newentry
\headword{lasiambar}%
\pos{n}%
\glosses{leopard shark}%
%------------------------------
\newentry
\headword{lat}%
\pos{n}%
\glosses{plank}%
%------------------------------
%------------------------------
\newentry
\headword{lauk}%
\pos{v}%
\glosses{smell}%
%------------------------------
\newentry
\headword{lauk}%
\pos{v}%
\glosses{appear}%
%------------------------------
%------------------------------
\newentry
\headword{laur}%
\pos{v}%
\glosses{be noisy}%
%------------------------------
\newentry
\headword{laur}%
\pos{vi}%
\glosses{boil; fish playing in water}%
%------------------------------
\newentry
\headword{laur}%
\pos{vi}%
\glosses{rising tide}%
%------------------------------
\newentry
\headword{laus}%
\pos{vi}%
\glosses{wide}%
%------------------------------
\newentry
\headword{laut}%
\pos{n}%
\glosses{sea}%
%------------------------------
%------------------------------
\newentry
\headword{lawalawat}%
\pos{n}%
\glosses{pouch}%
%------------------------------
\newentry
\headword{lawalawat}%
\pos{v}%
\glosses{pouch}%
%------------------------------
\newentry
\headword{lawan}%
\pos{n}%
\glosses{grouper}%
%------------------------------
\newentry
\headword{lawan}%
\pos{n}%
\glosses{k.o. small bamboo}%
%------------------------------
\newentry
\headword{lawarun}%
\pos{n}%
\glosses{womb}%
%------------------------------
\newentry
\headword{lawat}%
\pos{v}%
\glosses{put away}%
%------------------------------
\newentry
\headword{lawilawi}%
\pos{n}%
\glosses{onion}%
%------------------------------
\newentry
\headword{lawuak}%
\pos{n}%
\glosses{fish scales}%
%------------------------------
\newentry
\headword{lawuak}%
\pos{v}%
\glosses{scale a fish}%
%------------------------------
\newentry
\headword{layier}%
\pos{vi}%
\glosses{itchy}%
%------------------------------
\newentry
\headword{leba}%
\pos{n}%
\glosses{imam}%
%------------------------------
\newentry
\headword{lebai}%
\pos{adv}%
\glosses{better}%
%------------------------------
\newentry
\headword{lebaleba}%
\pos{v}%
\glosses{carry on shoulders}%
%------------------------------
%------------------------------
\newentry
\headword{lebe}%
\pos{v}%
\glosses{more; past; higher}%
%------------------------------
%------------------------------
%------------------------------
\newentry
\headword{leis}%
\pos{n}%
\glosses{stripe}%
%------------------------------
\newentry
\headword{leit}%
\pos{n}%
\glosses{king}%
%------------------------------
\newentry
\headword{leit pas}%
\pos{n}%
\glosses{queen}%
%------------------------------
\newentry
\headword{lek}%
\pos{n}%
\glosses{goat}%
%------------------------------
\newentry
\headword{lek nabonabon}%
\pos{n}%
\glosses{lemongrass}%
%------------------------------
%------------------------------
%------------------------------
\newentry
\headword{leki}%
\pos{n}%
\glosses{monkey}%
%------------------------------
%------------------------------
\newentry
\headword{lele}%
\pos{v}%
\glosses{fly off}%
%------------------------------
\newentry
\headword{leluk}%
\pos{v}%
\glosses{come}%
%------------------------------
\newentry
\headword{lem}%
\pos{n}%
\glosses{axe}%
%------------------------------
\newentry
\headword{lem}%
\pos{v}%
\glosses{beckon}%
%------------------------------
\newentry
\headword{lemat}%
\pos{n}%
\glosses{bamboo string}%
%------------------------------
\newentry
\headword{lembaga}%
\pos{n}%
\glosses{prison}%
%------------------------------
\newentry
\headword{Lempang}%
\pos{n}%
\glosses{Kei}%
%------------------------------
\newentry
\headword{lempuang}%
\pos{n}%
\glosses{island}%
%------------------------------
\newentry
\headword{Lempuangemun}%
\pos{n}%
\glosses{Cassowary Island}%
%------------------------------
%------------------------------
%------------------------------
\newentry
\headword{Lempuangtumun}%
\pos{n}%
\glosses{Dog Island}%
%------------------------------
\newentry
\headword{lemyar}%
\pos{n}%
\glosses{stone axe}%
%------------------------------
\newentry
\headword{leng}%
\pos{n}%
\glosses{village; place}%
%------------------------------
\newentry
\headword{leng dek}%
\pos{phrs}%
\glosses{earthquake}%
%------------------------------
%------------------------------
\newentry
\headword{lenggalengga}%
\pos{n}%
\glosses{chilli}%
%------------------------------
%------------------------------
%------------------------------
\newentry
\headword{Lenggon}%
\pos{n}%
\glosses{Karas Darat}%
%------------------------------
%------------------------------
%------------------------------
\newentry
\headword{Lengleng}%
\pos{n}%
\glosses{Lengleng}%
%------------------------------
\newentry
\headword{lepalepa}%
\pos{n}%
\glosses{wooden canoe}%
%------------------------------
\newentry
\headword{lepir}%
\pos{n}%
\glosses{rafter}%
%------------------------------
\newentry
\headword{lerang}%
\pos{n}%
\glosses{white gum}%
%------------------------------
\newentry
\headword{les}%
\pos{n}%
\glosses{stem}%
%------------------------------
%------------------------------
\newentry
\headword{letma}%
\pos{vt}%
\glosses{cut.branch}%
%------------------------------
\newentry
\headword{lewat}%
\pos{v}%
\glosses{pass; go.on}%
%------------------------------
%------------------------------
\newentry
\headword{licing}%
\pos{vi}%
\glosses{slippery; smooth}%
%------------------------------
\newentry
\headword{lidan}%
\pos{n}%
\glosses{friend}%
%------------------------------
\newentry
\headword{lim}%
\pos{n}%
\glosses{navel}%
%------------------------------
%------------------------------
\newentry
\headword{linggis}%
\pos{n}%
\glosses{tool}%
%------------------------------
%------------------------------
%------------------------------
\newentry
\headword{liti}%
\pos{n}%
\glosses{bracelet}%
%------------------------------
\newentry
\headword{lo}%
\pos{v}%
\glosses{consent; like}%
%------------------------------
\newentry
\headword{Loflof}%
\pos{n}%
\glosses{Loflof}%
%------------------------------
\newentry
\headword{lohar}%
\pos{n}%
\glosses{midday}%
%------------------------------
\newentry
\headword{loi}%
\pos{adv}%
\glosses{fast}%
%------------------------------
%------------------------------
\newentry
\headword{Lokpon}%
\pos{n}%
\glosses{Lokpon}%
%------------------------------
\newentry
\headword{loku}%
\pos{v}%
\glosses{catch}%
%------------------------------
\newentry
\headword{lokul}%
\pos{n}%
\glosses{bark}%
%------------------------------
\newentry
\headword{lolok}%
\pos{n}%
\glosses{leaf}%
%------------------------------
\newentry
\headword{lolok}%
\pos{n}%
\glosses{money}%
%------------------------------
\newentry
\headword{lolouk}%
\pos{n}%
\glosses{hole}%
%------------------------------
\newentry
\headword{loncing}%
\pos{n}%
\glosses{watch}%
%------------------------------
%------------------------------
\newentry
\headword{lopalopa}%
\pos{n}%
\glosses{envelope}%
%------------------------------
\newentry
\headword{lopteng}%
\pos{n}%
\glosses{bull/reef shark}%
%------------------------------
\newentry
\headword{lorap}%
\pos{n}%
\glosses{foundation}%
%------------------------------
%------------------------------
\newentry
\headword{los}%
\pos{n}%
\glosses{harbour}%
%------------------------------
\newentry
\headword{losing}%
\pos{n}%
\glosses{dozen}%
%------------------------------
\newentry
\headword{lot}%
\pos{n}%
\glosses{sinker}%
%------------------------------
%------------------------------
\newentry
\headword{loup}%
\pos{n}%
\glosses{butcherbird}%
%------------------------------
\newentry
\headword{loup}%
\pos{n}%
\glosses{fruit}%
%------------------------------
\newentry
\headword{lu}%
\pos{vi}%
\glosses{cold}%
%------------------------------
\newentry
\headword{luam}%
\pos{v}%
\glosses{sick}%
%------------------------------
\newentry
\headword{luam}%
\pos{v}%
\glosses{murky sea}%
%------------------------------
%------------------------------
\newentry
\headword{luk}%
\pos{v}%
\glosses{come}%
%------------------------------
\newentry
\headword{lusi}%
\pos{n}%
\glosses{eagle}%
%------------------------------
\newentry
\headword{lusi muaun}%
\pos{n}%
\glosses{k.o. tree}%
%------------------------------
\newentry
\headword{lusi pep jiejie}%
\pos{n}%
\glosses{eagle}%
%------------------------------
\end{letter}
\begin{letter}{m}
%------------------------------
\newentry
\headword{m'm}%
\pos{int}%
\glosses{interjection of agreement}%
%------------------------------
\newentry
\headword{ma=}%
\pos{gramm}%
\glosses{causative}%
%------------------------------
\newentry
\headword{ma}%
\pos{pro}%
\glosses{he/she/it}%
%------------------------------
\newentry
\headword{ma cicaun}%
\pos{n}%
\glosses{lastborn}%
%------------------------------
\newentry
\headword{ma he me}%
\pos{phrs}%
\glosses{that's it}%
%------------------------------
\newentry
\headword{ma temun}%
\pos{n}%
\glosses{firstborn}%
%------------------------------
\newentry
\headword{mabuk}%
\pos{v}%
\glosses{drunk}%
%------------------------------
%------------------------------
\newentry
\headword{madong}%
\pos{v}%
\glosses{stretch (out)}%
%------------------------------
%------------------------------
\newentry
\headword{magarip}%
\pos{n}%
\glosses{magrib}%
%------------------------------
\newentry
\headword{-mahap}%
\pos{n}%
\glosses{all}%
%------------------------------
\newentry
\headword{mahar}%
\pos{n}%
\glosses{dowry}%
%------------------------------
%------------------------------
\newentry
\headword{Mahem}%
\pos{n}%
\glosses{Fakfak people}%
%------------------------------
\newentry
\headword{mahutak}%
\pos{pro}%
\glosses{he/she/it alone}%
%------------------------------
%------------------------------
\newentry
\headword{main}%
\pos{pro}%
\glosses{his/her/its}%
%------------------------------
\newentry
\headword{mais}%
\pos{v}%
\glosses{spoiled}%
%------------------------------
%------------------------------
\newentry
\headword{mal}%
\pos{n}%
\glosses{rabbitfish}%
%------------------------------
\newentry
\headword{mal}%
\pos{n}%
\glosses{big loin cloth}%
%------------------------------
\newentry
\headword{Malai}%
\pos{n}%
\glosses{Malay}%
%------------------------------
%------------------------------
\newentry
\headword{malaikat}%
\pos{n}%
\glosses{angel; curse}%
%------------------------------
\newentry
\headword{Malaimang}%
\pos{n}%
\glosses{Indonesian}%
%------------------------------
%------------------------------
\newentry
\headword{malam}%
\pos{n}%
\glosses{prayers}%
%------------------------------
\newentry
\headword{malaouk}%
\pos{v}%
\glosses{turn over}%
%------------------------------
\newentry
\headword{malawan}%
\pos{v}%
\glosses{quarrel}%
%------------------------------
%------------------------------
\newentry
\headword{malelin}%
\pos{vt}%
\glosses{keep still}%
%------------------------------
\newentry
\headword{maliap}%
\pos{n}%
\glosses{rabbitfish}%
%------------------------------
\newentry
\headword{maling}%
\pos{v}%
\glosses{move to side}%
%------------------------------
\newentry
\headword{malkesi}%
\pos{n}%
\glosses{rabbitfish}%
%------------------------------
\newentry
\headword{malor}%
\pos{n}%
\glosses{loincloth}%
%------------------------------
\newentry
\headword{malu}%
\pos{v}%
\glosses{swear}%
%------------------------------
%------------------------------
\newentry
\headword{mam}%
\pos{n}%
\glosses{insect found in rice or flour; heteroptera}%
%------------------------------
\newentry
\headword{mama}%
\pos{n}%
\glosses{mama}%
%------------------------------
\newentry
\headword{mama}%
\pos{n}%
\glosses{uncle}%
%------------------------------
\newentry
\headword{mama caun}%
\pos{n}%
\glosses{uncle}%
%------------------------------
\newentry
\headword{mama temun}%
\pos{n}%
\glosses{uncle}%
%------------------------------
%------------------------------
\newentry
\headword{mambara}%
\pos{v}%
\glosses{stand}%
%------------------------------
%------------------------------
\newentry
\headword{mambon}%
\pos{v}%
\glosses{there is}%
%------------------------------
\newentry
\headword{Mamika}%
\pos{n}%
\glosses{Kaimana people}%
%------------------------------
\newentry
\headword{mamor}%
\pos{n}%
\glosses{hornbill}%
%------------------------------
%------------------------------
\newentry
\headword{mamun}%
\pos{v}%
\glosses{leave}%
%------------------------------
\newentry
\headword{man}%
\pos{n}%
\glosses{handle}%
%------------------------------
\newentry
\headword{manadu}%
\pos{n}%
\glosses{taro}%
%------------------------------
\newentry
\headword{mang}%
\pos{n}%
\glosses{language; voice}%
%------------------------------
\newentry
\headword{mang}%
\pos{vi}%
\glosses{bitter}%
%------------------------------
%------------------------------
%------------------------------
\newentry
\headword{manggamangga}%
\pos{n}%
\glosses{gong}%
%------------------------------
\newentry
\headword{manggang}%
\pos{vt}%
\glosses{hang}%
%------------------------------
\newentry
\headword{manggaren}%
\pos{v}%
\glosses{crawl}%
%------------------------------
\newentry
\headword{manggi}%
\pos{n}%
\glosses{fish}%
%------------------------------
%------------------------------
\newentry
\headword{mangmang}%
\pos{n}%
\glosses{k.o. tree}%
%------------------------------
\newentry
\headword{Mania}%
\pos{n}%
\glosses{Mania}%
%------------------------------
\newentry
\headword{Maniem}%
\pos{n}%
\glosses{Maniem}%
%------------------------------
\newentry
\headword{maniktambang}%
\pos{n}%
\glosses{brush turkey}%
%------------------------------
\newentry
\headword{maniktapuri}%
\pos{n}%
\glosses{crowned pigeon}%
%------------------------------
\newentry
\headword{manman}%
\pos{n}%
\glosses{k.o. fish}%
%------------------------------
%------------------------------
\newentry
\headword{manyuor}%
\pos{v}%
\glosses{adjust}%
%------------------------------
\newentry
\headword{maorek}%
\pos{v}%
\glosses{break down}%
%------------------------------
\newentry
\headword{maouk}%
\pos{v}%
\glosses{spit out}%
%------------------------------
\newentry
\headword{mara}%
\pos{v}%
\glosses{move towards land}%
%------------------------------
\newentry
\headword{marain}%
\pos{pro}%
\glosses{he/she/it alone}%
%------------------------------
%------------------------------
\newentry
\headword{maraok}%
\pos{vt}%
\glosses{crush}%
%------------------------------
\newentry
\headword{maraouk}%
\pos{v}%
\glosses{put}%
%------------------------------
%------------------------------
\newentry
\headword{mararak}%
\pos{v}%
\glosses{dry}%
%------------------------------
%------------------------------
%------------------------------
\newentry
\headword{marau}%
\pos{n}%
\glosses{gold}%
%------------------------------
%------------------------------
%------------------------------
\newentry
\headword{marmar}%
\pos{v}%
\glosses{walk}%
%------------------------------
%------------------------------
\newentry
\headword{marok}%
\pos{v}%
\glosses{joke}%
%------------------------------
%------------------------------
%------------------------------
\newentry
\headword{marua}%
\pos{v}%
\glosses{move towards sea}%
%------------------------------
\newentry
\headword{marum}%
\pos{v}%
\glosses{slice}%
%------------------------------
%------------------------------
\newentry
\headword{marur}%
\pos{n}%
\glosses{mucus}%
%------------------------------
%------------------------------
\newentry
\headword{Mas}%
\pos{n}%
\glosses{Mas}%
%------------------------------
%------------------------------
\newentry
\headword{masa}%
\pos{v}%
\glosses{dry in the sun}%
%------------------------------
\newentry
\headword{masak}%
\pos{v}%
\glosses{lift}%
%------------------------------
\newentry
\headword{masal}%
\pos{n}%
\glosses{flying fish}%
%------------------------------
%------------------------------
%------------------------------
\newentry
\headword{masalaboung}%
\pos{v}%
\glosses{cut}%
%------------------------------
\newentry
\headword{masaouk}%
\pos{v}%
\glosses{drag}%
%------------------------------
%------------------------------
\newentry
\headword{masara}%
\pos{v}%
\glosses{move towards land}%
%------------------------------
\newentry
\headword{masarut}%
\pos{v}%
\glosses{tear}%
%------------------------------
\newentry
\headword{masawin}%
\pos{n}%
\glosses{centipede}%
%------------------------------
\newentry
\headword{maser}%
\sensenr{}%
\pos{n}%
\glosses{star}%
\sensenr{}%
\pos{n}%
\glosses{starfish}%
%------------------------------
\newentry
\headword{masikit}%
\pos{n}%
\glosses{mosque}%
%------------------------------
%------------------------------
\newentry
\headword{masing}%
\pos{n}%
\glosses{sea cucumber}%
%------------------------------
%------------------------------
\newentry
\headword{masinul}%
\pos{n}%
\glosses{dew}%
%------------------------------
\newentry
\headword{masir}%
\pos{v}%
\glosses{weed}%
%------------------------------
\newentry
\headword{masoi}%
\sensenr{}%
\pos{n}%
\glosses{cuckoo}%
\sensenr{}%
\pos{n}%
\glosses{massoi tree}%
%------------------------------
\newentry
\headword{masok}%
\pos{vi}%
\glosses{tied too tight}%
%------------------------------
\newentry
\headword{masoki}%
\pos{v}%
\glosses{shove}%
%------------------------------
\newentry
\headword{masriku}%
\pos{n}%
\glosses{k.o. bird}%
%------------------------------
\newentry
\headword{masu}%
\pos{v}%
\glosses{search fish with light}%
%------------------------------
\newentry
\headword{masuk}%
\pos{v}%
\glosses{enter}%
%------------------------------
%------------------------------
\newentry
\headword{Mata}%
\pos{n}%
\glosses{Fakfak person}%
%------------------------------
\newentry
\headword{mata bulang}%
\pos{n}%
\glosses{shell}%
%------------------------------
\newentry
\headword{mata dimdim}%
\pos{n}%
\glosses{firefly}%
%------------------------------
%------------------------------
\newentry
\headword{matan sena}%
\pos{n}%
\glosses{k.o. fish}%
%------------------------------
\newentry
\headword{matur}%
\pos{vt}%
\glosses{drop}%
%------------------------------
\newentry
\headword{mau}%
\pos{v}%
\glosses{want}%
%------------------------------
%------------------------------
\newentry
\headword{Mauka}%
\pos{n}%
\glosses{Mauka}%
%------------------------------
\newentry
\headword{maulcie}%
\pos{vi}%
\glosses{crooked}%
%------------------------------
\newentry
\headword{maulma}%
\pos{vt}%
\glosses{bend}%
%------------------------------
%------------------------------
%------------------------------
\newentry
\headword{mawal}%
\pos{vi}%
\glosses{thick}%
%------------------------------
\newentry
\headword{mawin}%
\pos{v}%
\glosses{feel good}%
%------------------------------
%------------------------------
\newentry
\headword{mayilma}%
\pos{vt}%
\glosses{flip}%
%------------------------------
\newentry
\headword{me}%
\pos{dem}%
\glosses{distal}%
%------------------------------
\newentry
\headword{me}%
\pos{gramm}%
\glosses{topic marker}%
%------------------------------
\newentry
\headword{mecua}%
\pos{v}%
\glosses{store; bury}%
%------------------------------
%------------------------------
\newentry
\headword{mei}%
\pos{v}%
\glosses{come}%
%------------------------------
\newentry
\headword{meja}%
\pos{n}%
\glosses{table}%
%------------------------------
\newentry
\headword{mel}%
\pos{n}%
\glosses{mile (sea-mile)}%
%------------------------------
%------------------------------
%------------------------------
\newentry
\headword{melebor}%
\pos{v}%
\glosses{get rid of}%
%------------------------------
\newentry
\headword{meleluo}%
\pos{v}%
\glosses{sit}%
%------------------------------
%------------------------------
%------------------------------
%------------------------------
\newentry
\headword{mena}%
\pos{cnj}%
\glosses{otherwise}%
%------------------------------
\newentry
\headword{mena}%
\pos{adv}%
\glosses{later}%
%------------------------------
\newentry
\headword{mencari}%
\pos{v}%
\glosses{make a living}%
%------------------------------
\newentry
\headword{mendak}%
\pos{dem}%
\glosses{like that}%
%------------------------------
%------------------------------
%------------------------------
\newentry
\headword{mengerti}%
\pos{v}%
\glosses{understand}%
%------------------------------
\newentry
\headword{mengga}%
\pos{dem}%
\glosses{distal lative}%
%------------------------------
%------------------------------
%------------------------------
%------------------------------
%------------------------------
\newentry
\headword{menyanyi}%
\pos{v}%
\glosses{sing}%
%------------------------------
\newentry
\headword{mera}%
\pos{cnj}%
\glosses{then}%
%------------------------------
\newentry
\headword{mera}%
\pos{int}%
\glosses{interjection}%
%------------------------------
\newentry
\headword{merar}%
\pos{n}%
\glosses{mole}%
%------------------------------
\newentry
\headword{meraraouk}%
\pos{v}%
\glosses{cause to snap}%
%------------------------------
%------------------------------
\newentry
\headword{merengguen}%
\pos{v}%
\glosses{heap; gather}%
%------------------------------
\newentry
\headword{meresawuo}%
\pos{v}%
\glosses{struggle}%
%------------------------------
%------------------------------
\newentry
\headword{mesan}%
\pos{n}%
\glosses{gravestone}%
%------------------------------
\newentry
\headword{mesang}%
\pos{n}%
\glosses{gills}%
%------------------------------
\newentry
\headword{mesang}%
\pos{n}%
\glosses{pulp}%
%------------------------------
%------------------------------
%------------------------------
%------------------------------
\newentry
\headword{met}%
\pos{dem}%
\glosses{distal (object form)}%
%------------------------------
%------------------------------
\newentry
\headword{metko}%
\pos{dem}%
\glosses{distal locative}%
%------------------------------
%------------------------------
\newentry
\headword{mia}%
\pos{v}%
\glosses{come}%
%------------------------------
\newentry
\headword{miabes}%
\pos{dem}%
\glosses{distal quantity}%
%------------------------------
\newentry
\headword{miarip}%
\pos{dem}%
\glosses{distal quantity}%
%------------------------------
%------------------------------
\newentry
\headword{miasen}%
\pos{dem}%
\glosses{distal degree}%
%------------------------------
%------------------------------
\newentry
\headword{mier}%
\pos{pro}%
\glosses{they}%
%------------------------------
\newentry
\headword{mikon}%
\pos{vi}%
\glosses{full}%
%------------------------------
\newentry
\headword{min}%
\pos{v}%
\glosses{sleep}%
%------------------------------
\newentry
\headword{min}%
\pos{n}%
\glosses{throat}%
%------------------------------
\newentry
\headword{minar}%
\pos{n}%
\glosses{larynx}%
%------------------------------
%------------------------------
\newentry
\headword{mindi}%
\pos{dem}%
\glosses{like that}%
%------------------------------
%------------------------------
\newentry
\headword{ming}%
\pos{n}%
\glosses{oil}%
%------------------------------
\newentry
\headword{minggalot}%
\pos{n}%
\glosses{bedroom}%
%------------------------------
\newentry
\headword{minggaruk}%
\pos{v}%
\glosses{snore}%
%------------------------------
\newentry
\headword{minggi}%
\pos{adv}%
\glosses{with that}%
%------------------------------
\newentry
\headword{minggu}%
\pos{n}%
\glosses{week; Sunday}%
%------------------------------
\newentry
\headword{mingtun}%
\pos{n}%
\glosses{palm oil}%
%------------------------------
%------------------------------
\newentry
\headword{mintolma}%
\pos{v}%
\glosses{cut throat}%
%------------------------------
\newentry
\headword{minum}%
\pos{v}%
\glosses{drink}%
%------------------------------
%------------------------------
%------------------------------
%------------------------------
%------------------------------
\newentry
\headword{*mir}%
\pos{clf}%
\glosses{classifier for canoes}%
%------------------------------
\newentry
\headword{mirik}%
\pos{n}%
\glosses{song}%
%------------------------------
\newentry
\headword{mirik}%
\pos{v}%
\glosses{sing}%
%------------------------------
\newentry
\headword{mirkon}%
\pos{qnt}%
\glosses{one}%
%------------------------------
\newentry
\headword{misilmisil}%
\pos{n}%
\glosses{cement floor}%
%------------------------------
%------------------------------
%------------------------------
%------------------------------
%------------------------------
\newentry
\headword{mo}%
\pos{int}%
\glosses{softener}%
%------------------------------
\newentry
\headword{mok}%
\pos{n}%
\glosses{mug}%
%------------------------------
\newentry
\headword{momar}%
\pos{n}%
\glosses{k.o. fish}%
%------------------------------
\newentry
\headword{mon}%
\pos{vi}%
\glosses{quick}%
%------------------------------
\newentry
\headword{monkaret}%
\pos{vi}%
\glosses{lazy}%
%------------------------------
%------------------------------
\newentry
\headword{mor}%
\pos{n}%
\glosses{sour}%
%------------------------------
\newentry
\headword{mor}%
\pos{vi}%
\glosses{sour}%
%------------------------------
\newentry
\headword{mormor}%
\pos{n}%
\glosses{fish}%
%------------------------------
\newentry
\headword{mormor}%
\pos{v}%
\glosses{hide}%
%------------------------------
%------------------------------
\newentry
\headword{mososor}%
\pos{vi}%
\glosses{diligent}%
%------------------------------
\newentry
\headword{mosun}%
\pos{n}%
\glosses{season}%
%------------------------------
\newentry
\headword{motor}%
\pos{n}%
\glosses{motor}%
%------------------------------
\newentry
\headword{mu}%
\pos{pro}%
\glosses{they}%
%------------------------------
\newentry
\headword{muap}%
\pos{v}%
\glosses{eat}%
%------------------------------
\newentry
\headword{muap}%
\pos{n}%
\glosses{food}%
%------------------------------
\newentry
\headword{muap sabur}%
\pos{n}%
\glosses{sago tree}%
%------------------------------
\newentry
\headword{muap sabur kunun}%
\pos{n}%
\glosses{sago flour}%
%------------------------------
\newentry
\headword{muap sabur sangganun}%
\pos{n}%
\glosses{sago grub}%
%------------------------------
\newentry
\headword{muapsabursanong}%
\pos{n}%
\glosses{sago leaf roof}%
%------------------------------
%------------------------------
\newentry
\headword{muawaruo}%
\pos{v}%
\glosses{cook}%
%------------------------------
\newentry
\headword{muawese}%
\pos{vi}%
\glosses{hungry}%
%------------------------------
\newentry
\headword{muawesese}%
\pos{v}%
\glosses{very hungry; many hungry people}%
%------------------------------
%------------------------------
\newentry
\headword{mudi}%
\pos{v}%
\glosses{throw}%
%------------------------------
\newentry
\headword{muhutak}%
\pos{pro}%
\glosses{they alone}%
%------------------------------
\newentry
\headword{muin}%
\pos{pro}%
\glosses{their}%
%------------------------------
\newentry
\headword{muisese}%
\pos{vi}%
\glosses{hungry}%
%------------------------------
\newentry
\headword{mujim}%
\pos{n}%
\glosses{muezzin}%
%------------------------------
%------------------------------
\newentry
\headword{muk}%
\pos{v}%
\glosses{rock; nod}%
%------------------------------
\newentry
\headword{muk}%
\pos{v}%
\glosses{throw}%
%------------------------------
\newentry
\headword{muka}%
\pos{n}%
\glosses{front}%
%------------------------------
\newentry
\headword{mukmuk}%
\pos{v}%
\glosses{rock}%
%------------------------------
\newentry
\headword{*mul}%
\pos{n}%
\glosses{side}%
%------------------------------
%------------------------------
%------------------------------
\newentry
\headword{muler}%
\pos{n}%
\glosses{waist}%
%------------------------------
\newentry
\headword{Mulmul}%
\pos{n}%
\glosses{Mulmul}%
%------------------------------
\newentry
\headword{mulun}%
\pos{n}%
\glosses{side}%
%------------------------------
\newentry
\headword{mulunggo}%
\pos{v}%
\glosses{beside}%
%------------------------------
\newentry
\headword{-mun}%
\pos{pro}%
\glosses{prohibitive}%
%------------------------------
%------------------------------
\newentry
\headword{mun}%
\pos{vi}%
\glosses{rotten}%
%------------------------------
\newentry
\headword{mun}%
\pos{n}%
\glosses{louse}%
%------------------------------
\newentry
\headword{mun}%
\pos{n}%
\glosses{lime; citrus}%
%------------------------------
\newentry
\headword{mun sunsun}%
\pos{n}%
\glosses{nit}%
%------------------------------
\newentry
\headword{Munak}%
\pos{n}%
\glosses{Munak}%
%------------------------------
\newentry
\headword{munaninggan}%
\pos{pro}%
\glosses{they all}%
%------------------------------
%------------------------------
\newentry
\headword{munin}%
\pos{n}%
\glosses{west}%
%------------------------------
\newentry
\headword{munmun}%
\pos{v}%
\glosses{louse}%
%------------------------------
\newentry
\headword{-mur}%
\pos{n}%
\glosses{plural}%
%------------------------------
\newentry
\headword{muradik}%
\pos{n}%
\glosses{imperial pigeon}%
%------------------------------
\newentry
\headword{murain}%
\pos{pro}%
\glosses{they alone}%
%------------------------------
\newentry
\headword{murkumkum}%
\pos{n}%
\glosses{k.o. imperial pigeon}%
%------------------------------
\newentry
\headword{mursambuk}%
\pos{n}%
\glosses{imperial pigeon}%
%------------------------------
\newentry
\headword{mus}%
\pos{v}%
\glosses{eat}%
%------------------------------
%------------------------------
\newentry
\headword{musing}%
\pos{n}%
\glosses{leopard torpedo}%
%------------------------------
\newentry
\headword{mustika}%
\pos{n}%
\glosses{pearl shell}%
%------------------------------
\newentry
\headword{mutam}%
\pos{n}%
\glosses{flea thing}%
%------------------------------
\newentry
\headword{mutil}%
\pos{n}%
\glosses{marbles}%
%------------------------------
%------------------------------
\newentry
\headword{na}%
\pos{v}%
\glosses{consume}%
%------------------------------
\newentry
\headword{na=}%
\pos{gramm}%
\glosses{causative}%
%------------------------------
\newentry
\headword{nabaca}%
\pos{v}%
\glosses{read}%
%------------------------------
\newentry
\headword{nabalas}%
\pos{v}%
\glosses{answer}%
%------------------------------
\newentry
\headword{nabaris}%
\pos{v}%
\glosses{line up}%
%------------------------------
%------------------------------
\newentry
\headword{naberuak}%
\pos{v}%
\glosses{drop}%
%------------------------------
\newentry
\headword{nabestai}%
\pos{adv}%
\glosses{well}%
%------------------------------
\newentry
\headword{nabestai bot}%
\pos{phrs}%
\glosses{be careful on your way}%
%------------------------------
\newentry
\headword{nabobar}%
\pos{v}%
\glosses{shiver}%
%------------------------------
\newentry
\headword{nabuka}%
\pos{v}%
\glosses{open}%
%------------------------------
%------------------------------
\newentry
\headword{nabulis}%
\pos{v}%
\glosses{roll}%
%------------------------------
\newentry
\headword{nacerita}%
\pos{v}%
\glosses{tell}%
%------------------------------
\newentry
\headword{nacoba}%
\pos{v}%
\glosses{try}%
%------------------------------
\newentry
\headword{nadadi}%
\pos{v}%
\glosses{weave}%
%------------------------------
\newentry
\headword{nadorong}%
\pos{v}%
\glosses{push}%
%------------------------------
\newentry
\headword{nadou}%
\pos{v}%
\glosses{nod}%
%------------------------------
\newentry
\headword{nafaduli}%
\pos{v}%
\glosses{care}%
%------------------------------
\newentry
\headword{nafafat}%
\pos{v}%
\glosses{slap with hand}%
%------------------------------
\newentry
\headword{nafikir}%
\pos{v}%
\glosses{think}%
%------------------------------
\newentry
\headword{nagaris}%
\pos{v}%
\glosses{draw line}%
%------------------------------
\newentry
\headword{nageiding}%
\pos{v}%
\glosses{grow seedling}%
%------------------------------
\newentry
\headword{nagepi}%
\pos{v}%
\glosses{put clothespin}%
%------------------------------
\newentry
\headword{naharen}%
\pos{n}%
\glosses{leftover}%
%------------------------------
\newentry
\headword{nahimat}%
\pos{vi}%
\glosses{thrifty}%
%------------------------------
\newentry
\headword{nahitung}%
\pos{v}%
\glosses{count}%
%------------------------------
\newentry
\headword{naiar}%
\pos{n}%
\glosses{lontar palm}%
%------------------------------
%------------------------------
\newentry
\headword{nain}%
\pos{adv}%
\glosses{like}%
%------------------------------
\newentry
\headword{*nak}%
\pos{clf}%
\glosses{classifier for fruit}%
%------------------------------
\newentry
\headword{nak}%
\pos{n}%
\glosses{fruit; root vegetable}%
%------------------------------
\newentry
\headword{nak=}%
\pos{gramm}%
\glosses{just}%
%------------------------------
\newentry
\headword{nakabung}%
\pos{v}%
\glosses{stiff muscles}%
%------------------------------
\newentry
\headword{nakafan}%
\pos{v}%
\glosses{wrap in cloth}%
%------------------------------
\newentry
\headword{nakal}%
\pos{n}%
\glosses{head}%
%------------------------------
\newentry
\headword{nakal pokpok}%
\pos{n}%
\glosses{fontanelle}%
%------------------------------
\newentry
\headword{nakirim}%
\pos{v}%
\glosses{send}%
%------------------------------
\newentry
\headword{nakucak}%
\pos{v}%
\glosses{rub, pulverise}%
%------------------------------
\newentry
\headword{nakukus}%
\pos{v}%
\glosses{steam}%
%------------------------------
\newentry
\headword{naladur}%
\pos{v}%
\glosses{massage}%
%------------------------------
\newentry
\headword{nalat}%
\pos{v}%
\glosses{die}%
%------------------------------
\newentry
\headword{naloli}%
\pos{v}%
\glosses{pestle}%
%------------------------------
\newentry
\headword{naluar}%
\pos{v}%
\glosses{slacken}%
%------------------------------
\newentry
\headword{*nam}%
\pos{n}%
\glosses{husband}%
%------------------------------
\newentry
\headword{nam}%
\pos{v}%
\glosses{puddle}%
%------------------------------
\newentry
\headword{namakin}%
\pos{v}%
\glosses{feel uncomfortable}%
%------------------------------
\newentry
\headword{naman}%
\pos{q}%
\glosses{who}%
%------------------------------
\newentry
\headword{naman}%
\pos{vi}%
\glosses{deep}%
%------------------------------
\newentry
\headword{namandi}%
\pos{v}%
\glosses{plane}%
%------------------------------
\newentry
\headword{namangadap}%
\pos{v}%
\glosses{face}%
%------------------------------
%------------------------------
\newentry
\headword{namasawuot}%
\pos{v}%
\glosses{chase}%
%------------------------------
\newentry
\headword{namasuk}%
\pos{v}%
\glosses{give back}%
%------------------------------
%------------------------------
\newentry
\headword{nambiain}%
\pos{n}%
\glosses{starfruit}%
%------------------------------
\newentry
\headword{namenyasal}%
\pos{v}%
\glosses{be sorry}%
%------------------------------
\newentry
\headword{namgon}%
\pos{vi}%
\glosses{married (woman)}%
%------------------------------
\newentry
\headword{namin}%
\pos{v}%
\glosses{put to bed}%
%------------------------------
%------------------------------
%------------------------------
%------------------------------
\newentry
\headword{namot}%
\pos{v}%
\glosses{block}%
%------------------------------
\newentry
\headword{namun}%
\pos{n}%
\glosses{her husband}%
%------------------------------
\newentry
\headword{namun caun}%
\pos{n}%
\glosses{second husband}%
%------------------------------
\newentry
\headword{namusi}%
\pos{v}%
\glosses{kiss}%
%------------------------------
%------------------------------
\newentry
\headword{-nan}%
\pos{n}%
\glosses{too}%
%------------------------------
\newentry
\headword{nanam}%
\pos{n}%
\glosses{east of Karas}%
%------------------------------
\newentry
\headword{nanaun}%
\pos{adv}%
\glosses{quite}%
%------------------------------
\newentry
\headword{nanetkon}%
\pos{v}%
\glosses{curse}%
%------------------------------
\newentry
\headword{nanggan}%
\pos{v}%
\glosses{sing}%
%------------------------------
\newentry
\headword{-naninggan}%
\pos{pro}%
\glosses{quantifying pronominal suffix; all}%
%------------------------------
\newentry
\headword{naon}%
\pos{qnt}%
\glosses{one}%
%------------------------------
\newentry
\headword{napaki}%
\pos{v}%
\glosses{use; wear}%
%------------------------------
\newentry
\headword{naparis}%
\pos{v}%
\glosses{claw; scratch}%
%------------------------------
%------------------------------
\newentry
\headword{napasang}%
\pos{v}%
\glosses{put up}%
%------------------------------
%------------------------------
\newentry
\headword{napinda}%
\pos{v}%
\glosses{move}%
%------------------------------
\newentry
\headword{napinjam}%
\pos{v}%
\glosses{lend; borrow}%
%------------------------------
\newentry
\headword{napinjang}%
\pos{v}%
\glosses{borrow}%
%------------------------------
\newentry
\headword{naputus}%
\pos{v}%
\glosses{break}%
%------------------------------
\newentry
\headword{*nar}%
\pos{clf}%
\glosses{classifier for round things}%
%------------------------------
\newentry
\headword{nar}%
\pos{n}%
\glosses{egg}%
%------------------------------
%------------------------------
\newentry
\headword{narabir}%
\pos{v}%
\glosses{shout}%
%------------------------------
\newentry
\headword{naram}%
\pos{v}%
\glosses{press}%
%------------------------------
\newentry
\headword{naramas}%
\pos{v}%
\glosses{squeeze}%
%------------------------------
\newentry
\headword{narampas}%
\pos{v}%
\glosses{grab}%
%------------------------------
\newentry
\headword{narari}%
\pos{v}%
\glosses{slice}%
%------------------------------
\newentry
\headword{naras}%
\pos{v}%
\glosses{fight}%
%------------------------------
\newentry
\headword{narasa}%
\pos{v}%
\glosses{taste}%
%------------------------------
\newentry
\headword{narasaun}%
\pos{n}%
\glosses{taste}%
%------------------------------
\newentry
\headword{narawi}%
\pos{v}%
\glosses{filter}%
%------------------------------
\newentry
\headword{narekin}%
\pos{v}%
\glosses{count}%
%------------------------------
\newentry
\headword{narer}%
\pos{v}%
\glosses{to plug}%
%------------------------------
\newentry
\headword{nares}%
\pos{v}%
\glosses{too heavy}%
%------------------------------
\newentry
\headword{Narkon}%
\pos{n}%
\glosses{Narkon}%
%------------------------------
\newentry
\headword{narkon}%
\pos{qnt}%
\glosses{one}%
%------------------------------
\newentry
\headword{naroman}%
\pos{v}%
\glosses{lean on chin}%
%------------------------------
\newentry
\headword{narorar}%
\pos{v}%
\glosses{lead}%
%------------------------------
\newentry
\headword{narorik}%
\pos{v}%
\glosses{run aground}%
%------------------------------
\newentry
\headword{naruba}%
\pos{v}%
\glosses{change}%
%------------------------------
\newentry
\headword{narun}%
\pos{n}%
\glosses{egg; seed}%
%------------------------------
\newentry
\headword{narur}%
\pos{v}%
\glosses{run aground}%
%------------------------------
%------------------------------
\newentry
\headword{nasabir}%
\pos{v}%
\glosses{excrete}%
%------------------------------
\newentry
\headword{nasalen}%
\pos{vi}%
\glosses{completely opened}%
%------------------------------
\newentry
\headword{nasalik}%
\pos{v}%
\glosses{circumcise}%
%------------------------------
\newentry
\headword{nasambian}%
\pos{v}%
\glosses{go to mosque; hold service}%
%------------------------------
\newentry
\headword{nasambung}%
\pos{v}%
\glosses{connect}%
%------------------------------
\newentry
\headword{nasandar}%
\pos{v}%
\glosses{dock}%
%------------------------------
%------------------------------
\newentry
\headword{nasangginggir}%
\pos{v}%
\glosses{sail close to the coast}%
%------------------------------
%------------------------------
\newentry
\headword{nasanggur}%
\pos{v}%
\glosses{rinse}%
%------------------------------
%------------------------------
%------------------------------
\newentry
\headword{nasawawi}%
\pos{v}%
\glosses{squint}%
%------------------------------
\newentry
\headword{naseduk}%
\pos{n}%
\glosses{k.o. illness}%
%------------------------------
\newentry
\headword{naseduk}%
\pos{v}%
\glosses{pull}%
%------------------------------
%------------------------------
\newentry
\headword{nasek}%
\pos{v}%
\glosses{break fast}%
%------------------------------
%------------------------------
\newentry
\headword{nasesak}%
\pos{vi}%
\glosses{high tide}%
%------------------------------
\newentry
\headword{nasibur}%
\pos{v}%
\glosses{recite}%
%------------------------------
%------------------------------
%------------------------------
\newentry
\headword{nasirang}%
\pos{v}%
\glosses{pour}%
%------------------------------
%------------------------------
\newentry
\headword{nasiwik}%
\pos{v}%
\glosses{open}%
%------------------------------
\newentry
\headword{nasomit}%
\pos{v}%
\glosses{boil}%
%------------------------------
%------------------------------
%------------------------------
\newentry
\headword{nasuarik}%
\pos{v}%
\glosses{scattered; split}%
%------------------------------
\newentry
\headword{nasuat}%
\pos{v}%
\glosses{tuck in}%
%------------------------------
\newentry
\headword{nasuena}%
\pos{n}%
\glosses{sugar}%
%------------------------------
\newentry
\headword{nasuk}%
\pos{v}%
\glosses{go backwards}%
%------------------------------
\newentry
\headword{nasula}%
\pos{n}%
\glosses{traditional dance}%
%------------------------------
\newentry
\headword{nasula}%
\pos{v}%
\glosses{dance}%
%------------------------------
%------------------------------
%------------------------------
%------------------------------
\newentry
\headword{nasusun}%
\pos{v}%
\glosses{stack}%
%------------------------------
\newentry
\headword{natada}%
\pos{v}%
\glosses{collect water}%
%------------------------------
\newentry
\headword{natanda}%
\pos{v}%
\glosses{sign}%
%------------------------------
\newentry
\headword{natangkis}%
\pos{v}%
\glosses{prevent}%
%------------------------------
\newentry
\headword{natawar}%
\pos{v}%
\glosses{use prayer water}%
%------------------------------
\newentry
\headword{natekin}%
\pos{v}%
\glosses{sign}%
%------------------------------
\newentry
\headword{natewa}%
\pos{v}%
\glosses{punch}%
%------------------------------
\newentry
\headword{natobat}%
\pos{v}%
\glosses{repent}%
%------------------------------
\newentry
\headword{natora}%
\pos{v}%
\glosses{get better}%
%------------------------------
\newentry
\headword{natuka}%
\pos{v}%
\glosses{peck}%
%------------------------------
\newentry
\headword{natukar}%
\pos{v}%
\glosses{prick}%
%------------------------------
\newentry
\headword{natulis}%
\pos{v}%
\glosses{write}%
%------------------------------
\newentry
\headword{natumis}%
\pos{v}%
\glosses{stir-fry}%
%------------------------------
\newentry
\headword{natunggu}%
\pos{v}%
\glosses{wait}%
%------------------------------
\newentry
\headword{nau=}%
\pos{gramm}%
\glosses{reciprocal}%
%------------------------------
\newentry
\headword{nauanona}%
\pos{v}%
\glosses{tidy; balance; clean wood}%
%------------------------------
\newentry
\headword{naubes}%
\pos{v}%
\glosses{make up}%
%------------------------------
\newentry
\headword{nauhalar}%
\pos{v}%
\glosses{get married}%
%------------------------------
\newentry
\headword{naukaia}%
\pos{v}%
\glosses{mate}%
%------------------------------
\newentry
\headword{naukia}%
\pos{vi}%
\glosses{siblings}%
%------------------------------
\newentry
\headword{naukiaka}%
\pos{vi}%
\glosses{siblings}%
%------------------------------
\newentry
\headword{naulanggos}%
\pos{vi}%
\glosses{crossed arms/legs}%
%------------------------------
\newentry
\headword{nauleluk}%
\pos{v}%
\glosses{meet}%
%------------------------------
\newentry
\headword{naun}%
\pos{n}%
\glosses{fruit}%
%------------------------------
\newentry
\headword{naun}%
\pos{n}%
\glosses{soil}%
%------------------------------
\newentry
\headword{naun kerkap}%
\pos{n}%
\glosses{clay}%
%------------------------------
\newentry
\headword{naunak}%
\pos{dv}%
\glosses{show}%
%------------------------------
%------------------------------
%------------------------------
\newentry
\headword{naunin}%
\pos{v}%
\glosses{mark; recognise}%
%------------------------------
\newentry
\headword{naunin}%
\pos{n}%
\glosses{sign}%
%------------------------------
\newentry
\headword{naupar}%
\pos{n}%
\glosses{morning}%
%------------------------------
%------------------------------
\newentry
\headword{naurar}%
\pos{v}%
\glosses{turn around; circle; play.music; wander}%
%------------------------------
%------------------------------
\newentry
\headword{nausair}%
\pos{v}%
\glosses{fight}%
%------------------------------
%------------------------------
%------------------------------
\newentry
\headword{nauwar}%
\pos{n}%
\glosses{news}%
%------------------------------
\newentry
\headword{nauwar tamandi}%
\pos{phrs}%
\glosses{how are you}%
%------------------------------
\newentry
\headword{nauware}%
\pos{vi}%
\glosses{lelah}%
%------------------------------
\newentry
\headword{nawali}%
\pos{v}%
\glosses{return}%
%------------------------------
\newentry
\headword{nawan}%
\pos{v}%
\glosses{serve}%
%------------------------------
\newentry
\headword{nawanen}%
\pos{v}%
\glosses{load}%
%------------------------------
\newentry
\headword{nawanggar}%
\pos{v}%
\glosses{wait}%
%------------------------------
%------------------------------
%------------------------------
\newentry
\headword{nawarak}%
\pos{v}%
\glosses{close off with plank}%
%------------------------------
\newentry
\headword{nawarar}%
\pos{vt}%
\glosses{wake someone up}%
%------------------------------
\newentry
\headword{nawarik}%
\pos{v}%
\glosses{show}%
%------------------------------
\newentry
\headword{nawarir}%
\pos{vi}%
\glosses{be high}%
%------------------------------
%------------------------------
\newentry
\headword{nawaruok}%
\pos{v}%
\glosses{unload}%
%------------------------------
\newentry
\headword{nawas}%
\pos{v}%
\glosses{carry}%
%------------------------------
\newentry
\headword{nawerar}%
\pos{v}%
\glosses{make; do}%
%------------------------------
\newentry
\headword{nawol}%
\pos{v}%
\glosses{stick onto}%
%------------------------------
\newentry
\headword{nawot}%
\pos{v}%
\glosses{layer}%
%------------------------------
\newentry
\headword{nayie}%
\pos{v}%
\glosses{be stuck}%
%------------------------------
\newentry
\headword{-ne}%
\pos{dem}%
\glosses{demonstrative suffix}%
%------------------------------
\newentry
\headword{neba}%
\pos{gramm}%
\glosses{placeholder}%
%------------------------------
\newentry
\headword{neba}%
\pos{q}%
\glosses{what}%
%------------------------------
\newentry
\headword{nebara paruo}%
\pos{phrs}%
\glosses{what are you doing}%
%------------------------------
\newentry
\headword{nebidangat}%
\pos{n}%
\glosses{halfbeak}%
%------------------------------
\newentry
\headword{nebir}%
\pos{n}%
\glosses{k.o. fish}%
%------------------------------
%------------------------------
%------------------------------
\newentry
\headword{neko}%
\pos{n}%
\glosses{inside}%
%------------------------------
%------------------------------
%------------------------------
\newentry
\headword{nemies}%
\pos{vi}%
\glosses{exceed}%
%------------------------------
%------------------------------
\newentry
\headword{nene}%
\pos{n}%
\glosses{cicada}%
%------------------------------
%------------------------------
%------------------------------
%------------------------------
\newentry
\headword{nenggap}%
\pos{q}%
\glosses{why; what}%
%------------------------------
%------------------------------
\newentry
\headword{*ner}%
\pos{n}%
\glosses{inside}%
%------------------------------
%------------------------------
%------------------------------
\newentry
\headword{nerun}%
\pos{n}%
\glosses{inside}%
%------------------------------
\newentry
\headword{nerunggo}%
\pos{n}%
\glosses{inside}%
%------------------------------
\newentry
\headword{newa}%
\pos{vi}%
\glosses{same}%
%------------------------------
\newentry
\headword{newas}%
\pos{v}%
\glosses{stranded}%
%------------------------------
%------------------------------
\newentry
\headword{newer}%
\pos{v}%
\glosses{pay}%
%------------------------------
\newentry
\headword{newer}%
\pos{n}%
\glosses{payment}%
%------------------------------
%------------------------------
%------------------------------
%------------------------------
%------------------------------
%------------------------------
%------------------------------
%------------------------------
%------------------------------
%------------------------------
%------------------------------
%------------------------------
%------------------------------
%------------------------------
%------------------------------
%------------------------------
%------------------------------
%------------------------------
%------------------------------
%------------------------------
%------------------------------
%------------------------------
%------------------------------
%------------------------------
%------------------------------
\newentry
\headword{nika}%
\pos{n}%
\glosses{fishing line}%
%------------------------------
\newentry
\headword{nika}%
\pos{n}%
\glosses{marriage}%
%------------------------------
\newentry
\headword{=nin}%
\pos{gramm}%
\glosses{negator}%
%------------------------------
\newentry
\headword{nina}%
\pos{n}%
\glosses{grandmother}%
%------------------------------
%------------------------------
\newentry
\headword{ninanus}%
\pos{n}%
\glosses{greatgrandmother}%
%------------------------------
\newentry
\headword{ning}%
\pos{n}%
\glosses{illness}%
%------------------------------
\newentry
\headword{ning}%
\pos{v}%
\glosses{ill}%
%------------------------------
%------------------------------
%------------------------------
%------------------------------
\newentry
\headword{nokin}%
\pos{v}%
\glosses{be silent}%
%------------------------------
\newentry
\headword{noknok}%
\pos{v}%
\glosses{whisper}%
%------------------------------
\newentry
\headword{nop}%
\pos{n}%
\glosses{bamboo}%
%------------------------------
\newentry
\headword{nu}%
\pos{v}%
\glosses{machine noise}%
%------------------------------
\newentry
\headword{numur}%
\pos{n}%
\glosses{number}%
%------------------------------
\newentry
\headword{nun}%
\pos{n}%
\glosses{sound}%
%------------------------------
\newentry
\headword{nung}%
\pos{v}%
\glosses{hide}%
%------------------------------
\newentry
\headword{nunggununggu}%
\pos{n}%
\glosses{snail}%
%------------------------------
\newentry
\headword{nunun}%
\pos{v}%
\glosses{close eyes}%
%------------------------------
\newentry
\headword{nyanyi}%
\pos{n}%
\glosses{song}%
%------------------------------
\end{letter}
\begin{letter}{o}
\newentry
\headword{o}%
\pos{int}%
\glosses{emphasis}%
%------------------------------
\newentry
\headword{=o}%
\pos{gramm}%
\glosses{conditional}%
%------------------------------
%------------------------------
\newentry
\headword{ododa}%
\pos{n}%
\glosses{gado-gado}%
%------------------------------
\newentry
\headword{ofin}%
\pos{n}%
\glosses{oven}%
%------------------------------
\newentry
\headword{oh}%
\pos{int}%
\glosses{interjection of surprise}%
%------------------------------
\newentry
\headword{oioi}%
\pos{n}%
\glosses{fish}%
%------------------------------
%------------------------------
\newentry
\headword{okmang}%
\pos{v}%
\glosses{scare; order}%
%------------------------------
\newentry
\headword{*ol}%
\pos{n}%
\glosses{leaf}%
%------------------------------
\newentry
\headword{olol}%
\pos{v}%
\glosses{collect}%
%------------------------------
%------------------------------
\newentry
\headword{olun}%
\pos{n}%
\glosses{leaf}%
%------------------------------
%------------------------------
\newentry
\headword{onda}%
\pos{n}%
\glosses{barracuda}%
%------------------------------
\newentry
\headword{ongkos}%
\pos{n}%
\glosses{expenses}%
%------------------------------
%------------------------------
%------------------------------
\newentry
\headword{opa}%
\pos{dem}%
\glosses{anaphoric}%
%------------------------------
\newentry
\headword{opa}%
\pos{adv}%
\glosses{earlier}%
%------------------------------
%------------------------------
\newentry
\headword{opa yuwa}%
\pos{adv}%
\glosses{today}%
%------------------------------
\newentry
\headword{opatun}%
\pos{adv}%
\glosses{now}%
%------------------------------
%------------------------------
\newentry
\headword{or}%
\pos{n}%
\glosses{back; tail}%
%------------------------------
\newentry
\headword{ora}%
\pos{n}%
\glosses{fish}%
%------------------------------
%------------------------------
\newentry
\headword{oras}%
\pos{n}%
\glosses{time}%
%------------------------------
\newentry
\headword{oronggos}%
\pos{n}%
\glosses{triggerfish}%
%------------------------------
\newentry
\headword{orun}%
\pos{n}%
\glosses{tail}%
%------------------------------
\newentry
\headword{os}%
\pos{n}%
\glosses{sand}%
%------------------------------
\newentry
\headword{os barikbarik}%
\pos{n}%
\glosses{nerite shell}%
%------------------------------
\newentry
\headword{os kibi}%
\pos{n}%
\glosses{sea cucumber}%
%------------------------------
\newentry
\headword{Os Tumun}%
\pos{n}%
\glosses{Os Tumun}%
%------------------------------
\newentry
\headword{osa}%
\pos{dem}%
\glosses{up}%
%------------------------------
\newentry
\headword{osatko}%
\pos{dem}%
\glosses{up there}%
%------------------------------
\newentry
\headword{osep}%
\pos{n}%
\glosses{beach}%
%------------------------------
\newentry
\headword{osie}%
\pos{v}%
\glosses{rest}%
%------------------------------
\newentry
\headword{oskeit}%
\pos{n}%
\glosses{on/at beach}%
%------------------------------
%------------------------------
\newentry
\headword{oskol}%
\pos{n}%
\glosses{beach edge}%
%------------------------------
\newentry
\headword{osmarera}%
\pos{n}%
\glosses{wrasse}%
%------------------------------
\newentry
\headword{ospulpul}%
\pos{n}%
\glosses{fish}%
%------------------------------
\newentry
\headword{Ostem}%
\pos{n}%
\glosses{Ostem}%
%------------------------------
%------------------------------
%------------------------------
\newentry
\headword{our}%
\pos{v}%
\glosses{fall}%
%------------------------------
\newentry
\headword{owa}%
\pos{dem}%
\glosses{over there}%
%------------------------------
\newentry
\headword{owandi}%
\pos{dem}%
\glosses{like here}%
%------------------------------
%------------------------------
\newentry
\headword{owangga}%
\pos{dem}%
\glosses{far distal lative}%
%------------------------------
\newentry
\headword{owatko}%
\pos{dem}%
\glosses{far distal locative}%
%------------------------------
\end{letter}
\begin{letter}{p}
\newentry
\headword{-p}%
\pos{v}%
\glosses{distributive}%
%------------------------------
\newentry
\headword{pabalet}%
\pos{n}%
\glosses{fly}%
%------------------------------
\newentry
\headword{pabie}%
\pos{v}%
\glosses{carry}%
%------------------------------
\newentry
\headword{pabiep}%
\pos{n}%
\glosses{club}%
%------------------------------
%------------------------------
\newentry
\headword{padamual}%
\pos{n}%
\glosses{pandanus}%
%------------------------------
%------------------------------
\newentry
\headword{paden}%
\pos{n}%
\glosses{pole}%
%------------------------------
\newentry
\headword{paden raor}%
\pos{n}%
\glosses{pole}%
%------------------------------
\newentry
\headword{paden tabur}%
\pos{n}%
\glosses{king post}%
%------------------------------
\newentry
\headword{padewaden}%
\pos{n}%
\glosses{poles}%
%------------------------------
\newentry
\headword{padi}%
\pos{n}%
\glosses{rice hull; rice.plant}%
%------------------------------
%------------------------------
\newentry
\headword{paer}%
\pos{n}%
\glosses{venus clam}%
%------------------------------
\newentry
\headword{pahercie}%
\pos{vi}%
\glosses{opened}%
%------------------------------
\newentry
\headword{paherma}%
\pos{v}%
\glosses{open}%
%------------------------------
\newentry
\headword{paisor}%
\pos{n}%
\glosses{sea current}%
%------------------------------
\newentry
\headword{paisor kesun}%
\pos{n}%
\glosses{spiral coral}%
%------------------------------
%------------------------------
\newentry
\headword{pak}%
\pos{n}%
\glosses{moonfish; spadefish}%
%------------------------------
\newentry
\headword{pak}%
\pos{n}%
\glosses{k.o. plant}%
%------------------------------
\newentry
\headword{pak}%
\pos{v}%
\glosses{chew}%
%------------------------------
\newentry
\headword{pak}%
\pos{n}%
\glosses{nail}%
%------------------------------
\newentry
\headword{pak}%
\pos{n}%
\glosses{moon; month}%
%------------------------------
\newentry
\headword{pak mangmang}%
\pos{n}%
\glosses{k.o. plant}%
%------------------------------
\newentry
\headword{pak talawak}%
\pos{n}%
\glosses{new moon}%
%------------------------------
\newentry
\headword{pak tubak}%
\pos{phrs}%
\glosses{full moon}%
%------------------------------
%------------------------------
\newentry
\headword{pakmang}%
\pos{v}%
\glosses{explode}%
%------------------------------
%------------------------------
\newentry
\headword{Pakpak}%
\pos{n}%
\glosses{Fakfak (town)}%
%------------------------------
\newentry
\headword{pakpak}%
\pos{v}%
\glosses{braid}%
%------------------------------
%------------------------------
\newentry
\headword{paksa}%
\pos{v}%
\glosses{force}%
%------------------------------
\newentry
\headword{paksanual}%
\pos{n}%
\glosses{fish}%
%------------------------------
\newentry
\headword{paku}%
\pos{v}%
\glosses{nail}%
%------------------------------
\newentry
\headword{palang}%
\pos{n}%
\glosses{fern}%
%------------------------------
\newentry
\headword{palawak}%
\pos{vi}%
\glosses{slippery}%
%------------------------------
\newentry
\headword{pale}%
\pos{v}%
\glosses{make a bamboo floor}%
%------------------------------
%------------------------------
\newentry
\headword{palom}%
\pos{v}%
\glosses{spit}%
%------------------------------
\newentry
\headword{palom}%
\pos{n}%
\glosses{spit}%
%------------------------------
\newentry
\headword{*pan}%
\pos{clf}%
\glosses{classifier for heaps}%
%------------------------------
\newentry
\headword{pan}%
\pos{n}%
\glosses{basket}%
%------------------------------
\newentry
\headword{*pan}%
\pos{n}%
\glosses{heap}%
%------------------------------
\newentry
\headword{panci}%
\pos{n}%
\glosses{pan}%
%------------------------------
\newentry
\headword{pandoki}%
\pos{n}%
\glosses{nypa palm}%
%------------------------------
\newentry
\headword{pang}%
\pos{n}%
\glosses{summit}%
%------------------------------
\newentry
\headword{pang}%
\pos{n}%
\glosses{washingtub}%
%------------------------------
\newentry
\headword{panggal}%
\pos{n}%
\glosses{spider conch}%
%------------------------------
\newentry
\headword{panggala}%
\pos{n}%
\glosses{cassava}%
%------------------------------
\newentry
\headword{panggala}%
\pos{vi}%
\glosses{swollen}%
%------------------------------
\newentry
\headword{panggat}%
\pos{n}%
\glosses{measure}%
%------------------------------
\newentry
\headword{panggat}%
\pos{v}%
\glosses{walk with big steps}%
%------------------------------
\newentry
\headword{panggat}%
\pos{n}%
\glosses{step}%
%------------------------------
\newentry
\headword{panggatpanggat}%
\pos{n}%
\glosses{caterpillar}%
%------------------------------
\newentry
\headword{panggawangga}%
\pos{n}%
\glosses{leech}%
%------------------------------
\newentry
\headword{panggut}%
\pos{n}%
\glosses{wood tool}%
%------------------------------
\newentry
\headword{paning}%
\pos{v}%
\glosses{ask; call}%
%------------------------------
\newentry
\headword{panok}%
\pos{n}%
\glosses{order; promise}%
%------------------------------
\newentry
\headword{panok}%
\pos{v}%
\glosses{order}%
%------------------------------
\newentry
\headword{panok mecuan}%
\pos{n}%
\glosses{confirmation}%
%------------------------------
\newentry
\headword{panokpanok}%
\pos{v}%
\glosses{ask permission to leave}%
%------------------------------
\newentry
\headword{panpan}%
\pos{n}%
\glosses{limpet shell}%
%------------------------------
%------------------------------
%------------------------------
\newentry
\headword{paos}%
\pos{n}%
\glosses{mud}%
%------------------------------
%------------------------------
\newentry
\headword{parai}%
\pos{n}%
\glosses{owl}%
%------------------------------
\newentry
\headword{parair}%
\pos{v}%
\glosses{split; break}%
%------------------------------
\newentry
\headword{parambura}%
\pos{n}%
\glosses{curse}%
%------------------------------
\newentry
\headword{paramua}%
\pos{v}%
\glosses{cut}%
%------------------------------
\newentry
\headword{paramuang}%
\pos{n}%
\glosses{crocodile}%
%------------------------------
\newentry
\headword{paran}%
\pos{v}%
\glosses{put up wall}%
%------------------------------
\newentry
\headword{paransik}%
\pos{vi}%
\glosses{near}%
%------------------------------
\newentry
\headword{parar}%
\pos{v}%
\glosses{wake.up}%
%------------------------------
\newentry
\headword{Parar}%
\pos{n}%
\glosses{Fatar}%
%------------------------------
\newentry
\headword{parara}%
\pos{v}%
\glosses{extend on floor}%
%------------------------------
%------------------------------
%------------------------------
%------------------------------
\newentry
\headword{pararuo}%
\pos{v}%
\glosses{fly}%
%------------------------------
\newentry
\headword{paras}%
\pos{n}%
\glosses{embers}%
%------------------------------
%------------------------------
\newentry
\headword{parein}%
\pos{dv}%
\glosses{sell}%
%------------------------------
\newentry
\headword{pareinun}%
\pos{n}%
\glosses{price}%
%------------------------------
\newentry
\headword{pareir}%
\pos{v}%
\glosses{follow}%
%------------------------------
%------------------------------
\newentry
\headword{parenta}%
\pos{v}%
\glosses{command}%
%------------------------------
%------------------------------
%------------------------------
%------------------------------
%------------------------------
%------------------------------
%------------------------------
%------------------------------
\newentry
\headword{paritman}%
\pos{v}%
\glosses{claw}%
%------------------------------
%------------------------------
\newentry
\headword{parlu}%
\pos{v}%
\glosses{need}%
%------------------------------
%------------------------------
\newentry
\headword{parok}%
\pos{n}%
\glosses{finger}%
%------------------------------
\newentry
\headword{paror}%
\pos{n}%
\glosses{raised platform}%
%------------------------------
%------------------------------
%------------------------------
%------------------------------
\newentry
\headword{parua}%
\pos{v}%
\glosses{pluck}%
%------------------------------
\newentry
\headword{paruak}%
\pos{v}%
\glosses{throw aside; throw away; drop}%
%------------------------------
\newentry
\headword{parun}%
\pos{n}%
\glosses{wing; fin}%
%------------------------------
\newentry
\headword{paruo}%
\pos{v}%
\glosses{do; make}%
%------------------------------
\newentry
\headword{paruok}%
\pos{v}%
\glosses{exit; fruit}%
%------------------------------
\newentry
\headword{paruowaruo}%
\pos{v}%
\glosses{do; make}%
%------------------------------
%------------------------------
%------------------------------
\newentry
\headword{pas}%
\pos{v}%
\glosses{fit}%
%------------------------------
\newentry
\headword{pas}%
\pos{n}%
\glosses{female; woman}%
%------------------------------
\newentry
\headword{pas}%
\pos{adv}%
\glosses{exactly}%
%------------------------------
\newentry
\headword{pasa}%
\pos{n}%
\glosses{rice}%
%------------------------------
\newentry
\headword{pasar}%
\pos{n}%
\glosses{market}%
%------------------------------
%------------------------------
\newentry
\headword{pasarom}%
\pos{n}%
\glosses{ambarella}%
%------------------------------
%------------------------------
\newentry
\headword{pasiem}%
\pos{n}%
\glosses{yellow taro}%
%------------------------------
\newentry
\headword{pasienggara bot}%
\pos{v}%
\glosses{defecate}%
%------------------------------
\newentry
\headword{pasiep}%
\pos{v}%
\glosses{expel}%
%------------------------------
\newentry
\headword{pasier}%
\pos{n}%
\glosses{sea}%
%------------------------------
\newentry
\headword{pasier up}%
\pos{phrs}%
\glosses{calm sea}%
%------------------------------
\newentry
\headword{pasierbol}%
\pos{n}%
\glosses{shore}%
%------------------------------
%------------------------------
%------------------------------
%------------------------------
\newentry
\headword{pasirwasir}%
\pos{n}%
\glosses{brackish}%
%------------------------------
\newentry
\headword{paskot}%
\pos{n}%
\glosses{widow}%
%------------------------------
%------------------------------
\newentry
\headword{pasor}%
\pos{v}%
\glosses{fry}%
%------------------------------
\newentry
\headword{pasparin}%
\pos{n}%
\glosses{bride price}%
%------------------------------
%------------------------------
\newentry
\headword{pat}%
\pos{v}%
\glosses{sew}%
%------------------------------
\newentry
\headword{pat}%
\pos{n}%
\glosses{wing}%
%------------------------------
\newentry
\headword{patin}%
\pos{n}%
\glosses{sore}%
%------------------------------
\newentry
\headword{patin}%
\pos{vi}%
\glosses{wounded}%
%------------------------------
\newentry
\headword{patin ter}%
\pos{n}%
\glosses{scar}%
%------------------------------
\newentry
\headword{Patipi}%
\pos{n}%
\glosses{Onin people}%
%------------------------------
%------------------------------
\newentry
\headword{pau}%
\pos{v}%
\glosses{cook}%
%------------------------------
\newentry
\headword{pau}%
\pos{n}%
\glosses{earth oven}%
%------------------------------
\newentry
\headword{pawan}%
\pos{n}%
\glosses{plank}%
%------------------------------
\newentry
\headword{pawan tabak}%
\pos{n}%
\glosses{canoe plank}%
%------------------------------
\newentry
\headword{payiem}%
\pos{v}%
\glosses{fill}%
%------------------------------
\newentry
\headword{-pe}%
\pos{n}%
\glosses{our (\textsc{in})}%
%------------------------------
\newentry
\headword{pearip}%
\pos{n}%
\glosses{water between roots}%
%------------------------------
\newentry
\headword{pebis}%
\pos{n}%
\glosses{woman}%
%------------------------------
\newentry
\headword{Pebis Ruomun}%
\pos{n}%
\glosses{Pebis Ruomun}%
%------------------------------
\newentry
\headword{peik}%
\pos{n}%
\glosses{k.o. tree}%
%------------------------------
\newentry
\headword{pel}%
\pos{v}%
\glosses{cut; split}%
%------------------------------
\newentry
\headword{*pel}%
\pos{clf}%
\glosses{classifier for bunches}%
%------------------------------
\newentry
\headword{pel}%
\pos{n}%
\glosses{bunch}%
%------------------------------
\newentry
\headword{pelalang}%
\pos{n}%
\glosses{hot water}%
%------------------------------
\newentry
\headword{pelelu}%
\pos{n}%
\glosses{cold evening wind}%
%------------------------------
\newentry
\headword{peler}%
\pos{n}%
\glosses{mast}%
%------------------------------
%------------------------------
\newentry
\headword{pen}%
\pos{vi}%
\glosses{tasty; sweet}%
%------------------------------
%------------------------------
%------------------------------
%------------------------------
\newentry
\headword{penyakit}%
\pos{n}%
\glosses{illness; curse}%
%------------------------------
\newentry
\headword{penyakit kat nanetkon}%
\pos{phrs}%
\glosses{curse}%
%------------------------------
\newentry
\headword{pep}%
\pos{n}%
\glosses{pig}%
%------------------------------
\newentry
\headword{Pepgor Karimun}%
\pos{n}%
\glosses{Pep Karimun}%
%------------------------------
\newentry
\headword{Pepmang}%
\pos{n}%
\glosses{Indonesian}%
%------------------------------
\newentry
\headword{per}%
\pos{n}%
\glosses{water}%
%------------------------------
\newentry
\headword{per iriskap}%
\pos{n}%
\glosses{drinking water}%
%------------------------------
\newentry
\headword{per kerkap}%
\pos{n}%
\glosses{tea}%
%------------------------------
\newentry
\headword{per kuskap}%
\pos{n}%
\glosses{coffee}%
%------------------------------
\newentry
\headword{per natawarten}%
\pos{n}%
\glosses{holy water}%
%------------------------------
\newentry
\headword{per paiwai}%
\pos{n}%
\glosses{dragonfly}%
%------------------------------
\newentry
\headword{per pasirwasir}%
\pos{n}%
\glosses{brackish water}%
%------------------------------
\newentry
\headword{per taluk}%
\pos{phrs}%
\glosses{flowing water}%
%------------------------------
%------------------------------
%------------------------------
\newentry
\headword{perbol}%
\pos{n}%
\glosses{river bank}%
%------------------------------
%------------------------------
%------------------------------
%------------------------------
\newentry
\headword{perki}%
\pos{n}%
\glosses{waterfall}%
%------------------------------
\newentry
\headword{perna}%
\pos{adv}%
\glosses{ever}%
%------------------------------
\newentry
\headword{pertam}%
\pos{n}%
\glosses{tears}%
%------------------------------
%------------------------------
%------------------------------
%------------------------------
\newentry
\headword{perusahan}%
\pos{n}%
\glosses{company}%
%------------------------------
\newentry
\headword{pes}%
\pos{v}%
\glosses{peel}%
%------------------------------
\newentry
\headword{pes}%
\pos{n}%
\glosses{peel}%
%------------------------------
\newentry
\headword{pesawat}%
\pos{n}%
\glosses{plane}%
%------------------------------
\newentry
\headword{pesawesa}%
\pos{n}%
\glosses{spatula}%
%------------------------------
\newentry
\headword{pespes}%
\pos{n}%
\glosses{leftovers}%
%------------------------------
\newentry
\headword{peti}%
\pos{n}%
\glosses{box}%
%------------------------------
%------------------------------
\newentry
\headword{pi}%
\pos{pro}%
\glosses{we (\textsc{in})}%
%------------------------------
\newentry
\headword{pier}%
\pos{pro}%
\glosses{we two (\textsc{in})}%
%------------------------------
\newentry
\headword{pihutak}%
\pos{pro}%
\glosses{we alone (\textsc{in})}%
%------------------------------
%------------------------------
\newentry
\headword{pikiran}%
\pos{n}%
\glosses{thoughts}%
%------------------------------
%------------------------------
\newentry
\headword{pin}%
\pos{pro}%
\glosses{our (\textsc{in})}%
%------------------------------
\newentry
\headword{pinaninggan}%
\pos{pro}%
\glosses{we all (\textsc{in})}%
%------------------------------
%------------------------------
\newentry
\headword{pingan}%
\pos{n}%
\glosses{plate}%
%------------------------------
\newentry
\headword{Pinggor}%
\pos{n}%
\glosses{Pinggor}%
%------------------------------
%------------------------------
%------------------------------
%------------------------------
%------------------------------
%------------------------------
\newentry
\headword{pirain}%
\pos{pro}%
\glosses{we alone (\textsc{in})}%
%------------------------------
\newentry
\headword{pirawilak}%
\pos{n}%
\glosses{k.o.tree}%
%------------------------------
\newentry
\headword{-pis}%
\pos{n}%
\glosses{side}%
%------------------------------
\newentry
\headword{Pisor}%
\pos{n}%
\glosses{Pisor}%
%------------------------------
\newentry
\headword{pitis}%
\pos{n}%
\glosses{money}%
%------------------------------
\newentry
\headword{pitisnaharen}%
\pos{n}%
\glosses{change}%
%------------------------------
%------------------------------
%------------------------------
%------------------------------
%------------------------------
\newentry
\headword{po}%
\pos{v}%
\glosses{anchor}%
%------------------------------
\newentry
\headword{po}%
\pos{n}%
\glosses{breadfruit}%
%------------------------------
\newentry
\headword{poalot}%
\pos{n}%
\glosses{mooring spot}%
%------------------------------
\newentry
\headword{poar}%
\pos{n}%
\glosses{k.o. shell}%
%------------------------------
%------------------------------
\newentry
\headword{pokpok}%
\pos{n}%
\glosses{small motor}%
%------------------------------
\newentry
\headword{pol}%
\pos{n}%
\glosses{sap; latex; gum}%
%------------------------------
\newentry
\headword{pol}%
\pos{v}%
\glosses{compact; smooth}%
%------------------------------
\newentry
\headword{polas}%
\pos{n}%
\glosses{slice}%
%------------------------------
\newentry
\headword{polas}%
\pos{v}%
\glosses{slice}%
%------------------------------
%------------------------------
\newentry
\headword{polkayak}%
\pos{n}%
\glosses{papaya}%
%------------------------------
\newentry
\headword{ponggul}%
\pos{n}%
\glosses{k.o. fish}%
%------------------------------
\newentry
\headword{porkang}%
\pos{n}%
\glosses{stone.hole}%
%------------------------------
\newentry
\headword{pos}%
\pos{n}%
\glosses{hole}%
%------------------------------
\newentry
\headword{posiwosi}%
\pos{n}%
\glosses{spear with one point}%
%------------------------------
\newentry
\headword{potma}%
\pos{v}%
\glosses{cut}%
%------------------------------
%------------------------------
%------------------------------
\newentry
\headword{potpot}%
\pos{n}%
\glosses{blue-spotted stingray}%
%------------------------------
\newentry
\headword{pouk}%
\pos{v}%
\glosses{float}%
%------------------------------
\newentry
\headword{poun}%
\pos{n}%
\glosses{bundle}%
%------------------------------
\newentry
\headword{*poup}%
\pos{n}%
\glosses{bundle}%
%------------------------------
\newentry
\headword{poup}%
\pos{v}%
\glosses{carry living being on back}%
%------------------------------
\newentry
\headword{*poup}%
\pos{clf}%
\glosses{classifier for bundles}%
%------------------------------
\newentry
\headword{poupkon}%
\pos{qnt}%
\glosses{one}%
%------------------------------
%------------------------------
\newentry
\headword{Pour}%
\pos{n}%
\glosses{Faor}%
%------------------------------
\newentry
\headword{pous}%
\pos{n}%
\glosses{fish}%
%------------------------------
%------------------------------
\newentry
\headword{powar}%
\pos{n}%
\glosses{horn shell}%
%------------------------------
%------------------------------
\newentry
\headword{pue}%
\pos{v}%
\glosses{hit}%
%------------------------------
\newentry
\headword{pueselet}%
\pos{n}%
\glosses{spider}%
%------------------------------
%------------------------------
\newentry
\headword{pukmang}%
\pos{v}%
\glosses{fall and make a sound}%
%------------------------------
\newentry
\headword{pul}%
\pos{n}%
\glosses{wing}%
%------------------------------
\newentry
\headword{pulem}%
\pos{v}%
\glosses{blink}%
%------------------------------
\newentry
\headword{pulisi}%
\pos{n}%
\glosses{police}%
%------------------------------
%------------------------------
%------------------------------
\newentry
\headword{pulkiet}%
\pos{n}%
\glosses{betel stem}%
%------------------------------
\newentry
\headword{pulma}%
\pos{vt}%
\glosses{pinch}%
%------------------------------
\newentry
\headword{pulor}%
\pos{n}%
\glosses{betel vine}%
%------------------------------
\newentry
\headword{pulpul}%
\pos{n}%
\glosses{butterfly}%
%------------------------------
\newentry
\headword{pulpulkon}%
\pos{v}%
\glosses{fly around}%
%------------------------------
\newentry
\headword{pulseka}%
\pos{n}%
\glosses{grasshopper}%
%------------------------------
%------------------------------
%------------------------------
\newentry
\headword{pungpunggat}%
\pos{n}%
\glosses{fish}%
%------------------------------
\newentry
\headword{*pur}%
\pos{clf}%
\glosses{classifier for pieces}%
%------------------------------
\newentry
\headword{pur}%
\pos{n}%
\glosses{betel fruit}%
%------------------------------
\newentry
\headword{puraman}%
\pos{qnt}%
\glosses{how many}%
%------------------------------
\newentry
\headword{purap}%
\pos{qnt}%
\glosses{fifty}%
%------------------------------
\newentry
\headword{purarar}%
\pos{vi}%
\glosses{messy}%
%------------------------------
%------------------------------
\newentry
\headword{purir}%
\pos{qnt}%
\glosses{twenty}%
%------------------------------
\newentry
\headword{purir ba kon}%
\pos{qnt}%
\glosses{twenty-one}%
%------------------------------
%------------------------------
\newentry
\headword{pururu}%
\pos{v}%
\glosses{fall}%
%------------------------------
\newentry
\headword{pus}%
\pos{v}%
\glosses{flower}%
%------------------------------
\newentry
\headword{pus}%
\pos{n}%
\glosses{foam}%
%------------------------------
\newentry
\headword{pus}%
\pos{n}%
\glosses{flower}%
%------------------------------
\newentry
\headword{pusing}%
\pos{v}%
\glosses{confused; bothered}%
%------------------------------
\newentry
\headword{pusir}%
\pos{n}%
\glosses{bow}%
%------------------------------
\newentry
\headword{*put}%
\pos{qnt}%
\glosses{tens of}%
%------------------------------
\newentry
\headword{putirie}%
\pos{qnt}%
\glosses{eighty}%
%------------------------------
\newentry
\headword{putkaninggonie}%
\pos{qnt}%
\glosses{ninety}%
%------------------------------
\newentry
\headword{putkaninggonie talin kaninggonie}%
\pos{qnt}%
\glosses{ninety-nine}%
%------------------------------
\newentry
\headword{putkansuor}%
\pos{qnt}%
\glosses{forty}%
%------------------------------
\newentry
\headword{putkansuor talinggon}%
\pos{qnt}%
\glosses{forty-one}%
%------------------------------
\newentry
\headword{putkaruok}%
\pos{qnt}%
\glosses{thirty}%
%------------------------------
%------------------------------
\newentry
\headword{putkaruok talinggansuor}%
\pos{qnt}%
\glosses{thirty-four}%
%------------------------------
\newentry
\headword{putkaruok talinggaruok}%
\pos{qnt}%
\glosses{thirty-three}%
%------------------------------
\newentry
\headword{putkaruok talinggon}%
\pos{qnt}%
\glosses{thirty-one}%
%------------------------------
\newentry
\headword{putkaruok talinir}%
\pos{qnt}%
\glosses{thirty-two}%
%------------------------------
%------------------------------
\newentry
\headword{putkon}%
\pos{qnt}%
\glosses{ten}%
%------------------------------
\newentry
\headword{putkon ba ap}%
\pos{qnt}%
\glosses{fifteen}%
%------------------------------
\newentry
\headword{putkon ba eir}%
\pos{qnt}%
\glosses{twelve}%
%------------------------------
\newentry
\headword{putkon ba karuok}%
\pos{qnt}%
\glosses{thirteen}%
%------------------------------
\newentry
\headword{putkon ba kon}%
\pos{qnt}%
\glosses{eleven}%
%------------------------------
%------------------------------
\newentry
\headword{putraman}%
\pos{qnt}%
\glosses{sixty}%
%------------------------------
\newentry
\headword{putramandalin}%
\pos{qnt}%
\glosses{seventy}%
%------------------------------
\end{letter}
\begin{letter}{r}
\newentry
\headword{=r}%
\pos{gramm}%
\glosses{plural imperative}%
%------------------------------
\newentry
\headword{ra}%
\pos{v}%
\glosses{move; install; become}%
%------------------------------
\newentry
\headword{ra}%
\pos{v}%
\glosses{hear}%
%------------------------------
%------------------------------
%------------------------------
\newentry
\headword{rak}%
\pos{n}%
\glosses{shelf}%
%------------------------------
%------------------------------
\newentry
\headword{ram}%
\pos{n}%
\glosses{coral reef}%
%------------------------------
\newentry
\headword{ram kolkemkem}%
\pos{n}%
\glosses{k.o. coral}%
%------------------------------
\newentry
\headword{ram parokparok}%
\pos{n}%
\glosses{k.o. coral}%
%------------------------------
\newentry
\headword{ram tomtom}%
\pos{n}%
\glosses{table coral}%
%------------------------------
\newentry
\headword{raman}%
\pos{qnt}%
\glosses{six}%
%------------------------------
\newentry
\headword{ramandalin}%
\pos{qnt}%
\glosses{seven}%
%------------------------------
%------------------------------
\newentry
\headword{rambu}%
\pos{n}%
\glosses{sejenis kuskus}%
%------------------------------
\newentry
\headword{rami}%
\pos{v}%
\glosses{busily}%
%------------------------------
\newentry
\headword{ramie}%
\pos{v}%
\glosses{pull; drag}%
%------------------------------
\newentry
\headword{ran mian}%
\pos{v}%
\glosses{back and forth}%
%------------------------------
\newentry
\headword{rane}%
\pos{v}%
\glosses{make noise}%
%------------------------------
\newentry
\headword{rang}%
\pos{n}%
\glosses{open sea}%
%------------------------------
\newentry
\headword{rangrang}%
\pos{vi}%
\glosses{lukewarm}%
%------------------------------
\newentry
\headword{ranti}%
\pos{n}%
\glosses{chain}%
%------------------------------
\newentry
\headword{raor}%
\pos{n}%
\glosses{middle}%
%------------------------------
\newentry
\headword{raorko}%
\pos{n}%
\glosses{in the middle}%
%------------------------------
\newentry
\headword{rap}%
\pos{v}%
\glosses{laugh}%
%------------------------------
\newentry
\headword{rapat}%
\sensenr{}%
\pos{v}%
\glosses{tight}%
\sensenr{}%
\pos{v}%
\glosses{meet}%
%------------------------------
%------------------------------
\newentry
\headword{Rarait}%
\pos{n}%
\glosses{Seram}%
%------------------------------
%------------------------------
%------------------------------
%------------------------------
\newentry
\headword{rarie}%
\pos{v}%
\glosses{make soft}%
%------------------------------
%------------------------------
\newentry
\headword{rasa}%
\pos{v}%
\glosses{like}%
%------------------------------
\newentry
\headword{rasemsem}%
\pos{n}%
\glosses{pimples}%
%------------------------------
%------------------------------
\newentry
\headword{rawarawa}%
\pos{v}%
\glosses{laugh}%
%------------------------------
%------------------------------
\newentry
\headword{regil}%
\pos{n}%
\glosses{beam}%
%------------------------------
%------------------------------
\newentry
\headword{reidak}%
\pos{qnt}%
\glosses{much}%
%------------------------------
\newentry
\headword{reidaksawe}%
\pos{vi}%
\glosses{too much}%
%------------------------------
\newentry
\headword{rein}%
\pos{qnt}%
\glosses{much}%
%------------------------------
\newentry
\headword{reingge}%
\pos{qnt}%
\glosses{not much}%
%------------------------------
\newentry
\headword{reirap}%
\pos{qnt}%
\glosses{five hundred}%
%------------------------------
\newentry
\headword{*reit}%
\pos{qnt}%
\glosses{hundred}%
%------------------------------
\newentry
\headword{reitkon}%
\pos{qnt}%
\glosses{one hundred}%
%------------------------------
%------------------------------
\newentry
\headword{reon}%
\pos{adv}%
\glosses{maybe}%
%------------------------------
\newentry
\headword{rep}%
\pos{v}%
\glosses{get}%
%------------------------------
%------------------------------
\newentry
\headword{rer}%
\pos{n}%
\glosses{k.o. tree}%
%------------------------------
\newentry
\headword{rer}%
\pos{v}%
\glosses{chat; tell a story}%
%------------------------------
\newentry
\headword{rer}%
\pos{n}%
\glosses{true conch}%
%------------------------------
\newentry
\headword{rer}%
\pos{n}%
\glosses{story}%
%------------------------------
%------------------------------
\newentry
\headword{rerer}%
\pos{n}%
\glosses{shell}%
%------------------------------
\newentry
\headword{resan}%
\pos{n}%
\glosses{hammer}%
%------------------------------
\newentry
\headword{-rip}%
\pos{dem}%
\glosses{demonstrative marker expressing degree}%
%------------------------------
\newentry
\headword{ripi}%
\pos{qnt}%
\glosses{thousand}%
%------------------------------
\newentry
\headword{ripion}%
\pos{qnt}%
\glosses{one thousend}%
%------------------------------
\newentry
\headword{ririn}%
\pos{vi}%
\glosses{tall}%
%------------------------------
\newentry
\headword{roba}%
\pos{n}%
\glosses{Wednesday}%
%------------------------------
\newentry
\headword{robaherkiem}%
\pos{v}%
\glosses{holiday}%
%------------------------------
\newentry
\headword{robaherpak}%
\pos{n}%
\glosses{name of a month}%
%------------------------------
%------------------------------
\newentry
\headword{rombongan}%
\pos{n}%
\glosses{group}%
%------------------------------
\newentry
\headword{rontang}%
\pos{n}%
\glosses{pie}%
%------------------------------
\newentry
\headword{ror}%
\pos{n}%
\glosses{tree; wood}%
%------------------------------
\newentry
\headword{ror buabua}%
\pos{n}%
\glosses{k.o. tree}%
%------------------------------
\newentry
\headword{ror garta}%
\pos{n}%
\glosses{rubber tree}%
%------------------------------
\newentry
\headword{ror iriskap}%
\pos{n}%
\glosses{gum tree; eucalyptus}%
%------------------------------
\newentry
\headword{ror kulun}%
\pos{n}%
\glosses{bark}%
%------------------------------
\newentry
\headword{ror soren}%
\pos{n}%
\glosses{unprocessed wood}%
%------------------------------
\newentry
\headword{ror tabur}%
\pos{n}%
\glosses{tree stump}%
%------------------------------
\newentry
\headword{rorap}%
\pos{n}%
\glosses{foundation}%
%------------------------------
\newentry
\headword{rorkarok}%
\pos{n}%
\glosses{branch}%
%------------------------------
%------------------------------
%------------------------------
\newentry
\headword{rouk}%
\pos{v}%
\glosses{fall over}%
%------------------------------
\newentry
\headword{roukmang}%
\pos{v}%
\glosses{call out}%
%------------------------------
\newentry
\headword{roung}%
\pos{n}%
\glosses{k.o. shell}%
%------------------------------
\newentry
\headword{roye}%
\pos{v}%
\glosses{turn}%
%------------------------------
%------------------------------
%------------------------------
\newentry
\headword{rua}%
\pos{v}%
\glosses{extinguish; kill}%
%------------------------------
\newentry
\headword{ruak}%
\pos{v}%
\glosses{fall.fruit}%
%------------------------------
\newentry
\headword{ruam}%
\pos{n}%
\glosses{sweat}%
%------------------------------
\newentry
\headword{ruan}%
\pos{vi}%
\glosses{swollen}%
%------------------------------
\newentry
\headword{ruar}%
\pos{n}%
\glosses{shark}%
%------------------------------
\newentry
\headword{ruar bodaren}%
\pos{n}%
\glosses{shark}%
%------------------------------
\newentry
\headword{ruar kanggir nungnung}%
\pos{n}%
\glosses{tawny nurse shark}%
%------------------------------
\newentry
\headword{ruar tagirigiri}%
\pos{n}%
\glosses{shark}%
%------------------------------
%------------------------------
\newentry
\headword{rum}%
\pos{n}%
\glosses{k.o. fish}%
%------------------------------
\newentry
\headword{rum timbang}%
\pos{n}%
\glosses{fish}%
%------------------------------
\newentry
\headword{ruma tangga}%
\pos{n}%
\glosses{household}%
%------------------------------
\newentry
\headword{ruma tangga}%
\pos{v}%
\glosses{have a family}%
%------------------------------
%------------------------------
%------------------------------
%------------------------------
%------------------------------
%------------------------------
\newentry
\headword{Rumbati}%
\pos{n}%
\glosses{Onin people}%
%------------------------------
\newentry
\headword{rumrum}%
\pos{n}%
\glosses{k.o. plant}%
%------------------------------
\newentry
\headword{ruo}%
\pos{v}%
\glosses{dig}%
%------------------------------
\newentry
\headword{ruo}%
\pos{vi}%
\glosses{cooked}%
%------------------------------
\newentry
\headword{ruom}%
\pos{n}%
\glosses{foothill}%
%------------------------------
\newentry
\headword{ruop}%
\pos{n}%
\glosses{plug}%
%------------------------------
\newentry
\headword{rup}%
\pos{v}%
\glosses{help}%
%------------------------------
%------------------------------
\newentry
\headword{*rur}%
\pos{clf}%
\glosses{classifier for skewers}%
%------------------------------
\newentry
\headword{rur}%
\pos{v}%
\glosses{skewer}%
%------------------------------
\newentry
\headword{rur}%
\pos{n}%
\glosses{casuarina (tree)}%
%------------------------------
\newentry
\headword{rurkon}%
\pos{qnt}%
\glosses{one}%
%------------------------------
\newentry
\headword{rusa}%
\pos{n}%
\glosses{deer}%
%------------------------------
\newentry
\headword{rusing}%
\pos{n}%
\glosses{mortar}%
%------------------------------
\newentry
\headword{rusinggain}%
\pos{n}%
\glosses{pestle for coconut and kanari nut}%
%------------------------------
\end{letter}
\begin{letter}{s}
\newentry
\headword{sa}%
\pos{vi}%
\glosses{dry}%
%------------------------------
\newentry
\headword{saban}%
\pos{n}%
\glosses{k.o. big bamboo}%
%------------------------------
\newentry
\headword{Sabaor}%
\pos{n}%
\glosses{Buruwai}%
%------------------------------
\newentry
\headword{sabar}%
\pos{n}%
\glosses{front (of a boat)}%
%------------------------------
\newentry
\headword{sabarak}%
\pos{n}%
\glosses{space under house}%
%------------------------------
\newentry
\headword{sabel}%
\pos{n}%
\glosses{cleared forest}%
%------------------------------
\newentry
\headword{sabet}%
\pos{n}%
\glosses{boil}%
%------------------------------
\newentry
\headword{sabtu}%
\pos{n}%
\glosses{Saturday}%
%------------------------------
%------------------------------
\newentry
\headword{sabur}%
\pos{v}%
\glosses{wear; dress}%
%------------------------------
\newentry
\headword{sabur}%
\pos{n}%
\glosses{soap}%
%------------------------------
%------------------------------
\newentry
\headword{sadawak}%
\pos{n}%
\glosses{machete}%
%------------------------------
\newentry
\headword{saerak}%
\pos{v}%
\glosses{negative existential; empty}%
%------------------------------
\newentry
\headword{=saet}%
\pos{adv}%
\glosses{exclusively}%
%------------------------------
\newentry
\headword{saidak}%
\pos{adv}%
\glosses{true}%
%------------------------------
\newentry
\headword{saier}%
\pos{n}%
\glosses{taboo; bad luck; offering}%
%------------------------------
\newentry
\headword{saimbumbu}%
\pos{n}%
\glosses{dragonfish; filefish}%
%------------------------------
\newentry
\headword{sair}%
\pos{n}%
\glosses{fish place}%
%------------------------------
\newentry
\headword{sair}%
\pos{v}%
\glosses{shoot with gun}%
%------------------------------
\newentry
\headword{sair}%
\pos{v}%
\glosses{bake}%
%------------------------------
\newentry
\headword{sair}%
\pos{n}%
\glosses{corner}%
%------------------------------
\newentry
\headword{sairarar}%
\pos{n}%
\glosses{lobster}%
%------------------------------
\newentry
\headword{sairarar ladok}%
\pos{n}%
\glosses{harlequin shrimp}%
%------------------------------
%------------------------------
%------------------------------
\newentry
\headword{sakarip}%
\pos{n}%
\glosses{dibble stick}%
%------------------------------
\newentry
\headword{sal}%
\pos{n}%
\glosses{roof wood}%
%------------------------------
\newentry
\headword{sala}%
\pos{vi}%
\glosses{wrong}%
%------------------------------
\newentry
\headword{salaboung}%
\pos{v}%
\glosses{broken}%
%------------------------------
%------------------------------
\newentry
\headword{salak}%
\pos{vi}%
\glosses{dented}%
%------------------------------
\newentry
\headword{salak}%
\pos{qnt}%
\glosses{ten thousand}%
%------------------------------
\newentry
\headword{salak}%
\pos{n}%
\glosses{dent}%
%------------------------------
%------------------------------
\newentry
\headword{salamat}%
\pos{n}%
\glosses{good wish}%
%------------------------------
%------------------------------
\newentry
\headword{salawat}%
\pos{v}%
\glosses{k.o. prayer}%
%------------------------------
\newentry
\headword{salawei}%
\pos{n}%
\glosses{cone shell}%
%------------------------------
\newentry
\headword{salir}%
\pos{v}%
\glosses{change}%
%------------------------------
\newentry
\headword{salout}%
\pos{n}%
\glosses{flycatcher}%
%------------------------------
\newentry
\headword{sama}%
\pos{v}%
\glosses{same}%
%------------------------------
\newentry
\headword{samameng}%
\pos{n}%
\glosses{civet cat}%
%------------------------------
\newentry
\headword{samar}%
\pos{n}%
\glosses{north-west}%
%------------------------------
%------------------------------
\newentry
\headword{samor}%
\pos{n}%
\glosses{bead}%
%------------------------------
\newentry
\headword{sampai}%
\pos{cnj}%
\glosses{until}%
%------------------------------
%------------------------------
\newentry
\headword{sampi}%
\pos{v}%
\glosses{arrive}%
%------------------------------
\newentry
\headword{sampi}%
\pos{adv}%
\glosses{until}%
%------------------------------
\newentry
\headword{samsik}%
\pos{vi}%
\glosses{thin}%
%------------------------------
\newentry
\headword{Samuret}%
\pos{n}%
\glosses{Mbaham people}%
%------------------------------
\newentry
\headword{sanam}%
\pos{n}%
\glosses{scabies; smallpox}%
%------------------------------
\newentry
\headword{sanamsanam}%
\pos{vi}%
\glosses{hairy}%
%------------------------------
%------------------------------
\newentry
\headword{sandal}%
\pos{n}%
\glosses{slippers}%
%------------------------------
\newentry
\headword{sanggan}%
\pos{n}%
\glosses{lid}%
%------------------------------
\newentry
\headword{sanggan}%
\pos{n}%
\glosses{beetle; grub}%
%------------------------------
\newentry
\headword{sangganggam}%
\pos{v}%
\glosses{spread}%
%------------------------------
\newentry
\headword{sangganun}%
\pos{n}%
\glosses{lid}%
%------------------------------
\newentry
\headword{sanggara}%
\pos{v}%
\glosses{search}%
%------------------------------
%------------------------------
%------------------------------
%------------------------------
%------------------------------
\newentry
\headword{sanggat}%
\pos{n}%
\glosses{outrigger}%
%------------------------------
\newentry
\headword{sanggeran}%
\pos{n}%
\glosses{sago}%
%------------------------------
\newentry
\headword{sanggie}%
\pos{v}%
\glosses{close}%
%------------------------------
\newentry
\headword{sanggien}%
\pos{n}%
\glosses{bird of paradise}%
%------------------------------
%------------------------------
\newentry
\headword{sanggotma}%
\pos{vt}%
\glosses{break off a branch; pick fruits}%
%------------------------------
\newentry
\headword{sanggoup}%
\pos{n}%
\glosses{branch}%
%------------------------------
\newentry
\headword{sanggoyie}%
\pos{vi}%
\glosses{broken branch}%
%------------------------------
\newentry
\headword{sanong}%
\pos{n}%
\glosses{sago palm leaves; palm roof}%
%------------------------------
\newentry
\headword{sansa}%
\pos{vi}%
\glosses{dry}%
%------------------------------
\newentry
\headword{sansan}%
\pos{v}%
\glosses{stop}%
%------------------------------
\newentry
\headword{sansan}%
\pos{n}%
\glosses{packed food}%
%------------------------------
\newentry
\headword{sanual}%
\pos{n}%
\glosses{humpback snapper}%
%------------------------------
\newentry
\headword{saor}%
\pos{v}%
\glosses{anchor}%
%------------------------------
\newentry
\headword{saor}%
\pos{n}%
\glosses{anchor}%
%------------------------------
\newentry
\headword{saouk}%
\pos{v}%
\glosses{emerge}%
%------------------------------
\newentry
\headword{sap}%
\pos{v}%
\glosses{paddle}%
%------------------------------
\newentry
\headword{sap}%
\pos{n}%
\glosses{stick}%
%------------------------------
\newentry
\headword{sapi}%
\pos{n}%
\glosses{cow}%
%------------------------------
%------------------------------
\newentry
\headword{sar kararok}%
\pos{n}%
\glosses{grouper}%
%------------------------------
\newentry
\headword{sara}%
\pos{v}%
\glosses{ascend; climb}%
%------------------------------
\newentry
\headword{sarakan}%
\pos{v}%
\glosses{pass on}%
%------------------------------
\newentry
\headword{sarakmang}%
\pos{v}%
\glosses{soft sound}%
%------------------------------
\newentry
\headword{saramburung}%
\pos{n}%
\glosses{nightjar, swift, martin, swallow}%
%------------------------------
\newentry
\headword{saramin}%
\pos{n}%
\glosses{glasses}%
%------------------------------
\newentry
\headword{saranggeit}%
\pos{n}%
\glosses{sea cucumber}%
%------------------------------
\newentry
\headword{saranggeit kuskapkap}%
\pos{n}%
\glosses{sea cucumber}%
%------------------------------
\newentry
\headword{saranggeit taraun}%
\pos{n}%
\glosses{sea cucumber}%
%------------------------------
\newentry
\headword{saraun}%
\pos{n}%
\glosses{hat}%
%------------------------------
\newentry
\headword{sarbal}%
\pos{n}%
\glosses{grouper}%
%------------------------------
\newentry
\headword{sare}%
\pos{v}%
\glosses{strand}%
%------------------------------
%------------------------------
%------------------------------
\newentry
\headword{sarem}%
\pos{n}%
\glosses{ginger}%
%------------------------------
\newentry
\headword{saria}%
\pos{n}%
\glosses{woodswallow}%
%------------------------------
\newentry
\headword{sarie}%
\pos{v}%
\glosses{chase; follow; hunt}%
%------------------------------
\newentry
\headword{sarieng}%
\pos{n}%
\glosses{hill; cliff}%
%------------------------------
\newentry
\headword{sarik}%
\pos{n}%
\glosses{cockspur coral tree}%
%------------------------------
\newentry
\headword{Sarik}%
\pos{n}%
\glosses{Sarik}%
%------------------------------
\newentry
\headword{sarim}%
\pos{n}%
\glosses{guava}%
%------------------------------
\newentry
\headword{sarit}%
\pos{n}%
\glosses{shoal}%
%------------------------------
%------------------------------
\newentry
\headword{sarouk}%
\pos{vi}%
\glosses{not good}%
%------------------------------
\newentry
\headword{saroum}%
\pos{v}%
\glosses{shoot}%
%------------------------------
\newentry
\headword{saroum}%
\pos{n}%
\glosses{shoot}%
%------------------------------
\newentry
\headword{sarsar}%
\pos{n}%
\glosses{k.o. sea cucumber}%
%------------------------------
\newentry
\headword{sarua}%
\pos{v}%
\glosses{shave; scrape}%
%------------------------------
\newentry
\headword{Saruar}%
\pos{n}%
\glosses{Saruar}%
%------------------------------
\newentry
\headword{sarun}%
\pos{n}%
\glosses{rice sieve}%
%------------------------------
\newentry
\headword{sarusarut}%
\pos{vi}%
\glosses{torn}%
%------------------------------
\newentry
\headword{sasarem}%
\pos{n}%
\glosses{wild ginger}%
%------------------------------
\newentry
\headword{sasat}%
\pos{adv}%
\glosses{go quickly}%
%------------------------------
\newentry
\headword{sasep}%
\pos{n}%
\glosses{squirrelfish, soldierfish, cardinalfish}%
%------------------------------
\newentry
\headword{saser}%
\pos{n}%
\glosses{outrigger}%
%------------------------------
\newentry
\headword{sasirip}%
\pos{n}%
\glosses{k.o. bamboo}%
%------------------------------
\newentry
\headword{sasul}%
\pos{v}%
\glosses{spoon}%
%------------------------------
\newentry
\headword{sasul}%
\pos{n}%
\glosses{spoon}%
%------------------------------
%------------------------------
\newentry
\headword{saun}%
\pos{n}%
\glosses{night}%
%------------------------------
\newentry
\headword{saun lat}%
\pos{adv}%
\glosses{late at night; in the middle of the night}%
%------------------------------
\newentry
\headword{saur}%
\pos{n}%
\glosses{morning prayer}%
%------------------------------
\newentry
\headword{sausaun}%
\pos{n}%
\glosses{darkness}%
%------------------------------
\newentry
\headword{sausaun}%
\pos{vi}%
\glosses{dark}%
%------------------------------
\newentry
\headword{sawalawala}%
\pos{n}%
\glosses{k.o. tree}%
%------------------------------
\newentry
\headword{sawaluo}%
\pos{v}%
\glosses{feel}%
%------------------------------
\newentry
\headword{sawarer}%
\pos{n}%
\glosses{tortoise}%
%------------------------------
%------------------------------
\newentry
\headword{Sawarersalot}%
\pos{n}%
\glosses{Sawarersalot}%
%------------------------------
\newentry
\headword{sawaun}%
\pos{n}%
\glosses{old}%
%------------------------------
\newentry
\headword{sawawien}%
\pos{n}%
\glosses{k.o. string}%
%------------------------------
\newentry
\headword{=sawe}%
\pos{adv}%
\glosses{too}%
%------------------------------
%------------------------------
%------------------------------
\newentry
\headword{sayang}%
\pos{n}%
\glosses{nutmeg}%
%------------------------------
\newentry
\headword{sayang bungaun}%
\pos{n}%
\glosses{mace}%
%------------------------------
\newentry
\headword{sayang naun}%
\pos{n}%
\glosses{nutmeg fruit}%
%------------------------------
\newentry
\headword{sayang tangun}%
\pos{n}%
\glosses{nutmeg}%
%------------------------------
\newentry
\headword{sayang teun}%
\pos{n}%
\glosses{nutmeg fruit}%
%------------------------------
\newentry
\headword{sayangar}%
\pos{n}%
\glosses{nutmeg garden}%
%------------------------------
\newentry
\headword{sayerun}%
\pos{n}%
\glosses{ritual}%
%------------------------------
%------------------------------
\newentry
\headword{se}%
\pos{part}%
\glosses{iam}%
%------------------------------
\newentry
\headword{se}%
\pos{n}%
\glosses{cuscus}%
%------------------------------
\newentry
\headword{se koyet}%
\pos{phrs}%
\glosses{finished}%
%------------------------------
%------------------------------
\newentry
\headword{sebua}%
\pos{n}%
\glosses{lizard}%
%------------------------------
\newentry
\headword{sebuaror}%
\pos{n}%
\glosses{medicinal plant}%
%------------------------------
%------------------------------
%------------------------------
\newentry
\headword{sedawak}%
\pos{n}%
\glosses{machete}%
%------------------------------
%------------------------------
%------------------------------
\newentry
\headword{sehingga}%
\pos{cnj}%
\glosses{so that; until}%
%------------------------------
\newentry
\headword{sei}%
\pos{v}%
\glosses{lean to side}%
%------------------------------
\newentry
\headword{seik}%
\pos{n}%
\glosses{k.o. fish}%
%------------------------------
\newentry
\headword{Seiman}%
\pos{n}%
\glosses{Seiman}%
%------------------------------
%------------------------------
\newentry
\headword{Sek}%
\pos{n}%
\glosses{Sek}%
%------------------------------
\newentry
\headword{sek}%
\pos{v}%
\glosses{fish at sea}%
%------------------------------
\newentry
\headword{sek}%
\pos{n}%
\glosses{k. o. fish}%
%------------------------------
\newentry
\headword{Sekar}%
\pos{n}%
\glosses{Sekar people}%
%------------------------------
%------------------------------
%------------------------------
%------------------------------
%------------------------------
\newentry
\headword{sekola}%
\pos{v}%
\glosses{go.to.school}%
%------------------------------
\newentry
\headword{sekola}%
\pos{n}%
\glosses{school}%
%------------------------------
\newentry
\headword{sektabai}%
\pos{n}%
\glosses{tobacco type}%
%------------------------------
\newentry
\headword{Selagur Wadan}%
\pos{n}%
\glosses{Selagur Wadan}%
%------------------------------
%------------------------------
\newentry
\headword{selasa}%
\pos{n}%
\glosses{Tuesday}%
%------------------------------
%------------------------------
\newentry
\headword{selet}%
\pos{n}%
\glosses{piece}%
%------------------------------
\newentry
\headword{seletkon}%
\pos{n}%
\glosses{piece}%
%------------------------------
\newentry
\headword{seletma}%
\pos{v}%
\glosses{cut off}%
%------------------------------
\newentry
\headword{selinku}%
\pos{v}%
\glosses{cheat}%
%------------------------------
\newentry
\headword{sem}%
\pos{v}%
\glosses{afraid}%
%------------------------------
\newentry
\headword{sembamsembam}%
\pos{n}%
\glosses{damselfish}%
%------------------------------
%------------------------------
\newentry
\headword{semen}%
\pos{n}%
\glosses{concrete}%
%------------------------------
\newentry
\headword{Semena}%
\pos{n}%
\glosses{Semena}%
%------------------------------
%------------------------------
\newentry
\headword{semerlak}%
\pos{n}%
\glosses{tree}%
%------------------------------
\newentry
\headword{sempang}%
\pos{v}%
\glosses{kick}%
%------------------------------
%------------------------------
\newentry
\headword{semsuk}%
\pos{n}%
\glosses{caught.with.fear}%
%------------------------------
%------------------------------
\newentry
\headword{-sen}%
\pos{dem}%
\glosses{demonstrative inflection expressing degree}%
%------------------------------
%------------------------------
%------------------------------
\newentry
\headword{senen}%
\pos{n}%
\glosses{Monday}%
%------------------------------
\newentry
\headword{seng}%
\pos{n}%
\glosses{roof}%
%------------------------------
\newentry
\headword{sengseng}%
\pos{n}%
\glosses{true conch}%
%------------------------------
%------------------------------
\newentry
\headword{sensor}%
\pos{n}%
\glosses{chainsaw}%
%------------------------------
%------------------------------
\newentry
\headword{sensur caun}%
\pos{n}%
\glosses{small chainsaw}%
%------------------------------
%------------------------------
%------------------------------
\newentry
\headword{sepatu}%
\pos{n}%
\glosses{shoe}%
%------------------------------
\newentry
\headword{sepe}%
\pos{n}%
\glosses{hat}%
%------------------------------
\newentry
\headword{sepeda}%
\pos{n}%
\glosses{bike}%
%------------------------------
%------------------------------
%------------------------------
%------------------------------
\newentry
\headword{ser}%
\sensenr{}%
\pos{n}%
\glosses{loft, attic}%
\sensenr{}%
\pos{n}%
\glosses{hook}%
%------------------------------
%------------------------------
\newentry
\headword{sere}%
\pos{n}%
\glosses{itchy.fish; itchy.plant}%
%------------------------------
\newentry
\headword{sere kokokteng}%
\pos{n}%
\glosses{sea fern}%
%------------------------------
\newentry
\headword{sere sorun}%
\pos{n}%
\glosses{hawkfish}%
%------------------------------
\newentry
\headword{sere taraun}%
\pos{n}%
\glosses{anemone fish}%
%------------------------------
\newentry
\headword{Serem}%
\pos{n}%
\glosses{Serem}%
%------------------------------
%------------------------------
%------------------------------
\newentry
\headword{serun}%
\pos{n}%
\glosses{rays}%
%------------------------------
\newentry
\headword{seser}%
\pos{n}%
\glosses{bridled monocle bream}%
%------------------------------
\newentry
\headword{seser}%
\pos{v}%
\glosses{peel with knife}%
%------------------------------
\newentry
\headword{seser serein}%
\pos{n}%
\glosses{sea.itch}%
%------------------------------
\newentry
\headword{set}%
\pos{n}%
\glosses{bait}%
%------------------------------
%------------------------------
%------------------------------
%------------------------------
\newentry
\headword{seur}%
\pos{v}%
\glosses{bounce off}%
%------------------------------
%------------------------------
\newentry
\headword{sewa}%
\pos{v}%
\glosses{pull}%
%------------------------------
\newentry
\headword{Sewa}%
\pos{n}%
\glosses{beach name}%
%------------------------------
\newentry
\headword{siabor}%
\pos{n}%
\glosses{signal goby}%
%------------------------------
\newentry
\headword{siada}%
\pos{n}%
\glosses{k.o. deep-water fish}%
%------------------------------
\newentry
\headword{sialar}%
\pos{n}%
\glosses{fish}%
%------------------------------
\newentry
\headword{siamar}%
\pos{vi}%
\glosses{not good at all}%
%------------------------------
%------------------------------
\newentry
\headword{sian}%
\pos{n}%
\glosses{widow(er)}%
%------------------------------
\newentry
\headword{siap}%
\pos{v}%
\glosses{ready}%
%------------------------------
%------------------------------
%------------------------------
\newentry
\headword{sie}%
\sensenr{}%
\pos{v}%
\glosses{sting}%
\sensenr{}%
\pos{v}%
\glosses{sharpen}%
%------------------------------
\newentry
\headword{*siep}%
\pos{n}%
\glosses{edge}%
%------------------------------
\newentry
\headword{siepsieun}%
\pos{n}%
\glosses{very edge}%
%------------------------------
\newentry
\headword{sietan}%
\pos{n}%
\glosses{ghost}%
%------------------------------
\newentry
\headword{sieun}%
\pos{n}%
\glosses{edge}%
%------------------------------
\newentry
\headword{sik}%
\pos{v}%
\glosses{sneeze}%
%------------------------------
\newentry
\headword{sika polipoli}%
\pos{n}%
\glosses{wagtail}%
%------------------------------
\newentry
\headword{sikan}%
\pos{n}%
\glosses{cat}%
%------------------------------
\newentry
\headword{sikasika}%
\pos{n}%
\glosses{leopard sea cucumber; teripang}%
%------------------------------
\newentry
\headword{sikekan}%
\pos{n}%
\glosses{shore birds with long feet}%
%------------------------------
%------------------------------
\newentry
\headword{siktak}%
\pos{v}%
\glosses{slow}%
%------------------------------
\newentry
\headword{siktak}%
\pos{cnj}%
\glosses{then}%
%------------------------------
\newentry
\headword{siktaktak}%
\pos{v}%
\glosses{slow}%
%------------------------------
\newentry
\headword{sikuki}%
\pos{n}%
\glosses{dove-like birds}%
%------------------------------
\newentry
\headword{sil}%
\pos{n}%
\glosses{big shell}%
%------------------------------
\newentry
\headword{Silak}%
\pos{n}%
\glosses{Silak}%
%------------------------------
%------------------------------
\newentry
\headword{sileng}%
\pos{n}%
\glosses{sea snake}%
%------------------------------
\newentry
\headword{*silep}%
\pos{n}%
\glosses{back}%
%------------------------------
\newentry
\headword{silepko}%
\pos{v}%
\glosses{behind}%
%------------------------------
\newentry
\headword{sililar}%
\pos{n}%
\glosses{pandanus}%
%------------------------------
\newentry
\headword{sin}%
\pos{n}%
\glosses{needle}%
%------------------------------
\newentry
\headword{sinara}%
\pos{n}%
\glosses{offering}%
%------------------------------
\newentry
\headword{singasingat}%
\pos{n}%
\glosses{ant}%
%------------------------------
\newentry
\headword{singgitkit}%
\pos{n}%
\glosses{small bird}%
%------------------------------
\newentry
\headword{singgoli}%
\pos{n}%
\glosses{sago pancake}%
%------------------------------
%------------------------------
%------------------------------
%------------------------------
\newentry
\headword{siput babi}%
\pos{n}%
\glosses{snail}%
%------------------------------
\newentry
\headword{sir}%
\pos{vi}%
\glosses{clear}%
%------------------------------
\newentry
\headword{sira}%
\pos{v}%
\glosses{salt}%
%------------------------------
\newentry
\headword{sira}%
\pos{n}%
\glosses{salt}%
%------------------------------
\newentry
\headword{siram}%
\pos{n}%
\glosses{sailfish}%
%------------------------------
%------------------------------
%------------------------------
%------------------------------
\newentry
\headword{sirarai}%
\pos{n}%
\glosses{twig broom}%
%------------------------------
\newentry
\headword{siriar}%
\pos{n}%
\glosses{oven}%
%------------------------------
\newentry
\headword{sirie}%
\pos{v}%
\glosses{order}%
%------------------------------
\newentry
\headword{sirisiri}%
\pos{n}%
\glosses{curtain}%
%------------------------------
\newentry
\headword{sisiapong}%
\pos{n}%
\glosses{lionfish}%
%------------------------------
\newentry
\headword{sisir}%
\pos{n}%
\glosses{comb}%
%------------------------------
\newentry
\headword{sisir}%
\pos{v}%
\glosses{comb}%
%------------------------------
\newentry
\headword{sitai}%
\pos{cnj}%
\glosses{later}%
%------------------------------
%------------------------------
%------------------------------
\newentry
\headword{siwani}%
\pos{n}%
\glosses{rat}%
%------------------------------
\newentry
\headword{so}%
\pos{n}%
\glosses{wood without bark}%
%------------------------------
\newentry
\headword{so}%
\pos{v}%
\glosses{peel wood}%
%------------------------------
%------------------------------
\newentry
\headword{sobas}%
\pos{n}%
\glosses{dawn}%
%------------------------------
%------------------------------
\newentry
\headword{sok}%
\pos{v}%
\glosses{tangled}%
%------------------------------
\newentry
\headword{soki}%
\pos{v}%
\glosses{run smooth}%
%------------------------------
\newentry
\headword{soksok}%
\pos{v}%
\glosses{hiccups}%
%------------------------------
\newentry
\headword{sol karek}%
\pos{n}%
\glosses{rattan}%
%------------------------------
%------------------------------
\newentry
\headword{solim}%
\pos{n}%
\glosses{k.o. small fish}%
%------------------------------
\newentry
\headword{som}%
\pos{n}%
\glosses{person}%
%------------------------------
\newentry
\headword{some}%
\pos{int}%
\glosses{encouragement}%
%------------------------------
\newentry
\headword{somganien}%
\pos{n}%
\glosses{k.o. plant}%
%------------------------------
\newentry
\headword{somin}%
\pos{vi}%
\glosses{dead}%
%------------------------------
\newentry
\headword{somkabas}%
\pos{n}%
\glosses{stranger}%
%------------------------------
\newentry
\headword{somsom}%
\pos{n}%
\glosses{k.o. tree}%
%------------------------------
%------------------------------
%------------------------------
%------------------------------
\newentry
\headword{sontum}%
\pos{n}%
\glosses{person}%
%------------------------------
\newentry
\headword{sontum warten}%
\pos{n}%
\glosses{witch, sorcerer}%
%------------------------------
\newentry
\headword{sontumahap}%
\pos{n}%
\glosses{all people}%
%------------------------------
\newentry
\headword{sontumkabas}%
\pos{n}%
\glosses{stranger}%
%------------------------------
%------------------------------
\newentry
\headword{sontur}%
\pos{n}%
\glosses{example}%
%------------------------------
%------------------------------
\newentry
\headword{sopsop}%
\pos{n}%
\glosses{hair pin}%
%------------------------------
\newentry
\headword{sor}%
\pos{n}%
\glosses{fish}%
%------------------------------
\newentry
\headword{sor kangun}%
\pos{n}%
\glosses{fishbone}%
%------------------------------
%------------------------------
\newentry
\headword{sor kinggirkinggir}%
\pos{n}%
\glosses{batfish}%
%------------------------------
\newentry
\headword{sor pespes}%
\pos{n}%
\glosses{fish leftovers}%
%------------------------------
\newentry
\headword{sor sira}%
\pos{n}%
\glosses{salty dried fish}%
%------------------------------
\newentry
\headword{sorbir}%
\pos{n}%
\glosses{fish}%
%------------------------------
%------------------------------
\newentry
\headword{Sorung}%
\pos{n}%
\glosses{Sorong}%
%------------------------------
%------------------------------
\newentry
\headword{soso}%
\pos{v}%
\glosses{stretch out}%
%------------------------------
\newentry
\headword{sou}%
\pos{v}%
\glosses{slide}%
%------------------------------
\newentry
\headword{souk}%
\pos{n}%
\glosses{rat}%
%------------------------------
\newentry
\headword{soul}%
\pos{vi}%
\glosses{loose}%
%------------------------------
%------------------------------
\newentry
\headword{sowil}%
\pos{n}%
\glosses{sideburns}%
%------------------------------
\newentry
\headword{Sowir}%
\pos{n}%
\glosses{Sowir}%
%------------------------------
%------------------------------
\newentry
\headword{suagi}%
\pos{n}%
\glosses{tuna}%
%------------------------------
\newentry
\headword{Suagibaba}%
\pos{n}%
\glosses{Suagibaba}%
%------------------------------
\newentry
\headword{suamin}%
\pos{n}%
\glosses{a snack made of ubi kayu}%
%------------------------------
\newentry
\headword{suan}%
\pos{n}%
\glosses{grater}%
%------------------------------
\newentry
\headword{suar}%
\pos{n}%
\glosses{ironwood}%
%------------------------------
\newentry
\headword{suara}%
\pos{n}%
\glosses{voice}%
%------------------------------
\newentry
\headword{suarkang}%
\pos{n}%
\glosses{hole}%
%------------------------------
\newentry
\headword{suban}%
\pos{v}%
\glosses{fish}%
%------------------------------
\newentry
\headword{subuman}%
\pos{n}%
\glosses{worm}%
%------------------------------
%------------------------------
%------------------------------
%------------------------------
\newentry
\headword{sudaka}%
\pos{n}%
\glosses{money in hand}%
%------------------------------
%------------------------------
\newentry
\headword{suelet}%
\pos{n}%
\glosses{fish net}%
%------------------------------
\newentry
\headword{suensik}%
\pos{vi}%
\glosses{light}%
%------------------------------
\newentry
\headword{Sui}%
\pos{n}%
\glosses{Sui}%
%------------------------------
\newentry
\headword{suk}%
\pos{n}%
\glosses{shell}%
%------------------------------
\newentry
\headword{suka}%
\pos{v}%
\glosses{like; want}%
%------------------------------
\newentry
\headword{sukaun ge}%
\pos{phrs}%
\glosses{does not want}%
%------------------------------
%------------------------------
\newentry
\headword{sumsik}%
\pos{vi}%
\glosses{light}%
%------------------------------
\newentry
\headword{sun}%
\pos{v}%
\glosses{tie a basket}%
%------------------------------
\newentry
\headword{sun}%
\pos{n}%
\glosses{basket rope}%
%------------------------------
\newentry
\headword{sunak}%
\pos{n}%
\glosses{medicinal plant}%
%------------------------------
\newentry
\headword{sungsung}%
\pos{n}%
\glosses{pants}%
%------------------------------
\newentry
\headword{suo}%
\pos{v}%
\glosses{cut a coconut; break}%
%------------------------------
\newentry
\headword{Suo}%
\pos{n}%
\glosses{Suo}%
%------------------------------
\newentry
\headword{suoktal}%
\pos{n}%
\glosses{ikan cikcak}%
%------------------------------
\newentry
\headword{suol}%
\pos{n}%
\glosses{back}%
%------------------------------
\newentry
\headword{suolkang}%
\pos{n}%
\glosses{backbone, spine}%
%------------------------------
\newentry
\headword{suolkasir}%
\pos{n}%
\glosses{spine}%
%------------------------------
\newentry
\headword{suolkerun}%
\pos{n}%
\glosses{smooth side leaf}%
%------------------------------
\newentry
\headword{suopkaling}%
\pos{n}%
\glosses{eel}%
%------------------------------
\newentry
\headword{suor}%
\pos{n}%
\glosses{bamboo comb}%
%------------------------------
\newentry
\headword{suor}%
\pos{n}%
\glosses{horns}%
%------------------------------
\newentry
\headword{suor}%
\pos{v}%
\glosses{prick on horn}%
%------------------------------
\newentry
\headword{suosuo}%
\pos{v}%
\glosses{break}%
%------------------------------
\newentry
\headword{supaya}%
\pos{cnj}%
\glosses{so that}%
%------------------------------
%------------------------------
%------------------------------
\newentry
\headword{susa}%
\pos{v}%
\glosses{difficult}%
%------------------------------
\newentry
\headword{susia}%
\pos{vi}%
\glosses{difficult}%
%------------------------------
\newentry
\headword{susumandu}%
\pos{n}%
\glosses{lizardfish}%
%------------------------------
%------------------------------
\newentry
\headword{susur}%
\pos{v}%
\glosses{fire burning}%
%------------------------------
\newentry
\headword{susurofa}%
\pos{n}%
\glosses{sea cucumber}%
%------------------------------
%------------------------------
\newentry
\headword{suwarma}%
\pos{v}%
\glosses{cut diagonally}%
%------------------------------
\newentry
\headword{=ta}%
\pos{gramm}%
\glosses{nonfinal}%
%------------------------------
\newentry
\headword{taba}%
\pos{n}%
\glosses{iron; wire}%
%------------------------------
\newentry
\headword{tabai}%
\pos{n}%
\glosses{tobacco; cigarette}%
%------------------------------
\newentry
\headword{*tabak}%
\pos{clf}%
\glosses{half}%
%------------------------------
\newentry
\headword{tabak}%
\pos{n}%
\glosses{cut}%
%------------------------------
\newentry
\headword{tabaktabak}%
\pos{vi}%
\glosses{small}%
%------------------------------
\newentry
\headword{tabalaki}%
\pos{n}%
\glosses{tamarind}%
%------------------------------
\newentry
\headword{tabalaki atan}%
\pos{n}%
\glosses{plant}%
%------------------------------
\newentry
\headword{tabalam}%
\pos{n}%
\glosses{snapper}%
%------------------------------
\newentry
\headword{tabaon}%
\pos{vi}%
\glosses{half}%
%------------------------------
\newentry
\headword{tabarak}%
\pos{v}%
\glosses{fall; crash}%
%------------------------------
\newentry
\headword{tabaruop}%
\pos{n}%
\glosses{lead}%
%------------------------------
\newentry
\headword{taberak}%
\pos{n}%
\glosses{jackfruit}%
%------------------------------
\newentry
\headword{tabili}%
\pos{n}%
\glosses{snail}%
%------------------------------
\newentry
\headword{tabom}%
\pos{n}%
\glosses{turtle}%
%------------------------------
%------------------------------
\newentry
\headword{tabul}%
\pos{n}%
\glosses{bamboo}%
%------------------------------
\newentry
\headword{tabuon}%
\pos{n}%
\glosses{small clam; sea snail}%
%------------------------------
\newentry
\headword{tabuonsal}%
\pos{n}%
\glosses{nerite shell}%
%------------------------------
\newentry
\headword{tabusik}%
\pos{vi}%
\glosses{short}%
%------------------------------
\newentry
\headword{tadon}%
\pos{v}%
\glosses{cough}%
%------------------------------
\newentry
\headword{tadorcie}%
\pos{vi}%
\glosses{pulled out}%
%------------------------------
\newentry
\headword{tadorma}%
\pos{vt}%
\glosses{pull with force}%
%------------------------------
\newentry
\headword{taer}%
\pos{n}%
\glosses{tree kangaroo}%
%------------------------------
\newentry
\headword{=taero}%
\pos{gramm}%
\glosses{even if}%
%------------------------------
\newentry
\headword{=taet}%
\pos{adv}%
\glosses{more}%
%------------------------------
\newentry
\headword{=taet}%
\pos{cnj}%
\glosses{again}%
%------------------------------
%------------------------------
\newentry
\headword{tagarar}%
\pos{v}%
\glosses{spread legs}%
%------------------------------
\newentry
\headword{tagier}%
\pos{vi}%
\glosses{heavy}%
%------------------------------
\newentry
\headword{tagir}%
\pos{n}%
\glosses{mackerel}%
%------------------------------
\newentry
\headword{tagir polas}%
\pos{n}%
\glosses{plant}%
%------------------------------
\newentry
\headword{tagur}%
\pos{n}%
\glosses{east; east wind; wet season}%
%------------------------------
\newentry
\headword{tagurep}%
\pos{n}%
\glosses{east-side}%
%------------------------------
\newentry
\headword{tagurewun}%
\pos{n}%
\glosses{Torresian imperial pidgeon}%
%------------------------------
\newentry
\headword{tagurpak}%
\pos{n}%
\glosses{east.season}%
%------------------------------
%------------------------------
\newentry
\headword{tahan}%
\pos{v}%
\glosses{last; hold in place}%
%------------------------------
%------------------------------
%------------------------------
%------------------------------
\newentry
\headword{tai-}%
\pos{n}%
\glosses{side}%
%------------------------------
\newentry
\headword{taikon}%
\pos{n}%
\glosses{half; side}%
%------------------------------
\newentry
\headword{taikongkong}%
\pos{n}%
\glosses{sea cucumber}%
%------------------------------
\newentry
\headword{-tain}%
\pos{pro}%
\glosses{alone}%
%------------------------------
\newentry
\headword{tair}%
\pos{n}%
\glosses{side}%
%------------------------------
\newentry
\headword{=tak}%
\pos{adv}%
\glosses{just}%
%------------------------------
\newentry
\headword{*tak}%
\pos{n}%
\glosses{thin and flat thing}%
%------------------------------
\newentry
\headword{*tak}%
\pos{clf}%
\glosses{classifier for leaves}%
%------------------------------
\newentry
\headword{takurera}%
\pos{n}%
\glosses{bilimbi}%
%------------------------------
\newentry
\headword{tal}%
\pos{n}%
\glosses{fence}%
%------------------------------
%------------------------------
%------------------------------
\newentry
\headword{talam}%
\pos{n}%
\glosses{tray}%
%------------------------------
\newentry
\headword{talawak}%
\pos{n}%
\glosses{east}%
%------------------------------
%------------------------------
\newentry
\headword{*talep}%
\pos{n}%
\glosses{outside}%
%------------------------------
\newentry
\headword{talepko}%
\pos{v}%
\glosses{outside}%
%------------------------------
%------------------------------
\newentry
\headword{taluk}%
\pos{v}%
\glosses{come out}%
%------------------------------
\newentry
\headword{tama}%
\pos{q}%
\glosses{question root; which}%
%------------------------------
\newentry
\headword{-taman}%
\pos{clf}%
\glosses{few}%
%------------------------------
%------------------------------
\newentry
\headword{tamandi}%
\pos{q}%
\glosses{how; how are you}%
%------------------------------
%------------------------------
\newentry
\headword{tamangga}%
\pos{q}%
\glosses{where to/from}%
%------------------------------
%------------------------------
\newentry
\headword{tamatil}%
\pos{n}%
\glosses{tomato}%
%------------------------------
\newentry
\headword{tamatko}%
\pos{q}%
\glosses{where}%
%------------------------------
\newentry
\headword{tamawis}%
\pos{q}%
\glosses{where to}%
%------------------------------
\newentry
\headword{tamba}%
\pos{v}%
\glosses{add}%
%------------------------------
%------------------------------
\newentry
\headword{Tamisen}%
\pos{n}%
\glosses{Antalisa}%
%------------------------------
%------------------------------
\newentry
\headword{tamun}%
\pos{n}%
\glosses{border}%
%------------------------------
\newentry
\headword{tan}%
\pos{n}%
\glosses{arm and hand}%
%------------------------------
\newentry
\headword{tan kasir}%
\pos{n}%
\glosses{wrist; finger joints}%
%------------------------------
\newentry
\headword{tan laus}%
\pos{n}%
\glosses{handpalm}%
%------------------------------
%------------------------------
%------------------------------
%------------------------------
%------------------------------
\newentry
\headword{Tanamera}%
\pos{n}%
\glosses{Tanamera}%
%------------------------------
%------------------------------
\newentry
\headword{tanbes}%
\pos{vi}%
\glosses{right; be righthanded}%
%------------------------------
\newentry
\headword{tanbes}%
\pos{n}%
\glosses{right hand; right side}%
%------------------------------
%------------------------------
\newentry
\headword{*tang}%
\pos{clf}%
\glosses{classifier for seeds}%
%------------------------------
%------------------------------
\newentry
\headword{*tang}%
\pos{n}%
\glosses{seed}%
%------------------------------
\newentry
\headword{tanggal}%
\pos{vi}%
\glosses{good luck with fishing}%
%------------------------------
\newentry
\headword{tanggal}%
\pos{n}%
\glosses{smaller birds of prey}%
%------------------------------
\newentry
\headword{tanggalip}%
\pos{n}%
\glosses{fingernail}%
%------------------------------
%------------------------------
\newentry
\headword{tanggarara}%
\pos{n}%
\glosses{ring}%
%------------------------------
\newentry
\headword{tanggarek}%
\pos{n}%
\glosses{little finger, pinky}%
%------------------------------
%------------------------------
\newentry
\headword{tanggo}%
\pos{v}%
\glosses{hold; carry}%
%------------------------------
\newentry
\headword{tanggon}%
\pos{n}%
\glosses{boxfish}%
%------------------------------
\newentry
\headword{tanggon}%
\pos{n}%
\glosses{year}%
%------------------------------
\newentry
\headword{tanggor}%
\pos{n}%
\glosses{mangrove}%
%------------------------------
\newentry
\headword{Tanggor}%
\pos{n}%
\glosses{Tanggor}%
%------------------------------
%------------------------------
\newentry
\headword{tanggul}%
\pos{n}%
\glosses{elbow}%
%------------------------------
\newentry
\headword{tangguorcie}%
\pos{vi}%
\glosses{opened}%
%------------------------------
\newentry
\headword{tangguorma}%
\pos{vt}%
\glosses{open}%
%------------------------------
%------------------------------
\newentry
\headword{tangkap}%
\pos{v}%
\glosses{record; catch}%
%------------------------------
%------------------------------
\newentry
\headword{tangun}%
\pos{n}%
\glosses{seed}%
%------------------------------
%------------------------------
%------------------------------
\newentry
\headword{tanisa}%
\pos{n}%
\glosses{medicinal plant}%
%------------------------------
\newentry
\headword{tanparoemun}%
\pos{n}%
\glosses{thumb}%
%------------------------------
\newentry
\headword{tanparok}%
\pos{n}%
\glosses{finger}%
%------------------------------
\newentry
\headword{tanparok penden}%
\pos{n}%
\glosses{ring finger}%
%------------------------------
\newentry
\headword{tanparok raorkadok}%
\pos{n}%
\glosses{middle finger}%
%------------------------------
%------------------------------
\newentry
\headword{tansahadat}%
\pos{n}%
\glosses{index finger}%
%------------------------------
%------------------------------
%------------------------------
%------------------------------
%------------------------------
\newentry
\headword{tantayuon}%
\pos{vi}%
\glosses{left; be lefthanded}%
%------------------------------
\newentry
\headword{tantayuon}%
\pos{n}%
\glosses{left hand; left side}%
%------------------------------
\newentry
\headword{taokang}%
\pos{n}%
\glosses{coconut shell}%
%------------------------------
\newentry
\headword{taon}%
\pos{qnt}%
\glosses{one}%
%------------------------------
\newentry
\headword{taot}%
\pos{v}%
\glosses{chisel}%
%------------------------------
\newentry
\headword{taot}%
\pos{n}%
\glosses{chisel}%
%------------------------------
\newentry
\headword{taouk}%
\pos{v}%
\glosses{lie}%
%------------------------------
\newentry
\headword{tapal}%
\pos{n}%
\glosses{cloth}%
%------------------------------
\newentry
\headword{tapar}%
\pos{n}%
\glosses{kangaroo}%
%------------------------------
\newentry
\headword{tapi}%
\pos{cnj}%
\glosses{but}%
%------------------------------
\newentry
\headword{tapong}%
\pos{n}%
\glosses{wheat flour}%
%------------------------------
\newentry
\headword{tapukan}%
\pos{n}%
\glosses{demon}%
%------------------------------
\newentry
\headword{tar}%
\pos{v}%
\glosses{coil}%
%------------------------------
\newentry
\headword{=tar}%
\pos{gramm}%
\glosses{plural imperative}%
%------------------------------
\newentry
\headword{tar}%
\pos{vi}%
\glosses{coiled}%
%------------------------------
\newentry
\headword{tar}%
\pos{n}%
\glosses{part of canoe}%
%------------------------------
\newentry
\headword{tara}%
\pos{v}%
\glosses{close}%
%------------------------------
\newentry
\headword{tara}%
\pos{n}%
\glosses{coconut scraper}%
%------------------------------
\newentry
\headword{*tara}%
\pos{n}%
\glosses{grandchild; grandparent}%
%------------------------------
\newentry
\headword{tara emnem}%
\pos{n}%
\glosses{grandmother}%
%------------------------------
\newentry
\headword{tara esnem}%
\pos{n}%
\glosses{grandfather}%
%------------------------------
%------------------------------
\newentry
\headword{tarakmang}%
\pos{vi}%
\glosses{startled}%
%------------------------------
\newentry
\headword{tarakok}%
\pos{n}%
\glosses{heron}%
%------------------------------
\newentry
\headword{tarakues}%
\pos{n}%
\glosses{k.o. string}%
%------------------------------
%------------------------------
\newentry
\headword{taram}%
\pos{n}%
\glosses{frigatebird}%
%------------------------------
\newentry
\headword{taraman}%
\pos{n}%
\glosses{fathom}%
%------------------------------
%------------------------------
\newentry
\headword{tarangin}%
\pos{n}%
\glosses{south}%
%------------------------------
\newentry
\headword{taraouk}%
\pos{vi}%
\glosses{break}%
%------------------------------
\newentry
\headword{tarapa}%
\pos{n}%
\glosses{shell}%
%------------------------------
\newentry
\headword{tarapa}%
\pos{n}%
\glosses{water container}%
%------------------------------
\newentry
\headword{tararapang}%
\pos{n}%
\glosses{heel}%
%------------------------------
\newentry
\headword{tararar}%
\pos{n}%
\glosses{surgeonfish}%
%------------------------------
%------------------------------
\newentry
\headword{taraun}%
\pos{n}%
\glosses{grandparent/child}%
%------------------------------
\newentry
\headword{taraun canam}%
\pos{n}%
\glosses{grandson}%
%------------------------------
\newentry
\headword{taraun pas}%
\pos{n}%
\glosses{granddaughter}%
%------------------------------
\newentry
\headword{tarian}%
\pos{v}%
\glosses{dance}%
%------------------------------
\newentry
\headword{tarima}%
\pos{v}%
\glosses{receive}%
%------------------------------
%------------------------------
\newentry
\headword{taru}%
\pos{v}%
\glosses{say!}%
%------------------------------
\newentry
\headword{taruo}%
\pos{v}%
\glosses{say}%
%------------------------------
%------------------------------
\newentry
\headword{Tarus}%
\pos{n}%
\glosses{place name}%
%------------------------------
%------------------------------
\newentry
\headword{tas}%
\pos{n}%
\glosses{bag}%
%------------------------------
\newentry
\headword{Tat}%
\pos{n}%
\glosses{Tat}%
%------------------------------
\newentry
\headword{tata}%
\pos{n}%
\glosses{grandfather}%
%------------------------------
\newentry
\headword{tata kolak}%
\pos{n}%
\glosses{Bomberai inlander}%
%------------------------------
%------------------------------
\newentry
\headword{tatanina}%
\pos{n}%
\glosses{grandmother; respected woman}%
%------------------------------
\newentry
\headword{tataninanus}%
\pos{n}%
\glosses{greatgrandmother}%
%------------------------------
\newentry
\headword{tatanus}%
\pos{n}%
\glosses{greatgrandfather}%
%------------------------------
\newentry
\headword{tatapang}%
\pos{n}%
\glosses{wagtails}%
%------------------------------
%------------------------------
\newentry
\headword{taukanggir}%
\pos{n}%
\glosses{coconut shell}%
%------------------------------
\newentry
\headword{taukon}%
\pos{qnt}%
\glosses{some}%
%------------------------------
\newentry
\headword{taun}%
\pos{n}%
\glosses{thin and flat thing}%
%------------------------------
\newentry
\headword{=tauna}%
\pos{cnj}%
\glosses{so}%
%------------------------------
\newentry
\headword{taungtaung}%
\pos{v}%
\glosses{bent}%
%------------------------------
\newentry
\headword{*taur}%
\pos{clf}%
\glosses{classifier for heaps}%
%------------------------------
\newentry
\headword{taur}%
\pos{n}%
\glosses{medicinal plant}%
%------------------------------
\newentry
\headword{taur}%
\pos{n}%
\glosses{placeholder for names}%
%------------------------------
\newentry
\headword{*taur}%
\pos{n}%
\glosses{heap}%
%------------------------------
\newentry
\headword{taurkon}%
\pos{qnt}%
\glosses{one}%
%------------------------------
%------------------------------
%------------------------------
%------------------------------
\newentry
\headword{tawara}%
\pos{v}%
\glosses{chop}%
%------------------------------
\newentry
\headword{tawie}%
\pos{v}%
\glosses{take from a hot place}%
%------------------------------
%------------------------------
%------------------------------
\newentry
\headword{Tawotkang}%
\pos{n}%
\glosses{Tawotkang}%
%------------------------------
\newentry
\headword{tawotma}%
\pos{vt}%
\glosses{fold}%
%------------------------------
%------------------------------
\newentry
\headword{tayuon}%
\pos{vi}%
\glosses{not good}%
%------------------------------
\newentry
\headword{te}%
\pos{n}%
\glosses{pus}%
%------------------------------
\newentry
\headword{=te}%
\pos{gramm}%
\glosses{imperative}%
%------------------------------
\newentry
\headword{-te}%
\pos{qnt}%
\glosses{distributive}%
%------------------------------
\newentry
\headword{=te}%
\pos{gramm}%
\glosses{nonfinal}%
%------------------------------
\newentry
\headword{=teba}%
\pos{gramm}%
\glosses{progressive}%
%------------------------------
\newentry
\headword{tebol}%
\pos{n}%
\glosses{reef edge}%
%------------------------------
\newentry
\headword{tebolsuban}%
\pos{v}%
\glosses{fish}%
%------------------------------
\newentry
\headword{*tebon}%
\pos{qnt}%
\glosses{all}%
%------------------------------
%------------------------------
\newentry
\headword{tebonggan}%
\pos{qnt}%
\glosses{all}%
%------------------------------
%------------------------------
\newentry
\headword{teir}%
\pos{n}%
\glosses{oyster}%
%------------------------------
\newentry
\headword{teir}%
\pos{v}%
\glosses{make a stone wall}%
%------------------------------
\newentry
\headword{teiran}%
\pos{n}%
\glosses{my neighbour}%
%------------------------------
\newentry
\headword{*teit}%
\pos{n}%
\glosses{neighbour; clan, relatives}%
%------------------------------
\newentry
\headword{teitei}%
\pos{v}%
\glosses{step on}%
%------------------------------
\newentry
\headword{tektek}%
\pos{n}%
\glosses{knife}%
%------------------------------
\newentry
\headword{tel}%
\pos{n}%
\glosses{shell}%
%------------------------------
\newentry
\headword{telebor}%
\pos{v}%
\glosses{fall}%
%------------------------------
%------------------------------
\newentry
\headword{telenggues}%
\pos{n}%
\glosses{fish}%
%------------------------------
%------------------------------
\newentry
\headword{telin}%
\pos{v}%
\glosses{stop; stay}%
%------------------------------
\newentry
\headword{telpon}%
\pos{v}%
\glosses{telephone}%
%------------------------------
\newentry
\headword{teltel}%
\pos{v}%
\glosses{move; rock}%
%------------------------------
\newentry
\headword{teltel}%
\pos{n}%
\glosses{vase shell}%
%------------------------------
\newentry
\headword{teltel}%
\pos{n}%
\glosses{root vegetable}%
%------------------------------
\newentry
\headword{*tem}%
\pos{n}%
\glosses{tree stem}%
%------------------------------
\newentry
\headword{teman}%
\pos{n}%
\glosses{friend}%
%------------------------------
\newentry
\headword{temgerun}%
\pos{n}%
\glosses{mountain top}%
%------------------------------
%------------------------------
%------------------------------
%------------------------------
\newentry
\headword{temtemun}%
\pos{n}%
\glosses{big one(s)}%
%------------------------------
\newentry
\headword{temun}%
\pos{n}%
\glosses{a big one}%
%------------------------------
\newentry
\headword{temun}%
\pos{vi}%
\glosses{big}%
%------------------------------
\newentry
\headword{ten}%
\pos{vi}%
\glosses{bad}%
%------------------------------
\newentry
\headword{-ten}%
\pos{adj}%
\glosses{attributive}%
%------------------------------
\newentry
\headword{tenaun}%
\pos{n}%
\glosses{keel}%
%------------------------------
%------------------------------
\newentry
\headword{=tenden}%
\pos{cnj}%
\glosses{so}%
%------------------------------
\newentry
\headword{tenenun}%
\pos{vi}%
\glosses{that have gone bad}%
%------------------------------
\newentry
\headword{teng}%
\pos{n}%
\glosses{leaf midrib}%
%------------------------------
\newentry
\headword{teng}%
\pos{n}%
\glosses{feather}%
%------------------------------
\newentry
\headword{tenggelele}%
\pos{n}%
\glosses{tenggelele ritual}%
%------------------------------
\newentry
\headword{tenggelele}%
\pos{v}%
\glosses{do tenggelele}%
%------------------------------
\newentry
\headword{tenggeles}%
\pos{n}%
\glosses{brahmini kite}%
%------------------------------
\newentry
\headword{tenggenggen}%
\pos{v}%
\glosses{blink}%
%------------------------------
\newentry
\headword{tengguen}%
\pos{vi}%
\glosses{gathered}%
%------------------------------
\newentry
\headword{tengguen}%
\pos{v}%
\glosses{heap}%
%------------------------------
%------------------------------
\newentry
\headword{tengun}%
\pos{n}%
\glosses{feather}%
%------------------------------
%------------------------------
\newentry
\headword{teok}%
\pos{vi}%
\glosses{foggy; snow}%
%------------------------------
\newentry
\headword{teok}%
\pos{n}%
\glosses{fog; snow}%
%------------------------------
\newentry
\headword{tep}%
\pos{v}%
\glosses{fruit}%
%------------------------------
\newentry
\headword{*tep}%
\pos{n}%
\glosses{fruit}%
%------------------------------
\newentry
\headword{tep}%
\pos{n}%
\glosses{deep sea}%
%------------------------------
\newentry
\headword{*tep}%
\pos{clf}%
\glosses{classifier for fruit}%
%------------------------------
%------------------------------
\newentry
\headword{tepeles}%
\pos{n}%
\glosses{jar}%
%------------------------------
\newentry
\headword{tepkon}%
\pos{qnt}%
\glosses{one}%
%------------------------------
\newentry
\headword{tepner}%
\pos{n}%
\glosses{deep seawater}%
%------------------------------
\newentry
\headword{ter}%
\pos{n}%
\glosses{mark; scar}%
%------------------------------
\newentry
\headword{ter}%
\pos{n}%
\glosses{tea}%
%------------------------------
%------------------------------
\newentry
\headword{terar}%
\pos{n}%
\glosses{coral stones}%
%------------------------------
\newentry
\headword{terar kararak}%
\pos{n}%
\glosses{low tide}%
%------------------------------
%------------------------------
\newentry
\headword{terarkeit}%
\pos{n}%
\glosses{on coral}%
%------------------------------
%------------------------------
%------------------------------
\newentry
\headword{termus}%
\pos{n}%
\glosses{thermos}%
%------------------------------
%------------------------------
%------------------------------
\newentry
\headword{terunggo}%
\pos{n}%
\glosses{at its place}%
%------------------------------
\newentry
\headword{terus}%
\pos{cnj}%
\glosses{then}%
%------------------------------
\newentry
\headword{terus}%
\pos{vi}%
\glosses{further; go on}%
%------------------------------
%------------------------------
\newentry
\headword{tete}%
\pos{n}%
\glosses{grandfather}%
%------------------------------
%------------------------------
\newentry
\headword{teteris}%
\pos{n}%
\glosses{sieve}%
%------------------------------
\newentry
\headword{tetetas}%
\pos{n}%
\glosses{drum}%
%------------------------------
\newentry
\headword{teun}%
\pos{n}%
\glosses{fruit}%
%------------------------------
\newentry
\headword{teya}%
\pos{n}%
\glosses{man}%
%------------------------------
%------------------------------
\newentry
\headword{tibobi}%
\pos{n}%
\glosses{k.o. tree}%
%------------------------------
\newentry
\headword{tiga}%
\pos{qnt}%
\glosses{three}%
%------------------------------
\newentry
\headword{tik}%
\pos{vi}%
\glosses{old; take long}%
%------------------------------
\newentry
\headword{tikninda}%
\pos{adv}%
\glosses{before too long}%
%------------------------------
\newentry
\headword{tim}%
\pos{n}%
\glosses{emperor fish}%
%------------------------------
\newentry
\headword{*tim}%
\pos{n}%
\glosses{edge}%
%------------------------------
\newentry
\headword{timbang}%
\pos{n}%
\glosses{forehead}%
%------------------------------
%------------------------------
%------------------------------
\newentry
\headword{Timinepnep}%
\pos{n}%
\glosses{beach name}%
%------------------------------
%------------------------------
\newentry
\headword{timun}%
\pos{n}%
\glosses{edge; tip}%
%------------------------------
\newentry
\headword{timun sobangun}%
\pos{n}%
\glosses{growing tip}%
%------------------------------
\newentry
\headword{ting}%
\pos{n}%
\glosses{jungle}%
%------------------------------
%------------------------------
%------------------------------
%------------------------------
\newentry
\headword{tiri}%
\pos{v}%
\glosses{run; sail; swim; cycle}%
%------------------------------
\newentry
\headword{tiri}%
\pos{n}%
\glosses{drum}%
%------------------------------
\newentry
\headword{Titibua Karimun}%
\pos{n}%
\glosses{name of a cape}%
%------------------------------
\newentry
\headword{to}%
\pos{int}%
\glosses{right}%
%------------------------------
%------------------------------
\newentry
\headword{tobutobur}%
\pos{v}%
\glosses{pall-bearing}%
%------------------------------
%------------------------------
\newentry
\headword{tok}%
\pos{int}%
\glosses{not yet}%
%------------------------------
\newentry
\headword{tok}%
\pos{adv}%
\glosses{yet; still; first}%
%------------------------------
\newentry
\headword{tok bes}%
\pos{phrs}%
\glosses{alive}%
%------------------------------
\newentry
\headword{tok tok}%
\pos{phrs}%
\glosses{not yet}%
%------------------------------
\newentry
\headword{tokatokan}%
\pos{n}%
\glosses{cornetfish}%
%------------------------------
\newentry
\headword{tokitoki}%
\pos{n}%
\glosses{gecko}%
%------------------------------
\newentry
\headword{tokta me}%
\pos{phrs}%
\glosses{soon}%
%------------------------------
\newentry
\headword{toktok}%
\sensenr{}%
\pos{vi}%
\glosses{lost}%
\sensenr{}%
\pos{vi}%
\glosses{alive}%
%------------------------------
\newentry
\headword{tol}%
\pos{n}%
\glosses{kingfisher}%
%------------------------------
\newentry
\headword{tolas}%
\pos{v}%
\glosses{break fast}%
%------------------------------
\newentry
\headword{tolaspak}%
\pos{n}%
\glosses{Ramadan month}%
%------------------------------
\newentry
\headword{tolcie}%
\pos{vi}%
\glosses{be cut}%
%------------------------------
\newentry
\headword{tolma}%
\pos{v}%
\glosses{cut string; take a shortcut}%
%------------------------------
%------------------------------
%------------------------------
%------------------------------
\newentry
\headword{Tomage}%
\pos{n}%
\glosses{Tomage}%
%------------------------------
\newentry
\headword{toman}%
\pos{n}%
\glosses{bag}%
%------------------------------
%------------------------------
%------------------------------
%------------------------------
\newentry
\headword{tompat}%
\pos{n}%
\glosses{place}%
%------------------------------
\newentry
\headword{tong}%
\pos{n}%
\glosses{barrel}%
%------------------------------
\newentry
\headword{Tonggarai}%
\pos{n}%
\glosses{place name}%
%------------------------------
%------------------------------
\newentry
\headword{Tonggatonggar}%
\pos{n}%
\glosses{Tonggatonggar}%
%------------------------------
\newentry
\headword{toni}%
\pos{v}%
\glosses{say; want; think}%
%------------------------------
\newentry
\headword{top}%
\pos{n}%
\glosses{fusilier}%
%------------------------------
%------------------------------
\newentry
\headword{torak}%
\pos{n}%
\glosses{part of canoe}%
%------------------------------
\newentry
\headword{torak}%
\pos{n}%
\glosses{fish}%
%------------------------------
\newentry
\headword{toras}%
\pos{vi}%
\glosses{cleared}%
%------------------------------
\newentry
\headword{torim}%
\pos{n}%
\glosses{eggplant}%
%------------------------------
%------------------------------
\newentry
\headword{Torkuran}%
\pos{n}%
\glosses{Tarak}%
%------------------------------
\newentry
\headword{tororo}%
\pos{vi}%
\glosses{opened wide}%
%------------------------------
\newentry
\headword{torpes}%
\pos{n}%
\glosses{shell}%
%------------------------------
\newentry
\headword{tot}%
\pos{n}%
\glosses{sea urchin}%
%------------------------------
\newentry
\headword{toungtoung}%
\pos{vi}%
\glosses{bulky}%
%------------------------------
\newentry
\headword{towari}%
\pos{n}%
\glosses{bachelor; young}%
%------------------------------
%------------------------------
%------------------------------
%------------------------------
\newentry
\headword{tu}%
\pos{v}%
\glosses{hit; pound}%
%------------------------------
\newentry
\headword{tua}%
\pos{v}%
\glosses{live}%
%------------------------------
\newentry
\headword{tua}%
\pos{n}%
\glosses{title for elder person}%
%------------------------------
\newentry
\headword{tuangga}%
\pos{n}%
\glosses{spine fish}%
%------------------------------
\newentry
\headword{tuaringgiar}%
\pos{vi}%
\glosses{old}%
%------------------------------
\newentry
\headword{tuaruar}%
\pos{v}%
\glosses{live}%
%------------------------------
%------------------------------
\newentry
\headword{tuatkur}%
\pos{n}%
\glosses{living place}%
%------------------------------
\newentry
\headword{tubak}%
\pos{v}%
\glosses{point and touch}%
%------------------------------
\newentry
\headword{Tuburasap}%
\pos{n}%
\glosses{Tuburuasa}%
%------------------------------
%------------------------------
%------------------------------
%------------------------------
%------------------------------
%------------------------------
\newentry
\headword{tumin}%
\pos{n}%
\glosses{watermelon}%
%------------------------------
%------------------------------
\newentry
\headword{tumteng}%
\pos{n}%
\glosses{bedbug}%
%------------------------------
%------------------------------
\newentry
\headword{tumun}%
\sensenr{}%
\pos{n}%
\glosses{child}%
\sensenr{}%
\pos{n}%
\glosses{small}%
%------------------------------
\newentry
\headword{tumun canam}%
\pos{n}%
\glosses{son}%
%------------------------------
\newentry
\headword{tumun caun}%
\pos{n}%
\glosses{small child}%
%------------------------------
\newentry
\headword{tumun miskinden}%
\pos{n}%
\glosses{orphan}%
%------------------------------
\newentry
\headword{tumun pas}%
\pos{n}%
\glosses{daughter}%
%------------------------------
%------------------------------
\newentry
\headword{=tun}%
\pos{adv}%
\glosses{very}%
%------------------------------
\newentry
\headword{tun}%
\pos{v}%
\glosses{shake}%
%------------------------------
\newentry
\headword{tunggarek}%
\pos{n}%
\glosses{k.o. string}%
%------------------------------
%------------------------------
\newentry
\headword{tunggin}%
\pos{n}%
\glosses{ridge pole}%
%------------------------------
%------------------------------
%------------------------------
\newentry
\headword{tup}%
\pos{n}%
\glosses{poisonous root used to catch fish}%
%------------------------------
\newentry
\headword{tup}%
\pos{v}%
\glosses{fish}%
%------------------------------
\newentry
\headword{tur}%
\pos{vi}%
\glosses{fall}%
%------------------------------
%------------------------------
\newentry
\headword{turing}%
\pos{n}%
\glosses{hill}%
%------------------------------
%------------------------------
\newentry
\headword{Tuwak}%
\pos{n}%
\glosses{Tuwak}%
%------------------------------
\end{letter}
\begin{letter}{u}
\newentry
\headword{u}%
\pos{n}%
\glosses{aunt}%
%------------------------------
\newentry
\headword{u}%
\pos{n}%
\glosses{fish}%
%------------------------------
%------------------------------
\newentry
\headword{uda}%
\pos{n}%
\glosses{rice sieve}%
%------------------------------
\newentry
\headword{uei}%
\pos{int}%
\glosses{interjection of surprise}%
%------------------------------
\newentry
\headword{ugar}%
\pos{n}%
\glosses{crab}%
%------------------------------
%------------------------------
%------------------------------
\newentry
\headword{ukir}%
\pos{v}%
\glosses{measure}%
%------------------------------
\newentry
\headword{ukuran}%
\pos{n}%
\glosses{size}%
%------------------------------
\newentry
\headword{ul}%
\pos{n}%
\glosses{urine}%
%------------------------------
\newentry
\headword{ulan}%
\pos{n}%
\glosses{aunt}%
%------------------------------
%------------------------------
\newentry
\headword{uli}%
\pos{n}%
\glosses{rudder; helmsman}%
%------------------------------
\newentry
\headword{ulpom}%
\pos{n}%
\glosses{bladder}%
%------------------------------
\newentry
\headword{ultom}%
\pos{n}%
\glosses{goatfish}%
%------------------------------
\newentry
\headword{ulur}%
\pos{v}%
\glosses{urinate}%
%------------------------------
\newentry
\headword{umat}%
\pos{n}%
\glosses{people}%
%------------------------------
%------------------------------
\newentry
\headword{-un}%
\sensenr{}%
\pos{pro}%
\glosses{his/her/its}%
\sensenr{}%
\pos{pro}%
\glosses{our (\textsc{ex})}%
%------------------------------
\newentry
\headword{-un}%
\pos{gramm}%
\glosses{nominaliser}%
%------------------------------
\newentry
\headword{un=}%
\pos{gramm}%
\glosses{reflexive}%
%------------------------------
\newentry
\headword{un kawer}%
\pos{n}%
\glosses{body fat}%
%------------------------------
\newentry
\headword{unana}%
\pos{n}%
\glosses{earthenware vase}%
%------------------------------
\newentry
\headword{unapi}%
\pos{n}%
\glosses{sea cucumber}%
%------------------------------
%------------------------------
%------------------------------
\newentry
\headword{unganggie}%
\pos{v}%
\glosses{lift oneself}%
%------------------------------
%------------------------------
%------------------------------
\newentry
\headword{Uninsinei}%
\pos{n}%
\glosses{Teluk Buruwai}%
%------------------------------
%------------------------------
\newentry
\headword{unkoryap}%
\pos{v}%
\glosses{divide}%
%------------------------------
\newentry
\headword{unmasir}%
\pos{v}%
\glosses{give birth}%
%------------------------------
\newentry
\headword{unsor}%
\pos{n}%
\glosses{orange-spotted trevally}%
%------------------------------
\newentry
\headword{untuk}%
\pos{cnj}%
\glosses{for}%
%------------------------------
\newentry
\headword{up}%
\pos{n}%
\glosses{calm}%
%------------------------------
\newentry
\headword{up}%
\pos{n}%
\glosses{plug}%
%------------------------------
\newentry
\headword{up}%
\pos{v}%
\glosses{kindle}%
%------------------------------
\newentry
\headword{upsa}%
\pos{v}%
\glosses{falling of leaves}%
%------------------------------
\newentry
\headword{ur}%
\pos{n}%
\glosses{wind}%
%------------------------------
%------------------------------
\newentry
\headword{ur kirun}%
\pos{n}%
\glosses{cloud}%
%------------------------------
\newentry
\headword{ur temun}%
\pos{phrs}%
\glosses{rough sea}%
%------------------------------
\newentry
\headword{uran}%
\pos{n}%
\glosses{debt}%
%------------------------------
\newentry
\headword{urap}%
\pos{n}%
\glosses{street}%
%------------------------------
\newentry
\headword{uren}%
\pos{n}%
\glosses{wave}%
%------------------------------
\newentry
\headword{ureren}%
\pos{vi}%
\glosses{wavy}%
%------------------------------
%------------------------------
%------------------------------
\newentry
\headword{urukmang}%
\pos{v}%
\glosses{suddenly move; suddenly  make sound}%
%------------------------------
%------------------------------
\newentry
\headword{us}%
\pos{n}%
\glosses{penis}%
%------------------------------
\newentry
\headword{us}%
\pos{v}%
\glosses{clean pandanus}%
%------------------------------
%------------------------------
\newentry
\headword{usar}%
\pos{vt}%
\glosses{erect; build house}%
%------------------------------
\newentry
\headword{usar}%
\pos{vi}%
\glosses{stand.up}%
%------------------------------
%------------------------------
%------------------------------
\newentry
\headword{usiep}%
\pos{n}%
\glosses{shoal}%
%------------------------------
\newentry
\headword{uspulpul}%
\pos{n}%
\glosses{picasso triggerfish}%
%------------------------------
\newentry
\headword{ut}%
\pos{v}%
\glosses{mark}%
%------------------------------
\newentry
\headword{ut}%
\pos{n}%
\glosses{mark}%
%------------------------------
\newentry
\headword{utkon}%
\pos{qnt}%
\glosses{some}%
%------------------------------
\newentry
\headword{utkon}%
\pos{adv}%
\glosses{alone; apart}%
%------------------------------
%------------------------------
\newentry
\headword{Utun}%
\pos{n}%
\glosses{Buton}%
%------------------------------
\end{letter}
\begin{letter}{w}
\newentry
\headword{wa}%
\pos{dem}%
\glosses{proximal}%
%------------------------------
\newentry
\headword{wais}%
\pos{n}%
\glosses{place}%
%------------------------------
\newentry
\headword{wak}%
\pos{n}%
\glosses{millipede}%
%------------------------------
\newentry
\headword{wakpol}%
\pos{n}%
\glosses{lizard}%
%------------------------------
\newentry
\headword{waktu}%
\pos{n}%
\glosses{time}%
%------------------------------
\newentry
\headword{waktu}%
\pos{cnj}%
\glosses{when}%
%------------------------------
\newentry
\headword{walaka}%
\pos{n}%
\glosses{Goromese}%
%------------------------------
\newentry
\headword{Walaka}%
\pos{n}%
\glosses{Gorom}%
%------------------------------
\newentry
\headword{Walakamang}%
\pos{n}%
\glosses{Goromese}%
%------------------------------
\newentry
\headword{walalom}%
\pos{n}%
\glosses{shore current}%
%------------------------------
%------------------------------
%------------------------------
\newentry
\headword{walawala}%
\pos{v}%
\glosses{throw}%
%------------------------------
\newentry
\headword{walor}%
\pos{n}%
\glosses{coconut leaf}%
%------------------------------
\newentry
\headword{walorteng}%
\pos{n}%
\glosses{coconut leaf midrib}%
%------------------------------
%------------------------------
\newentry
\headword{wam}%
\pos{v}%
\glosses{roll pandanus}%
%------------------------------
\newentry
\headword{wam}%
\pos{n}%
\glosses{roll}%
%------------------------------
\newentry
\headword{Wambar}%
\pos{n}%
\glosses{Wambar}%
%------------------------------
\newentry
\headword{*wan}%
\pos{n}%
\glosses{time}%
%------------------------------
\newentry
\headword{wandi}%
\pos{dem}%
\glosses{like this}%
%------------------------------
\newentry
\headword{wandiwandi}%
\pos{n}%
\glosses{small plug}%
%------------------------------
\newentry
\headword{wane}%
\pos{dem}%
\glosses{proximal}%
%------------------------------
\newentry
\headword{wang}%
\pos{n}%
\glosses{dugong}%
%------------------------------
\newentry
\headword{wangga}%
\pos{dem}%
\glosses{proximal lative}%
%------------------------------
\newentry
\headword{Wanggaruar}%
\pos{n}%
\glosses{Wanggaruar}%
%------------------------------
\newentry
\headword{wanggon}%
\pos{adv}%
\glosses{once}%
%------------------------------
\newentry
\headword{wanggongon}%
\pos{adv}%
\glosses{seldom}%
%------------------------------
\newentry
\headword{wangguwanggus}%
\pos{n}%
\glosses{flute}%
%------------------------------
%------------------------------
%------------------------------
%------------------------------
\newentry
\headword{Wanim}%
\pos{n}%
\glosses{Wanim}%
%------------------------------
%------------------------------
\newentry
\headword{wanir}%
\pos{adv}%
\glosses{twice}%
%------------------------------
%------------------------------
%------------------------------
\newentry
\headword{Wap}%
\pos{n}%
\glosses{Sanggalabai}%
%------------------------------
\newentry
\headword{war}%
\pos{v}%
\glosses{use sorcery}%
%------------------------------
\newentry
\headword{war}%
\pos{v}%
\glosses{fish}%
%------------------------------
\newentry
\headword{war}%
\pos{n}%
\glosses{sorcery}%
%------------------------------
\newentry
\headword{war}%
\pos{n}%
\glosses{shark}%
%------------------------------
\newentry
\headword{war pasierkip}%
\pos{n}%
\glosses{whale shark}%
%------------------------------
\newentry
\headword{waranggeit}%
\pos{n}%
\glosses{k.o. coral}%
%------------------------------
\newentry
\headword{wariam}%
\pos{n}%
\glosses{k.o. fish}%
%------------------------------
\newentry
\headword{waring}%
\pos{n}%
\glosses{file clam}%
%------------------------------
\newentry
\headword{warkangkang}%
\pos{n}%
\glosses{goosebumps}%
%------------------------------
\newentry
\headword{warkasom}%
\pos{n}%
\glosses{starfish}%
%------------------------------
\newentry
\headword{warkin}%
\pos{n}%
\glosses{tide}%
%------------------------------
\newentry
\headword{warkin garos}%
\pos{phrs}%
\glosses{low tide}%
%------------------------------
\newentry
\headword{warkin kararak}%
\pos{phrs}%
\glosses{low tide}%
%------------------------------
\newentry
\headword{warkin laur}%
\pos{phrs}%
\glosses{high tide}%
%------------------------------
\newentry
\headword{warkin nasesak}%
\pos{phrs}%
\glosses{high tide}%
%------------------------------
\newentry
\headword{warkin tararup}%
\pos{phrs}%
\glosses{low tide}%
%------------------------------
%------------------------------
%------------------------------
\newentry
\headword{warpas}%
\pos{n}%
\glosses{sorceress}%
%------------------------------
%------------------------------
\newentry
\headword{warum}%
\pos{n}%
\glosses{trash}%
%------------------------------
\newentry
\headword{waruo}%
\pos{v}%
\glosses{wash; bathe}%
%------------------------------
\newentry
\headword{warwar}%
\pos{n}%
\glosses{k.o. plant}%
%------------------------------
\newentry
\headword{wasorak}%
\pos{n}%
\glosses{ark clam}%
%------------------------------
\newentry
\headword{wat}%
\pos{n}%
\glosses{coconut}%
%------------------------------
\newentry
\headword{wat kabur}%
\pos{n}%
\glosses{young coconut}%
%------------------------------
\newentry
\headword{wat karoraun}%
\pos{n}%
\glosses{coconut}%
%------------------------------
\newentry
\headword{wat kawaren}%
\pos{v}%
\glosses{scrape coconut}%
%------------------------------
\newentry
\headword{wat kerkapkap}%
\pos{n}%
\glosses{coconut}%
%------------------------------
\newentry
\headword{wat pasor}%
\pos{n}%
\glosses{fried coconut}%
%------------------------------
\newentry
\headword{wat sarenden}%
\pos{n}%
\glosses{coconut}%
%------------------------------
\newentry
\headword{wat sasul}%
\pos{n}%
\glosses{green coconut}%
%------------------------------
\newentry
\headword{watko}%
\pos{dem}%
\glosses{proximal locative}%
%------------------------------
\newentry
\headword{watman}%
\pos{n}%
\glosses{sea cucumber}%
%------------------------------
\newentry
\headword{watwat}%
\pos{n}%
\glosses{tree}%
%------------------------------
%------------------------------
\newentry
\headword{weinun}%
\pos{adv}%
\glosses{too}%
%------------------------------
\newentry
\headword{wel}%
\pos{n}%
\glosses{top shell}%
%------------------------------
\newentry
\headword{Welangguni}%
\pos{n}%
\glosses{Arguni people}%
%------------------------------
\newentry
\headword{welawela}%
\pos{n}%
\glosses{wedding rite}%
%------------------------------
\newentry
\headword{wele}%
\pos{n}%
\glosses{vegetables}%
%------------------------------
\newentry
\headword{welenggap}%
\pos{vi}%
\glosses{blue; green}%
%------------------------------
\newentry
\headword{wenawena}%
\pos{n}%
\glosses{bee; honey}%
%------------------------------
\newentry
\headword{wenawena eun}%
\pos{n}%
\glosses{beehive}%
%------------------------------
\newentry
\headword{wenggam}%
\pos{n}%
\glosses{rust}%
%------------------------------
%------------------------------
\newentry
\headword{Werpati}%
\pos{n}%
\glosses{Werpati}%
%------------------------------
\newentry
\headword{Werwaras}%
\pos{n}%
\glosses{Werwaras}%
%------------------------------
\newentry
\headword{werwer}%
\pos{n}%
\glosses{ikan julung}%
%------------------------------
\newentry
\headword{westal}%
\pos{n}%
\glosses{hair}%
%------------------------------
\newentry
\headword{weswes}%
\pos{n}%
\glosses{shell}%
%------------------------------
\newentry
\headword{wewar}%
\pos{n}%
\glosses{axe}%
%------------------------------
\newentry
\headword{wie}%
\pos{n}%
\glosses{mango tree; mango}%
%------------------------------
\newentry
\headword{wien}%
\pos{n}%
\glosses{fishing line}%
%------------------------------
\newentry
\headword{wienar}%
\pos{n}%
\glosses{parrotfish}%
%------------------------------
\newentry
\headword{wienar saruam}%
\pos{n}%
\glosses{longnose parrotfish}%
%------------------------------
\newentry
\headword{wienar tebolkin}%
\pos{n}%
\glosses{roundhead parrotfish}%
%------------------------------
%------------------------------
\newentry
\headword{wiercie}%
\pos{vi}%
\glosses{unstuck; be open}%
%------------------------------
\newentry
\headword{wierun}%
\pos{n}%
\glosses{stalk}%
%------------------------------
\newentry
\headword{wiet}%
\pos{n}%
\glosses{stalk}%
%------------------------------
\newentry
\headword{wilak}%
\pos{n}%
\glosses{sea}%
%------------------------------
%------------------------------
\newentry
\headword{winyal}%
\pos{v}%
\glosses{fish}%
%------------------------------
\newentry
\headword{wirma}%
\pos{v}%
\glosses{open}%
%------------------------------
\newentry
\headword{wis}%
\pos{adv}%
\glosses{yesterday}%
%------------------------------
\newentry
\headword{wise}%
\pos{adv}%
\glosses{long ago}%
%------------------------------
\newentry
\headword{wiseme}%
\pos{adv}%
\glosses{a long time ago}%
%------------------------------
%------------------------------
\newentry
\headword{wol}%
\sensenr{}%
\pos{n}%
\glosses{family}%
\sensenr{}%
\pos{n}%
\glosses{tree}%
%------------------------------
\newentry
\headword{wolnelebor}%
\pos{v}%
\glosses{call names}%
%------------------------------
%------------------------------
%------------------------------
\newentry
\headword{wororoi}%
\pos{n}%
\glosses{parrot}%
%------------------------------
\newentry
\headword{wowa}%
\pos{n}%
\glosses{aunt}%
%------------------------------
\newentry
\headword{wowa caun}%
\pos{n}%
\glosses{aunt}%
%------------------------------
\newentry
\headword{wowa temun}%
\pos{n}%
\glosses{aunt}%
%------------------------------
%------------------------------
%------------------------------
%------------------------------
\newentry
\headword{wuar sisiarun}%
\pos{n}%
\glosses{grasshopper}%
%------------------------------
%------------------------------
%------------------------------
\newentry
\headword{wuong}%
\pos{v}%
\glosses{whistle}%
%------------------------------
\newentry
\headword{wuorma}%
\pos{v}%
\glosses{cut down a tree; cut}%
%------------------------------
\newentry
\headword{wuorwuor}%
\pos{v}%
\glosses{dream}%
%------------------------------
%------------------------------
%------------------------------
\end{letter}
\begin{letter}{y}
\newentry
\headword{ya}%
\pos{int}%
\glosses{yes}%
%------------------------------
\newentry
\headword{ya aula}%
\pos{int}%
\glosses{interjection of incredulity}%
%------------------------------
%------------------------------
\newentry
\headword{yaban}%
\pos{n}%
\glosses{wind}%
%------------------------------
\newentry
\headword{yakarek taraun}%
\pos{n}%
\glosses{grouper}%
%------------------------------
\newentry
\headword{yakop}%
\pos{n}%
\glosses{cockatoo}%
%------------------------------
\newentry
\headword{yakop leirun}%
\pos{n}%
\glosses{palm cockatoo}%
%------------------------------
\newentry
\headword{yakop posun}%
\pos{n}%
\glosses{placename}%
%------------------------------
\newentry
\headword{yal}%
\pos{n}%
\glosses{paddle}%
%------------------------------
\newentry
\headword{yal}%
\pos{v}%
\glosses{paddle}%
%------------------------------
\newentry
\headword{yalyal}%
\pos{v}%
\glosses{paddle}%
%------------------------------
\newentry
\headword{yam}%
\pos{v}%
\glosses{have sex}%
%------------------------------
%------------------------------
\newentry
\headword{yap}%
\pos{n}%
\glosses{black potato}%
%------------------------------
\newentry
\headword{yap}%
\pos{v}%
\glosses{divide}%
%------------------------------
\newentry
\headword{yap seran}%
\pos{n}%
\glosses{yam}%
%------------------------------
\newentry
\headword{yar}%
\pos{n}%
\glosses{stone}%
%------------------------------
\newentry
\headword{yar kangkang}%
\pos{n}%
\glosses{sharp rock}%
%------------------------------
\newentry
\headword{yar loupkaning}%
\pos{n}%
\glosses{coral}%
%------------------------------
\newentry
\headword{yaralus}%
\pos{n}%
\glosses{gravel}%
%------------------------------
\newentry
\headword{yarkanyuot}%
\pos{n}%
\glosses{bivalve}%
%------------------------------
\newentry
\headword{yarkawaram}%
\pos{n}%
\glosses{k.o. triggerfish}%
%------------------------------
\newentry
\headword{yarmunmun}%
\pos{n}%
\glosses{boulder}%
%------------------------------
\newentry
\headword{yarnener}%
\pos{n}%
\glosses{coral shore}%
%------------------------------
\newentry
\headword{yarpan}%
\pos{n}%
\glosses{rock}%
%------------------------------
\newentry
\headword{yarpos}%
\pos{n}%
\glosses{rock hole}%
%------------------------------
\newentry
\headword{Yarpos Kon}%
\pos{n}%
\glosses{Yarpos Kon}%
%------------------------------
\newentry
\headword{yartep}%
\pos{n}%
\glosses{sardine}%
%------------------------------
%------------------------------
\newentry
\headword{yasin}%
\pos{n}%
\glosses{koran verse}%
%------------------------------
\newentry
\headword{yatal}%
\pos{n}%
\glosses{stone wall}%
%------------------------------
\newentry
\headword{yawarnak}%
\pos{n}%
\glosses{plant}%
%------------------------------
\newentry
\headword{yawe}%
\pos{dem}%
\glosses{down}%
%------------------------------
\newentry
\headword{yawengga}%
\pos{dem}%
\glosses{from/to down}%
%------------------------------
\newentry
\headword{yawet}%
\pos{dem}%
\glosses{down object form}%
%------------------------------
\newentry
\headword{yawetko}%
\pos{dem}%
\glosses{down there}%
%------------------------------
\newentry
\headword{yawir}%
\pos{n}%
\glosses{lime}%
%------------------------------
\newentry
\headword{ye}%
\pos{cnj}%
\glosses{or}%
%------------------------------
%------------------------------
\newentry
\headword{yecie}%
\pos{v}%
\glosses{return}%
%------------------------------
%------------------------------
\newentry
\headword{yes}%
\pos{n}%
\glosses{stomach worm}%
%------------------------------
\newentry
\headword{yes}%
\pos{n}%
\glosses{medicine}%
%------------------------------
\newentry
\headword{yeso}%
\pos{int}%
\glosses{I don't know}%
%------------------------------
\newentry
\headword{yie}%
\pos{v}%
\glosses{swim}%
%------------------------------
\newentry
\headword{yies}%
\pos{n}%
\glosses{plant}%
%------------------------------
\newentry
\headword{yo}%
\pos{int}%
\glosses{yes}%
%------------------------------
%------------------------------
%------------------------------
%------------------------------
%------------------------------
\newentry
\headword{yopyop}%
\pos{n}%
\glosses{hibiscus}%
%------------------------------
%------------------------------
\newentry
\headword{Yorre}%
\pos{n}%
\glosses{Yorre}%
%------------------------------
%------------------------------
\newentry
\headword{yu-}%
\pos{dem}%
\glosses{demonstrative prefix}%
%------------------------------
%------------------------------
%------------------------------
%------------------------------
%------------------------------
\newentry
\headword{yume}%
\pos{dem}%
\glosses{distal}%
%------------------------------
\newentry
\headword{yumene}%
\pos{dem}%
\glosses{distal}%
%------------------------------
%------------------------------
\newentry
\headword{yuol}%
\pos{n}%
\glosses{day}%
%------------------------------
\newentry
\headword{yuol tama}%
\pos{q}%
\glosses{when}%
%------------------------------
\newentry
\headword{yuolyuol}%
\pos{vi}%
\glosses{light; shine}%
%------------------------------
\newentry
\headword{yuon}%
\pos{vt}%
\glosses{rub; clean}%
%------------------------------
\newentry
\headword{yuon}%
\pos{n}%
\glosses{rag}%
%------------------------------
\newentry
\headword{yuon}%
\pos{n}%
\glosses{sun}%
%------------------------------
\newentry
\headword{yuon ba mintolmaretkon}%
\pos{phrs}%
\glosses{may the sun pull out your liver}%
%------------------------------
\newentry
\headword{yuon daruk}%
\pos{n}%
\glosses{sunset}%
%------------------------------
\newentry
\headword{yuon monpak}%
\pos{n}%
\glosses{dry season}%
%------------------------------
\newentry
\headword{yuon nawariri}%
\pos{n}%
\glosses{noon}%
%------------------------------
\newentry
\headword{yuon sara}%
\pos{phrs}%
\glosses{sunrise}%
%------------------------------
\newentry
\headword{yuopyuop}%
\pos{n}%
\glosses{hibiscus}%
%------------------------------
\newentry
\headword{yuor}%
\pos{n}%
\glosses{grass}%
%------------------------------
\newentry
\headword{yuor}%
\pos{n}%
\glosses{day}%
%------------------------------
\newentry
\headword{yuor}%
\pos{v}%
\glosses{right}%
%------------------------------
\newentry
\headword{yuor}%
\pos{int}%
\glosses{true}%
%------------------------------
\newentry
\headword{yuorsik}%
\pos{vi}%
\glosses{straight}%
%------------------------------
\newentry
\headword{yuot}%
\pos{n}%
\glosses{snail}%
%------------------------------
\newentry
\headword{yuwa}%
\pos{dem}%
\glosses{proximal}%
%------------------------------
\newentry
\headword{yuwandi}%
\pos{adv}%
\glosses{like this}%
%------------------------------
\newentry
\headword{yuwane}%
\pos{dem}%
\glosses{proximal}%
%------------------------------
\newentry
\headword{yuwatko}%
\pos{dem}%
\glosses{proximal locative}%
%------------------------------
\newentry
\headword{yuyui}%
\pos{n}%
\glosses{sea cucumber}%
\end{letter}

\twocolumn[\section*{Wordlist English-Kalamang}]

\noindent\english{a big one}
  \kalamang{temun}  
\english{a little}
  \kalamang{bolon}  
\english{a long time ago}
  \kalamang{wiseme}  
\english{above}
  \kalamang{keitko}  
\english{achilles heel}
  \kalamang{korkies}  
\english{add}
  \kalamang{koder};  \kalamang{tamba}  
\english{adjust}
  \kalamang{manyuor}  
\english{adopted}
  \kalamang{keitkeit}  
\english{afraid}
  \kalamang{sem}  
\english{afternoon}
  \kalamang{asar};  \kalamang{ginggir};  \kalamang{go ginggir}  
\english{again}
  \kalamang{kodaet};  \kalamang{koi};  \kalamang{=taet}  
\english{aged}
  \kalamang{giriaun}  
\english{agr}
  \kalamang{m'm}  
\english{algae}
  \kalamang{lamut}  
\english{alive}
  \kalamang{tok bes};  \kalamang{toktok}  
\english{all}
  \kalamang{-gan};  \kalamang{-mahap};  \kalamang{*tebon};  \kalamang{tebonggan}  
\english{all people}
  \kalamang{sontumahap}  
\english{alone}
  \kalamang{-ahutak};  \kalamang{-tain}  
\english{alone; apart}
  \kalamang{utkon}  
\english{ambarella}
  \kalamang{pasarom}  
\english{anchor}
  \kalamang{po};  \kalamang{saor};  \kalamang{saor}  
\english{and then}
  \kalamang{eba metko}  
\english{anemone fish}
  \kalamang{sere taraun}  
\english{angel; curse}
  \kalamang{malaikat}  
\english{angelfish}
  \kalamang{kubalbal}  
\english{angry}
  \kalamang{kacok};  \kalamang{kademor};  \kalamang{koewa};  \kalamang{kouran}  
\english{ankle}
  \kalamang{korkasir}  
\english{ankle bone}
  \kalamang{korkancing}  
\english{another}
  \kalamang{kabas}  
\english{another place}
  \kalamang{kolpis}  
\english{answer}
  \kalamang{jawab};  \kalamang{nabalas}  
\english{ant}
  \kalamang{singasingat}  
\english{Antalisa}
  \kalamang{Tamisen}  
\english{antidote; resistant}
  \kalamang{anti}  
\english{anus}
  \kalamang{kahamanpos}  
\english{anything}
  \kalamang{don konkon}  
\english{anyway; whatever; poor}
  \kalamang{kasian}  
\english{appear}
  \kalamang{lauk}  
\english{approximately}
  \kalamang{-konggap}  
\english{Arguni people}
  \kalamang{Welangguni}  
\english{aril; mace}
  \kalamang{kosarun}  
\english{ark clam}
  \kalamang{wasorak}  
\english{arm and hand}
  \kalamang{tan}  
\english{armpit}
  \kalamang{keleng}  
\english{arrange}
  \kalamang{ator}  
\english{arrive}
  \kalamang{sampi}  
\english{arrow}
  \kalamang{karop}  
\english{Aru islands}
  \kalamang{Dobu}  
\english{as long as}
  \kalamang{asal}  
\english{ascend; climb}
  \kalamang{sara}  
\english{ashes}
  \kalamang{koep}  
\english{ask}
  \kalamang{gerket}  
\english{ask permission to leave}
  \kalamang{panokpanok}  
\english{ask; call}
  \kalamang{paning}  
\english{asparagus bean}
  \kalamang{kawuok kahahen}  
\english{at an angle}
  \kalamang{kaling}  
\english{at its place}
  \kalamang{terunggo}  
\english{(at) night}
  \kalamang{go saun}  
\english{aunt}
  \kalamang{ema caun};  \kalamang{ema temun};  \kalamang{u};  \kalamang{ulan};  \kalamang{wowa};  \kalamang{wowa caun};  \kalamang{wowa temun}  
\english{avocado}
  \kalamang{afukat}  
\english{axe}
  \kalamang{lem};  \kalamang{wewar}  
\english{bachelor; young}
  \kalamang{towari}  
\english{back}
  \kalamang{*ep};  \kalamang{*silep};  \kalamang{suol}  
\english{back and forth}
  \kalamang{ran mian}  
\english{back of boat}
  \kalamang{kudakuda}  
\english{back of knee}
  \kalamang{kor kawar mat kalot}  
\english{back; tail}
  \kalamang{or}  
\english{backbone, spine}
  \kalamang{suolkang}  
\english{backside}
  \kalamang{epkadok}  
\english{bad}
  \kalamang{ten}  
\english{bag}
  \kalamang{tas};  \kalamang{toman}  
\english{bait}
  \kalamang{set}  
\english{bake}
  \kalamang{sair} 
\english{ball}
  \kalamang{bola};  \kalamang{gol}  
\english{bamboo}
  \kalamang{daluang};  \kalamang{gaus}; \kalamang{gous}; \kalamang{karuan};  \kalamang{kasut};  \kalamang{nop};  \kalamang{tabul}  
\english{bamboo comb}
  \kalamang{suor}  
\english{bamboo floor}
  \kalamang{kanggaran}  
\english{bamboo string}
  \kalamang{lemat}  
\english{bamboo wall}
  \kalamang{kayakat}  
\english{banana}
  \kalamang{im}  
\english{banana leaf}
  \kalamang{imol}  
\english{banana sap}
  \kalamang{im polun}  
\english{Banda Islands}
  \kalamang{Andan}  
\english{bark}
  \kalamang{boukbouk};  \kalamang{gobukbuk};  \kalamang{lokul};  \kalamang{ror kulun}  
\english{barracuda}
  \kalamang{onda}  
\english{barrel}
  \kalamang{dorom};  \kalamang{tong}  
\english{basket}
  \kalamang{karanjang};  \kalamang{kiem};  \kalamang{pan}  
\english{basket rope}
  \kalamang{sun}  
\english{basket; sieve}
  \kalamang{kurera}  
\english{bastard valerian}
  \kalamang{bunga kupukupu}  
\english{batfish}
  \kalamang{sor kinggirkinggir}  
\english{bathe}
  \kalamang{boubou}  
\english{be angry}
  \kalamang{bonaras}  
\english{be full}
  \kalamang{kabor}  
\english{be lazy}
  \kalamang{jabul}  
\english{be noisy}
  \kalamang{laur}  
\english{be shy}
  \kalamang{karames}  
\english{be silent}
  \kalamang{nokin}  
\english{be sorry}
  \kalamang{namenyasal}  
\english{be stuck}
  \kalamang{nayie}  
\english{be tired}
  \kalamang{kanggir pop}  
\english{be careful on your way}
  \kalamang{nabestai bot}  
\english{be cut}
  \kalamang{tolcie}  
\english{be high}
  \kalamang{nawarir}  
\english{beach}
  \kalamang{osep}  
\english{beach edge}
  \kalamang{oskol}  
\english{beach name}
  \kalamang{Sewa}  
\english{beach.name}
  \kalamang{Timinepnep}  
\english{bead}
  \kalamang{samor}  
\english{beam}
  \kalamang{balak};  \kalamang{regil}  
\english{bean}
  \kalamang{kawuok}  
\english{beard}
  \kalamang{jangkut}  
\english{because}
  \kalamang{karena}  
\english{because; after all}
  \kalamang{habis}  
\english{beckon}
  \kalamang{lem}  
\english{become}
  \kalamang{jadi}  
\english{bedbug}
  \kalamang{tumteng}  
\english{bedroom}
  \kalamang{minggalot}  
\english{bee-eater}
  \kalamang{giringgining};  \kalamang{karopkarop}  
\english{bee; honey}
  \kalamang{wenawena}  
\english{beehive}
  \kalamang{wenawena eun}  
\english{beer}
  \kalamang{bir}  
\english{beetle; grub}
  \kalamang{sanggan}  
\english{before too long}
  \kalamang{tikninda}  
\english{behind}
  \kalamang{epko};  \kalamang{silepko}  
\english{bench}
  \kalamang{banku}  
\english{bend}
  \kalamang{maulma}  
\english{bend down; kneel}
  \kalamang{eiruk}  
\english{bent}
  \kalamang{taungtaung}  
\english{beside}
  \kalamang{mulunggo}  
\english{betel}
  \kalamang{kaul}  
\english{betel fruit}
  \kalamang{pur}  
\english{betel nut}
  \kalamang{buok teun}  
\english{betel stem}
  \kalamang{pulkiet}  
\english{betel vine}
  \kalamang{pulor}  
\english{betel; betel nut}
  \kalamang{buok}  
\english{better}
  \kalamang{lebai}  
\english{big}
  \kalamang{aremun};  \kalamang{emun};  \kalamang{temun} 
\english{big bamboo type}
  \kalamang{saban}  
\english{big heap}
  \kalamang{bungbung}  
\english{big loin cloth}
  \kalamang{mal}  
\english{big one(s)}
  \kalamang{temtemun}  
\english{big.shell}
  \kalamang{sil}  
\english{bike}
  \kalamang{sepeda}  
\english{bilimbi}
  \kalamang{takurera}  
\english{bira}
  \kalamang{bira}  
\english{bird}
  \kalamang{kasamin};  \kalamang{kasamin naun getgetkadok};  \kalamang{kedederet};  \kalamang{kolu welek}  
\english{bird of paradise}
  \kalamang{sanggien}  
\english{bite}
  \kalamang{koraruo}  
\english{bitter}
  \kalamang{mang}  
\english{bivalve}
  \kalamang{yarkanyuot}  
\english{black}
  \kalamang{kuskap}  
\english{black ant}
  \kalamang{donenet}  
\english{black butcherbird}
  \kalamang{kotipol}  
\english{black potato}
  \kalamang{yap}  
\english{black-spotted stingray}
  \kalamang{kamel kir}  
\english{bladder}
  \kalamang{ulpom}  
\english{blind}
  \kalamang{kanggir saun}  
\english{blink}
  \kalamang{delepdelep};  \kalamang{pulem};  \kalamang{tenggenggen}  
\english{block}
  \kalamang{*kies};  \kalamang{namot}  
\english{blood}
  \kalamang{karyak}  
\english{blossom}
  \kalamang{bunga arun}  
\english{blouze; shirt}
  \kalamang{ladan}  
\english{blow}
  \kalamang{kou}  
\english{blue-spotted stingray}
  \kalamang{potpot}  
\english{blue; green}
  \kalamang{welenggap}  
\english{blunt}
  \kalamang{karaonggis}  
\english{blurry}
  \kalamang{katem}  
\english{body}
  \kalamang{eren};  \kalamang{kaden}  
\english{body fat}
  \kalamang{un kawer}  
\english{body hair}
  \kalamang{kadenenen}  
\english{boil}
  \kalamang{nasomit};  \kalamang{sabet}  
\english{boil}
  \kalamang{laur}  
\english{boiler}
  \kalamang{dandang}  
\english{Bomberai inlander}
  \kalamang{tata kolak}  
\english{bone}
  \kalamang{kang}  
\english{book}
  \kalamang{buk}  
\english{border}
  \kalamang{kerap};  \kalamang{tamun}  
\english{borrow}
  \kalamang{napinjang}  
\english{bother}
  \kalamang{komister}  
\english{bottle}
  \kalamang{botal}  
\english{bottom}
  \kalamang{*elak};  \kalamang{elaun};  \kalamang{kahaman}  
\english{boulder}
  \kalamang{yarmunmun}  
\english{bounce}
  \kalamang{dalangdalang}  
\english{bounce off}
  \kalamang{seur}  
\english{bow}
  \kalamang{pusir}  
\english{bow planks}
  \kalamang{kirawat}  
\english{box}
  \kalamang{peti}  
\english{boxfish}
  \kalamang{tanggon}  
\english{bracelet}
  \kalamang{liti}  
\english{brackish}
  \kalamang{pasirwasir}  
\english{brackish water}
  \kalamang{per pasirwasir}  
\english{brahmini kite}
  \kalamang{tenggeles}  
\english{braid}
  \kalamang{pakpak}  
\english{branch}
  \kalamang{karok};  \kalamang{rorkarok};  \kalamang{sanggoup}  
\english{branch; stem}
  \kalamang{kawat}  
\english{breadfruit}
  \kalamang{po}  
\english{break}
  \kalamang{kawar};  \kalamang{naputus};  \kalamang{suosuo};  \kalamang{taraouk}  
\english{break down}
  \kalamang{maorek}  
\english{break fast}
  \kalamang{nasek};  \kalamang{tolas}  
\english{break off a branch; pick fruits}
  \kalamang{sanggotma}  
\english{break; fold}
  \kalamang{kawarma}  
\english{breast}
  \kalamang{am}  
\english{breast milk}
  \kalamang{am perun}  
\english{brick}
  \kalamang{bataku}  
\english{bride price}
  \kalamang{pasparin}  
\english{bridled monocle bream}
  \kalamang{seser}  
\english{bring}
  \kalamang{bon};  \kalamang{kuet};  \kalamang{kuru}  
\english{broken}
  \kalamang{kararcie};  \kalamang{salaboung}  
\english{broken branch}
  \kalamang{sanggoyie}  
\english{brown; grey}
  \kalamang{kowewep}  
\english{brush turkey}
  \kalamang{maniktambang}  
\english{bubble}
  \kalamang{gelembung};  \kalamang{karabubu}  
\english{bud}
  \kalamang{galip}  
\english{bulky}
  \kalamang{toungtoung}  
\english{bull/reef shark}
  \kalamang{lopteng}  
\english{bunch}
  \kalamang{pel}  
\english{bundle}
  \kalamang{poun};  \kalamang{*poup}  
\english{Burewun}
  \kalamang{Burewun}  
\english{burn}
  \kalamang{dinan};  \kalamang{komaruk};  \kalamang{komelek}  
\english{Buruwai}
  \kalamang{Sabaor}  
\english{bury}
  \kalamang{dan}  
\english{busily}
  \kalamang{rami}  
\english{but}
  \kalamang{ba};  \kalamang{tapi}  
\english{butcherbirds}
  \kalamang{loup}  
\english{Buton}
  \kalamang{Utun}  
\english{butterfly}
  \kalamang{pulpul}  
\english{butterflyfish}
  \kalamang{kilibobang}  
\english{calf of leg}
  \kalamang{kormul}  
\english{call out}
  \kalamang{roukmang}  
\english{call names}
  \kalamang{wolnelebor}  
\english{call; call out}
  \kalamang{gonggung}  
\english{calm}
  \kalamang{up}  
\english{calm sea}
  \kalamang{pasier up}  
\english{can}
  \kalamang{belek};  \kalamang{bisa}  
\english{candlenut}
  \kalamang{komeri}  
\english{cane}
  \kalamang{kom}  
\english{cannonball tree}
  \kalamang{kai taul}  
\english{cannot}
  \kalamang{eranun}  
\english{canoe}
  \kalamang{et}  
\english{canoe plank}
  \kalamang{pawan tabak}  
\english{cap}
  \kalamang{kawier}  
\english{cape}
  \kalamang{kariemun}  
\english{care}
  \kalamang{nafaduli}  
\english{carry}
  \kalamang{nawas};  \kalamang{pabie}  
\english{carry living being on back}
  \kalamang{poup}  
\english{carry on back}
  \kalamang{bitko}  
\english{carry on shoulders}
  \kalamang{lebaleba}  
\english{carve}
  \kalamang{kies}  
\english{cassava}
  \kalamang{panggala}  
\english{cassowary}
  \kalamang{kasawari}  
\english{casuarina (tree)}
  \kalamang{rur}  
\english{cat}
  \kalamang{sikan}  
\english{catapult}
  \kalamang{kataperor}  
\english{catch}
  \kalamang{loku}  
\english{caterpillar}
  \kalamang{panggatpanggat}  
\english{caught with fear}
  \kalamang{semsuk}  
\english{cause to snap}
  \kalamang{meraraouk}  
\english{cave}
  \kalamang{iar}  
\english{cement floor}
  \kalamang{misilmisil}  
\english{centipede}
  \kalamang{masawin}  
\english{chain}
  \kalamang{ranti}  
\english{chainsaw}
  \kalamang{sensor}  
\english{chair}
  \kalamang{kadera}  
\english{change}
  \kalamang{kosalir};  \kalamang{kowat};  \kalamang{naruba};  \kalamang{pitisnaharen};  \kalamang{salir}  
\english{charcoal}
  \kalamang{kus}  
\english{chase}
  \kalamang{namasawuot}  
\english{chase; follow; hunt}
  \kalamang{sarie}  
\english{chat; tell a story}
  \kalamang{rer}  
\english{chayote}
  \kalamang{labu siam}  
\english{cheat}
  \kalamang{selinku}  
\english{check}
  \kalamang{cek}  
\english{cheek}
  \kalamang{koliep}  
\english{chest}
  \kalamang{aknar}  
\english{chestnut}
  \kalamang{gayam}  
\english{chew}
  \kalamang{pak}  
\english{chew betel}
  \kalamang{buokbuok}  
\english{chewy; tense}
  \kalamang{dong}  
\english{chicken}
  \kalamang{kokok}  
\english{chicken egg}
  \kalamang{kokok narun}  
\english{child}
  \kalamang{tumun}  
\english{chilli}
  \kalamang{lenggalengga}  
\english{chisel}
  \kalamang{taot};  \kalamang{taot}  
\english{chiton}
  \kalamang{gawawi}  
\english{chop}
  \kalamang{dakdak};  \kalamang{tawara}  
\english{christian}
  \kalamang{kawir}  
\english{cicada}
  \kalamang{nene}  
\english{circle}
  \kalamang{kokarap}  
\english{circumcise}
  \kalamang{nasalik}  
\english{civet cat}
  \kalamang{samameng}  
\english{clam}
  \kalamang{kanyuot}  
\english{clamp}
  \kalamang{kowaram}  
\english{claw}
  \kalamang{paritman}  
\english{claw; scratch}
  \kalamang{naparis}  
\english{clay}
  \kalamang{naun kerkap}  
\english{clean pandanus}
  \kalamang{us}  
\english{clear}
  \kalamang{gelas};  \kalamang{go sir};  \kalamang{kalar};  \kalamang{sir}  
\english{clear land}
  \kalamang{amdir komaruk}  
\english{cleared}
  \kalamang{toras}  
\english{cleared forest}
  \kalamang{sabel}  
\english{climb}
  \kalamang{koyos}  
\english{close}
  \kalamang{komasasuk};  \kalamang{koyen};  \kalamang{sanggie};  \kalamang{tara}  
\english{close eyes}
  \kalamang{nunun}  
\english{close off with plank}
  \kalamang{nawarak}  
\english{close roof}
  \kalamang{komanggangguop}  
\english{cloth}
  \kalamang{donselet};  \kalamang{tapal}  
\english{cloud}
  \kalamang{kierun};  \kalamang{ur kirun}  
\english{clove tree}
  \kalamang{cengki}  
\english{clown triggerfish}
  \kalamang{kawaramleit}  
\english{club}
  \kalamang{kabiep};  \kalamang{pabiep}  
\english{cockatoo}
  \kalamang{yakop}  
\english{cockroach}
  \kalamang{dudin}  
\english{cockspur coral tree}
  \kalamang{sarik}  
\english{coconut}
  \kalamang{wat};  \kalamang{wat karoraun};  \kalamang{wat kerkapkap};  \kalamang{wat sarenden}  
\english{coconut leaf}
  \kalamang{walor}  
\english{coconut leaf midrib}
  \kalamang{walorteng} 
\english{coconut scraper}
  \kalamang{tara}  
\english{coconut shell}
  \kalamang{taokang};  \kalamang{taukanggir}  
\english{coffee}
  \kalamang{kofir};  \kalamang{per kuskap}  
\english{coil}
  \kalamang{tar}  
\english{coiled}
  \kalamang{tar}  
\english{cold}
  \kalamang{lu}  
\english{cold evening wind}
  \kalamang{pelelu}  
\english{collar bone}
  \kalamang{aknar kangun}  
\english{collect}
  \kalamang{olol}  
\english{collect water}
  \kalamang{natada}  
\english{comb}
  \kalamang{sisir};  \kalamang{sisir}  
\english{come}
  \kalamang{luk};  \kalamang{mei};  \kalamang{mia}  
\english{come.out}
  \kalamang{taluk}  
\english{command}
  \kalamang{parenta}  
\english{compact; smooth}
  \kalamang{pol}  
\english{company}
  \kalamang{perusahan}  
\english{completely opened}
  \kalamang{nasalen}  
\english{concave side}
  \kalamang{keitpis}  
\english{concha}
  \kalamang{kelkam taun}  
\english{concrete}
  \kalamang{semen}  
\english{condition}
  \kalamang{go};  \kalamang{kondisi}  
\english{cone shell}
  \kalamang{salawei}  
\english{confirmation}
  \kalamang{panok mecuan}  
\english{confused; bothered}
  \kalamang{pusing}  
\english{connect}
  \kalamang{nasambung}  
\english{consent; like}
  \kalamang{lo}  
\english{consume}
  \kalamang{na}  
\english{container}
  \kalamang{bak}  
\english{continue}
  \kalamang{ajar};  \kalamang{langjut}  
\english{convex side}
  \kalamang{akpis}  
\english{cook}
  \kalamang{kuar};  \kalamang{muawaruo};  \kalamang{pau}  
\english{cooked}
  \kalamang{ruo}  
\english{cooking.utensil}
  \kalamang{karam}  
\english{coral}
  \kalamang{yar loupkaning}  
\english{coral reef}
  \kalamang{ram}  
\english{coral stones}
  \kalamang{terar}  
\english{coral.shore}
  \kalamang{yarnener}  
\english{cormorant}
  \kalamang{lamora kasamin}  
\english{corner}
  \kalamang{sair}  
\english{cornetfish}
  \kalamang{tokatokan}  
\english{cotton}
  \kalamang{kai kawas};  \kalamang{kapas}  
\english{cough}
  \kalamang{tadon}  
\english{count}
  \kalamang{nahitung};  \kalamang{narekin}  
\english{cousin}
  \kalamang{dudan}  
\english{cover; dress}
  \kalamang{komasabur}  
\english{cow}
  \kalamang{sapi}  
\english{crab}
  \kalamang{keluer};  \kalamang{ugar}  
\english{crawl}
  \kalamang{guadang};  \kalamang{manggaren}  
\english{crawl; slither}
  \kalamang{gare}  
\english{crocodile}
  \kalamang{paramuang}  
\english{crooked}
  \kalamang{maulcie}  
\english{cross-cousin}
  \kalamang{korap}  
\english{crossed arms/legs}
  \kalamang{naulanggos}  
\english{crow}
  \kalamang{goras}  
\english{crowned pigeon}
  \kalamang{maniktapuri}  
\english{crush}
  \kalamang{maraok}  
\english{crushed}
  \kalamang{karaok}  
\english{cry}
  \kalamang{ecua};  \kalamang{eruap}  
\english{cry for}
  \kalamang{koecuan}  
\english{cuckoo}
  \kalamang{masoi}  
\english{cucumber}
  \kalamang{komurkomur}  
\english{cup}
  \kalamang{cangkir}  
\english{curse}
  \kalamang{kout};  \kalamang{nanetkon};  \kalamang{parambura};  \kalamang{penyakit kat nanetkon}  
\english{curtain}
  \kalamang{sirisiri}  
\english{cuscus}
  \kalamang{se}  
\english{custom}
  \kalamang{kalour}  
\english{cut}
  \kalamang{karok};  \kalamang{korkor};  \kalamang{masalaboung};  \kalamang{paramua};  \kalamang{potma};  \kalamang{tabak}  
\english{cut a coconut; break}
  \kalamang{suo}  
\english{cut diagonally}
  \kalamang{suwarma}  
\english{cut down a tree; cut}
  \kalamang{wuorma}  
\english{cut off}
  \kalamang{seletma}  
\english{cut out}
  \kalamang{kortaptap}  
\english{cut throat}
  \kalamang{mintolma}  
\english{cut branch}
  \kalamang{letma}  
\english{cut string; take a shortcut}
  \kalamang{tolma}  
\english{cut; split}
  \kalamang{pel}
\english{damselfish}
  \kalamang{sembamsembam}  
\english{dance}
  \kalamang{nasula};  \kalamang{tarian}  
\english{dangle}
  \kalamang{dek}  
\english{dark}
  \kalamang{sausaun}  
\english{darkness}
  \kalamang{sausaun}  
\english{darter}
  \kalamang{kedua}  
\english{daughter}
  \kalamang{tumun pas}  
\english{dawn}
  \kalamang{sobas}  
\english{day}
  \kalamang{go yuol};  \kalamang{hari};  \kalamang{yuol};  \kalamang{yuor}  
\english{day after tomorrow}
  \kalamang{keirko}  
\english{day before yesterday}
  \kalamang{keitar}  
\english{dead}
  \kalamang{korek};  \kalamang{somin}  
\english{deaf}
  \kalamang{kelkam toktok}  
\english{debt}
  \kalamang{uran}  
\english{deep}
  \kalamang{karan};  \kalamang{naman}  
\english{deep sea}
  \kalamang{tep}  
\english{deep seawater}
  \kalamang{tepner}  
\english{deer}
  \kalamang{rusa}  
\english{defecate}
  \kalamang{kiet};  \kalamang{kietkiet};  \kalamang{pasienggara bot}  
\english{demon}
  \kalamang{tapukan}  
\english{dent}
  \kalamang{salak}  
\english{dented}
  \kalamang{salak}  
\english{descend}
  \kalamang{bara}  
\english{destroy}
  \kalamang{fakurat}  
\english{dew}
  \kalamang{masinul}  
\english{dibble stick}
  \kalamang{sakarip}  
\english{die}
  \kalamang{lalat};  \kalamang{nalat}  
\english{difficult}
  \kalamang{susa};  \kalamang{susia}  
\english{dig}
  \kalamang{ruo}  
\english{diligent}
  \kalamang{mososor}  
\english{directly}
  \kalamang{langsung}  
\english{dirt}
  \kalamang{kotur}  
\english{dirty}
  \kalamang{kotur}  
\english{disappeared}
  \kalamang{gosomin}  
\english{disease}
  \kalamang{kaman}  
\english{dish}
  \kalamang{holang}  
\english{distal demonstrative}
  \kalamang{ime};  \kalamang{imene};  \kalamang{me};  \kalamang{yume};  \kalamang{yumene}  
\english{disturb; mix}
  \kalamang{koyal}  
\english{dive}
  \kalamang{ar}  
\english{divide}
  \kalamang{unkoryap};  \kalamang{yap}  
\english{do tenggelele}
  \kalamang{tenggelele}  
\english{do; make}
  \kalamang{paruo};  \kalamang{paruowaruo}  
\english{do; try}
  \kalamang{bonasau}  
\english{dock}
  \kalamang{nasandar}  
\english{does not want}
  \kalamang{sukaun ge}  
\english{dog}
  \kalamang{bal}  
\english{dolphin}
  \kalamang{kurap}  
\english{door}
  \kalamang{anggas}  
\english{doorpost}
  \kalamang{anggas padenun}  
\english{dove-like birds}
  \kalamang{sikuki}  
\english{down}
  \kalamang{yawe}  
\english{dowry}
  \kalamang{mahar}  
\english{dozen}
  \kalamang{losing}  
\english{drag}
  \kalamang{masaouk}  
\english{dragonfish; filefish}
  \kalamang{saimbumbu}  
\english{dragonfly}
  \kalamang{per paiwai}  
\english{draw a line; mark}
  \kalamang{kasep}  
\english{draw line}
  \kalamang{nagaris}  
\english{dream}
  \kalamang{wuorwuor}  
\english{drill}
  \kalamang{bor};  \kalamang{bor}  
\english{drink}
  \kalamang{minum}  
\english{drinking water}
  \kalamang{per iriskap}  
\english{drivers}
  \kalamang{hukat narun}  
\english{drizzle, light rain}
  \kalamang{kalis sasarawe}  
\english{drool}
  \kalamang{ewarom}  
\english{drop}
  \kalamang{cici};  \kalamang{komamun};  \kalamang{lapas};  \kalamang{matur};  \kalamang{naberuak}  
\english{drum}
  \kalamang{kawaret};  \kalamang{tetetas};  \kalamang{tiri}  
\english{drunk}
  \kalamang{mabuk}  
\english{dry}
  \kalamang{kararak};  \kalamang{karuar};  \kalamang{mararak};  \kalamang{sa};  \kalamang{sansa}  
\english{dry in the sun}
  \kalamang{masa}  
\english{dry season}
  \kalamang{yuon monpak}  
\english{drying rack}
  \kalamang{karuar} 
\english{duck}
  \kalamang{bebak}  
\english{dugong}
  \kalamang{wang}  
\english{durian}
  \kalamang{duran}  
\english{dusk}
  \kalamang{go kerkap}  
\english{eagle}
  \kalamang{lusi};  \kalamang{lusi pep jiejie}  
\english{ear}
  \kalamang{kelkam}  
\english{ear opening}
  \kalamang{kelkampos}  
\english{earlier}
  \kalamang{opa}  
\english{earlobe}
  \kalamang{kelkam elaun}  
\english{early morning}
  \kalamang{go dung}  
\english{earrings}
  \kalamang{anting};  \kalamang{gigiwang}  
\english{earth oven}
  \kalamang{pau}  
\english{earthenware vase}
  \kalamang{unana}  
\english{earthquake}
  \kalamang{leng dek}  
\english{earthworm}
  \kalamang{kalabet}  
\english{earwax}
  \kalamang{kolkiet}  
\english{east}
  \kalamang{talawak}  
\english{east of Karas}
  \kalamang{nanam}  
\english{east-side}
  \kalamang{tagurep}  
\english{east season}
  \kalamang{tagurpak}  
\english{east; east wind; wet season}
  \kalamang{tagur}  
\english{eastern koel}
  \kalamang{isak}  
\english{eat}
  \kalamang{bolkoyal};  \kalamang{muap};  \kalamang{mus}  
\english{edge}
  \kalamang{*as};  \kalamang{asun};  \kalamang{duk};  \kalamang{*siep};  \kalamang{sieun};  \kalamang{*tim};  \kalamang{timun}  
\english{eel}
  \kalamang{baluku};  \kalamang{gulas};  \kalamang{kawarsuop};  \kalamang{suopkaling}  
\english{egg}
  \kalamang{nar}  
\english{egg; seed}
  \kalamang{narun}  
\english{eggplant}
  \kalamang{torim}  
\english{eight}
  \kalamang{irie}  
\english{eighty}
  \kalamang{putirie}  
\english{elbow}
  \kalamang{tanggul} 
\english{eleven}
  \kalamang{putkon ba kon}  
\english{embers}
  \kalamang{din songsong};  \kalamang{paras}  
\english{emerge}
  \kalamang{saouk}  
\english{emperor fish}
  \kalamang{tim}  
\english{empty place}
  \kalamang{go saerak}  
\english{English}
  \kalamang{Inggrismang}  
\english{entangled}
  \kalamang{kolkol}  
\english{enter}
  \kalamang{masuk}  
\english{envelope}
  \kalamang{lopalopa}  
\english{erect}
  \kalamang{usar}  
\english{Esa Tanggiun}
  \kalamang{Esa Tanggiun}  
\english{even if}
  \kalamang{biar};  \kalamang{=taero}  
\english{ever}
  \kalamang{perna}  
\english{exactly}
  \kalamang{pas}  
\english{example}
  \kalamang{sontur}  
\english{exceed}
  \kalamang{nemies}  
\english{exclusively}
  \kalamang{=saet}  
\english{excrete}
  \kalamang{nasabir}  
\english{exit; fruit}
  \kalamang{paruok}  
\english{expel}
  \kalamang{ar};  \kalamang{eis};  \kalamang{pasiep}  
\english{expenses}
  \kalamang{ongkos}  
\english{explode}
  \kalamang{dumang};  \kalamang{pakmang}  
\english{extend on floor}
  \kalamang{parara}  
\english{extinguish; kill}
  \kalamang{rua}  
\english{eye}
  \kalamang{kanggir}  
\english{eyelashes}
  \kalamang{kanggir nenen}  
\english{eyelid}
  \kalamang{kanggir pulun}  
\english{face}
  \kalamang{kanggirar}; \kalamang{konasur};  \kalamang{namangadap}  
\english{facial hair}
  \kalamang{konenen}  
\english{faeces}
  \kalamang{kiet}  
\english{Fakfak people}
  \kalamang{Mahem}  
\english{Fakfak person}
  \kalamang{Mata}  
\english{Fakfak (town)}
  \kalamang{Pakpak}  
\english{fall}
  \kalamang{our};  \kalamang{pururu};  \kalamang{telebor};  \kalamang{tur}  
\english{fall over}
  \kalamang{rouk}  
\english{falling of fruit}
  \kalamang{ruak}  
\english{falling and making a sound}
  \kalamang{pukmang}  
\english{fall; crash}
  \kalamang{tabarak}  
\english{falling of leaves}
  \kalamang{upsa}  
\english{family}
  \kalamang{wol}  
\english{family name}
  \kalamang{fam}  
\english{fan}
  \kalamang{kawiawi}  
\english{Faor}
  \kalamang{Pour}  
\english{far; tall; long}
  \kalamang{kahen}  
\english{fart}
  \kalamang{kietpo}  
\english{fast}
  \kalamang{loi}  
\english{Fatar}
  \kalamang{Parar}  
\english{father; adult man; uncle}
  \kalamang{esa}  
\english{fathom}
  \kalamang{taraman}  
\english{feather}
  \kalamang{teng};  \kalamang{tengun}  
\english{feel}
  \kalamang{sawaluo}  
\english{feel cold}
  \kalamang{kawes}  
\english{feel good}
  \kalamang{mawin}  
\english{feel uncomfortable}
  \kalamang{namakin}  
\english{female infant}
  \kalamang{enemtumun}  
\english{female; woman}
  \kalamang{pas}  
\english{fence}
  \kalamang{tal}  
\english{fern}
  \kalamang{palang}  
\english{few}
  \kalamang{etaman};  \kalamang{-taman}  
\english{fibre boat}
  \kalamang{fiber}  
\english{fifteen}
  \kalamang{putkon ba ap}  
\english{fifty}
  \kalamang{purap}  
\english{fig}
  \kalamang{kul}  
\english{fight}
  \kalamang{langka};  \kalamang{naras};  \kalamang{nausair}  
\english{file clam}
  \kalamang{waring}  
\english{fill}
  \kalamang{payiem}  
\english{film; photo}
  \kalamang{foto}  
\english{filter}
  \kalamang{narawi}  
\english{find; meet}
  \kalamang{koluk}  
\english{finger}
  \kalamang{parok};  \kalamang{tanparok}  
\english{fingernail}
  \kalamang{tanggalip}  
\english{finish}
  \kalamang{koyet}  
\english{finished}
  \kalamang{habis};  \kalamang{se koyet}  
\english{fire}
  \kalamang{din}  
\english{fire burning}
  \kalamang{susur}  
\english{firefly}
  \kalamang{elam};  \kalamang{karop};  \kalamang{mata dimdim}  
\english{fireplace}
  \kalamang{didir}  
\english{firewood}
  \kalamang{kaipur}  
\english{firewood}
  \kalamang{kai}  
\english{first}
  \kalamang{giarun}  
\english{firstborn}
  \kalamang{ma temun}  
\english{fish}
  \kalamang{sor} 
\english{fish at sea}
  \kalamang{sek}  
\english{fish cage}
  \kalamang{karamba}  
\english{fish in low water}
  \kalamang{kosa}  
\english{fish leftovers}
  \kalamang{sor pespes}  
\english{fish net}
  \kalamang{hukat};  \kalamang{suelet}  
\english{fish scales}
  \kalamang{lawuak}  
\english{fish trap}
  \kalamang{fer};  \kalamang{gawar}  
\english{fish place}
  \kalamang{sair}
\english{fishbone}
  \kalamang{sor kangun} 
\english{fishing hook}
  \kalamang{kaling}  
\english{fishing line}
  \kalamang{nika};  \kalamang{wien}  
\english{fit}
  \kalamang{pas}  
\english{five}
  \kalamang{ap}  
\english{five hundred}
  \kalamang{reirap}  
\english{flames}
  \kalamang{din paras}  
\english{flank}
  \kalamang{kirun}  
\english{flea thing}
  \kalamang{mutam}  
\english{flee}
  \kalamang{kiem}  
\english{flip}
  \kalamang{mayilma}  
\english{flipped}
  \kalamang{koyelcie}  
\english{float}
  \kalamang{pouk}  
\english{flower}
  \kalamang{bunga};  \kalamang{pus}
\english{flowing water}
  \kalamang{per taluk}  
\english{flute}
  \kalamang{wangguwanggus}  
\english{fly}
  \kalamang{pabalet};  \kalamang{pararuo}  
\english{fly around}
  \kalamang{pulpulkon}  
\english{fly off}
  \kalamang{lele}  
\english{flycatcher}
  \kalamang{salout}  
\english{flying fish}
  \kalamang{masal}  
\english{foam}
  \kalamang{pus}  
\english{fog; snow}
  \kalamang{teok}  
\english{foggy; snow}
  \kalamang{teok}  
\english{fold}
  \kalamang{kang};  \kalamang{kawetkawet};  \kalamang{tawotma}  
\english{follow}
  \kalamang{pareir}  
\english{fontanelle}
  \kalamang{nakal pokpok}  
\english{food}
  \kalamang{muap}  
\english{foothill}
  \kalamang{ruom}  
\english{footprint}
  \kalamang{kolkom}  
\english{footsole}
  \kalamang{korel}  
\english{for}
  \kalamang{untuk}  
\english{force}
  \kalamang{paksa}  
\english{forehead}
  \kalamang{timbang}  
\english{forget}
  \kalamang{konawaruo}  
\english{forty}
  \kalamang{putkansuor}  
\english{forty-one}
  \kalamang{putkansuor talinggon}  
\english{foundation}
  \kalamang{lorap};  \kalamang{rorap}  
\english{four}
  \kalamang{kansuor}  
\english{Friday}
  \kalamang{ariemun}  
\english{fried coconut}
  \kalamang{wat pasor}  
\english{friend}
  \kalamang{lidan};  \kalamang{teman}  
\english{frigatebird}
  \kalamang{taram}  
\english{frog}
  \kalamang{kareng}  
\english{from outside}
  \kalamang{kolga}  
\english{front}
  \kalamang{borara};  \kalamang{kalaor};  \kalamang{muka}  
\english{front (of a boat)}
  \kalamang{sabar}  
\english{fruit}
  \kalamang{loup};  \kalamang{naun}; \kalamang{nak};  \kalamang{tep};  \kalamang{*tep};  \kalamang{teun}  
\english{fruit bat}
  \kalamang{kuek}  
\english{fruit set}
  \kalamang{kababur}  
\english{fruit-dove}
  \kalamang{kuk}  
\english{fry}
  \kalamang{pasor}  
\english{frying pan}
  \kalamang{kaling}  
\english{full}
  \kalamang{mikon}  
\english{full moon}
  \kalamang{pak tubak}  
\english{further}
  \kalamang{terus}  
\english{fusilier}
  \kalamang{top}  
\english{gado-gado}
  \kalamang{ododa}  
\english{gallbladder}
  \kalamang{iem}  
\english{game}
  \kalamang{kanggeirun};  \kalamang{kanggeit}  
\english{garden}
  \kalamang{amdir}  
\english{garlic}
  \kalamang{bawang iriskapten}  
\english{gathered}
  \kalamang{tengguen}  
\english{gecko}
  \kalamang{tokitoki}  
\english{get}
  \kalamang{rep}  
\english{get better}
  \kalamang{natora}  
\english{get married}
  \kalamang{halar};  \kalamang{nauhalar}  
\english{get rid of}
  \kalamang{melebor}  
\english{get; buy}
  \kalamang{jie}  
\english{ghost}
  \kalamang{sietan}  
\english{giant}
  \kalamang{laksasa}  
\english{giant trevally}
  \kalamang{imanana}  
\english{gills}
  \kalamang{mesang}  
\english{ginger}
  \kalamang{sarem}  
\english{ginger-like root}
  \kalamang{gulasi}  
\english{give}
  \kalamang{∅}  
\english{give back}
  \kalamang{namasuk}  
\english{give birth}
  \kalamang{amkeit};  \kalamang{unmasir}  
\english{give packed food}
  \kalamang{kosansan}  
\english{glass}
  \kalamang{don pernanan};  \kalamang{gelas};  \kalamang{ginana}  
\english{glasses}
  \kalamang{saramin}  
\english{glassy sweeper}
  \kalamang{kapapet}  
\english{go}
  \kalamang{bo}  
\english{go backwards}
  \kalamang{nasuk}  
\english{go quickly}
  \kalamang{sasat}  
\english{go to mosque; hold service}
  \kalamang{nasambian}  
\english{go out}
  \kalamang{keluar}  
\english{go to school}
  \kalamang{sekola}  
\english{goal}
  \kalamang{bet}  
\english{goat}
  \kalamang{lek}  
\english{goatfish}
  \kalamang{ultom}  
\english{gold}
  \kalamang{marau}  
\english{golden trevally}
  \kalamang{irausi}  
\english{gong}
  \kalamang{kulikuli};  \kalamang{manggamangga}  
\english{good}
  \kalamang{bes}  
\english{good luck with fishing}
  \kalamang{tanggal}  
\english{good wish}
  \kalamang{salamat}  
\english{goosebumps}
  \kalamang{warkangkang}  
\english{Gorom}
  \kalamang{Walaka}  
\english{Goromese}
  \kalamang{walaka};  \kalamang{Walakamang}  
\english{grab}
  \kalamang{kasorma};  \kalamang{konggelem};  \kalamang{narampas}  
\english{grandchild; grandparent}
  \kalamang{*tara}  
\english{granddaughter}
  \kalamang{taraun pas}  
\english{grandfather}
  \kalamang{tara esnem};  \kalamang{tata};  \kalamang{tete}  
\english{grandfather; man}
  \kalamang{esnem}  
\english{grandmother}
  \kalamang{nina};  \kalamang{tara emnem}  
\english{grandmother; respected woman}
  \kalamang{tatanina}  
\english{grandson}
  \kalamang{taraun canam}  
\english{grass}
  \kalamang{kaman};  \kalamang{kucai};  \kalamang{yuor}  
\english{grasshopper}
  \kalamang{kurera};  \kalamang{pulseka};  \kalamang{wuar sisiarun}  
\english{grate}
  \kalamang{kir}  
\english{grater}
  \kalamang{suan}  
\english{grave}
  \kalamang{kubir}  
\english{gravel}
  \kalamang{yaralus}  
\english{gravestone}
  \kalamang{mesan}  
\english{graveyard}
  \kalamang{kubirar}  
\english{greatgrandfather}
  \kalamang{tatanus}  
\english{greatgrandmother}
  \kalamang{ninanus};  \kalamang{tataninanus}  
\english{greatgrandparent}
  \kalamang{kanus}  
\english{greedy}
  \kalamang{kir}  
\english{green bean}
  \kalamang{balikawuok}  
\english{green beans}
  \kalamang{boncis}  
\english{green coconut}
  \kalamang{wat sasul}  
\english{group}
  \kalamang{dong};  \kalamang{rombongan}  
\english{grouper}
  \kalamang{kabaruap};  \kalamang{kabaruap kerkapkap};  \kalamang{kabaruap kuskap};  \kalamang{lameli};  \kalamang{lawan};  \kalamang{sar kararok};  \kalamang{sarbal};  \kalamang{yakarek taraun}  
\english{grow}
  \kalamang{kos}  
\english{grow seedling}
  \kalamang{nageiding}  
\english{growing tip}
  \kalamang{timun sobangun}  
\english{growth}
  \kalamang{kosun}  
\english{grub}
  \kalamang{laluon}  
\english{guava}
  \kalamang{sarim}  
\english{gum tree; eucalyptus}
  \kalamang{ror iriskap}  
\english{gums}
  \kalamang{gierkawer}  
\english{hair}
  \kalamang{westal}  
\english{hair pin}
  \kalamang{sopsop}  
\english{hairy}
  \kalamang{sanamsanam}  
\english{half}
  \kalamang{*tabak};  \kalamang{tabaon}  
\english{half-dry}
  \kalamang{emsan}  
\english{half; side}
  \kalamang{taikon}  
\english{halfbeak}
  \kalamang{nebidangat}  
\english{Halmahera}
  \kalamang{Almahera}  
\english{hammer}
  \kalamang{resan}  
\english{hand}
  \kalamang{kosiaur}  
\english{handle}
  \kalamang{kein};  \kalamang{man}  
\english{handpalm}
  \kalamang{tan laus}  
\english{hang}
  \kalamang{gang};  \kalamang{ganggang};  \kalamang{manggang}  
\english{happy}
  \kalamang{gampang}  
\english{harbour}
  \kalamang{los}  
\english{hard}
  \kalamang{karan}  
\english{harlequin shrimp}
  \kalamang{sairarar ladok}  
\english{harvest fruit}
  \kalamang{koser}  
\english{hat}
  \kalamang{saraun};  \kalamang{sepe}  
\english{have a family}
  \kalamang{ruma tangga}  
\english{have sex}
  \kalamang{yam}  
\english{hawkfish}
  \kalamang{sere sorun}  
\english{head}
  \kalamang{nakal}  
\english{heap}
  \kalamang{*pan};  \kalamang{*taur};  \kalamang{tengguen}  
\english{heap; gather}
  \kalamang{merengguen}  
\english{hear}
  \kalamang{ra}  
\english{hear; listen}
  \kalamang{kelua}  
\english{heat in fire}
  \kalamang{balama}  
\english{heavy}
  \kalamang{tagier}  
\english{heel}
  \kalamang{tararapang}  
\english{helmeted friarbird}
  \kalamang{kokoak}  
\english{help}
  \kalamang{rup}  
\english{hermit crab}
  \kalamang{karor}  
\english{heron}
  \kalamang{doka};  \kalamang{tarakok}  
\english{hey-ho}
  \kalamang{geigar}  
\english{hibiscus}
  \kalamang{yopyop};  \kalamang{yuopyuop}  
\english{hiccups}
  \kalamang{soksok}  
\english{hide}
  \kalamang{mormor};  \kalamang{nung}  
\english{high tide}
  \kalamang{warkin laur};  \kalamang{warkin nasesak}  
\english{high tide}
  \kalamang{nasesak}  
\english{hill}
  \kalamang{turing}  
\english{hill; cliff}
  \kalamang{sarieng}  
\english{his wife}
  \kalamang{kieun}  
\english{hit}
  \kalamang{burbur};  \kalamang{duk};  \kalamang{kararma};  \kalamang{pue} 
\english{hit with tool}
  \kalamang{koyak}  
\english{hit; pound}
  \kalamang{tu}  
\english{hit; touch}
  \kalamang{kosara}  
\english{hold}
  \kalamang{iar};  \kalamang{kinkin}  
\english{hold; carry}
  \kalamang{tanggo}  
\english{hole}
  \kalamang{lolouk};  \kalamang{pos};  \kalamang{suarkang}  
\english{holiday}
  \kalamang{robaherkiem}  
\english{holy water}
  \kalamang{per natawarten}  
\english{hook}
  \kalamang{ser}  
\english{horn shell}
  \kalamang{powar}  
\english{hornbill}
  \kalamang{mamor}  
\english{horns}
  \kalamang{suor}  
\english{horse}
  \kalamang{kuda};  \kalamang{lajarang}  
\english{hot}
  \kalamang{lalang}  
\english{hot water}
  \kalamang{pelalang}  
\english{hour; o'clock}
  \kalamang{jam}  
\english{house}
  \kalamang{kewe}  
\english{house post}
  \kalamang{kewe padenun}  
\english{household}
  \kalamang{ruma tangga}  
\english{how are you}
  \kalamang{nauwar tamandi}  
\english{how many}
  \kalamang{puraman}  
\english{how; how are you}
  \kalamang{tamandi}  
\english{hug}
  \kalamang{koup}  
\english{humpback snapper}
  \kalamang{sanual}  
\english{hundred}
  \kalamang{*reit}  
\english{hungry}
  \kalamang{kabor lalang};  \kalamang{muawese};  \kalamang{muisese}  
\english{husband}
  \kalamang{*nam}  
\english{husks}
  \kalamang{kupkup}  
\english{I don't know}
  \kalamang{yeso}  
\english{ibis}
  \kalamang{kurua}  
\english{ice}
  \kalamang{es}  
\english{if}
  \kalamang{kalau}  
\english{if not}
  \kalamang{get me}  
\english{ill}
  \kalamang{ning}  
\english{illness}
  \kalamang{barala};  \kalamang{ning}  
\english{illness; curse}
  \kalamang{penyakit}  
\english{imam}
  \kalamang{leba}   
\english{imperial pigeon}
  \kalamang{muradik};  \kalamang{mursambuk}  
\english{imperial pigeon type}
  \kalamang{murkumkum}  
\english{in the light}
  \kalamang{goraruo}  
\english{in the middle}
  \kalamang{raorko}   
\english{index finger}
  \kalamang{tansahadat}  
\english{Indonesian}
  \kalamang{Malaimang};  \kalamang{Pepmang}  
\english{infant}
  \kalamang{au}  
\english{insect found in rice or flour}
  \kalamang{mam}  
\english{inside}
  \kalamang{neko};  \kalamang{*ner};  \kalamang{nerun};  \kalamang{nerunggo}  
\english{inside canoe}
  \kalamang{kanggurun}  
\english{inside of a tree}
  \kalamang{*kun}  
\english{inside of pili nut}
  \kalamang{kanaisasen}  
\english{inside of a tree}
  \kalamang{kunun}  
\english{intercourse}
  \kalamang{kasuo}  
\english{intestines}
  \kalamang{kabun}  
\english{iron}
  \kalamang{kamung}  
\english{iron; wire}
  \kalamang{taba}  
\english{ironwood}
  \kalamang{suar}    
\english{island}
  \kalamang{lempuang}  
\english{it doesn't matter}
  \kalamang{don konkonin}  
\english{it's cloudy}
  \kalamang{go git}  
\english{itchy}
  \kalamang{layier}  
\english{itchy fish; itchy plant}
  \kalamang{sere}  
\english{jackfruit}
  \kalamang{taberak}  
\english{jar}
  \kalamang{tepeles}  
\english{Java; Javanese}
  \kalamang{Jawa}  
\english{jaw; chin}
  \kalamang{kanggus}  
\english{jew's harp}
  \kalamang{gonggong}  
\english{jinn}
  \kalamang{jim}  
\english{joint}
  \kalamang{*kasir}  
\english{joke}
  \kalamang{marok}  
\english{journey}
  \kalamang{bot}  
\english{jump}
  \kalamang{dalang}  
\english{jump over}
  \kalamang{korabir}  
\english{jungle}
  \kalamang{ting}  
\english{just}
  \kalamang{nak=};  \kalamang{=tak}  
\english{just a little}
  \kalamang{bolodak}  
\english{k.o. bamboo}
  \kalamang{sasirip}  
\english{k.o. banana}
  \kalamang{im pawan};  \kalamang{im sarawuar};  \kalamang{im selen};  \kalamang{im sepatu};  \kalamang{im sontum};  \kalamang{im yuol putkansuor}  
\english{k.o. bird}
  \kalamang{masriku}  
\english{k.o. coarse woven mat}
  \kalamang{el}   
\english{k.o. coconut}
  \kalamang{garawi}  
\english{k.o. coral}
  \kalamang{ram kolkemkem};  \kalamang{ram parokparok}; \kalamang{waranggeit} 
\english{k.o. fish}
  \kalamang{alar};  \kalamang{birbir};  \kalamang{bugar}\kalamang{dadir};  \kalamang{golip};  \kalamang{gorip};  \kalamang{goyas};  \kalamang{gual};  \kalamang{kamandi}; \kalamang{kanaisasen};  \kalamang{kanas};  \kalamang{kanas kolkol};  \kalamang{kas};  \kalamang{kelikeli};  \kalamang{kir};  \kalamang{kolpanggat};  \kalamang{kous};  \kalamang{koutpol};  \kalamang{kul};  \kalamang{labis};  \kalamang{labor};  \kalamang{manggi};  \kalamang{manman};  \kalamang{momar};  \kalamang{mormor};  \kalamang{nebir};  \kalamang{oioi};  \kalamang{ora};  \kalamang{ospulpul};  \kalamang{paksanual};  \kalamang{pous};  \kalamang{pungpunggat};  \kalamang{rum};  \kalamang{rum timbang}; \kalamang{sek};  \kalamang{seik};  \kalamang{sialar};  \kalamang{sorbir};  \kalamang{suban};  \kalamang{tebolsuban};  \kalamang{telenggues};  \kalamang{torak};  \kalamang{tup};  \kalamang{u};  \kalamang{war};  \kalamang{wariam}  ;  \kalamang{winyal}    
\english{k.o. illness}
  \kalamang{inamurin};  \kalamang{langgulanggur};  \kalamang{naseduk}  
\english{k.o. pandanus}
  \kalamang{kamual}  
\english{k.o. plant}
  \kalamang{gogit};  \kalamang{gos};  \kalamang{kahalongkahalong};  \kalamang{kai kala}; \kalamang{kasuop}; \kalamang{katuk};  \kalamang{pak};  \kalamang{pak mangmang};  \kalamang{rumrum};  \kalamang{somganien};  \kalamang{warwar}  
\english{k.o. prayer}
  \kalamang{salawat}  
\english{k.o. sea cucumber}
  \kalamang{sarsar}  
\english{k.o. shell}
  \kalamang{daria};  \kalamang{poar};  \kalamang{roung}  
\english{k.o. small fish}
  \kalamang{koltengteng}  
\english{k.o. tree}
  \kalamang{jojon};  \kalamang{kabun};  \kalamang{kawalawalan};  \kalamang{langgar};  \kalamang{lusi muaun};  \kalamang{mangmang};  \kalamang{ror buabua}; \kalamang{pirawilak};  \kalamang{somsom}; \kalamang{tibobi}
\english{k.o. triggerfish}
  \kalamang{kawaram boldinggap};  \kalamang{yarkawaram} 
\english{Kaimana}
  \kalamang{Kemana}  
\english{Kaimana.people}
  \kalamang{Mamika}  
\english{Kalamang language}
  \kalamang{Kalamangmang}  
\english{kangaroo}
  \kalamang{tapar}  
\english{kapok tree}
  \kalamang{kapuk}   
\english{Karas Darat}
  \kalamang{Lenggon}  
\english{Karas Island}
  \kalamang{Kalamang lempuang}  
\english{Karas inhabitant; Karas Island}
  \kalamang{Kalamang}  
\english{kawin}
  \kalamang{kasut}  
\english{keel}
  \kalamang{tenaun}  
\english{keep still}
  \kalamang{malelin}  
\english{keep watch}
  \kalamang{jaga}  
\english{Kei}
  \kalamang{Lempang}  
\english{ketjap}
  \kalamang{kecap}  
\english{key}
  \kalamang{koser}  
\english{Kiaba}
  \kalamang{Kiaba}  
\english{kick}
  \kalamang{sempang}  
\english{kidneys}
  \kalamang{kale}  
\english{Kilimala language}
  \kalamang{Kueimang}  
\english{kindle}
  \kalamang{up}  
\english{king}
  \kalamang{leit}  
\english{king post}
  \kalamang{paden tabur}  
\english{kingfisher}
  \kalamang{tol}  
\english{kiss}
  \kalamang{namusi}  
\english{kitchen}
  \kalamang{didiras}  
\english{knee}
  \kalamang{korpak}  
\english{knife}
  \kalamang{tektek}  
\english{know}
  \kalamang{gonggin}   
\english{koran verse}
  \kalamang{yasin}  
\english{Kuhl's stingray}
  \kalamang{isis}  
\english{lake}
  \kalamang{karep}  
\english{lamp}
  \kalamang{don yuolyuol};  \kalamang{lampur}  
\english{land-side}
  \kalamang{keit}  
\english{language; voice}
  \kalamang{mang}  
\english{large intestines}
  \kalamang{kietpak}  
\english{larvae}
  \kalamang{kipkip}  
\english{larynx}
  \kalamang{minar}  
\english{last; hold in place}
  \kalamang{tahan}  
\english{lastborn}
  \kalamang{ma cicaun}
\english{late at night; in the middle of the night}
  \kalamang{saun lat}  
\english{later}
  \kalamang{mena};  \kalamang{sitai}  
\english{laugh}
  \kalamang{rap};  \kalamang{rawarawa}  
\english{laugh at}
  \kalamang{komayeki}  
\english{layer}
  \kalamang{nawot}  
\english{lazy}
  \kalamang{barahala};  \kalamang{monkaret}  
\english{lead}
  \kalamang{narorar};  \kalamang{tabaruop}  
\english{leaf}
  \kalamang{lolok};  \kalamang{*ol};  \kalamang{olun}  
\english{leaf midrib}
  \kalamang{teng}  
\english{lean to side}
  \kalamang{sei}  
\english{lean.on.chin}
  \kalamang{naroman}  
\english{learn}
  \kalamang{belajar}  
\english{leave}
  \kalamang{mamun}  
\english{leech}
  \kalamang{panggawangga}  
\english{left; be lefthanded; left hand; left side}
  \kalamang{tantayuon}  
\english{leftover}
  \kalamang{naharen}  
\english{leftovers}
  \kalamang{pespes}  
\english{leg}
  \kalamang{kor}  
\english{lelah}
  \kalamang{nauware}  
\english{lemongrass}
  \kalamang{lek nabonabon}  
\english{lend; borrow}
  \kalamang{napinjam}  
\english{leopard sea cucumber; teripang}
  \kalamang{sikasika}  
\english{leopard shark}
  \kalamang{lasiambar}  
\english{leopard torpedo}
  \kalamang{musing}  
\english{less}
  \kalamang{kurang}  
\english{lick}
  \kalamang{kobelen}  
\english{lid}
  \kalamang{sanggan};  \kalamang{sangganun}  
\english{lie}
  \kalamang{taouk}  
\english{lie; joke}
  \kalamang{kidi}  
\english{life}
  \kalamang{hidup}  
\english{lift}
  \kalamang{ganggie};  \kalamang{masak}  
\english{lift oneself}
  \kalamang{unganggie}  
\english{light}
  \kalamang{suensik};  \kalamang{sumsik}  
\english{light; shine}
  \kalamang{yuolyuol}  
\english{lightning}
  \kalamang{godelep}  
\english{like}
  \kalamang{nain};  \kalamang{rasa}  
\english{like here}
  \kalamang{owandi}  
\english{like that}
  \kalamang{mendak};  \kalamang{mindi}  
\english{like this}
  \kalamang{wandi};  \kalamang{yuwandi}  
\english{like; want}
  \kalamang{suka}  
\english{lime}
  \kalamang{yawir}  
\english{lime; citrus}
  \kalamang{mun}  
\english{limpet shell}
  \kalamang{panpan}  
\english{line up}
  \kalamang{nabaris}  
\english{lionfish}
  \kalamang{sisiapong}  
\english{lip}
  \kalamang{bolkul}  
\english{little finger, pinky}
  \kalamang{tanggarek}  
\english{live}
  \kalamang{hidup};  \kalamang{tua};  \kalamang{tuaruar}  
\english{living place}
  \kalamang{tuatkur}  
\english{lizard}
  \kalamang{irong};  \kalamang{sebua};  \kalamang{wakpol}  
\english{lizardfish}
  \kalamang{susumandu}  
\english{load}
  \kalamang{nawanen}  
\english{lobster}
  \kalamang{sairarar}  
\english{lock}
  \kalamang{koser}    
\english{loft, attic}
  \kalamang{ser}  
\english{loincloth}
  \kalamang{malor} 
\english{long ago}
  \kalamang{wise}  
\english{longnose parrotfish}
  \kalamang{wienar saruam}  
\english{lontar palm}
  \kalamang{naiar}  
\english{look}
  \kalamang{kome}  
\english{look around}
  \kalamang{guanggarien}  
\english{look for trouble}
  \kalamang{alanganrep}  
\english{loose}
  \kalamang{asaskon};  \kalamang{soul}  
\english{lorikeet}
  \kalamang{keirkeir}  
\english{lost}
  \kalamang{toktok}  
\english{louse}
  \kalamang{mun};  \kalamang{munmun}  
\english{low}
  \kalamang{garos}  
\english{low tide}
  \kalamang{terar kararak};  \kalamang{warkin garos};  \kalamang{warkin kararak};  \kalamang{warkin tararup}  
\english{lukewarm}
  \kalamang{rangrang}  
\english{lungs}
  \kalamang{gawar}  
\english{mace}
  \kalamang{sayang bungaun}  
\english{machete}
  \kalamang{sadawak};  \kalamang{sedawak}  
\english{machine noise}
  \kalamang{nu}  
\english{mackerel}
  \kalamang{tagir}  
\english{magrib}
  \kalamang{magarip}  
\english{maintain}
  \kalamang{keit}  
\english{maize}
  \kalamang{keteles}  
\english{make a hole}
  \kalamang{durma}  
\english{make a sound}
  \kalamang{ar}  
\english{make a stone wall}
  \kalamang{teir}  
\english{make floor}
  \kalamang{kowarara}  
\english{make noise}
  \kalamang{genggalong};  \kalamang{rane}  
\english{make rim}
  \kalamang{kangjie}  
\english{make up}
  \kalamang{naubes}  
\english{make bamboo floor}
  \kalamang{pale}  
\english{make a living}
  \kalamang{mencari}  
\english{make soft}
  \kalamang{rarie}  
\english{make; do}
  \kalamang{nawerar}  
\english{Malakuli}
  \kalamang{Distrik}  
\english{Malay}
  \kalamang{Malai}  
\english{male infant}
  \kalamang{esnemtumun}  
\english{male tree}
  \kalamang{kumkum}  
\english{mama}
  \kalamang{mama}  
\english{man}
  \kalamang{-ca};  \kalamang{-cam};  \kalamang{teya}  
\english{man; male}
  \kalamang{canam}  
\english{mango tree; mango}
  \kalamang{wie}  
\english{mangosteen}
  \kalamang{gain}  
\english{mangrove}
  \kalamang{tanggor}   
\english{manta ray}
  \kalamang{kamel muradik}  
\english{many}
  \kalamang{imbuang}  
\english{marbles}
  \kalamang{mutil}  
\english{mark}
  \kalamang{ut}
\english{mark; recognise}
  \kalamang{naunin}  
\english{mark; scar}
  \kalamang{ter}  
\english{market}
  \kalamang{pasar}  
\english{marriage}
  \kalamang{halar};  \kalamang{nika}  
\english{married}
  \kalamang{kion}  
\english{married (woman)}
  \kalamang{namgon}  
\english{marrow}
  \kalamang{kangun nerunggo}  
\english{marry}
  \kalamang{kion}  
\english{marungga}
  \kalamang{kai modar}  
\english{Mas}
  \kalamang{Sewa}  
\english{massage}
  \kalamang{naladur}  
\english{massoi tree}
  \kalamang{masoi}  
\english{mast}
  \kalamang{peler}  
\english{mat}
  \kalamang{irar};  \kalamang{kalifan}  
\english{mate}
  \kalamang{naukaia} 
\english{may the sun pull out your liver}
  \kalamang{yuon ba mintolmaretkon}  
\english{maybe}
  \kalamang{gen};  \kalamang{reon}  
\english{Mbaham people}
  \kalamang{Samuret}  
\english{measure}
  \kalamang{panggat};  \kalamang{ukir}  
\english{meat}
  \kalamang{dagim}  
\english{medicinal plant}
  \kalamang{kerker};  \kalamang{kulkabok};  \kalamang{sebuaror};  \kalamang{sunak};  \kalamang{taur}  
\english{medicinal.plant}
  \kalamang{tanisa}  
\english{medicine}
  \kalamang{kai; yes}  
\english{medicine man}
  \kalamang{kamanget}  
\english{meet}
  \kalamang{ketemu};  \kalamang{nauleluk};  \kalamang{rapat}  
\english{memory}
  \kalamang{ingatan}  
\english{men}
  \kalamang{esmumur}  
\english{messy}
  \kalamang{purarar}  
\english{midday}
  \kalamang{daruon};  \kalamang{lohar}  
\english{middle}
  \kalamang{raor}  
\english{middle finger}
  \kalamang{tanparok raorkadok}  
\english{mile (sea-mile)}
  \kalamang{mel}  
\english{millipede}
  \kalamang{wak}  
\english{miss}
  \kalamang{kotipol}  
\english{mix}
  \kalamang{campur}  
\english{mole}
  \kalamang{merar}  
\english{Monday}
  \kalamang{senen}  
\english{money}
  \kalamang{lolok};  \kalamang{pitis}  
\english{monitor lizard}
  \kalamang{iwora}  
\english{monkey}
  \kalamang{leki}  
\english{month}
  \kalamang{dilurpak}  
\english{month name}
  \kalamang{hajiwak};  \kalamang{robaherpak}  
\english{moon; month}
  \kalamang{pak}  
\english{moonfish; spadefish}
  \kalamang{pak}  
\english{more}
  \kalamang{=taet}  
\english{more; past; higher}
  \kalamang{lebe}  
\english{morning}
  \kalamang{naupar}  
\english{morning prayer}
  \kalamang{saur}  
\english{mortar}
  \kalamang{rusing}  
\english{mosque}
  \kalamang{masikit}  
\english{mosquito}
  \kalamang{kalkalet}  
\english{mosquito net}
  \kalamang{kolambu}  
\english{mother; aunt; adult woman}
  \kalamang{ema}  
\english{motor}
  \kalamang{motor}  
\english{motor boat}
  \kalamang{jonsong}  
\english{mountain top}
  \kalamang{temgerun}  
\english{mountain; mainland}
  \kalamang{kolak}  
\english{mouth}
  \kalamang{kanggur}  
\english{mouth; rim}
  \kalamang{bol}  
\english{move}
  \kalamang{iar};  \kalamang{napinda}  
\english{move away}
  \kalamang{dikolko}  
\english{move along path; install; become}
  \kalamang{ra}  
\english{move to side}
  \kalamang{maling}  
\english{move towards land}
  \kalamang{mara};  \kalamang{masara}  
\english{move towards sea}
  \kalamang{marua}  
\english{move diagonally up}
  \kalamang{era}  
\english{move; rock}
  \kalamang{teltel}  
\english{much}
  \kalamang{fakurat};  \kalamang{reidak};  \kalamang{rein}  
\english{mucus}
  \kalamang{marur}  
\english{mud}
  \kalamang{paos}  
\english{muezzin}
  \kalamang{mujim}  
\english{mug}
  \kalamang{mok}   
\english{murky sea}
  \kalamang{luam}  
\english{mushroom}
  \kalamang{kawien}  
\english{must}
  \kalamang{harus} 
\english{nail}
  \kalamang{pak};  \kalamang{paku}  
\english{name}
  \kalamang{in}   
\english{narrow}
  \kalamang{kou}  
\english{naughty}
  \kalamang{kelkam}  
\english{navel}
  \kalamang{lim}  
\english{near}
  \kalamang{kokir};  \kalamang{paransik}  
\english{neckbone}
  \kalamang{komanggasir}  
\english{necklace}
  \kalamang{kalung}  
\english{need}
  \kalamang{farlu; parlu}  
\english{needle}
  \kalamang{sin}  
\english{needlefish}
  \kalamang{kamuamual}  
\english{negative existential; empty}
  \kalamang{saerak}  
\english{neighbour; clan, relatives}
  \kalamang{*teit}  
\english{nerite shell}
  \kalamang{os barikbarik};  \kalamang{tabuonsal}  
\english{nest}
  \kalamang{eun}  
\english{net}
  \kalamang{dari}  
\english{new}
  \kalamang{giar}  
\english{New Guinea Rosewood}
  \kalamang{kayu nani}  
\english{new moon}
  \kalamang{pak talawak}  
\english{news}
  \kalamang{nauwar}   
\english{night}
  \kalamang{saun}  
\english{nightjar}
  \kalamang{saramburung}  
\english{nine}
  \kalamang{kaninggonie}  
\english{ninety}
  \kalamang{putkaninggonie}  
\english{ninety-nine}
  \kalamang{putkaninggonie talin kaninggonie}  
\english{nipple}
  \kalamang{am belun}  
\english{nit}
  \kalamang{mun sunsun}  
\english{nod}
  \kalamang{nadou}  
\english{noon}
  \kalamang{yuon nawariri}  
\english{noose}
  \kalamang{dedesi}  
\english{normal}
  \kalamang{biasa}  
\english{north}
  \kalamang{kuawi}  
\english{north-west}
  \kalamang{samar}  
\english{nose}
  \kalamang{bustang}  
\english{nostril}
  \kalamang{bustang posun}  
\english{not good}
  \kalamang{sarouk};  \kalamang{tayuon}  
\english{not good at all}
  \kalamang{siamar}  
\english{not know}
  \kalamang{komahal}  
\english{not much}
  \kalamang{reingge}  
\english{not reach; not enough}
  \kalamang{kokour}  
\english{not yet}
  \kalamang{tok};  \kalamang{tok tok}  
\english{not; no}
  \kalamang{ge}  
\english{nothing}
  \kalamang{ge mera}  
\english{now}
  \kalamang{opatun}  
\english{number}
  \kalamang{anka};  \kalamang{numur}  
\english{nutmeg}
  \kalamang{sayang};  \kalamang{sayang tangun}  
\english{nutmeg fruit}
  \kalamang{sayang naun};  \kalamang{sayang teun}  
\english{nutmeg garden}
  \kalamang{sayangar}  
\english{nypa palm}
  \kalamang{pandoki}  
\english{octopus}
  \kalamang{kurera}  
\english{offering}
  \kalamang{buoksarun};  \kalamang{sinara}  
\english{office}
  \kalamang{gantor}  
\english{oil}
  \kalamang{ming}  
\english{old}
  \kalamang{sawaun};  \kalamang{tuaringgiar}  
\english{old woman}
  \kalamang{emnem}  
\english{old; take long}
  \kalamang{tik}  
\english{older or respected woman}
  \kalamang{enem}  
\english{on coral}
  \kalamang{terarkeit}  
\english{on top of; above}
  \kalamang{kerunggo}  
\english{on/at beach}
  \kalamang{oskeit}  
\english{once}
  \kalamang{wanggon}  
\english{one}
  \kalamang{epkon};  \kalamang{etkon};  \kalamang{kieskon};  \kalamang{kodak};  \kalamang{kon};  \kalamang{mirkon};  \kalamang{naon};  \kalamang{narkon};  \kalamang{poupkon};  \kalamang{rurkon};  \kalamang{taon};  \kalamang{taurkon};  \kalamang{tepkon}  
\english{one hundred}
  \kalamang{reitkon}  
\english{one more}
  \kalamang{kodaet}  
\english{one string}
  \kalamang{alkon}  
\english{one thousend}
  \kalamang{ripion}  
\english{Onin people}
  \kalamang{Patipi};  \kalamang{Rumbati}  
\english{onion}
  \kalamang{lawilawi}  
\english{only}
  \kalamang{hanya}  
\english{open}
  \kalamang{kahetma};  \kalamang{nabuka};  \kalamang{nasiwik};  \kalamang{paherma};  \kalamang{tangguorma};  \kalamang{wirma}  
\english{open limb}
  \kalamang{borma}  
\english{open sea}
  \kalamang{rang}  
\english{opened}
  \kalamang{kasawircie};  \kalamang{pahercie};  \kalamang{tangguorcie}  
\english{opened wide}
  \kalamang{tororo}  
\english{or}
  \kalamang{atau};  \kalamang{ye}  
\english{orange-lined triggerfish}
  \kalamang{kulpanggat}  
\english{orange-spotted trevally}
  \kalamang{unsor}  
\english{orchid}
  \kalamang{kamaser}  
\english{order}
  \kalamang{panok};  \kalamang{sirie}  
\english{order; promise}
  \kalamang{panok}  
\english{orphan}
  \kalamang{tumun miskinden}  
\english{other}
  \kalamang{kabas}  
\english{otherwise}
  \kalamang{mena}  
\english{our betel}
  \kalamang{buokpe}  
\english{out}
  \kalamang{kol}  
\english{outrigger}
  \kalamang{sanggat};  \kalamang{saser}  
\english{outside}
  \kalamang{*kol};  \kalamang{kolko};  \kalamang{*talep};  \kalamang{talepko}  
\english{oven}
  \kalamang{ofin};  \kalamang{siriar}  
\english{over there}
  \kalamang{owa}  
\english{owl}
  \kalamang{kumbai};  \kalamang{parai}  
\english{oyster}
  \kalamang{teir}  
\english{packed food}
  \kalamang{sansan}  
\english{paddle}
  \kalamang{sap};  \kalamang{yal}; \kalamang{yalyal}  
\english{paint}
  \kalamang{cat}  
\english{palala}
  \kalamang{kababa}  
\english{pall-bearing}
  \kalamang{tobutobur}  
\english{palm cockatoo}
  \kalamang{fikfika};  \kalamang{yakop leirun}  
\english{palm oil}
  \kalamang{mingtun}  
\english{pan}
  \kalamang{panci}  
\english{pandanus}
  \kalamang{padamual};  \kalamang{sililar}  
\english{pandanus leaf}
  \kalamang{bunga rampi}  
\english{pants}
  \kalamang{sungsung}  
\english{papaya}
  \kalamang{polkayak}  
\english{parent in law; child in law}
  \kalamang{ketan}  
\english{parrot}
  \kalamang{busbus};  \kalamang{kastupi};  \kalamang{keir};  \kalamang{wororoi}  
\english{parrotfish}
  \kalamang{kuotpol};  \kalamang{wienar}  
\english{part of canoe}
  \kalamang{tar};  \kalamang{torak}  
\english{part of outrigger}
  \kalamang{keibar}  
\english{pass}
  \kalamang{kuang}  
\english{pass on}
  \kalamang{sarakan}  
\english{pass; go on}
  \kalamang{lewat}  
\english{pastry}
  \kalamang{kue};  \kalamang{kukis}  
\english{pay}
  \kalamang{newer}  
\english{payment}
  \kalamang{newer}  
\english{pearl shell}
  \kalamang{mustika}  
\english{peck}
  \kalamang{natuka}  
\english{peel}
  \kalamang{kasotma};  \kalamang{kawaruan};  \kalamang{kawotma};  \kalamang{pes}
\english{peel with knife}
  \kalamang{seser}  
\english{peel wood}
  \kalamang{so}  
\english{penis}
  \kalamang{us}  
\english{people}
  \kalamang{umat}
\english{periwinkle shell}
  \kalamang{goras}  
\english{person}
  \kalamang{-et};  \kalamang{som};  \kalamang{sontum}  
\english{pestle}
  \kalamang{naloli}  
\english{pestle for coconut and kanari nut}
  \kalamang{rusinggain}  
\english{picasso triggerfish}
  \kalamang{uspulpul}  
\english{pick; weave}
  \kalamang{kajie}  
\english{picture}
  \kalamang{gambar}  
\english{pie}
  \kalamang{rontang}  
\english{piece}
  \kalamang{selet};  \kalamang{seletkon}  
\english{pig}
  \kalamang{pep}  
\english{pili nut}
  \kalamang{kanai}  
\english{pimples}
  \kalamang{rasemsem}  
\english{pinch}
  \kalamang{pulma}  
\english{pineapple}
  \kalamang{kainasu}  
\english{place}
  \kalamang{go};  \kalamang{tompat};  \kalamang{wais}  
\english{placeholder}
  \kalamang{neba}  
\english{placeholder for names}
  \kalamang{taur}  
\english{plane}
  \kalamang{desil};  \kalamang{iskap};  \kalamang{namandi};  \kalamang{pesawat}  
\english{planing tool}
  \kalamang{desil}  
\english{plank}
  \kalamang{lat};  \kalamang{pawan}  
\english{plank in boat}
  \kalamang{gading}  
\english{planks roof}
  \kalamang{kahaminpat}  
\english{plant}
  \kalamang{balkawuok};  \kalamang{biawas};  \kalamang{karek ewun saerak};  \kalamang{kies koladok};  \kalamang{koya};  \kalamang{langsa};  \kalamang{tabalaki atan};  \kalamang{tagir polas};  \kalamang{yawarnak};  \kalamang{yies}  
\english{plate}
  \kalamang{pingan}  
\english{play}
  \kalamang{kanggei}  
\english{pluck}
  \kalamang{parua}  
\english{plug}
  \kalamang{ruop};  \kalamang{up}  
\english{plumb rule}
  \kalamang{kalolang}  
\english{point and touch}
  \kalamang{tubak}  
\english{poisonous root used to catch fish}
  \kalamang{tup}  
\english{pole}
  \kalamang{paden};  \kalamang{paden raor}  
\english{poles}
  \kalamang{padewaden}  
\english{police}
  \kalamang{pulisi}  
\english{pond; bay}
  \kalamang{arep}  
\english{porridge}
  \kalamang{bubir} 
\english{pouch}
  \kalamang{lawalawat} 
\english{pour onto}
  \kalamang{konggareor}  
\english{pour; dump; spill}
  \kalamang{gareor}  
\english{prawn type}
  \kalamang{kiel}  
\english{prayer}
  \kalamang{doa};  \kalamang{kelelet}  
\english{prayers}
  \kalamang{malam}  
\english{pregnant}
  \kalamang{kaborko}  
\english{press}
  \kalamang{naram}  
\english{prevent}
  \kalamang{natangkis}  
\english{price}
  \kalamang{pareinun}  
\english{prick}
  \kalamang{natukar}  
\english{prick on horn}
  \kalamang{suor}  
\english{prison}
  \kalamang{lembaga}  
\english{prohibition}
  \kalamang{keraira}  
\english{promise}
  \kalamang{janji}  
\english{prox}
  \kalamang{wane};  \kalamang{yuwane}  
\english{puddle}
  \kalamang{nam}  
\english{pufferfish}
  \kalamang{kabasar}  
\english{pull}
  \kalamang{kasawirma};  \kalamang{naseduk};  \kalamang{sewa}  
\english{pull out}
  \kalamang{darua};  \kalamang{dorma}  
\english{pull with force}
  \kalamang{tadorma}  
\english{pull; drag}
  \kalamang{ramie}  
\english{pulled out}
  \kalamang{dorcie};  \kalamang{tadorcie}  
\english{pulp}
  \kalamang{mesang}  
\english{punch}
  \kalamang{natewa}  
\english{pupil; eyeball}
  \kalamang{kanggirnar}  
\english{pus}
  \kalamang{te}  
\english{push}
  \kalamang{nadorong}  
\english{push; bring}
  \kalamang{deir}  
\english{put}
  \kalamang{maraouk}  
\english{put away}
  \kalamang{lawat}  
\english{put clothespin}
  \kalamang{nagepi}  
\english{put to bed}
  \kalamang{namin}  
\english{put up}
  \kalamang{napasang}  
\english{put up wall}
  \kalamang{paran}  
\english{quail}
  \kalamang{kokok ladok}  
\english{quality}
  \kalamang{kualitek}  
\english{quarrel}
  \kalamang{malawan}  
\english{queen}
  \kalamang{leit pas}  
\english{quick}
  \kalamang{mon}  
\english{quite}
  \kalamang{ko=};  \kalamang{nanaun}  
\english{rabbitfish}
  \kalamang{mal};  \kalamang{maliap};  \kalamang{malkesi}  
\english{rafter}
  \kalamang{gunting};  \kalamang{lepir}  
\english{rag}
  \kalamang{yuon}  
\english{rain}
  \kalamang{kalis};  \kalamang{kalis}  
\english{rainbow}
  \kalamang{kalis tanggir}  
\english{raised platform}
  \kalamang{paror}  
\english{Ramadan month}
  \kalamang{tolaspak}  
\english{rat}
  \kalamang{siwani};  \kalamang{souk}  
\english{rattan}
  \kalamang{sol karek}  
\english{rays}
  \kalamang{serun}  
\english{reach}
  \kalamang{kobes}  
\english{read}
  \kalamang{nabaca}  
\english{ready}
  \kalamang{kalar};  \kalamang{siap}  
\english{receive}
  \kalamang{tarima}  
\english{recite}
  \kalamang{nasibur}  
\english{record}
  \kalamang{kamera}  
\english{record; catch}
  \kalamang{tangkap}  
\english{red}
  \kalamang{kerkap}  
\english{red ant}
  \kalamang{karebar}  
\english{red onion}
  \kalamang{bawang kerkapten}  
\english{reef edge}
  \kalamang{tebol}  
\english{remember}
  \kalamang{konenen}  
\english{repent}
  \kalamang{natobat}  
\english{report}
  \kalamang{lapor}  
\english{rest}
  \kalamang{istirahat};  \kalamang{osie}  
\english{return}
  \kalamang{nawali};  \kalamang{yecie}  
\english{ribs}
  \kalamang{kirkangkang}  
\english{rice}
  \kalamang{pasa}  
\english{rice hull; rice plant}
  \kalamang{padi}  
\english{rice sieve}
  \kalamang{sarun};  \kalamang{uda}  
\english{rice package}
  \kalamang{kowar}  
\english{ridge pole}
  \kalamang{tunggin}  
\english{right}
  \kalamang{to};  \kalamang{yuor}  
\english{right; be righthanded; right hand; right side}
  \kalamang{tanbes}  
\english{ring}
  \kalamang{tanggarara}  
\english{ring finger}
  \kalamang{tanparok penden}  
\english{rinse}
  \kalamang{komurkomur};  \kalamang{nasanggur}  
\english{rinse mouth}
  \kalamang{kanggursau}  
\english{ripe}
  \kalamang{iren}  
\english{rising tide}
  \kalamang{laur}  
\english{ritual}
  \kalamang{koramtolma};  \kalamang{sayerun}  
\english{river bank}
  \kalamang{perbol}  
\english{river; lake}
  \kalamang{kat}  
\english{road}
  \kalamang{borun}  
\english{rock}
  \kalamang{mukmuk};  \kalamang{yarpan}  
\english{rock hole}
  \kalamang{yarpos}  
\english{rock; nod}
  \kalamang{muk}  
\english{roll}
  \kalamang{nabulis};  \kalamang{wam}  
\english{roll pandanus}
  \kalamang{wam}  
\english{roof}
  \kalamang{seng}  
\english{roof wood}
  \kalamang{sal}  
\english{room}
  \kalamang{kalot}  
\english{root}
  \kalamang{*kiel};  \kalamang{kielun}  
\english{root vegetable}
  \kalamang{kalip};  \kalamang{teltel}  
\english{rope}
  \kalamang{karek}  
\english{rotten}
  \kalamang{is};  \kalamang{kap};  \kalamang{mun}  
\english{rough}
  \kalamang{kasar}  
\english{rough sea}
  \kalamang{ur temun}  
\english{rough side leaf}
  \kalamang{kabor elaun}  
\english{round}
  \kalamang{iwang}  
\english{roundhead parrotfish}
  \kalamang{wienar tebolkin}  
\english{roving coral grouper}
  \kalamang{kabaruap kotamtam}  
\english{rub, pulverise}
  \kalamang{nakucak}  
\english{rub; clean}
  \kalamang{yuon}  
\english{rubber tree}
  \kalamang{ror garta}  
\english{rudder; helmsman}
  \kalamang{uli}  
\english{run}
  \kalamang{kararu; kiem}  
\english{run away from}
  \kalamang{kokiem}  
\english{run away with woman}
  \kalamang{girgir}  
\english{run smooth}
  \kalamang{soki}  
\english{run aground}
  \kalamang{narorik};  \kalamang{narur}  
\english{run; sail; swim; cycle}
  \kalamang{tiri}  
\english{rust}
  \kalamang{wenggam}  
\english{sack}
  \kalamang{elkin};  \kalamang{goni}  
\english{sago}
  \kalamang{sanggeran}  
\english{sago flour}
  \kalamang{muap sabur kunun}  
\english{sago grub}
  \kalamang{muap sabur sangganun}  
\english{sago leaf roof}
  \kalamang{muapsabursanong}  
\english{sago palm leaves; palm roof}
  \kalamang{sanong}  
\english{sago pancake}
  \kalamang{singgoli}  
\english{sago tree}
  \kalamang{muap sabur}  
\english{sail}
  \kalamang{kier};  \kalamang{kinggir}  
\english{sail close to the coast}
  \kalamang{nasangginggir}  
\english{sailfish}
  \kalamang{siram}  
\english{salt}
  \kalamang{sira};  \kalamang{sira}  
\english{salty dried fish}
  \kalamang{sor sira}  
\english{same}
  \kalamang{newa};  \kalamang{sama}  
\english{sand}
  \kalamang{os}  
\english{sand mound}
  \kalamang{kiel kierun}  
\english{sap}
  \kalamang{emun}  
\english{sap; latex; gum}
  \kalamang{pol}  
\english{sardine}
  \kalamang{yartep}  
\english{sarong}
  \kalamang{kadok}  
\english{Saturday}
  \kalamang{sabtu}  
\english{saw}
  \kalamang{aragadi}  
\english{say}
  \kalamang{taruo}  
\english{say!}
  \kalamang{taru}  
\english{say; want; think}
  \kalamang{toni}  
\english{scabies}
  \kalamang{kapis}  
\english{scabies; smallpox}
  \kalamang{sanam}  
\english{scale a fish}
  \kalamang{lawuak}  
\english{scar}
  \kalamang{patin ter}  
\english{scare; order}
  \kalamang{okmang}  
\english{scattered; split}
  \kalamang{nasuarik}  
\english{school}
  \kalamang{sekola}  
\english{scrape coconut}
  \kalamang{wat kawaren}  
\english{scrape; feel itchy}
  \kalamang{koyal}  
\english{scratch}
  \kalamang{kaware}  
\english{scream}
  \kalamang{arekmang};  \kalamang{genggueng}  
\english{scrubfowl}
  \kalamang{geries emun}  
\english{sea}
  \kalamang{laut};  \kalamang{pasier};  \kalamang{wilak}  
\english{sea bird}
  \kalamang{kaskas}  
\english{sea cucumber}
  \kalamang{guap};  \kalamang{kariakibi};  \kalamang{kibi};  \kalamang{kibi karek};  \kalamang{masing};  \kalamang{os kibi};  \kalamang{saranggeit};  \kalamang{saranggeit kuskapkap};  \kalamang{saranggeit taraun};  \kalamang{susurofa};  \kalamang{taikongkong};  \kalamang{unapi};  \kalamang{watman};  \kalamang{yuyui}  
\english{sea current}
  \kalamang{paisor}  
\english{sea fern}
  \kalamang{sere kokokteng}  
\english{sea sand}
  \kalamang{bayas}  
\english{sea snake}
  \kalamang{sileng}  
\english{sea urchin}
  \kalamang{tot}  
\english{sea-side}
  \kalamang{ak}    
\english{seam}
  \kalamang{arat}  
\english{search}
  \kalamang{sanggara}  
\english{search fish with light}
  \kalamang{masu}  
\english{season}
  \kalamang{mosun}  
\english{second husband}
  \kalamang{namun caun}  
\english{second wife}
  \kalamang{kieun caun}  
\english{see}
  \kalamang{kome}  
\english{see; look}
  \kalamang{kona}  
\english{seed}
  \kalamang{dowi};  \kalamang{*tang};  \kalamang{tangun}  
\english{seedling}
  \kalamang{iun}  
\english{sejenis kuskus}
  \kalamang{rambu}  
\english{sejenis siput gerai}
  \kalamang{kolkemkem}  
\english{seldom}
  \kalamang{wanggongon}  
\english{sell}
  \kalamang{parein}  
\english{send}
  \kalamang{kama};  \kalamang{kiempanait};  \kalamang{nakirim}  
\english{sense}
  \kalamang{akal}  
\english{Seram}
  \kalamang{Rarait}  
\english{serve}
  \kalamang{nawan}  
\english{seven}
  \kalamang{ramandalin}  
\english{seventy}
  \kalamang{putramandalin}  
\english{sew}
  \kalamang{pat}  
\english{sew leaves}
  \kalamang{gaim}  
\english{shade}
  \kalamang{kiek}  
\english{shadow}
  \kalamang{git};  \kalamang{kiekter}  
\english{shake}
  \kalamang{tun}  
\english{shark}
  \kalamang{ruar};  \kalamang{ruar bodaren};  \kalamang{ruar tagirigiri};  \kalamang{war}  
\english{sharp}
  \kalamang{belbel};  \kalamang{kang}  
\english{sharp rock}
  \kalamang{yar kangkang}  
\english{sharpen}
  \kalamang{sie}  
\english{shave; scrape}
  \kalamang{sarua}  
\english{shearwater}
  \kalamang{kornambi}  
\english{shelf}
  \kalamang{rak}  
\english{shell}
  \kalamang{daladala};  \kalamang{dokadoka};  \kalamang{kalsum};  \kalamang{kerker};  \kalamang{ko};  \kalamang{mata bulang};  \kalamang{rerer};  \kalamang{suk};  \kalamang{tarapa};  \kalamang{tel};  \kalamang{torpes};  \kalamang{weswes}  
\english{shin}
  \kalamang{korus}  
\english{ship}
  \kalamang{kapal}  
\english{shirt}
  \kalamang{kabai}  
\english{shiver}
  \kalamang{nabobar}  
\english{shoal}
  \kalamang{sarit};  \kalamang{usiep}  
\english{shoe}
  \kalamang{sepatu}  
\english{shoot}
  \kalamang{karop};  \kalamang{saroum};  \kalamang{saroum}  
\english{shoot with gun}
  \kalamang{sair}  
\english{shore}
  \kalamang{pasierbol}  
\english{shore birds with long feet}
  \kalamang{sikekan}  
\english{shore, land, inland}
  \kalamang{kibis}  
\english{shore.current}
  \kalamang{walalom}  
\english{short}
  \kalamang{tabusik}  
\english{short of breath}
  \kalamang{asokmang}  
\english{shoulder}
  \kalamang{bekiem}  
\english{shoulder blade}
  \kalamang{bekiemkang}  
\english{shout}
  \kalamang{narabir}  
\english{shove}
  \kalamang{masoki}  
\english{shovel}
  \kalamang{eskop}  
\english{show}
  \kalamang{balaok};  \kalamang{naunak};  \kalamang{nawarik}  
\english{shrimp}
  \kalamang{kokada}  
\english{sibling}
  \kalamang{*dun};  \kalamang{*kia}  
\english{sibling-in-law}
  \kalamang{dauk}  
\english{siblings}
  \kalamang{naukia};  \kalamang{naukiaka}  
\english{sick}
  \kalamang{kaden lalang};  \kalamang{luam}  
\english{side}
  \kalamang{-dok};  \kalamang{kirarun};  \kalamang{*mul};  \kalamang{mulun};  \kalamang{-pis};  \kalamang{tai-};  \kalamang{tair}  
\english{side; kidneys}
  \kalamang{kir}  
\english{side; part}
  \kalamang{-kadok}  
\english{sideburns}
  \kalamang{sowil}  
\english{sieve}
  \kalamang{kurera};  \kalamang{teteris}  
\english{sign}
  \kalamang{natanda};  \kalamang{natekin};  \kalamang{naunin}  
\english{signal goby}
  \kalamang{siabor}  
\english{sing}
  \kalamang{menyanyi};  \kalamang{mirik};  \kalamang{nanggan}  
\english{sink}
  \kalamang{dare}  
\english{sinker}
  \kalamang{lot}  
\english{sit}
  \kalamang{meleluo}  
\english{sit and do nothing}
  \kalamang{doka}  
\english{six}
  \kalamang{raman}  
\english{sixty}
  \kalamang{putraman}  
\english{size}
  \kalamang{ukuran}  
\english{skewer}
  \kalamang{kotam};  \kalamang{rur}  
\english{skewer; stab; fit}
  \kalamang{komain}  
\english{skin}
  \kalamang{kulun}  
\english{skin dirt}
  \kalamang{lamut}  
\english{skinny}
  \kalamang{karaonggis}  
\english{sky}
  \kalamang{kisileng}  
\english{slacken}
  \kalamang{naluar}  
\english{slap with hand}
  \kalamang{nafafat}  
\english{slave}
  \kalamang{ke}  
\english{sleep}
  \kalamang{min}  
\english{slice}
  \kalamang{korot};  \kalamang{marum};  \kalamang{narari};  \kalamang{polas};  \kalamang{polas}  
\english{slide}
  \kalamang{sou}  
\english{slimy}
  \kalamang{kanggarom}  
\english{slippers}
  \kalamang{sandal}  
\english{slippery}
  \kalamang{palawak}  
\english{slippery; smooth}
  \kalamang{licing}  
\english{slow}
  \kalamang{siktak};  \kalamang{siktaktak}  
\english{small}
  \kalamang{caun};  \kalamang{cicaun};  \kalamang{kinkin};  \kalamang{kinkinun};  \kalamang{tabaktabak};  \kalamang{tumun}  
\english{small bamboo type}
  \kalamang{lawan}  
\english{small chainsaw}
  \kalamang{sensur caun}  
\english{small child}
  \kalamang{tumun caun}  
\english{small clam; sea snail}
  \kalamang{tabuon}  
\english{small loin cloth}
  \kalamang{kewa}  
\english{small motor}
  \kalamang{pokpok}  
\english{small one}
  \kalamang{cicaun}  
\english{small unripe fruit}
  \kalamang{kaburun}  
\english{small plug}
  \kalamang{wandiwandi}  
\english{smaller birds of prey}
  \kalamang{tanggal}  
\english{smell}
  \kalamang{gawar};  \kalamang{lauk}  
\english{smoke}
  \kalamang{diguar}  
\english{smooth side leaf}
  \kalamang{suolkerun}  
\english{snail}
  \kalamang{nunggununggu};  \kalamang{siput babi};  \kalamang{tabili};  \kalamang{yuot}  
\english{snake}
  \kalamang{kip}  
\english{snapped}
  \kalamang{kawarcie}  
\english{snapper}
  \kalamang{tabalam}  
\english{sneeze}
  \kalamang{sik}  
\english{snore}
  \kalamang{minggaruk}  
\english{so}
  \kalamang{jadi};  \kalamang{=tauna};  \kalamang{=tenden}  
\english{so that}
  \kalamang{eba};  \kalamang{supaya}  
\english{soap}
  \kalamang{sabur}  
\english{soft}
  \kalamang{kalawen}  
\english{soft coral}
  \kalamang{lam}  
\english{soft sound}
  \kalamang{sarakmang}  
\english{soft; fine}
  \kalamang{halus}  
\english{soil}
  \kalamang{naun}  
\english{some}
  \kalamang{ikon};  \kalamang{taukon};  \kalamang{utkon}  
\english{son}
  \kalamang{tumun canam}  
\english{song}
  \kalamang{mirik};  \kalamang{nyanyi}  
\english{soon}
  \kalamang{tokta me}  
\english{sorceress}
  \kalamang{warpas}  
\english{sorcery}
  \kalamang{war}  
\english{sore}
  \kalamang{patin}  
\english{Sorong}
  \kalamang{Sorung}  
\english{sound}
  \kalamang{ar};  \kalamang{nun}  
\english{sour}
  \kalamang{mor};  \kalamang{mor}  
\english{soursop}
  \kalamang{duran walanda}  
\english{south}
  \kalamang{tarangin}  
\english{space under a house}
  \kalamang{sabarak}  
\english{spatula}
  \kalamang{pesawesa}  
\english{speak}
  \kalamang{ewa};  \kalamang{ewawa}  
\english{speak Kalamang}
  \kalamang{Kalamangmang}  
\english{spear}
  \kalamang{gala}  
\english{spear with one point}
  \kalamang{posiwosi}  
\english{spice}
  \kalamang{daun salam};  \kalamang{kai manis}  
\english{spider}
  \kalamang{pueselet}  
\english{spider conch}
  \kalamang{panggal}  
\english{spinach}
  \kalamang{bayam}  
\english{spine}
  \kalamang{suolkasir}  
\english{spine fish}
  \kalamang{tuangga}  
\english{spiral coral}
  \kalamang{paisor kesun}  
\english{spirit}
  \kalamang{arwa}  
\english{spit}
  \kalamang{palom};  \kalamang{palom}  
\english{spit at}
  \kalamang{koalom}  
\english{spit out}
  \kalamang{maouk}  
\english{split; break}
  \kalamang{parair}  
\english{spoiled}
  \kalamang{mais}  
\english{spoon}
  \kalamang{sasul};  \kalamang{sasul}  
\english{spread}
  \kalamang{sangganggam}  
\english{spread legs}
  \kalamang{tagarar}  
\english{sprinkle}
  \kalamang{kowarwak}  
\english{squash}
  \kalamang{kasabiti}  
\english{squeeze}
  \kalamang{naramas}  
\english{squid}
  \kalamang{konyak}  
\english{squint}
  \kalamang{nasawawi}  
\english{squirrelfish, soldierfish, cardinalfish}
  \kalamang{sasep}  
\english{stab}
  \kalamang{konamin}  
\english{stack}
  \kalamang{nasusun}  
\english{stairs}
  \kalamang{but}  
\english{stalk}
  \kalamang{*gor};  \kalamang{gorun};  \kalamang{wierun};  \kalamang{wiet}  
\english{stand}
  \kalamang{mambara}  
\english{stand up}
  \kalamang{usar}  
\english{star}
  \kalamang{maser}  
\english{starfish}
  \kalamang{maser};  \kalamang{warkasom}  
\english{starfruit}
  \kalamang{nambiain}  
\english{startle}
  \kalamang{kotarakmang}  
\english{startled}
  \kalamang{tarakmang}  
\english{stay}
  \kalamang{gocie}  
\english{steal}
  \kalamang{eksuet};  \kalamang{kuek}  
\english{steam}
  \kalamang{nakukus}  
\english{stem}
  \kalamang{*ar};  \kalamang{arun};  \kalamang{les}  
\english{step}
  \kalamang{panggat}  
\english{step on}
  \kalamang{teitei}  
\english{stick}
  \kalamang{sap}  
\english{stick onto}
  \kalamang{konawol};  \kalamang{nawol}  
\english{sticky}
  \kalamang{kopol}  
\english{stiff muscles}
  \kalamang{nakabung}  
\english{sting}
  \kalamang{sie}  
\english{stingray}
  \kalamang{kamel}  
\english{stir-fry}
  \kalamang{natumis}  
\english{stomach}
  \kalamang{kabor}  
\english{stomach illness}
  \kalamang{kabornar}  
\english{stomach worm}
  \kalamang{yes}  
\english{stone}
  \kalamang{yar}  
\english{stone axe}
  \kalamang{lemyar}  
\english{stone wall}
  \kalamang{yatal}  
\english{stone hole}
  \kalamang{porkang}  
\english{stop}
  \kalamang{istop};  \kalamang{sansan}  
\english{stop; stay}
  \kalamang{telin}  
\english{store; bury}
  \kalamang{mecua}  
\english{story}
  \kalamang{cerita};  \kalamang{rer}  
\english{stove}
  \kalamang{kamfor}  
\english{straight}
  \kalamang{yuorsik}  
\english{strand}
  \kalamang{sare}  
\english{stranded}
  \kalamang{newas}  
\english{stranger}
  \kalamang{kolet};  \kalamang{somkabas};  \kalamang{sontumkabas}  
\english{street}
  \kalamang{istrat};  \kalamang{urap}  
\english{stretch out}
  \kalamang{soso}  
\english{stretch (out)}
  \kalamang{madong}  
\english{string}
  \kalamang{*al};  \kalamang{kangkanggarek}  
\english{string.type}
  \kalamang{sawawien}  
\english{stripe}
  \kalamang{istrep};  \kalamang{leis}  
\english{striped eel catfish}
  \kalamang{kadam}  
\english{strong}
  \kalamang{kuat}  
\english{struggle}
  \kalamang{meresawuo}  
\english{stupid}
  \kalamang{boda}  
\english{suck; smoke}
  \kalamang{kosom}  
\english{suddenly move; sudden sound}
  \kalamang{urukmang}  
\english{sugar}
  \kalamang{don penpen};  \kalamang{nasuena}  
\english{sugar palm}
  \kalamang{cok}  
\english{sugar; white cloth}
  \kalamang{don iriskap}  
\english{summit}
  \kalamang{pang}  
\english{sun}
  \kalamang{yuon}  
\english{sunday}
  \kalamang{ahat}  
\english{Sunday}
  \kalamang{hari minggu}  
\english{sunrise}
  \kalamang{yuon sara}  
\english{sunset}
  \kalamang{yuon daruk}  
\english{surgeonfish}
  \kalamang{karariem};  \kalamang{tararar}  
\english{swallow}
  \kalamang{dareok}  
\english{swear}
  \kalamang{malu}  
\english{sweat}
  \kalamang{ruam}  
\english{sweep}
  \kalamang{gokabara};  \kalamang{kabara}  
\english{swim}
  \kalamang{yie}  
\english{swollen}
  \kalamang{panggala};  \kalamang{ruan}  
\english{t-shirt}
  \kalamang{kous}  
\english{table}
  \kalamang{meja}  
\english{table coral}
  \kalamang{ram tomtom}  
\english{taboo; bad luck; offering}
  \kalamang{saier}  
\english{tail}
  \kalamang{orun}  
\english{take care of}
  \kalamang{cam}  
\english{take out}
  \kalamang{kolo}  
\english{take from hot}
  \kalamang{tawie}  
\english{talk together}
  \kalamang{garung}  
\english{tall}
  \kalamang{ririn}  
\english{tamarind}
  \kalamang{tabalaki}  
\english{Tana Besar}
  \kalamang{Gowien}  
\english{tangled}
  \kalamang{sok}  
\english{Tarak}
  \kalamang{Torkuran}  
\english{taro}
  \kalamang{manadu}  
\english{tarpaulin}
  \kalamang{farlak}  
\english{taste}
  \kalamang{narasa};  \kalamang{narasaun}  
\english{tasty; sweet}
  \kalamang{pen}  
\english{tawny nurse shark}
  \kalamang{ruar kanggir nungnung} 
\english{tea}
  \kalamang{per kerkap};  \kalamang{ter}  
\english{teach}
  \kalamang{ajar}  
\english{teacher}
  \kalamang{guru}  
\english{tear}
  \kalamang{masarut}  
\english{tears}
  \kalamang{pertam}  
\english{teeth}
  \kalamang{gier}  
\english{telephone}
  \kalamang{telpon}  
\english{tell}
  \kalamang{cerita};  \kalamang{nacerita}  
\english{Teluk Buruwai}
  \kalamang{Uninsinei}  
\english{ten}
  \kalamang{putkon}  
\english{ten thousand}
  \kalamang{salak}  
\english{tenggelele}
  \kalamang{tenggelele}  
\english{tens of}
  \kalamang{*put}  
\english{tent}
  \kalamang{garumbang}  
\english{terrace}
  \kalamang{istup}  
\english{testicles}
  \kalamang{elkin narun}  
\english{that have gone bad}
  \kalamang{tenenun}  
\english{that's it}
  \kalamang{ma he me}  
\english{The Netherlands}
  \kalamang{Beladar}  
\english{the small one(s)}
  \kalamang{kinkinun}  
\english{then}
  \kalamang{baru};  \kalamang{eba};  \kalamang{koi};  \kalamang{mera};  \kalamang{siktak};  \kalamang{terus}  
\english{there is}
  \kalamang{mambon}  
\english{thermos}
  \kalamang{termus}  
\english{thick}
  \kalamang{mawal}  
\english{thief}
  \kalamang{eksuet}  
\english{thigh}
  \kalamang{kolkiem}  
\english{thin}
  \kalamang{samsik}  
\english{thin and flat thing}
  \kalamang{*tak};  \kalamang{taun}  
\english{thing}
  \kalamang{don}  
\english{things; clothes}
  \kalamang{dodon}  
\english{think}
  \kalamang{kona};  \kalamang{nafikir}  
\english{thirteen}
  \kalamang{putkon ba karuok}  
\english{thirty}
  \kalamang{putkaruok}  
\english{thirty-four}
  \kalamang{putkaruok talinggansuor}  
\english{thirty-one}
  \kalamang{putkaruok talinggon}  
\english{thirty-three}
  \kalamang{putkaruok talinggaruok}  
\english{thirty-two}
  \kalamang{putkaruok talinir}  
\english{that}
  \kalamang{me}
\english{this}
  \kalamang{wa}
\english{thorn}
  \kalamang{*kang};  \kalamang{kangun}  
\english{thorns}
  \kalamang{kangkangun}  
\english{thoughts}
  \kalamang{pikiran}  
\english{thousand}
  \kalamang{ripi}  
\english{thread}
  \kalamang{kawas}  
\english{three}
  \kalamang{karuok};  \kalamang{tiga}  
\english{thrifty}
  \kalamang{nahimat}  
\english{throat}
  \kalamang{min}  
\english{throat and neck}
  \kalamang{komang}  
\english{throw}
  \kalamang{mudi};  \kalamang{muk};  \kalamang{walawala}  
\english{throw aside; throw away; drop}
  \kalamang{paruak}  
\english{thumb}
  \kalamang{tanparoemun}  
\english{thunder}
  \kalamang{godarung}  
\english{Thursday}
  \kalamang{kamis}  
\english{tide}
  \kalamang{warkin}  
\english{tidy; balance; clean wood}
  \kalamang{nauanona}  
\english{tie}
  \kalamang{kanie}  
\english{tie a basket}
  \kalamang{sun}  
\english{tied too tight}
  \kalamang{masok}  
\english{tight}
  \kalamang{rapat}  
\english{tilefish}
  \kalamang{bintulak}  
\english{time}
  \kalamang{oras};  \kalamang{waktu};  \kalamang{*wan}  
\english{to plug}
  \kalamang{narer}  
\english{to pour}
  \kalamang{nasirang}  
\english{tobacco type}
  \kalamang{sektabai}  
\english{tobacco; cigarette}
  \kalamang{tabai}  
\english{today}
  \kalamang{opa yuwa}  
\english{toenails}
  \kalamang{kortanggalip}  
\english{toes}
  \kalamang{korparokparok} 
\english{tomato}
  \kalamang{tamatil}  
\english{tomorrow}
  \kalamang{kasur}   
\english{tongs}
  \kalamang{kowaram}  
\english{tongue}
  \kalamang{belen}  
\english{too}
  \kalamang{-nan};  \kalamang{=sawe};  \kalamang{weinun}  
\english{too heavy}
  \kalamang{nares}  
\english{too much}
  \kalamang{reidaksawe}  
\english{too tight}
  \kalamang{langgour}  
\english{too; any; even}
  \kalamang{=barak}  
\english{tool}
  \kalamang{linggis}  
\english{top}
  \kalamang{keirun};  \kalamang{*keit};  \kalamang{keitko}  
\english{top shell}
  \kalamang{wel}  
\english{torn}
  \kalamang{sarusarut}  
\english{torresian imperial pidgeon}
  \kalamang{tagurewun}  
\english{tortoise}
  \kalamang{kanung};  \kalamang{sawarer}  
\english{towel}
  \kalamang{handuk}  
\english{tradition}
  \kalamang{adat}  
\english{traditional dance}
  \kalamang{nasula}  
\english{trash}
  \kalamang{warum}  
\english{tray}
  \kalamang{talam}  
\english{treat}
  \kalamang{kamang}  
\english{tree}
  \kalamang{cam};  \kalamang{damar lelak};  \kalamang{girawar};  \kalamang{kasor};  \kalamang{semerlak};  \kalamang{watwat};  \kalamang{wol}  
\english{tree fern}
  \kalamang{iwala}  
\english{tree kangaroo}
  \kalamang{taer}  
\english{tree stem}
  \kalamang{*tem}  
\english{tree stump}
  \kalamang{ror tabur}  
\english{tree trunk; base}
  \kalamang{ewun}  
\english{tree; wood}
  \kalamang{ror}  
\english{triggerfish}
  \kalamang{kawaram};  \kalamang{oronggos}  
\english{true}
  \kalamang{haidak};  \kalamang{saidak};  \kalamang{yuor}  
\english{true conch}
  \kalamang{rer};  \kalamang{sengseng}  
\english{try}
  \kalamang{coba};  \kalamang{nacoba}  
\english{Tuburuasa}
  \kalamang{Tuburasap}  
\english{tuck in}
  \kalamang{nasuat}  
\english{Tuesday}
  \kalamang{selasa}  
\english{tuna}
  \kalamang{suagi}  
\english{turban shell}
  \kalamang{ang}  
\english{turmeric}
  \kalamang{barang}  
\english{turn}
  \kalamang{barotma};  \kalamang{roye}  
\english{turn around; circle; play music; wander}
  \kalamang{naurar}  
\english{turn back to}
  \kalamang{kosilep}  
\english{turn over}
  \kalamang{malaouk}  
\english{turtle}
  \kalamang{kerar};  \kalamang{tabom}  
\english{tusk}
  \kalamang{gelemun}  
\english{tusk shell}
  \kalamang{kasom}  
\english{twelve}
  \kalamang{putkon ba eir}  
\english{twenty}
  \kalamang{purir}  
\english{twenty-one}
  \kalamang{purir ba kon}  
\english{twice}
  \kalamang{wanir}  
\english{twig broom}
  \kalamang{sirarai}  
\english{two}
  \kalamang{eir};  \kalamang{-ier}  
\english{two-pointed spear}
  \kalamang{kasalong}  
\english{umbilical ovula}
  \kalamang{kaituki}  
\english{unable to do}
  \kalamang{alangan}  
\english{uncle}
  \kalamang{esa caun};  \kalamang{esa temun};  \kalamang{mama};  \kalamang{mama caun};  \kalamang{mama temun}  
\english{under}
  \kalamang{elao}  
\english{under side foot}
  \kalamang{korlaus}  
\english{understand}
  \kalamang{mengerti}  
\english{unload}
  \kalamang{nawaruok}  
\english{unprocessed wood}
  \kalamang{ror soren}  
\english{unripe}
  \kalamang{kalomun};  \kalamang{kangguar}  
\english{unstuck; be open}
  \kalamang{wiercie}  
\english{until}
  \kalamang{bo};  \kalamang{sampai};  \kalamang{sampi}  
\english{untroubled}
  \kalamang{komisternin}  
\english{up}
  \kalamang{osa}  
\english{up there}
  \kalamang{osatko}  
\english{urinate}
  \kalamang{ulur}  
\english{urine}
  \kalamang{ul}  
\english{use prayer water}
  \kalamang{natawar}  
\english{use plumb rule}
  \kalamang{kalolang}  
\english{use sorcery}
  \kalamang{war}  
\english{use; wear}
  \kalamang{napaki}  
\english{vagina}
  \kalamang{kar}  
\english{vase}
  \kalamang{gusi}  
\english{vase shell}
  \kalamang{teltel}  
\english{vegetable}
  \kalamang{kanggin};  \kalamang{kangginwele}  
\english{vegetables}
  \kalamang{wele}  
\english{vein}
  \kalamang{kaden kies}  
\english{veins}
  \kalamang{kaden kieskies}  
\english{venus clam}
  \kalamang{paer}  
\english{very}
  \kalamang{-kaning};  \kalamang{=tun}  
\english{very far}
  \kalamang{kahahen}  
\english{very hungry; many hungry people}
  \kalamang{muawesese}  
\english{very much}
  \kalamang{bareireimun}  
\english{very white}
  \kalamang{iriskapkap}  
\english{very young}
  \kalamang{kalomlomun}  
\english{very edge}
  \kalamang{siepsieun}  
\english{village building}
  \kalamang{gedung}  
\english{village; place}
  \kalamang{leng}  
\english{voice}
  \kalamang{suara}  
\english{vomit}
  \kalamang{emguk};  \kalamang{emguk}  
\english{wagtail}
  \kalamang{sika polipoli}  
\english{wagtails}
  \kalamang{tatapang}  
\english{waist}
  \kalamang{muler}  
\english{wait}
  \kalamang{natunggu};  \kalamang{nawanggar}  
\english{wake someone up}
  \kalamang{nawarar}  
\english{wake up}
  \kalamang{parar}  
\english{walk}
  \kalamang{korgi marmar};  \kalamang{marmar}  
\english{walk with big steps}
  \kalamang{panggat}  
\english{wall}
  \kalamang{goparar}  
\english{want}
  \kalamang{mau}  
\english{wash face}
  \kalamang{kanggisawuo}  
\english{wash; bathe}
  \kalamang{waruo}  
\english{washingtub}
  \kalamang{pang}  
\english{washtub}
  \kalamang{bintang}  
\english{wasp}
  \kalamang{kier}  
\english{wasp nest; beehive}
  \kalamang{gowienkier}  
\english{waspfish}
  \kalamang{kambanau}  
\english{watch}
  \kalamang{kabarua};  \kalamang{loncing}  
\english{water}
  \kalamang{per}  
\english{water between roots}
  \kalamang{pearip}  
\english{water container}
  \kalamang{tarapa}  
\english{waterfall}
  \kalamang{perki}  
\english{watermelon}
  \kalamang{tumin}  
\english{wave}
  \kalamang{uren}  
\english{wavy}
  \kalamang{ureren}  
\english{weak}
  \kalamang{kememe}  
\english{weak as a result of not eating}
  \kalamang{kaloum}  
\english{wear}
  \kalamang{korko}  
\english{wear; dress}
  \kalamang{sabur}  
\english{weave}
  \kalamang{nadadi}  
\english{weave up}
  \kalamang{kowam}  
\english{wedding rite}
  \kalamang{welawela}  
\english{Wednesday}
  \kalamang{roba}  
\english{weed}
  \kalamang{masir}  
\english{week; Sunday}
  \kalamang{minggu}  
\english{weight watcher}
  \kalamang{kibibal}  
\english{well}
  \kalamang{nabestai}  
\english{west}
  \kalamang{daru};  \kalamang{kemanur};  \kalamang{kemanurep};  \kalamang{munin}  
\english{west-wind season}
  \kalamang{kemanurpak}  
\english{wet}
  \kalamang{kamen}  
\english{whale}
  \kalamang{kowam}  
\english{whale shark}
  \kalamang{war pasierkip}  
\english{what}
  \kalamang{ha};  \kalamang{neba}  
\english{what are you doing}
  \kalamang{nebara paruo}  
\english{wheat flour}
  \kalamang{tapong}  
\english{when}
  \kalamang{waktu};  \kalamang{yuol tama}  
\english{where}
  \kalamang{tamatko} 
\english{where from}
  \kalamang{tamangga}
\english{where to}
  \kalamang{tamawis; tamangga}  
\english{which one}
  \kalamang{kon tama}  
\english{whisper}
  \kalamang{noknok}  
\english{whistle}
  \kalamang{filoit};  \kalamang{wuong}  
\english{whistle-call; message}
  \kalamang{kodi}  
\english{white}
  \kalamang{iriskap}  
\english{white gum}
  \kalamang{lerang}  
\english{white person}
  \kalamang{guarten}  
\english{who}
  \kalamang{naman}  
\english{why; what}
  \kalamang{nenggap}  
\english{wide}
  \kalamang{laus}  
\english{widow}
  \kalamang{paskot}  
\english{widower}
  \kalamang{kamun}  
\english{widow(er)}
  \kalamang{sian}  
\english{wife}
  \kalamang{*kiar}  
\english{wild ginger}
  \kalamang{sasarem}  
\english{wild sugarcane}
  \kalamang{kalap};  \kalamang{kusukusu}  
\english{wild/forest breadfruit}
  \kalamang{katawengga}  
\english{wind}
  \kalamang{ur};  \kalamang{yaban}  
\english{window}
  \kalamang{jendela}  
\english{window frame}
  \kalamang{kosin}  
\english{wing}
  \kalamang{pat};  \kalamang{pul}  
\english{wing; fin}
  \kalamang{parun}  
\english{witch, sorcerer}
  \kalamang{sontum warten}  
\english{with a hole in it}
  \kalamang{durcie}  
\english{with that}
  \kalamang{minggi}  
\english{woman}
  \kalamang{pebis}  
\english{womb}
  \kalamang{lawarun}  
\english{women}
  \kalamang{emumur}  
\english{wood}
  \kalamang{langgan}  
\english{wood tool}
  \kalamang{panggut}  
\english{wood without bark}
  \kalamang{so}  
\english{wooden canoe}
  \kalamang{lepalepa}  
\english{woodswallow}
  \kalamang{saria}  
\english{work}
  \kalamang{karajang} 
\english{worm}
  \kalamang{iban};  \kalamang{subuman}  
\english{wounded}
  \kalamang{patin}  
\english{wrap}
  \kalamang{karap}; \kalamang{kies};  \kalamang{kokies}  
\english{wrap in cloth}
  \kalamang{kafan};  \kalamang{nakafan}  
\english{wrasse}
  \kalamang{osmarera}  
\english{wring}
  \kalamang{golma}  
\english{wrist; finger joints}
  \kalamang{tan kasir}  
\english{write}
  \kalamang{natulis}  
\english{wrong}
  \kalamang{sala}  
\english{yam}
  \kalamang{yap seran}  
\english{Yarpos Kon}
  \kalamang{Yarpos Kon}  
\english{yawn}
  \kalamang{gelem}  
\english{year}
  \kalamang{tanggon}  
\english{yellow}
  \kalamang{baranggap}  
\english{yellow taro}
  \kalamang{pasiem}  
\english{yes}
  \kalamang{a'a};  \kalamang{esie};  \kalamang{ya};  \kalamang{yo}  
\english{yesterday}
  \kalamang{wis}  
\english{yet; still; first}
  \kalamang{tok} 
\english{you know}
  \kalamang{kan; to}  
\english{young coconut}
  \kalamang{wat kabur}  

  \onecolumn
  \egroup


\chead{}

\chapter{Maps}\is{place names}\is{toponyms|see{place names}}
\label{ch:maps}
Map~\ref{fig:karasnarrnames} displays those names mentioned in the traditional narratives (§\ref{sec:narr}). Map~\ref{fig:karasnames} shows all recorded coastal names of the big Karas island, and a few additional place names. For a bigger version, zoom in on the digital version of this page, or download the map from the Kalamang archive.\footnote{At \url{http://hdl.handle.net/10050/00-0000-0000-0004-1BEC-7}.} 

\begin{figure}[ht]  
	\centering
	\includegraphics[width=\textwidth]{Images/Karas_Islands_w_names_narratives.pdf}
	\caption[Place names of Karas mentioned in traditional narratives]{Place names of Karas mentioned in traditional narratives}\label{fig:karasnarrnames}	
\end{figure}


\begin{figure}[!ht]  
	\centering
	\includegraphics[angle=90, width=\textwidth]{Images/Karas_Islands_w_names.pdf}
	\caption[Coastal place names of Karas]{Coastal place names of Karas}\label{fig:karasnames}
\end{figure}
