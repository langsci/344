\chapter{Multiclausal constructions}
\label{ch:compl}
Chapter~\ref{ch:clause} dealt with the structure of the simple Kalamang clause. This chapter explores how such clauses are combined into multiclausal constructions. §\ref{sec:clacomb} describes different ways to combine main clauses in cooordination-like ways, followed by subordinate or subordinate-like clause constructions: §\ref{sec:complclause} on complement clauses, §\ref{sec:appreh} on apprehensive clauses and §\ref{sec:condclause} on conditional clauses.

\section{Coordination-like clause combining}\is{clause combining|(}
\label{sec:clacomb}
In this section, the four ways of combining main clauses are described: asyndetic conjunction is addressed in §\ref{sec:asyndetic}, conjunctions in §\ref{sec:clauseconj}, tail-head linkage in §\ref{sec:tailhead} and non-final clauses in §\ref{sec:nfin}.

\subsection{Asyndetic conjunction}\is{conjuction!asyndetic}
\label{sec:asyndetic}
Asyndetic conjunction is the juxtaposition\is{juxtaposition} of two clauses without an overt coordination. This is the most common way to conjoin clauses, and is not connected to a special function. The relation between asyndetically conjoined clauses is rather expressed through intonation\is{intonation}. If the first clause ends in a high boundary tone (H\%), this indicates that it is a non-final clause (§\ref{sec:nonfinal}), which combined with asyndetic conjunction indicates a sequential relationship between the clauses. This is illustrated in Figure~\ref{fig:kurubotatko}, where the speaker relates a sequence of events from the day before. While the second and third clause contain words that indicate we are dealing with sequential events (\textit{koyet} `finished', see §\ref{sec:compl}, and \textit{koi} `then; again', see §\ref{sec:seqconj}), the first clause is truly asyndetically conjoined to the second. It remains unclear what the difference is between asyndetically conjoined clauses and clauses conjoined with sequential conjunctions (§\ref{sec:seqconj}).\is{apposition}\is{juxtaposition}\is{asyndetic conjunction}\is{clause combining}\is{prosody!clause-final}

\begin{figure}[p]
	\includegraphics[width=.95\textwidth]{Images/kurubotatko}
	\caption[Asyndetic conjunction of sequential events with a high boundary tone]{Asyndetic conjunction of sequential events with a high boundary tone}\label{fig:kurubotatko}
\end{figure}

A low boundary tone on the first clause (L\%) indicates a declarative clause. The following clause typically starts a new paragraph in a story. There may or may not be a sequential relationship between the clauses. In the narrative section in Figure~\ref{fig:anseparar}, which precedes the section in Figure~\ref{fig:kurubotatko} in the same narrative, there is a sequential relationship between the juxtaposed clauses. A new paragraph in the story is started: the events of a new day are related.

\begin{figure}[p]
	\includegraphics[width=.95\textwidth]{Images/anseparar}
	\caption[Asyndetic conjunction of sequential events with a low boundary tone]{Asyndetic conjunction of sequential events with a low boundary tone}\label{fig:anseparar}
\end{figure}
%sounds also saved in archive > data > prosody cuts.

\subsection{Clause combining with conjunctions}
\label{sec:clauseconj}
Kalamang conjunctions (introduced in §\ref{sec:wconj}) can be categorised into four functional types, described in turn: sequential, alternative, reason and consequence, and condition conjunctions. Kalamang makes use of both indigenous forms and more recent borrowings from Indonesian, Papuan Malay or a neighbouring Austronesian\is{Austronesian!loan} language (here referred to as ``AN borrowings/loans'' for the sake of simplicity). While all conjunctions may be and frequently are used on their own, combinations of indigenous and borrowed conjunctions with seemingly the same semantics in the same clause are also common. For most of these combinations, the semantics and pragmatics do not seem to differ with respect to the use of a single conjunction. Indigenous conjunctions are more frequent than borrowed ones. In general, when conjunctions are combined, the indigenous conjunctions precede the AN-derived conjunctions.

In the following sections, the four different conjunction types are described. Within each section, I first present the indigenous conjunctions, followed by the borrowed conjunctions, concluding with combinations of indigenous and borrowed conjunctions.

\subsubsection{Sequential events}\is{conjunction!sequential}
\label{sec:seqconj}
Two conjunctions mark sequential events. The Kalamang sequential conjunction is \textit{eba} `then', supplemented by AN loan word \textit{terus} `then'.

\textit{Eba} `then' is placed between the two clauses it links, as in~(\ref{exe:ebaan}). Intonationally, it belongs to the second clause. In~(\ref{exe:ebakuru}), \textit{eba} `then' is used to connect three clauses in a row. \textit{Eba} is frequently shortened to \textit{ba}, illustrated in~(\ref{exe:lebaik}). \textit{Eba} is also used in some conditional clauses, see §\ref{sec:condclause}.

\begin{exe}
	\ex \gll an watko komet=et \textbf{eba} an jie=te kain=bon cocok=et\\
    \textsc{1sg} \textsc{prox.loc} look={\glet} then \textsc{1sg} get={\glte} \textsc{2sg.poss=com} fit={\glet}\\
    \glt `I look here, then I get [one picture] and match with yours.' \jambox*{\href{http://hdl.handle.net/10050/00-0000-0000-0004-1BE8-0}{[stim43\_0:54]}}
    \label{exe:ebaan}
	\ex \gll kai-rep=teba se langsung=i gayam=at kajien \textbf{eba} kuru luk=ta \textbf{eba} se paramuan se di=ra kuar=ta metko\\
    firewood-get={\glteba} {\glse} directly={\gli} chestnut=\textsc{obj} pick then bring come={\glta} then {\glse} cut {\glse} \textsc{caus}=go cook={\glta} \textsc{dist.loc}\\
    \glt `[She] went to get firewood and pick chestnuts, then coming back, then [she] cut [them], having cooked [them] there...' \jambox*{\href{http://hdl.handle.net/10050/00-0000-0000-0004-1BA2-F}{[conv11\_0:47]}}
    \label{exe:ebakuru}
	\ex \glll Lebai ka mia \textbf{ba} pier minum-minumet.\\
	 lebai ka mia \textbf{eba} pier minum∼minum=et\\
	better \textsc{2sg} come then \textsc{2du} drink∼\textsc{red}={\glet}\\
	\glt `You better come here, then we drink [alcohol].' \jambox*{\href{http://hdl.handle.net/10050/00-0000-0000-0004-1BB0-D}{[stim12\_1:23]}}
	\label{exe:lebaik}
\end{exe}

Of the borrowed conjunctions the most frequent is \textit{terus} `then', illustrated in~(\ref{exe:terterus}).

\begin{exe}
	\ex \gll ma terima \textbf{terus} ma se ecien=i kewe=ko\\
	\textsc{3sg} receive then \textsc{3sg} {\glse} return={\gli} house=\textsc{loc}\\
	\glt `He receives [it] and then he returns home.' \jambox*{\href{http://hdl.handle.net/10050/00-0000-0000-0004-1BB0-D}{[stim12\_5:49]}}
	\label{exe:terterus}
\end{exe}

Another borrowed conjunction is \textit{lalu} `then', which is rather infrequent, perhaps because it is a conjunction associated with Malay varieties further west in Indonesia \parencite[][431]{donohue2011}.
	
One combination of an indigenous and a borrowed conjunction marks sequential events: \textit{baru eba} `then'. In~(\ref{exe:barueba}), this combination marks a new paragraph with a change of subject in the story that the speaker is telling, rather than linking events, actions or states sequentially. Data are lacking to confirm that the double use of these conjunctions has a clearly distinct function.

\begin{exe}
	\ex \glll Mindiet me ka marmartebare, kamun sarain. Ah. \textbf{Baru} \textbf{eba}, sepeun opa...\\
	 mindi=et me ka marmar=teba=te ka-mun sara=in ah baru eba sepe-un opa\\
	like\_that={\glet} {\glme} \textsc{2sg} walk={\glteba}=\textsc{imp} \textsc{2sg-proh} ascend=\textsc{proh} \textsc{int} then then hat-\textsc{3poss} {\glopa}\\
	\glt `Like that you just walk, don't you get up [on your bike]. Ah. Then, that hat of his...' \jambox*{\href{http://hdl.handle.net/10050/00-0000-0000-0004-1BD3-9}{[stim33\_1:27]}}
	\label{exe:barueba}
\end{exe}

Sequential events can also be marked with the help of the completive aspectual construction \textit{=i koyet} (see §\ref{sec:compl} and~§\ref{sec:ikoyetsvc}). Simultaneous events are expressed in \is{predicate!complex}complex predicates, see Chapter~\ref{ch:svc}.

\subsubsection{Disjunctive}\is{conjunction!disjunctive}
\label{sec:ye}
The Kalamang disjunctive conjunction is \textit{ye} `or'. In addition, \textit{atau} `or' is borrowed from AN.

\textit{Ye} `or' can be used to coordinate NPs or clauses. An example with \textit{ye} as clause coordinator is given in~(\ref{exe:udabon}). \textit{Ye} typically follows both the items it coordinates, and belongs to the preceding item intonationally.

\begin{exe}
	\ex \gll uda=bon=a melelu \textbf{ye} neba=bon=a melelu \textbf{ye}\\
	rice\_sieve=\textsc{com=foc} sit or what=\textsc{com=foc} sit or\\
	\glt `Sitting with a rice sieve or sitting with what?' \jambox*{\href{http://hdl.handle.net/10050/00-0000-0000-0004-1BE7-5}{[stim42\_6:36]}}
	\label{exe:udabon}
\end{exe}

The AN loan \textit{atau} `or' is used as a disjunctive coordinator for both NPs and clauses. An example coordinating clauses is given in~(\ref{exe:naunkon}). It is never used on both clauses it connects, unlike \textit{ye} `or'.

\begin{exe}
	\ex \gll som-kon ∅-te kon∼kon \textbf{atau} som-kon ∅-te naun-kon\\
	person-one give-\textsc{distr} one∼\textsc{distr} or person-one give-\textsc{distr} fruit-one\\
	\glt `Gave each person one, or gave each person one fruit.' \jambox*{\href{http://hdl.handle.net/10050/00-0000-0000-0004-1BD1-D}{[stim31\_2:48]}}
	\label{exe:naunkon}
\end{exe}

A combination of \textit{ye} `or' and \textit{atau} `or' is attested eight times in the natural spoken corpus. This is exemplified in~(\ref{exe:ruaye}). This double use of the disjunctive coordinator does not seem to have a special function.

\begin{exe}
	\ex \gll ma pi=at=a ruat=kin \textbf{ye} \textbf{atau} pi=at tamandi=kin\\
	\textsc{3sg} \textsc{1pl.incl=obj=foc} kill=\textsc{vol} or or \textsc{1pl.incl=obj} how=\textsc{vol}\\
	\glt `Does he want to kill us, or what does he want to do with us?' \jambox*{\href{http://hdl.handle.net/10050/00-0000-0000-0004-1BAD-2}{[narr29\_8:56]}}
	\label{exe:ruaye}
\end{exe}


\subsubsection{Adversative}\is{conjunction!adversative}
\label{sec:ba}
The Kalamang adversative conjunction is \textit{ba} `but'. It is also used to conjoin \is{numeral}numerals between 11 and 29, as described in §\ref{sec:cardnum}. A second aversative, \textit{tapi} `but', is borrowed from AN.

In~(\ref{exe:goba}), \textit{ba} is used to express opposition between the position of a figurine next to a tree in one picture versus another picture. Intonationally, \textit{ba} belongs to the first clause.

\begin{exe}
	\ex \gll kon wa me ror kanggirar-un=ko \textbf{ba} tanbes-pis=ko\\
	one \textsc{prox} {\glme} tree face-\textsc{3poss}=\textsc{loc} but right-side=\textsc{loc}\\
	\glt `This one he's facing the tree, but on the right side.' \jambox*{\href{http://hdl.handle.net/10050/00-0000-0000-0004-1BC8-8}{[stim26\_4:09]}}
	\label{exe:goba}
	\ex \gll ma mu=at komet∼komet \textbf{ba} mu nokidak\\
	\textsc{3sg} \textsc{3pl=obj} look∼\textsc{prog} but \textsc{3pl} be\_silent\\
	\glt `He watches them but they are silent.' \jambox*{\href{http://hdl.handle.net/10050/00-0000-0000-0004-1BD1-D}{[stim31\_3:16]}}
	\label{exe:noki}
\end{exe}

The combination of \textit{ba} `but' and \textit{tapi} `but' is also attested, without a different meaning. It has six corpus occurrences, all by the same speaker. (\ref{exe:batap}) contains yet a third AN adversative coordinator: \textit{sedangkan} `but; whereas'. This conjunction is only found in the speech of this single speaker within the corpus.

\begin{exe}
	\ex \gll ma koi bo ror kodaet=ko ma koi pareir=taet \textbf{ba} \textbf{tapi} goras wa me \textbf{sedangkan} sor-un ma lapas=nin\\
	\textsc{3sg} then go tree one\_more=\textsc{loc} \textsc{3sg} then follow=again but but crow \textsc{prox} {\glme} but fish-\textsc{3poss} \textsc{3sg} drop.\textsc{mly=neg}\\
	\glt `Then he goes to another tree, he follows again, but the crow doesn't drop its fish.' \jambox*{\href{http://hdl.handle.net/10050/00-0000-0000-0004-1C9B-8}{[stim3\_0:54]}}
	\label{exe:batap}
\end{exe}

\subsubsection{Consequence and reason}\is{conjunction!consequence}
\label{sec:consconj}
Consequence is expressed with a construction involving non-final \textit{=ta} and conjunction \textit{eba} `then', whereas reason clauses are marked with clitic \textit{=tauna} `so' or \textit{=tenden} `so'. In addition, several reason and consequence conjunctions have been borrowed from AN. For the use of locative distal demonstrative \textit{metko} together with \textit{eba}, see §\ref{sec:medisc}.

Sequential conjunction \textit{eba} `then' can be understood as `so that' when it follows non-final marker \textit{=ta} (§\ref{sec:ta}), indicating consequence. It is then invariably used in its short form \textit{ba}. Consider~(\ref{exe:kawettt}) and~(\ref{exe:mangmangg}). A reading where \textit{(e)ba} means `then' is possible, but a consequential reading is more suitable, and is also reflected by informants' translation of these clauses to AN with use of \textit{supaya} `so that'.

\begin{exe}
	\ex \gll an se dodon-an met kuru marua metko=\textbf{ta} \textbf{eba} kawet∼kawet sambil garung=et\\
	\textsc{1sg} {\glse} clothing-\textsc{1sg.poss} \textsc{dist.obj} bring move\_seawards \textsc{dist.loc}={\glta} so\_that fold∼\textsc{iter} while chat={\glet}\\
	\glt `I brought those clothes of mine to the sea there, so that [I could] fold while chatting.' \jambox*{\href{http://hdl.handle.net/10050/00-0000-0000-0004-1BA3-3}{[conv10\_3:28]}}
	\label{exe:kawettt}
	\ex \gll kalamang-mang ewa=\textbf{ta} \textbf{eba} ma tangkap=et\\
	Kalamang-language speak={\glta} so\_that \textsc{3sg} record={\glet}\\
	\glt `Speak Kalamang so that she can record.' \jambox*{\href{http://hdl.handle.net/10050/00-0000-0000-0004-1BBD-5}{[conv12\_4:38]}}
	\label{exe:mangmangg}
\end{exe}

The clitic \textit{=tauna} links reason to result. It most commonly attaches to \is{demonstrative}demonstrative forms, and then usually the distal form. However, the clitic also has a few occurrences on transitive and intransitive verbs. (\ref{exe:mebarauna}) illustrates \textit{=tauna} on the distal demonstrative \textit{me}, which is also marked with \is{focus}focus marker \textit{=ba}. The example is from a story about diving for lobsters. \textit{=tauna} serves to link a reason (the subject recognises a good diving spot) to a result (the subject quickly catches a lobster). (\ref{exe:kararaktauna}) shows \textit{=tauna} on a stative intransitive verb, indicating the reason for suggesting another sailing route. In~(\ref{exe:narauna}), the clitic attaches to a transitive verb, indicating that the fact that the subject (a crow) ate rotten fish is taken not so much as the reason but as proof that it has degraded itself to eating rotten food. That example also shows the combination of a Kalamang and a borrowed conjunction with a slightly different meaning (\textit{sehingga} `until; so that; with the result that').

\begin{exe}
	\ex \gll Mel se dalang=i bara mungkin yar-un naunin=ten me=ba=\textbf{tauna} ma se jie kuru sara\\
	Mel {\glse} jump={\gli} descend maybe stone-\textsc{3poss} recognise-\textsc{ten} \textsc{dist}={\glba}=so \textsc{3sg} {\glse} get bring ascend\\
	\glt `Mel jumped down, maybe he recognised his stone, so he got [a lobster] and brought [it] up.' \jambox*{\href{http://hdl.handle.net/10050/00-0000-0000-0004-1BAE-4}{[narr44\_15:24]}}
	\label{exe:mebarauna}
	\ex \gll warkin kararak=\textbf{tauna} ge=et pi osa=ka terus=i marat=et\\
	tide dry=so no={\glet} we \textsc{up=lat} go\_further={\gli} move\_landwards={\glet}\\
	\glt `The tide is low, so why don't we continue from up there towards land.' \jambox*{\href{http://hdl.handle.net/10050/00-0000-0000-0004-1B6E-D}{[conv25\_3:40]}}
	\label{exe:kararaktauna}
	\ex \gll ka don yuwa=at=a na=\textbf{tauna} sehingga don mun=ten wandi=et ka bisa na=ta\\
	\textsc{2sg} thing \textsc{prox=obj=foc} eat=so so\_that thing rotten=\textsc{at} like\_this={\glet} \textsc{2sg} can eat={\glta}\\
	\glt `You ate this thing, [how has it come so far] that you can eat rotten things like this?' \jambox*{\href{http://hdl.handle.net/10050/00-0000-0000-0004-1B91-5}{[narr39\_7:35]}}
	\label{exe:narauna}
\end{exe}

A clitic on the predicate, \textit{=tende(n)} `so', also links reason to result, as illustrated in~(\ref{exe:wartenden}) and~(\ref{exe:naubestenden}). Although \textit{=tende(n)} and \textit{=tauna} attach to different constituents, no significant difference in meaning is apparent from the current corpus.\footnote{Compare Papuan Malay \textit{jadi} as a sentence-final particle that indicates the reason for the sentence, according to \textcite{donohue2011}.}

\begin{exe}
	\ex \gll in bo war=\textbf{tenden} in=nan kaden-un koi kememe\\
	\textsc{1pl.excl} go fish=so \textsc{1pl.excl}=too body-\textsc{1pl.excl.poss} again weak\\
	\glt `We went fishing so our bodies are tired.' \jambox*{\href{http://hdl.handle.net/10050/00-0000-0000-0004-1BA2-F}{[conv11\_5:43]}}
	\label{exe:wartenden}
	\ex \gll mier se nau=bes=\textbf{tenden} ma kiun=at jaga to\\
	\textsc{3du} {\glse} \textsc{recp}=good=so \textsc{3sg} wife.\textsc{3poss=obj} watch right\\
	\glt `They have made up with each other, so he is taking care of his wife, right.' \jambox*{\href{http://hdl.handle.net/10050/00-0000-0000-0004-1BAA-C}{[stim7\_26:23]}}
	\label{exe:naubestenden}
\end{exe}

Otherwise, conjunctions marking reason or consequence are mainly borrowed from AN\is{Austronesian!loan}: \textit{jadi} `so', \textit{karena} `because' and \textit{supaya} `so that', exemplified in~(\ref{exe:jadi}) to~(\ref{exe:supaya}), all occur frequently in the natural spoken corpus. \textit{Sehingga} `until; so that; with the result that' only has a few unclear occurrences besides~(\ref{exe:narauna}), where it is combined with a Kalamang conjunction.

\begin{exe}
	\ex \gll kiun ketiga tum∼tum karuok weinun \textbf{jadi} pebis-un karuok-gan tum∼tum tebonggan kaninggonie\\
	wife.\textsc{3poss} third child∼\textsc{pl} three too so woman-\textsc{3poss} three-all child∼\textsc{pl} all nine\\
	\glt `His third wife also had three children, so his three women had nine children in total.' \jambox*{\href{http://hdl.handle.net/10050/00-0000-0000-0004-1C98-7}{[narr26\_19:30]}}
	\label{exe:jadi}
	\ex \gll mu se koi wat pes=at di=kahalong siun=ko \textbf{karena} kahalong siun kang\\
	\textsc{3pl} {\glse} then coconut peel=\textsc{obj} \textsc{caus}=spear edge\textsc{.3poss=loc} because spear edge.\textsc{3poss} sharp\\
	\glt `Then they put the coconut skin on the points of the two-pointed spear, because the two-pointed spear has sharp points.' \jambox*{\href{http://hdl.handle.net/10050/00-0000-0000-0004-1BDC-D}{[conv8\_3:13]}}
	\label{exe:karena}
	\ex \gll ka muap=at kuet=nin ka-mun langsung=i bo=in \textbf{supaya} pi tok muap-sanggara eba marei\\
	\textsc{2sg} food=\textsc{obj} bring=\textsc{neg} \textsc{2sg=proh} directly={\gli} go-\textsc{proh} so\_that \textsc{1pl.incl} first food-search then move\_landwards.\textsc{imp}\\
	\glt `You didn't bring food, don't you go directly, so that we go food-searching first, then go towards land!' \jambox*{\href{http://hdl.handle.net/10050/00-0000-0000-0004-1BA3-3}{[conv10\_7:05]}}
	\label{exe:supaya}
\end{exe}
	
Combinations of the Kalamang construction \textit{=ta} {\glta} + \textit{(e)ba} `so that' and \textit{supaya} `so that' are frequently found. The 13 occurrences from natural speech do not indicate a difference between the use of one or two conjunctions.

\begin{exe}
	\ex \gll mu wat tu=\textbf{ta} \textbf{eba} \textbf{supaya} naramas=te mu per-un=at nan=et\\
	\textsc{3pl} \textsc{prox.obj} pound={\glta} so\_that so\_that squeeze={\glte} \textsc{3pl} liquid-\textsc{3poss=obj} consume={\glet}\\
	\glt `They pound this so that [they can] squeeze [it] and they drink its liquid.' \jambox*{\href{http://hdl.handle.net/10050/00-0000-0000-0004-1BB1-3}{[narr34\_3:58]}}
	\label{exe:basup}
\end{exe}

Combinations of \textit{=tauna} and borrowed \textit{jadi} `so' are uncommon but attested. All three occurrences are around pauses and/or repairs, so they seem to be used as \is{filler}fillers. Consider~(\ref{exe:merauna}).

\ea \glll Maː nebai koyet muː jadi merauna opa an sirie ma Binkur esun temun...\\
	ma neba=i koyet mu \textbf{jadi} me=\textbf{tauna} opa an sirie ma Binkur esun temun\\
	\textsc{3sg} \textsc{ph}={\gli} finish \textsc{3pl} so \textsc{dist}=so {\glopa} \textsc{1sg} send \textsc{3sg} Binkur father.\textsc{3poss} big\\
	\glt `After he did whatsit they, eh, so that, I sent him, Binkur's father...' \jambox*{\href{http://hdl.handle.net/10050/00-0000-0000-0004-1B9F-F}{[conv9\_30:30]}}
	\label{exe:merauna}
\z

\subsubsection{Concessive}\is{conjunction!concessive}
Concessive constructions are formed with a dedicated clitic \textit{=taero}, described in §\ref{sec:condmood}. In addition to that, a conjunction borrowed from AN is used: \textit{biar} `even if', which precedes the concessive clause. 

\begin{exe}
	\ex \gll sayang-saran=i koyet \textbf{biar} kolak=ko mu kaluar\\
	nutmeg-ascend={\gli} finish even\_if mountain=\textsc{loc} \textsc{3pl} exit\\
	\glt `After harvesting the nutmeg, even if they're in the mountains, they come out.' \jambox*{\href{http://hdl.handle.net/10050/00-0000-0000-0004-1BF1-6}{[narr12\_7:48]}}
	\ex \gll \textbf{kalau} mat kuru masarat=nin=et me pi barat=nin\\
	if \textsc{3sg.obj} bring move\_landwards=\textsc{neg}={\glet} {\glme} \textsc{1pl.incl} descend=\textsc{neg}\\
	\glt `If they don't bring him towards land, we don't go down.' \jambox*{\href{http://hdl.handle.net/10050/00-0000-0000-0004-1BC3-B}{[conv7\_4:22]}}
\end{exe}	

\subsubsection{Conditional}\is{conditional}
\label{sec:condconj}
Conditional clauses are formed with help of a topicalised clause with \textit{=et me} (described in §\ref{sec:condclause}). The AN conjunction \textit{kalau} `if' (pronounced /kalo/ or /kalu/, but I adhere to the Indonesian spelling here) is also used. \textit{Kalau} precedes the clause that presents the condition. The conjunction \textit{kalau} and the Kalamang strategy with a topicalised clause may be combined, as in~(\ref{exe:kaloloi}).

\begin{exe}
	\ex \gll \textbf{kalau} loi∼loi=tun=\textbf{et} \textbf{me} eranun ka sitak sitak sitak\\
	if quick∼\textsc{ints=ints}={\glet} {\glme} cannot \textsc{2sg} slow slow slow\\
	\glt `If [you do it] too quickly, it's not possible, you [have to do it] slowly, slowly, slowly.' \jambox*{\href{http://hdl.handle.net/10050/00-0000-0000-0004-1BA6-6}{[conv13\_1:09]}}
	\label{exe:kaloloi}
\end{exe}

\subsection{Tail-head linkage}\is{clause-chaining|see also{tail-head linkage}}\is{tail-head linkage}
\label{sec:tailhead}
Tail-head linkage is the \is{repetition}repetition of the last (part of a) clause in a clause chain at the beginning of the next clause chain \parencite{devries2005}. This is a common clause-combining device in Kalamang. The amount of material repeated varies from the entire clause to just the predicate, but the latter is the most common. When tail-head linkage is not achieved by conjoining, it may be combined with the construction \textit{=i koyet} (which expresses completive aspect but is only used to link clauses, and translates as `after'; see §\ref{sec:compl}). Another clause-linking device, non-final marker \textit{=ta} (§\ref{sec:ta}), is also seen on the recapitulated predicate, confirming de Vries' (2005: 372) observation that clause-linking strategies used elsewhere in the language are also employed in tail-head linkage. All tail-head linkage in Kalamang is used for sequential events or actions.

All examples given in this section contain the last part of a clause chain ending in a low boundary tone (indicated by a full stop), and the entire next clause chain from the repeated part to the next low boundary tone. Rising intonation\is{intonation} is indicated by a comma. %as in the rest of the thesis, so maybe not point out again.

(\ref{exe:svthl}) shows the repetition of the entire clause (in b), consisting of subject and predicate.

\ea
	\ea\glll Koi go yuolet \textbf{ma} \textbf{koi} \textbf{maruaret}. \\
	koi go yuol=et ma koi maruat=et\\
	then condition day={\glet} \textsc{3sg} again move\_seawards={\glet}\\
	\glt `When it was day, he went towards sea again.'
	\label{exe:yuoletmaruaret}
	\ex \glll \textbf{Ma} \textbf{koi} \textbf{maruaret}, mindi weinun ma era ma pewun karuarten met nani koyet,\\
	ma koi maruat=et mindi weinun ma era ma pep-un karuar=ten met nan=i koyet\\ 
	\textsc{3sg} again move\_seawards={\glet} like\_that too \textsc{3sg} ascend \textsc{3sg} pig-\textsc{3poss} smoke\_dry=\textsc{at} \textsc{dist.obj} consume={\gli} finish\\
	\glt `He went towards sea again, like that too he came up and after he ate their smoked pig,'
	\ex \glll a emun gounat koyet kieri koyet ma he ecien.\\
	a emun go-un=at koyet kiet=i koyet ma se ecien\\
	\textsc{hes} mother.\textsc{3poss} place-\textsc{3poss=obj} finish defecate={\gli} finish \textsc{3sg} {\glse} return\\	
	\glt `after defecating in their mother's place, he went back.'	\jambox*{\href{http://hdl.handle.net/10050/00-0000-0000-0004-1BDB-C}{[narr28\_1:54]}}
	\z
	\label{exe:svthl}
\z

(\ref{exe:ovthl}) shows the repetition of object\is{object} and predicate (in b). Note that the object and predicate at the end of the first chain are followed by an afterthought \textit{balgi to} `with dogs, right'. This afterthought is not repeated at the beginning of the next clause chain. Repetition of the object is not obligatory, as~(\ref{exe:paningi}) shows.

\ea
	\ea \glll Mier bore \textbf{pewat} \textbf{sanggara}, balgi to.\\
	mier bo=te pep=at sanggara bal=ki to\\
	\textsc{3du} go={\glte} pig=\textsc{obj} search dog=\textsc{ins} right\\
	\glt `They two went hunting [lit. searching] pigs, with a dog, right.
	
	\ex \glll \textbf{Pewat} \textbf{sanggara}, ma era, emnem muawunat nani koyet,\\
	pep=at sanggara ma era emnem muap-un=at nan=i koyet\\
	pig=\textsc{obj} search \textsc{3sg} ascend mother food-\textsc{3poss=obj} consume={\gli} finish\\
	`Searching for pigs, he went up, after eating the mother's food,
	
	\ex \glll ma he koi kietkieri koyet, ma he yecie.\\
	ma se koi kiet∼kiet=i koyet ma se yecie\\
	\textsc{3sg} {\glse} again defecate∼\textsc{red}={\gli} finish \textsc{3sg} {\glse} return\\
	\glt `after defecating again, he returned.' \jambox*{\href{http://hdl.handle.net/10050/00-0000-0000-0004-1BDB-C}{[narr28\_0:49]}}

	\z
	\label{exe:ovthl}
\z

(\ref{exe:predthl}) shows the repetition of just the predicate.

\begin{exe}
	\ex 
	\begin{xlist}
		\ex \glll Mindia bo nani koyet bal se \textbf{taouk}.\\
		mindia bo nan=i koyet bal se taouk\\
		like\_that=\textsc{foc} go consume={\gli} finish dog {\glse} lie\_down\\
		\glt `After going eating like that, the dog lies down.'
		
		\ex \glll \textbf{Taouk}, goras opa me naminyasa: ``Aduh!''\\
		tauk goras opa me naminyasa aduh\\
		lie\_down crow {\glopa} {\glme} regret \textsc{int.mly}\\
		\glt `Lies down, that crow regrets: ``Ah!''' \jambox*{\href{http://hdl.handle.net/10050/00-0000-0000-0004-1C9B-8}{[stim3\_2:33]}}

	\end{xlist}
\label{exe:predthl}
\end{exe} 

Otherwise, repetition of the predicate is either done by marking the repeated part with completive aspect \textit{=i koyet} (examples~\ref{exe:samsii} and~\ref{exe:paningi}), or by marking it with non-final \textit{=ta} (§\ref{sec:ta}, example~\ref{exe:subanda}), creating a link not only between the tail and the head, but also between the head and the following clause.

\ea
	\ea \glll Manyori koyet ma yorsik an koi desili paruo \textbf{samsik}.\\
		manyor=i koyet ma yorsik an koi desil=i paruo samsik\\
		adjust={\gli} finish \textsc{3sg} straight \textsc{1sg} then plane={\gli} make thin\\
		\glt `After adjusting it's straight and then I plane it to make it thin.'
		
		\ex \glll \textbf{Samsi} \textbf{koyet} an koliepliunat dikolko.\\
		samsik=i koyet an koliep∼liep-un=at di=kolko\\
		thin={\gli} finish \textsc{1sg} cheeck∼\textsc{pl-3poss=obj} \textsc{caus}=move\_out\\
		\glt `After it's thin, I get rid of its sides [lit. cheeks].' \jambox*{\href{http://hdl.handle.net/10050/00-0000-0000-0004-1BDD-5}{[narr42\_5:43]}}
	\end{xlist}
	\label{exe:samsii}
	
	\ex 
	\ea \glll An bo mu erunat \textbf{paning}.\\
		an bo mu et-un=at paning\\
		\textsc{1sg} go \textsc{3pl} canoe-\textsc{3poss=obj} ask\\
		\glt `I went to ask them for their canoe.'
		
		\ex \glll \textbf{Paningi} \textbf{koyet}, mu he lo,\\
		paning=i koyet mu se lo\\
		ask={\gli} finish \textsc{3sg} {\glse} consent\\
		\glt `After asking, they consented,'
		
		\ex \glll an se kuru mian bo \textbf{seranat} \textbf{sanggaran},\\
		an se kuru mian bo set-an=at sanggaran\\
		\textsc{1sg} {\glse} bring come go bait-\textsc{1sg.poss}\\
		\glt `I brought [the canoe to my place] and went searching my bait,'
		
		\ex \glll \textbf{seranat} \textbf{sanggarani} \textbf{koyet}, nikan, yalan,\\ 
		set-an=at sanggaran=i koyet nika-an yal-an\\
		bait-\textsc{1sg.poss=obj} search={\gli} finish line-\textsc{1sg.poss} paddle-\textsc{1sg.poss}\\
		\glt `after searching my bait, my line, my paddle,'
		
		\ex \glll met kuru baran an se bot.\\
		met kuru baran an se bot\\
		\textsc{dist.obj} bring descend \textsc{1sg} {\glse} go\\
		\glt `brought that down, I went.' \jambox*{\href{http://hdl.handle.net/10050/00-0000-0000-0004-1C99-E}{[narr8\_0:08]}}
			\z
	\label{exe:paningi}
	
	\ex
	\ea \glll An se koi ma tebolsuban.\\
	an se koi ma tebolsuban\\
	\textsc{1sg} {\glse} then move\_landwards fish\_at\_reef\_edge\\
	\glt `Then I moved towards land to fish at the reef edge.'
	
	\ex \glll Ma tepnerga marua, \textbf{tebolsuban}. \\
	ma tepner=ka marua tebolsuban\\
	move\_landwards deep\_seawater-\textsc{lat} move\_seawards fish\_at\_reef\_edge\\
	\glt `Moved towards land from the deep seawater, moved towards sea, fished at the reef edge.'
	
	\ex \glll \textbf{Tebolsubanda}, kabaruawan erir karuok.\\
	tebolsuban=ta kabaruap-an et-eir karuok\\
	fish\_at\_reef\_edge={\glta} grouper-\textsc{1sg.poss} \textsc{clf\_an}-two three\\
	\glt `Fishing at the reef edge, I had two or three groupers.' \jambox*{\href{http://hdl.handle.net/10050/00-0000-0000-0004-1C99-E}{[narr8\_0:42]}}
	
	\z
	\label{exe:subanda}
\z


(\ref{exe:oskitko}) shows the repetition of just the location in the predicate. This may have a scene-setting function, but there are not enough examples in the corpus to make a conclusive analysis.

\ea
	\ea\glll In langganat pararani koyet bo turusi bo \textbf{oskitko}.\\
		in langgan=at pararan=i koyet bo turus=i bo os-keit=ko\\
		\textsc{1pl.excl} wood=\textsc{obj} extend={\gli} finish go further={\gli} go beach-top=\textsc{loc}\\
		\glt `After extending the wood we went further to the beach.'
		
		\ex \glll	\textbf{Oskitko}, in muat paning.\\
		os-keit=ko in mu=at paning\\
		beach-top=\textsc{loc} \textsc{1pl.excl} \textsc{3pl=obj} ask\\
		\glt `On the beach we asked them.' \jambox*{\href{http://hdl.handle.net/10050/00-0000-0000-0004-1BB4-6}{[narr14\_4:21]}}
		
	\z
	\label{exe:oskitko}
\z

\subsection{Non-final clauses}\is{non-final|(}
\label{sec:nfin}
Two clitics on the predicate mark the predicate as non-final across clauses: \textit{=te} and \textit{=ta}. These are versatile clause combiners that do not specify the exact relationship between propositions.

\subsubsection{Non-final \textit{=te}}
\label{sec:te}
The predicate clitic \textit{=te} marks a predicate as non-final. It can be found on predicates followed by a clause with the same arguments, in which case the arguments are typically left out and the two predicates directly follow each other. In~(\ref{exe:tankin}), \textit{tankinkin} `to shake hands' is followed by \textit{ecie} `to return'. The clause following a predicate with \textit{=te} may also have different arguments. In~(\ref{exe:mure}) and~(\ref{exe:gonggungde}), \textit{=te} occurs on the last verb of a clause, which is followed by a clause with another subject, but which expresses an event that is related to the event in the first clause. The relation is typically temporal (sequential\is{sequential}) and slightly causal, but note that \textit{=te} is never obligatory, and that sequentiality and causality may be expressed in other ways (as touched upon earlier in §~\ref{sec:clacomb}). \textit{=te} is a versatile way of tying states and events together, without forcing a particular reading of the exact relation between the propositions.

\begin{exe}
	\ex \gll sontum se tankinkin=\textbf{te} ecie-p∼cie-p\\
	person {\glse} shake\_hands={\glte} return-{\glp}∼\textsc{distr}-{\glp}\\
	\glt `People shook hands and returned.' \jambox*{\href{http://hdl.handle.net/10050/00-0000-0000-0004-1B85-F}{[narr5\_5:32]}}
	\label{exe:tankin}
	\ex \gll ka mu ∅=\textbf{te} mu na\\
	\textsc{2sg} \textsc{3pl} give={\glte} \textsc{3pl} consume\\
	\glt `You give [the food] to them, they eat. \jambox*{\href{http://hdl.handle.net/10050/00-0000-0000-0004-1BA2-F}{[conv11\_5:18]}}
	\label{exe:mure}
	\ex \gll kiun=at me ma gonggung=\textbf{te} ma kirarun=ko\\
	wife.\textsc{3poss=obj} {\glme} \textsc{3sg} call={\glte} \textsc{3sg} side=\textsc{loc}\\
	\glt `His wife he calls, she [comes and sits] beside him.' \jambox*{\href{http://hdl.handle.net/10050/00-0000-0000-0004-1BAA-C}{[stim7\_28:27]}}
	\label{exe:gonggungde}
\end{exe}

Non-final \textit{=te} is very common on \textit{bo} `to go' when it has the meaning `to turn', `to become', or `until'. The predicates that follow \textit{bo=te} in~(\ref{exe:borenain}) and~(\ref{exe:boreyuol}) are nominal. \textit{Bo} is also used with non-final \textit{=te} in its original sense, `to go', when combined with a location predicate, as in~(\ref{exe:sikanbo}).

\begin{exe}												
	\ex
	\label{exe:borenain}
	{\gll se bo=\textbf{te} nain panggala naun=kap\\
		\textsc{iam} go=\textsc{\glte} like cassava fruit=\textsc{sim}\\
		\glt `[It's] already becoming as big as a cassava.' \jambox*{\href{http://hdl.handle.net/10050/00-0000-0000-0004-1BBD-5}{[conv12\_6:51]}}
	}
	\ex \gll bo=\textbf{te} yuol me eba metko nene toni o an kona\\
	go={\glte} day \textsc{dist} then \textsc{dist.loc} grandmother say \textsc{surpr} \textsc{1sg} see\\
	\glt `Until that day, only then grandmother said: ``Oh, I see.''' \jambox*{\href{http://hdl.handle.net/10050/00-0000-0000-0004-1BC3-B}{[conv7\_10:41]}}
	\label{exe:boreyuol}
	\ex \gll sikan bo=\textbf{te} tepeles nerun=ko\\
	cat go={\glte} jar inside=\textsc{loc}\\
	\glt `The cat goes inside the jar.' \jambox*{\href{http://hdl.handle.net/10050/00-0000-0000-0004-1BAF-5}{[stim20\_1:21]}}
	\label{exe:sikanbo}
\end{exe}

The difference between non-final \textit{=te} and \textit{eba} `then' (§\ref{sec:seqconj}) or tail-head linking with \mbox{\textit{=i koyet}} (§\ref{sec:tailhead}) is that the latter two strategies are clause-chaining\is{clause-chaining} strategies, and non-final \textit{=te} only links one event or action to another. 

The difference between predicate linker \textit{=i} (§\ref{sec:mvci}) and non-final \textit{=te} is that \textit{=i} is exclusively used within the clause, which means that the verbs linked by \textit{=i} must have the same arguments, whereas \textit{=te} relates states and events across clauses. Moreover, verb linker \textit{=i} is used in complex predicates with certain semantics such as those expressing complex motion, while \textit{=te} is not associated with specific verb semantics.

\subsubsection{Non-final \textit{=ta}}
\label{sec:ta}
The predicate clitic \textit{=ta} apparently has a roughly similar function to non-final \textit{=te}: it relates two states or events. Therefore, I tentatively analyse it as a non-final marker, but distinguish it from \textit{=te}, because it occurs in a few contexts in which non-final \textit{=te} is infrequent. Since I do not yet have a conclusive analysis of it, I make do with describing the distribution in detail. 

Non-final \textit{=ta} is certainly not a phonological variant of non-final \textit{=te}, as it may be found in the same environments. Compare \textit{tankinkin=te} `shake hands' (from \textit{tan} `arm and hand' and \textit{kinkin} `hold') in~(\ref{exe:tankin}) above with \textit{kinkin=ta} `hold' here:

\begin{exe}
	\ex \gll ma kinkin=\textbf{ta} metko\\
	\textsc{3sg} hold={\glta} \textsc{dist.loc}\\
	\glt `He holds [it] there.' \jambox*{\href{http://hdl.handle.net/10050/00-0000-0000-0004-1BE7-5}{[stim42\_8:06]}}
	\label{exe:kinkinda}
\end{exe}

Non-final \textit{=ta} occurs in same-subject environments, as illustrated in~(\ref{exe:nasesak}), but much less so than non-final \textit{=te}.

\begin{exe}
	\ex \gll an yie=\textbf{ta} kajie ba kat kan nasesak\\
	\textsc{1sg} swim={\glta} pick but lake right high\_tide\\
	\glt `I was swimming picking [chestnuts] but the water in the lake was high, right.' \jambox*{\href{http://hdl.handle.net/10050/00-0000-0000-0004-1BA2-F}{[conv11\_1:10]}}
	\label{exe:nasesak}
\end{exe}	

Non-final \textit{=ta} frequently occurs in combinaion with \textit{ba} `but', as in~(\ref{exe:gergettaba}), and the (distinct) sequential marker \textit{(e)ba} `then', as in~(\ref{exe:tonikalau}) and~(\ref{exe:pakutet}). This combination is very rare for non-final \textit{=te}.

\begin{exe}
	\ex \gll an gerket=\textbf{ta} \textbf{ba} mu toni mu=a koluk=ta Tami mu sabarak-un=ko\\
	\textsc{1sg} ask={\glta} but \textsc{3pl} say \textsc{3pl=foc} find={\glta} Tami \textsc{3pl} under\_house-\textsc{3poss=loc}\\
	\glt `I asked but they said they just found it under Tami's house.' \jambox*{\href{http://hdl.handle.net/10050/00-0000-0000-0004-1BCE-D}{[conv4\_6:13]}}
	\label{exe:gergettaba}
	\ex \gll an toni kalau ki=konggo=a garung=et an se dodon-an met kuru marua metko=\textbf{ta} \textbf{(e)ba} kawet∼kawet sambil garung=et\\
	\textsc{1sg} say if \textsc{2pl=an.loc=foc} chat={\glet} \textsc{1sg} {\glse} clothing-\textsc{1sg.poss} \textsc{dist.obj} bring move\_seawards \textsc{dist.loc}={\glta} then fold∼\textsc{iter} simultaneously chat={\glet}\\
	\glt `I said if you are chatting at yours, I bring my clothing down there, then fold while chatting.' \jambox*{\href{http://hdl.handle.net/10050/00-0000-0000-0004-1BA3-3}{[conv10\_3:25]}}
	\label{exe:tonikalau}
	\ex \gll pi pakut=et tahan=\textbf{ta} \textbf{(e)ba} bisa yorsik=\textbf{ta} \textbf{ba} bisa kit-kadok di=rat=et\\
	\textsc{1pl.incl} nail={\glet} endure={\glta} then can straight={\glta} then can top-side \textsc{caus}=move={\glet}\\
	\glt `If we nail steadily then [we] can make it straight, then [we] can install the top.' \jambox*{\href{http://hdl.handle.net/10050/00-0000-0000-0004-1BB3-0}{[narr7\_3:58]}}
	\label{exe:pakutet}
\end{exe}

Non-final \textit{=ta} is also very frequently followed by a variant of distal \is{demonstrative!distal}demonstrative \textit{me} (§\ref{sec:me}) or \is{topic}topic marker \textit{me} (§\ref{sec:discme}). Again, this is hardly found with non-final \textit{=te}.

\begin{exe}
	\ex \gll yuol me Sek=a in bara os payiem=\textbf{ta} me an tang tama-n=i kajie\\
	day \textsc{dist} Sek=\textsc{foc} \textsc{1pl.excl} descend sand fill={\glta} {\glme} \textsc{1sg} seed \textsc{q}-{\glnn}={\gli} pick\\
	\glt `That day [at] Sek, we went down to fill sand, I picked I-don't-know-how-many seeds.' \jambox*{\href{http://hdl.handle.net/10050/00-0000-0000-0004-1BA2-F}{[conv11\_4:36]}}
	\ex \gll mindi bo=te tete se somin=\textbf{ta} met se ecien=i masarat=kin\\
	like\_that go={\glte} grandfather {\glse} die={\glta} \textsc{dist.obj} {\glse} return={\gli} move\_landwards=\textsc{vol}\\
	\glt `Like that until grandfather had died, then [we] wanted to go back towards land.' \jambox*{\href{http://hdl.handle.net/10050/00-0000-0000-0004-1BC3-B}{[conv7\_10:14]}}
	\ex 
	{\gll o kusukusu toni tok nakal-ca tok kuskap=\textbf{ta} ime tok tok\\
	\textsc{emph} cuscus say not\_yet head-\textsc{2sg.poss} still black={\glte} \textsc{dist} not\_yet not\_yet\\
	\glt `The cuscus says: ``Not yet, your head is still black, not yet.''' \jambox*{\href{http://hdl.handle.net/10050/00-0000-0000-0004-1BC1-0}{[narr19\_15:04]}}
	}
\end{exe}

\textit{=ta} is not found on numbers, and not with locations except for \textit{metko} `there' (example~\ref{exe:kinkinda}), whereas non-final \textit{=te} is frequently found followed by predicative locatives.

In contrast to non-final \textit{=te}, non-final \textit{=ta} is often seen following the lative clitic \textit{=ka}, as in~(\ref{exe:tabo}). Non-final \textit{=ta} surfaces as \textit{ra} after vowels, making it easily confused with the directional verb \textit{ra} `to move (away)'. (\ref{exe:ranora}) shows parallel clauses which help distinguish the allomorph \textit{=ra} from the verb \textit{ra} `move away'. \textit{Ra} is followed by other directional verbs with which it shouldn't be compatible (e.g. \textit{sara} `to ascend'). Moreover, the morpheme does not carry stress when following \textit{=ka}, whereas \textit{ra} `to go; to move away' does. 

\begin{exe}
	\ex \gll ma=bon kiun=bon pasar=ka=\textbf{ta} bo don-jiet=kin\\
	\textsc{3sg}=\textsc{com} wife.\textsc{3poss}=\textsc{com} market=\textsc{lat}={\glta} go thing-buy=\textsc{vol}\\
	`He and his wife want to go to the market to buy stuff.' \jambox*{\href{http://hdl.handle.net/10050/00-0000-0000-0004-1BAA-C}{[stim7\_27:15]}}
	\label{exe:tabo}
	\ex 
	\begin{xlist}
		\ex \gll Abdula emun se masara {\ob}...{\cb} ma se gen koi \textbf{mengga} \textbf{sara}\\
		Abdula mother.\textsc{3poss} {\glse} move\_landwards {} \textsc{3sg} {\glse} maybe then \textsc{dist.lat} ascend\\
		\glt `Abdula's mother had gone inland, maybe she had gone up from there already.' \jambox*{\href{http://hdl.handle.net/10050/00-0000-0000-0004-1BA3-3}{[conv10\_6:25]}}
		\ex \gll Unyil emun koi etuat-mang \textbf{mengga=ra} \textbf{sara}\\
		Unyil mother.\textsc{3poss} then cry-voice \textsc{dist.lat}-? ascend\\
		\glt `Then Unyil's mother came up crying.' \jambox*{\href{http://hdl.handle.net/10050/00-0000-0000-0004-1BC3-B}{[conv7\_0:23]}}
	\end{xlist}
	\label{exe:ranora}
\end{exe}

The combination lative + \textit{bo} `to go' (given in~\ref{exe:tabo}), without adding non-final \textit{=ta} in between, seems to be ungrammatical. Lative + \textit{=ta} + another verb expressing movement is also attested, but there, the omission of \textit{=ta} is grammatical, as illustrated in~(\ref{exe:gata}) and~(\ref{exe:wanggab}) with \textit{bara} `to descend'. Complex lative constructions are further described in §\ref{sec:mvcgoal}.

\begin{exe}
	\ex \gll masikit mul=ka=\textbf{ta} bara\\
	mosque side=\textsc{lat}={\glta} descend\\
	\glt `[They] go down from the side of the mosque.' \jambox*{\href{http://hdl.handle.net/10050/00-0000-0000-0004-1BCF-3}{[stim42\_0:12]}}
	\label{exe:gata}
	\ex \gll ki wangga barat=et\\
	\textsc{2pl} \textsc{prox.lat} descend={\glet}\\
	\glt `You go down here.' \jambox*{\href{http://hdl.handle.net/10050/00-0000-0000-0004-1BC3-B}{[narr2\_2:37]}}
	\label{exe:wanggab}
\end{exe}\is{non-final|)}
%also masikit=ka mengga bara
\is{clause combining|)}

\section{Complement clauses}\is{complement clause}
\label{sec:complclause}
\label{sec:speech}
Complement clauses are subordinate\is{clause!subordinate} clauses that function as an argument of the main clause. All Kalamang complement clauses are direct\is{direct speech} and reported speech\is{reported speech} that function as the object of the main clause, introduced by various speech and perception verbs, \is{iamitive}iamitive \textit{se}, \is{demonstrative!manner}demonstrative \textit{wandi} `like this' and quotative \textit{eh}. 

Complement clauses containing speech\is{speech verb} or thought may be introduced by the verb \textit{toni} `say; think', certain speech or perception verbs\is{perception verb}, iamitive \textit{se}, demonstrative \textit{wandi} `like this' or \is{interjection}interjection \textit{eh}, all of which follow the subject and precede the quoted or reported speech. Some of these may be combined. Direct speech may be also given without introduction by linguistic material. \is{quotative marker} 

\textit{Toni} always introduces complement clauses and is a dependent verb\is{verb!dependent}: it cannot be inflected and it cannot be the predicate of a simple clause. \textit{Toni} can introduce direct speech, as in~(\ref{exe:mutonio}), as well as reported speech, as in~(\ref{exe:matonimu}). The distinction can be deduced from the use of pronouns in context. In~(\ref{exe:mutonio}), we know that both third-person plural\is{plural!pronominal} \textit{mu} and first-person plural exclusive \textit{in} refer to the same group of people. The \is{interjection}interjection \textit{o} is a further clue that this is a direct speech report. (\ref{exe:matonimu}) must be reported speech, because it is clear from the context that the third-person singular \textit{ma} is someone else than the third-person plural \textit{mu}.

\begin{exe}
	\ex \gll mu toni o in hukat=at kon-i koluk\\
	\textsc{3pl} say \textsc{int} \textsc{1pl.excl} net=\textsc{obj} one-\textsc{objqnt} find\\
	\glt `They said: ``O, we found a net.''' \jambox*{\href{http://hdl.handle.net/10050/00-0000-0000-0004-1BCE-D}{[conv4\_5:22]}}
	\label{exe:mutonio}
	\ex \gll ma toni mu paruak ma se komet∼komet hukat yuwa me tok giar=ten\\
	\textsc{3sg} say \textsc{3pl} throw\_away \textsc{3sg} {\glse} look∼\textsc{prog} net \textsc{prox} {\glme} still new=\textsc{at}\\
	\glt `She said that they threw [it] away, she had been looking, this net was still new.' \jambox*{\href{http://hdl.handle.net/10050/00-0000-0000-0004-1BCE-D}{[conv4\_5:51]}}
	\label{exe:matonimu}
\end{exe}

In many cases, like~(\ref{exe:kamanan}), it remains underspecified whether direct or reported speech is intended, unless the speech is set apart with the use of a different voice, imitating the source.

\begin{exe}
	\ex \gll ma toni kaman=nan mambon\\
	\textsc{3sg} say grass=too \textsc{exist}\\
	\glt `He said: ``There is grass, too.''' \jambox*{\href{http://hdl.handle.net/10050/00-0000-0000-0004-1BD7-2}{[narr3\_3:26]}}
	\label{exe:kamanan}
\end{exe}

The speech verb \textit{taruo} `to say' can only be used in combination with \textit{toni} to introduce speech. 

\begin{exe}
	\ex \gll kiun=a \textbf{taruo} \textbf{toni} mu=nan se ma go-un=at ruon\\
	wife-\textsc{3poss=foc} say say \textsc{3pl}=too {\glse} move\_seawards place-\textsc{3poss=obj} dig\\
	\glt `His wife said that that they also wanted to go down to dig out their place.' \jambox*{\href{http://hdl.handle.net/10050/00-0000-0000-0004-1BA3-3}{[conv10\_20:41]}}
	\label{exe:taruotoni}
\end{exe}

Other speech and perception verbs, like \textit{gonggin} `to know', \textit{gerket} `to ask', \textit{konawaruo} `to forget' and \textit{narasa} `to feel', may introduce speech, thought or sentiment independently or in complex predicates (§\ref{sec:mvctoni}) together with \textit{toni}. Examples of \textit{gerket} `to ask' with and without \textit{toni} are given below.

\begin{exe}
	\ex \gll mu se nau=gerket \textbf{nau=gerket} \textbf{toni} deh ma watko=nin\\
	\textsc{3pl} {\glse} \textsc{recp}=ask \textsc{recp}=ask say \textsc{int.pej} \textsc{3sg} \textsc{prox.loc}=\textsc{neg}\\
	\glt `They asked each other (saying): ``Ah, he isn't here.''' \jambox*{\href{http://hdl.handle.net/10050/00-0000-0000-0004-1BCF-3}{[narr28\_12:29]}}
	\label{exe:gerketoni}
	\ex \gll in se \textbf{gerket} mu se maruan ye\\
	\textsc{1sg.ex} {\glse} ask \textsc{3pl} {\glse} move\_seawards or\\
	\glt `We asked: ``Have they come down yet?'' \jambox*{\href{http://hdl.handle.net/10050/00-0000-0000-0004-1BDB-C}{[narr2\_1:06]}}
	\label{exe:gerketmu}
\end{exe}

These verbs specify what kind of speech, thought or sentiment is introduced by \textit{toni} `say; think; want'. Without a speech or perception verb, the standard reading of \textit{toni} is `say', or sometimes `think', as illustrated in~(\ref{exe:tonin}) below.

The intransitive verb \textit{nafikir} `to think' may introduce thought without help of \textit{toni}. The few natural speech corpus examples all introduce direct speech, illustrated in~(\ref{exe:nafikireh}). There are no combinations of \textit{nafikir} and \textit{toni}. Reported thought is introduced with \textit{toni}, as in~(\ref{exe:tonin}).

\begin{exe}
	\ex \gll ma se \textbf{nafikir} eh tamandi=ten=a bo leng kon=ko=et\\
	\textsc{3sg} {\glse} think \textsc{quot} how={\glten}=\textsc{foc} go village one=\textsc{loc}={\glet}\\
	\glt `He thought: ``How do [I] go to that one village?''' \jambox*{\href{http://hdl.handle.net/10050/00-0000-0000-0004-1BC1-0}{[narr19\_6:06]}}
	\label{exe:nafikireh}
	\ex \gll ma \textbf{toni} in se lalat to\\
	\textsc{3sg} think \textsc{1pl.excl} {\glse} dead right\\
	\glt `She thought we were dead, you know.' \jambox*{\href{http://hdl.handle.net/10050/00-0000-0000-0004-1BBB-2}{[narr40\_3:51]}}
	\label{exe:tonin}
\end{exe}

Another use of \textit{toni} is in combination with irrealis marker \textit{=kin}, where \textit{toni} means `want' or marks future tense, described in §\ref{sec:kin} and~§\ref{sec:mvctoni}.

A shorter way of introducing speech or thought is with iamitive \textit{se} (which was seen in combination with verbs in~\ref{exe:gerketoni}, \ref{exe:gerketmu} and \ref{exe:nafikireh}).

\begin{exe}
	\ex \gll eba an \textbf{se} inye ma Onco=ba\\
	then \textsc{1sg} {\glse} \textsc{int.pej} \textsc{3sg} Onco=\textsc{foc}\\
	\glt `Then I'm like: ``Aaah, it's Onco!''' \jambox*{\href{http://hdl.handle.net/10050/00-0000-0000-0004-1B9F-F}{[conv9\_14:39]}}
	\label{exe:anseinye}
\end{exe}

A variant is with the addition of demonstrative \textit{wandi} `like this', as in~(\ref{exe:wandiema}).

\begin{exe}
	\ex \gll an se \textbf{wandi} eh ema\\
	\textsc{1sg} {\glse} like\_this hey aunt\\
	\glt `I went like this: ``Hey aunt!''' \jambox*{\href{http://hdl.handle.net/10050/00-0000-0000-0004-1BBB-2}{[narr40\_4:56]}}
	\label{exe:wandiema}
\end{exe}	

These can both be combined with a speech verb, as in~(\ref{exe:wandig}).

\begin{exe}
	\ex \gll mu se \textbf{wandi} \textbf{gerket} an ∅=te\\
	\textsc{3pl} {\glse} like\_this ask \textsc{1sg} \textsc{give=imp}\\
	\glt `They asked like this: ``Give me!''' \jambox*{\href{http://hdl.handle.net/10050/00-0000-0000-0004-1BCF-3}{[narr28\_12:22]}}
	\label{exe:wandig}
\end{exe}	

The \is{interjection}interjection \textit{eh} can be used on its own to introduce speech, but may also be used with other devices, as in~(\ref{exe:nafikireh}). In the majority of the corpus instances, it follows \textit{toni}. It is not always clear whether \textit{eh} introduces quoted speech or is part of it. In~(\ref{exe:wandiema}) above, \textit{eh} is translated as `hey' and is part of the quoted speech, because that was more likely in that particular context. In~(\ref{exe:toknein}), it is more likely that \textit{eh} introduces speech, as it is used in an exchange between two people who already have each other's attention.

\begin{exe}
	\ex \gll eh mama se ruon eh ruon ba ki-mun tok na=in\\
	\textsc{quot} mother {\glse} cooked \textsc{quot} cooked but \textsc{2pl-proh} yet eat=\textsc{proh}\\
	\glt `{``}Mom, it's cooked.'' ``Yes, but don't you guys eat [it] yet!''' \jambox*{\href{http://hdl.handle.net/10050/00-0000-0000-0004-1BA2-F}{[conv11\_3:41]}}
	\label{exe:toknein}
\end{exe}	

Direct speech that is referred to without being introduced by linguistic material is common in narratives or short narratives within conversation\is{conversation}. This requires some empathy from the speaker and some shared background knowledge for the addressee to understand what is going on. It is often used when there is a change of whose speech is reported. The first quote is then usually introduced, but the speaker switch is not. In~(\ref{exe:yoin}), the addressee can recognise the speaker switch, for example, by the word \textit{yo} `yes'. (Of course, intonation\is{intonation} and \is{pitch}pitch also play a role in marking quoted speech or a change of speaker. This domain remains a fruitful area for further description.)

\begin{exe}
	\ex \gll Mas toni eh pi tiri ra komet=et yo in se tiri ra\\
	Mas say \textsc{quot} \textsc{1pl.incl} sail move\_path look={\glet} yes \textsc{1pl.excl} {\glse} sail move\_path\\
	\glt `Mas said: ``We sail that way to look.'' ``Yes.'' We sailed that way.' \jambox*{\href{http://hdl.handle.net/10050/00-0000-0000-0004-1BC6-C}{[narr17\_0:51]}}
	\label{exe:yoin}
\end{exe}

Complement clauses are negated in the same way as main clauses, by adding negator \textit{=nin} to the predicate. Consider the examples with \textit{toni} `say' and \textit{kona} `think' in examples~(\ref{exe:matoni}) and~(\ref{exe:ankona}), respectively.

\ea \label{exe:matoni}
\gll ma toni sor nat=\textbf{nin}\\
\textsc{3sg} say fish consume=\textsc{neg}\\
\glt `She said the fish didn't bite. \jambox*{\href{http://hdl.handle.net/10050/00-0000-0000-0004-1BA4-1}{[conv16\_15:45]}}
\z 

\ea \label{exe:ankona}
\gll an kona in barat=\textbf{nin}\\
\textsc{1sg} think \textsc{1pl.excl} descend=\textsc{neg}\\
\glt `I thought we didn't go down.' \jambox*{\href{http://hdl.handle.net/10050/00-0000-0000-0004-1BA4-1}{[conv16\_28:32]}}
\z


\section{Apprehensive constructions}
\label{sec:appreh}
Apprehensive constructions are complex clause constructions expressing fear or apprehension, ``a judgement of undesirable possibility'' \parencite[][224]{verstraete2005}. Kalamang has two types of apprehensive constructions: precautionary constructions and apprehensive constructions with a dedicated morpheme.\is{fear}\is{apprehensive}\is{precautionary}

Precautionary constructions are a way to express a warning against an undesirable event. They consist of a clause expressing a precaution taken (the precautionary situation in \citealt[][298]{lichtenberk1995}) and a clause describing an expected undesirable situation (the apprehension-causing situation in \citealt[][298]{lichtenberk1995}). Precautionary constructions are made with clause linker \textit{mena} `otherwise; in case' (homonymous to a temporal adverb meaning `later').\footnote{\textcite{faller2018} suggest that temporal-anaphoric adverbs are often diachronic sources of apprehensive markers. This also seems to be the case for Kalamang \textit{mena} `later; otherwise; in case', which corresponds to local Malay \textit{nanti} `later', also used in precautionary constructions \parencite[][352]{sneddon2012}.} There are no data to clearly demonstrate that \textit{mena} introduces a subordinate clause, but there is a clear pragmatic dependency between the two clauses it links (see \citealt{faller2018} for a discussion of pragmatic dependency in sentences linked with apprehensive markers cross-linguistically).

Most precautionary constructions are combined with a prohibition, such as in~(\ref{exe:menamalai}), where the precaution taken is not asking a Malay-speaking woman about something, and the undesirable outcome is that she comes to the house using Indonesian (while the speaker and addressee are trying to have a conversation in Kalamang).

\begin{exe}
	\ex \gll ma-mun se koi gerket=in \textbf{mena} ma koi bara malaimang=taet mu me pep mangun met=a\\
	\textsc{3sg-proh} {\glse} again ask=\textsc{proh} otherwise \textsc{3sg} again descend Malay=more \textsc{3pl} {\glme} pig language-\textsc{3poss} \textsc{dist.obj=foc}\\
	\glt `Don't ask again, otherwise she will come down again speaking more Malay, their pig's language.' \jambox*{\href{http://hdl.handle.net/10050/00-0000-0000-0004-1B9F-F}{[conv9\_12:31]}}
	\label{exe:menamalai}
\end{exe}

A prohibition does not need to be explicitly uttered in order for a precautionary construction to be made. (\ref{exe:menakus}) is uttered in the context of a discussion about closing the door during a video recording. The person who is against closing the door uses \textit{mena} to indicate that leaving the door open is the precaution to be taken against the undesirable outcome of having very dark faces in the recording. There is no special meaning associated with the repetition of \textit{mena}.

\begin{exe}
	\ex \gll go sausaun=et me \textbf{mena} kanggirar-pe \textbf{mena} kuskap\\
	condition dark={\glet} {\glme} otherwise eye-\textsc{1pl.incl.poss} otherwise black\\
	\glt `[It's not good] if it is dark, otherwise our faces are black.' \jambox*{\href{http://hdl.handle.net/10050/00-0000-0000-0004-1B9F-F}{[conv9\_17:16]}}
	\label{exe:menakus}
\end{exe}

Precautionary constructions can also be of the in-case type \parencite{lichtenberk1995}. In this type, the speaker warns against a possible but not necessarily expected undesirable outcome.

\begin{exe}
	\ex \gll kawir-ca=at kuet=te \textbf{mena} yuon lalang\\
	hat-\textsc{2sg.poss=obj} bring=\textsc{imp} in\_case sun hot\\
	\glt `Bring your hat, in case the sun is hot!' \jambox*{[elic]}
	\ex \gll pain=at kuet=te \textbf{mena} kalis urun\\
	umbrella=\textsc{obj} bring=\textsc{imp} in\_case rain fall\\
	\glt `Bring an umbrella in case it rains!' \jambox*{[elic]}
\end{exe}	

%
%All precautionary examples except (1) are in combination with the prohibitive. Mena comes between the two clauses, except in (4), where it follows the subject of the second clause. This is uncommon but attested and most probably grammatical.
%
%Precautionary
%1 gosausaunet me mena kanggirarpe mena kuskap
%2 perfect: kamun kinkinin mena ma piat ruan
%3 beautiful: mamun se koi gerkerin mena ma koi bara malaimangdaet mu me pep mangun mera
%4 ki-mun mabu-in, polisi mena piat hukum
%
%Temporal
%`then': ma mara marok o an toni taikon wilak elak yuwatko, mena manan se mengga yali marua
%`later': bes santia kuru era kuar gayam opa me, ma toni mena an kuru era kuaret eba metko

Kalamang also has a dedicated apprehensive mood clitic \textit{=re}, which is attached to the subject of a clause which expresses some kind of danger. Apprehensive mood is described in §\ref{sec:apprmood}.

\begin{exe}
	\ex \gll ka kolko=te wat=\textbf{re} kat kosarat=et\\
	\textsc{2sg} move\_out=\textsc{imp} coconut=\textsc{appr} \textsc{2sg.obj} hit={\glet}\\
	\glt `Move aside, or a coconut might hit you!' \jambox*{\href{http://hdl.handle.net/10050/00-0000-0000-0004-1C60-A}{[elic\_app\_4]}}
	\label{exe:watree}
\end{exe}


\section{Conditional clauses}
\label{sec:condclause}
There are two strategies for forming conditional clauses. The first is with a dedicated conditional clitic \textit{=o}/\textit{=ero} `if' or concessive \textit{=taero} `even if' on the condition, described in §\ref{sec:condmood}. The enclitic \textit{=o} is exemplified in~(\ref{exe:waroobes}). The Malay loan \textit{kalau} `if' was discussed in §\ref{sec:condconj}.

\begin{exe}
	\ex \gll jadi tanaman pun demekian wat=\textbf{o} bes im=\textbf{o} bes sayang=\textbf{o} bes\\
    so plant even thus coconut=\textsc{cond} good banana=\textsc{cond} good nutmeg=\textsc{cond} good\\
    \glt `So whichever plant [we grow], whether it's coconut, banana or nutmeg, it's good.' \jambox*{\href{http://hdl.handle.net/10050/00-0000-0000-0004-1BD0-8}{[narr13\_2:50]}}
\label{exe:waroobes}
\end{exe}
%also elicited ka nanero na, ka muawero muap(te). 

The second strategy for conditionals makes use of a clause-initial scene-setting topic\is{topic}, typically with a time adverbial\is{time adverbial}, as in~(\ref{exe:saunet})\footnote{for the similarities between conditionals and topics see \textcite{haiman1978}.}. It may be marked with \is{topic}topic marker \textit{me} and/or irrealis \textit{=et} (§\ref{sec:et}). The scene-setter is prosodically separate from the main clause. It ends on a high \is{pitch}pitch and is optionally followed by a pause. The scene-setter \textit{kasuret me} in~(\ref{exe:kasuret}), translated as `tomorrow', can be more literally translated as `when it is tomorrow'. The conditional clause may also be followed by sequential conjunction \textit{eba} `then', as in~(\ref{exe:sabeba}). An example with negative condition \textit{ge=et me} `if not' is given in~(\ref{exe:getmehe}).

\begin{exe}
	\ex \glll Nene opa me, go\_saunet, ma war.\\
	nene opa me \textbf{go\_saun=et} ma war\\
	grandmother {\glopa} {\glme} evening={\glet} \textsc{3sg} fish\\
	\glt `That grandmother, when it was evening, she went fishing.' \jambox*{\href{http://hdl.handle.net/10050/00-0000-0000-0004-1BDF-0}{[narr27\_0:09]}}
	\label{exe:saunet}
	\ex \glll Kasuret me kabon, Ambunbon, Serambon, tok bo rorpotma.\\
	\textbf{kasur=et} \textbf{me} ka=bon Ambun=bon Seram=bon tok bo ror-potma\\
	tomorrow={\glet} {\glme} \textsc{2sg=com} Ambun=\textsc{com} Seram=\textsc{com} first go wood-cut\\
	\glt `Tomorrow, you, Ambun and Seram first go wood-cutting.' \jambox*{\href{http://hdl.handle.net/10050/00-0000-0000-0004-1BB3-0}{[narr7\_4:09]}}
	\label{exe:kasuret}
	\ex \glll In tok mara mengga hari sabtuet eba in maruaret.\\
	in tok mara mengga hari \textbf{sabtu=et} \textbf{eba} in maruat=et\\
	\textsc{1pl.excl} first move\_landwards \textsc{dist.lat} day Saturday={\glet} then \textsc{1pl.excl} move\_seawards={\glet}\\
	\glt `We go towards land first, when it's Saturday, then we come towards sea.' \jambox*{\href{http://hdl.handle.net/10050/00-0000-0000-0004-1BCF-3}{[narr2\_12:55]}}
	\label{exe:sabeba}
	\ex \gll al-un kinkinun \textbf{ge=et} \textbf{me} se rasa\\
	string-\textsc{3poss} small no={\glet} {\glme} {\glse} like\\
	\glt `It has small strings, if not, it would have been good already.' \jambox*{\href{http://hdl.handle.net/10050/00-0000-0000-0004-1BB7-9}{[conv19\_17:36]}}
	\label{exe:getmehe}
\end{exe}	

Below are three examples of conditional clauses with irrealis \textit{=et} that are not time adverbials. (\ref{exe:haletme}) is taken from a recording where the speaker teaches the addressee how to weave a pandanus leaf envelope. (\ref{exe:raweteba}) again contains conjunction \textit{eba} `then'. The combination of \textit{=et} and \textit{eba} does not necessarily imply a conditional reading; it must be deduced from the context. Conditional clauses with irrealis \textit{=et} can be translated with either `if' or `when'. In some contexts, such as in~(\ref{exe:orkoet}), either reading seems appropriate.

\begin{exe}
	\ex \gll komahal=\textbf{et} \textbf{me} ukir=te\\
	not\_understand={\glet} {\glme} measure=\textsc{imp}\\
	\glt `If [you] don't understand, measure.' \jambox*{\href{http://hdl.handle.net/10050/00-0000-0000-0004-1BB5-B}{[conv17\_8:52]}}
	\label{exe:haletme}
	\ex \gll ma rap=\textbf{et} \textbf{eba} gier-un iriskap\\
	\textsc{3sg} laugh={\glet} then tooth-\textsc{3poss} white\\
	\glt `When he laughs, his teeth are white.' \jambox*{\href{http://hdl.handle.net/10050/00-0000-0000-0004-1BC1-0}{[narr19\_8:02]}}
	\label{exe:raweteba}
	\ex \gll an or=\textbf{et} mu toni sabar-kadok=a iren, an sabar=ko=\textbf{et} mu toni or-kadok=a iren\\
	\textsc{1sg} back={\glet} \textsc{3pl} say front-side=\textsc{foc} white \textsc{1sg} front=\textsc{loc}={\glet} \textsc{3pl} say back-side=\textsc{foc} white\\
	\glt `If/when I'm in the back they say the one in the front is white, if/when I'm in the front they say the one in the back is white.' \jambox*{\href{http://hdl.handle.net/10050/00-0000-0000-0004-1BC1-0}{[narr19\_6:10]}}
	\label{exe:orkoet}
\end{exe}

Conditional constructions with \textit{=et} and \textit{eba} may be expanded with distal locative \textit{metko} `there'. These are described in §\ref{sec:medisc}.

%also: newer=et eba ra-r=et. (and ki newer=et eba in kahetma-r=et) if you pay, you can go gon. but need not be conditional: pi buokbu=et eba pi garung=et is purely sequential.
