\chapter{Quantifiers}\is{numeral|(}\is{quantifier|(}
\label{ch:quant}
Quantifiers are words that indicate the quantity of the referent of the NP. They were introduced in §\ref{sec:wcquant}. As described in §\ref{sec:nmodqnt}, quantifiers take the first slot after the noun in the NP. This chapter looks at the properties of quantifiers beyond their distributional properties. It considers cardinal, ordinal and collective numerals (§\ref{sec:cardnum}), numeral classifiers (§\ref{sec:clf}), non-numeral quantifiers (§\ref{sec:quantall}), and the inflections quantifiers may carry (§\ref{sec:quantinfl}).

\section{Numerals}\is{numeral}
\label{sec:cardnum}
Cardinal numerals are numerals that are used in counting, which express exact quantities. There are unique terms for numerals 1--9 (possibly with the exception of `seven'; see below). Tens are formed with \textit{put-} and a numeral from 1 to 9. Numerals 11–19 and 21–29 are formed with \textit{putkon} `ten' or \textit{purir} `twenty', followed by the linker \textit{ba}, followed by a numeral from 1 to 9. Numerals 31–39, 41–49 etc. are formed with the linker \textit{talin}. Hundreds are formed with \textit{reit}, thousands with \textit{ripi}, millions with \textit{juta} and billions with \textit{miliar}. An overview of the building blocks of Kalamang numerals is given in Table~\ref{tab:numsall}, where `+' stands for another numeral. A space between a number or linker and `+' indicates separate phonological words, whereas a lack of a space indicates that the numeral is a single phonological word. For example, `one thousand', formed with \textit{ripi} and \textit{kon}, is one phonological word: \textit{ripion}, while `one million', formed with \textit{juta} and \textit{kon}, consists of two phonological words: \textit{juta kon}. To express `zero', the negative existential \textit{saerak} is used.

\begin{table}[ht]
	\caption{Numeral quantifiers}
	\label{tab:numsall}
		\begin{tabular}{l l}
			\lsptoprule 
			1 & \textit{kon}\\
			2 & \textit{eir}\\
			3 & \textit{karuok}\\
			4 & \textit{kansuor}\\
			5 & \textit{ap}\\
			6 & \textit{raman}\\
			7 & \textit{ramandalin}\\
			8 & \textit{irie}\\
			9 & \textit{kaninggonie}\\
			10& \textit{putkon} \\
			11--19 & \textit{putkon ba +} \\
			20 & \textit{purir}\\
			21--29 & \textit{purir ba +}\\
			31+ & \textit{put+ talin+}\\
			tens of& \textit{put+} \\
			hundreds of& \textit{reit+} \\
			thousands of& \textit{ripi+} \\
			millions of & \textit{juta +}\\
			billions of & \textit{miliar +}\\\lspbottomrule 
		\end{tabular}
	
\end{table}

%\textit{Put-} for multiples of ten, as well as a number seven derived from six, are areal features of the Austronesian\is{Austronesian} languages in West Bomberai\is{West Bomberai}, found as close as in Uruangnirin\is{Uruangnirin!loan} \parencite{schapper2013}. Note that Iha\is{Iha} and Mbaham\is{Mbaham}, Kalamang's sister languages, have a quinary system \parencite{flassy1987,flassy1992}. %Also some of the TAP lang's, see vol 1 and 2 and peraps the AP chapter on numerals.
%Possible cognate forms with Iha and Mbaham are one (Mbaham \textit{oqono}), two (Iha \textit{rik}, Mbaham \textit{wriik}) and eight (Iha \textit{teri}). Cf. also proto-AP *nuk `one' and *araqu `two' \parencite{klamer2017}, which is reminiscent of \textit{karuok} `three'. Note also Uruangnirin \textit{taran-sa} `seven' (\textit{sa} `one') and \textit{teri-nua} `eight' (\textit{nua} `two'). \textit{Taran-} and \textit{raman} seem related, meaning `six' (proto-AP *talam `six', \citealt{klamer2017}). \textit{Teri-} in Uruangnirin `eight' and Iha \textit{teri} `eight' seem also related, and remind a bit of Kalamang \textit{irie} `eight'. An overview is given in Table~\ref{tab:numcogn}.
%
%\begin{table}
%	\caption{Possible cognates with Kalamang numerals}
%	\label{tab:numcogn}
%	
%		\footnotesize
%		\begin{tabular}{l l l l l l}
%			\hline 
%			& Kalamang& Iha & Mbaham & proto-AP & Uruangnirin\\ \hline 
%			1 & \textit{kon} &&\textit{oqono}&*nuk(u?) &\\
%			2 & \textit{eir} & \textit{rik} & \textit{wriik} & *araqu&\\
%			6 & \textit{raman} &&&*talam&\\
%			7 & \textit{raman-dalin} &&&& \textit{taran-sa} `seven'\\ 
%			8 & \textit{irie} & \textit{teri} & &&\textit{teri-nua} `eight'\\ \hline		
%		\end{tabular}
%	
%\end{table}

Numerals like \textit{ramandalin} and \textit{kaninggonie} are likely decomposable into the morphemes \textit{raman-talin} and \textit{kanin-kon-ie} (cf. \textit{ir-ie} `eight'). Of these, the numerals \textit{raman} `six' and \textit{kon} `one' are easily recognisable. \textit{Talin} may mean something like `further' or `extra', which explains both its use as a linker for numerals higher than thirty and its use in `seven'.\footnote{7 = 6 + 1 is found in several languages of the area, among which are the Aru languages, Onin, Sekar and Uruangnirin \parencite{schapper2013}.} The meaning and origin of \textit{kanin} and \textit{ie} are unknown. Note that the conjunction \textit{ba}, used in numerals between 11 and 29, is a common conjunction in Kalamang, not limited to numerals (see §\ref{sec:ba}).

The higher numerals are all loans from Austronesian\is{Austronesian}. \textit{Reit-} `hundred' is related to PMP *RaCus, and \textit{ripi-} `thousand' is related to PMP *Ribu. Cf. also Iha, Mbaham and Uruangnirin, which all use \textit{rati} for `hundred' and \textit{ripi} for `thousand'. Kalamang \textit{juta} `million' and \textit{miliar} `billion' are unchanged loans from Malay. The collective numeral \textit{salak} might be related to Indonesian \textit{se-laksa} `ten thousand', in East-Indonesian pronunciation \textit{sa-laksa} (\textit{se-} and \textit{sa-} from \textit{satu} `one').

The base for numerals between 11 and 99 is \textit{put-}. Numerals 11–19 and 21–29 are formed as \textit{put-} + numeral + \textit{ba} + numeral. For numbers higher than thirty, the linker for the tens and the ones is \textit{talin}, so that we get \textit{put-} + numeral + \textit{talin-} + numeral. A few clarifying examples are given in~(\ref{exe:tens}).

\begin{exe}
	\ex
	\begin{tabbing}
		\hspace*{1.5cm}\=\hspace*{5cm}\=\kill
		11 \> \textit{putkon ba kon}\\
		23 \> \textit{purir ba karuok}\\
		35 \> \textit{putkaruok talinap}\\
		57 \> \textit{purap talinramandalin}\\
		98 \> \textit{putkaninggonie talinirie}
	\end{tabbing}
	\label{exe:tens}
\end{exe} 

More complex and higher numerals are formed as follows. The number is divided into millions, thousands, hundreds and tens, which are given in that order. In tens and hundreds of thousands, the thousands are grouped. That is, 72,000 is not rendered as `seventy thousand and two thousand', but as `seventy and two thousand'. Linkers \textit{talin} and \textit{ba} are used only for tens and ones (including tens and ones of thousands). \textit{Reit}, \textit{ripi}, \textit{juta} and \textit{miliar} cannot stand on their own. For example, `one hundred' is \textit{reitkon}, not *\textit{reit}. Note that complex high numerals — although speakers have no trouble producing them — had to be elicited and are rarely if ever used in daily life. 

\begin{exe}
	\ex
	\begin{footnotesize}
	\begin{tabbing}
		\hspace*{2cm}\=\hspace*{5cm}\=\kill
		2456\> \textit{ripir reitkansuor purap talinraman}\\
		\> `two thousand four hundred fifty and six'\\
		8721\> \textit{ripirie reitramandalin purir ba kon}\\
		\> `eight thousand seven hundred twenty and one'\\
		72,568 \> \textit{ripi putramandalin talinir reirap putraman talinirie}\\
		\> `seventy and two thousand five hundred sixty and eight'\\
		526,389 \> \textit{ripi reirap purir ba raman reitkaruok putirie talinkaninggonie}\\
		\> `five hundred and twenty and six thousand three hundred eighty \\
		\> and nine'\\
		1,500,000 \> \textit{juta kon ripi reirap}\\
		\> `one million five hundred thousand'
	\end{tabbing} 
	\end{footnotesize}
	\label{exe:compnums}
\end{exe}

In six-digit numerals, no distinction is made between the ten thousands and the tens, i.e. numerals such as 520,000 and 500,020 are expressed in the same way: \textit{ripi reirap purir} `five hundred twenty thousand' (lit. `thousand five hundred twenty').\footnote{There might be an intonational difference, but this was not tested for.} The difference between numbers from 1000 to 1999 (with \textit{ripion} `one thousand') and those involving thousands (\textit{ripi} `thousand') is illustrated in~(\ref{exe:contrnums}).

\begin{exe}
	\ex
	\begin{tabbing}
		\hspace*{2cm}\=\hspace*{5cm}\=\kill
		1050\> \textit{ripion purap}\\
		\> `one thousand fifty'\\
		50,000\> \textit{ripi purap}\\
		\> `fifty thousand'\\
		1100 \> \textit{ripion reitkon}\\
		\> `one thousand one hundred'\\
		100,000 \> \textit{ripi reitkon}\\
		\> `one hundred thousand'
	\end{tabbing} 
	\label{exe:contrnums}
\end{exe}

Years (as in `the year 1973') are expressed in the same way as numerals, preceded by \textit{tanggon} `year'. To say `X years', the numeral is suffixed to the noun.

There are no ordinal\is{numeral!ordinal} numbers that are derived from cardinal numbers. `First' is expressed with the verb \textit{borara} `to be first'. There are no ways to say second, third, etc.: all subsequent entities following `first' are \textit{pareirun} `the following' (nominalised from \textit{pareir} `to follow'). The last one in a sequence can be referred to with the root \textit{ep-} `behind'. One can say, for example, \textit{an epka} `I come last' (lit. `I come from the back') with a lative marker on \textit{ep-}, or \textit{tumunan epko/epkadok} `my child is the last' with a locative marker or \textit{-kadok} `side' on \textit{ep-}. An illustration is given in~(\ref{exe:ord}).

\begin{exe}
	\ex
	{\gll ma koi \textbf{ep=ka} luk=ta me eh \textbf{boraran}\\
		\textsc{3sg} then back=\textsc{lat} come={\glte} {\glme} \textsc{hes} first\\
		\glt `He came last, I mean, first.' \jambox*{\href{http://hdl.handle.net/10050/00-0000-0000-0004-1B9F-F}{[conv9\_23:37]}}
	}
	\label{exe:ord}
\end{exe}

A collective numeral\is{numeral!collective} indicates that several entities are seen as a unit and not as individuals. There is one collective numeral in Kalamang: \textit{salak} `ten thousand'. Not used in counting, \textit{salak} is a collective numeral used, for example, for trading goods such as nutmeg. An example is given in~(\ref{exe:salak}).

\begin{exe}
	\ex
	{\gll pi bo rep=et me sampi \textbf{salak}\\
		\textsc{1pl.incl} go get={\glet} {\glme} until ten\_thousand\\
		\glt `We harvest up to ten thousand.' \jambox*{\href{http://hdl.handle.net/10050/00-0000-0000-0004-1BF1-6}{[narr12\_3:46]}}
	}
	\label{exe:salak}
\end{exe}

\textit{Salak} may be combined with a cardinal as in~(\ref{exe:salaon}).

\begin{exe}
	\ex
	{\gll musim kon-i me \textbf{salak-kon}=et\\
		season one-\textsc{objqnt} {\glme} ten\_thousand-one=\glet\\
		\glt `One season, ten thousand.' \jambox*{\href{http://hdl.handle.net/10050/00-0000-0000-0004-1BF1-6}{[narr12\_3:49]}}
	}
	\label{exe:salaon}
\end{exe}

The form \textit{salak} is probably related to Indonesian \textit{sa-laksa} `ten thousand'.


\subsection{Classifiers}\is{classifier|(}
\label{sec:clf}




\begin{table}[t]
	\small  
	\caption{Classifiers}
	\label{tab:class}
		\begin{tabularx}{\textwidth}{llQp{2cm}}
			\lsptoprule
			classifier & gloss & used for & also used\newline  for/as\\
			\midrule 
			\textit{al-} & \textsc{clf\_strip} & strips or strings of (natural) material & part-of-whole\\
			\textit{ar-} & \textsc{clf\_stem} & all trees, plants and rope, as well as \textit{kewe} `house' and \textit{paden} `pole'&part-of-whole\\
			\textit{ep-} & \textsc{clf\_group} & groups of animates, e.g. a school of fish, a group of children&\\
			\textit{et-} & \textsc{clf\_an} & all animals, including fish and birds&\\
			\textit{kis-}& \textsc{clf\_long} & long thin things, such as cigarettes, strips of leaf for weaving, and construction materials like planks and beams&\\
			\textit{mir-}&  \textsc{clf\_canoe} &\textit{et} `canoe'&\\
			\textit{nak-}& \textsc{clf\_fruit1} & certain fruits, vegetables and roots, such as citrus fruit, breadfruit, aubergine, tomato and carrot&part-of-whole\\
			\textit{nar-}& \textsc{clf\_round} & small oval or round objects, such as eggs, seeds and candy&\\ %not pow, cf. kokoknar.
			\textit{pel-}& \textsc{clf\_comb} & `comb', for bananas&\\
			\textit{poup-}& \textsc{clf\_bundle} & bundles of e.g. long green beans&part-of-whole\\
			\textit{pur-}& \textsc{clf\_piece} & pieces of e.g. fish, vegetable or wood&\\
			\textit{rur-}& \textsc{clf\_skewer} & strung or skewered things, e.g. fish on a string or skewer& verb\newline `to skewer'\\
			\textit{tabak-} & \textsc{clf\_half} & things cut cross-wise, containers filled half, half-smoked cigarettes& noun `shortly cut piece'\\
			\textit{tak-}& \textsc{clf\_leaf} & thin, flat things such as leaves, sheets of paper, paper money, planks, triplex board and corrugated iron&part-of-whole\\
			\textit{tang-}& \textsc{clf\_seed} & nuts and some fruits and legumes, such as tomato, pili nuts, nutmeg, tamarind, beans, peanuts and Tahitian chestnut&part-of-whole\\
			%			\textit{taur}& for heaps of e.g. coconuts & part-of-whole\\ seems to be indep noun.
			\textit{tep-}& \textsc{clf\_fruit2} & `fruit', for e.g. bananas, nutmeg, mangoes, rose-apple&part-of-whole\\
			\lspbottomrule
		\end{tabularx}
	
\end{table}

A classifier gives information about the classification of a noun. Kalamang classifiers, which are numeral prefixes, occupy the quantifier slot together with a numeral when modifying certain classes of nouns. When those nouns are modified by a numeral, the use of a classifier is obligatory. They can also be prefixed to the question word \textit{puraman} `how many'. There are 16 classifiers, listed in Table~\ref{tab:class}. Those that are also bound roots that express part-whole relations (§\ref{sec:inal}) are marked accordingly. There are two unique classifiers (i.e. classifiers that apply to only one noun: \citealt{grinevald2007}): \textit{mir-}\footnote{The West Bomberai language Mbaham has a word \textit{muur} `branch' \parencite{flassy1987} and a transport classifier \textit{mu-} \parencite{cottet2015}, and Iha has a classifier \textit{mur} for boats, `motor' and branches (Katherine Walker, p.c.).} for the noun \textit{et} `canoe' and \textit{pel-} for the noun \textit{im} `banana'.\footnote{While I elicited negative grammaticality judgments for \textit{mir-} with, for example, other modes of transport or other things made of wood, I have not had the chance to test \textit{pel-} in combination with nouns other than `banana'. In other words, the status of \textit{pel-} as a unique classifier is based on its lack of appearance in combination with nouns other than `banana' in the current corpus.} An example with the classifier \textit{kis-} for long thin things on the numeral \textit{kon} `one', modifying the object \textit{tabai} `cigarette', is given in~(\ref{exe:kiskon}). An example with the fruit classifier \textit{nak-} and \textit{puraman} `how many' is given in~(\ref{exe:watnak}).\is{noun!classification}\is{gender*}
%flassy iha -mur bendah yang pipih
%falssy baham muur batang



\begin{exe}
	\ex
	{\gll ma se tabai=at \textbf{kis}-kon-i jien\\
	\textsc{3sg} {\glse} cigarette=\textsc{obj} \textsc{clf\_long}-one-\textsc{objqnt} get\\
	\glt `He got one cigarette.' \jambox*{\href{http://hdl.handle.net/10050/00-0000-0000-0004-1BD7-2}{[narr3\_12:04]}}}
	\label{exe:kiskon}
	\ex \gll wat \textbf{nak}-puraman-i mindi kajie\\
	coconut \textsc{clf\_fruit1}-how\_many-\textsc{objqnt} like\_that pick\\
	\glt `We picked up I-don't-know-how-many coconuts like that.' \jambox*{\href{http://hdl.handle.net/10050/00-0000-0000-0004-1BA2-F}{[conv11\_4:50]}}
	\label{exe:watnak}
\end{exe}


Many nouns that are modified with a numeral are not attested with a classifier. Examples include all nouns referring to persons (unless they are in a group and following each other, in which case the group classifier \textit{ep-} is used), shells as in~(\ref{exe:kanye}), landscape features like \textit{lempuang} `island' and celestial bodies like \textit{pak} `moon'. The latter two categories are perhaps not surprising, since they are less likely to be quantified with an exact number. Other nouns associate with more than one classifier (though not at the same time), depending on which characteristic of the nominal referent is being emphasised. This way, classifiers help specify whether we are talking about the leaves, the stem or the fruit of a plant, or whether we are talking about a halved fish, fish as single entities, fish on a string or schools of fish. To take another example, \textit{mun} `lime' can be modified with the classifier for halves \textit{tabak-} if it is cut cross-wise, or with the fruit classifier \textit{nak-} if it is whole. Consider also the examples with \textit{sayang} `nutmeg' in~(\ref{exe:sayangya}):

\begin{exe}
	\ex 
	\begin{xlist}
		\ex \gll sayang \textbf{ar}-kon\\
		nutmeg \textsc{clf\_stem}-one\\
		\glt `one nutmeg tree'
		\ex \gll sayang \textbf{tang}-kon\\
		nutmeg \textsc{clf\_seed}-one\\
		\glt `one nutmeg [seed]'
		\ex \gll sayang \textbf{tep}-kon\\
		nutmeg \textsc{clf\_fruit2}-one\\
		\glt `one nutmeg [fruit]'
	\end{xlist}
	\label{exe:sayangya}
\end{exe}
%elicit 2020

There are three classifiers for plant products: \textit{nak-}, \textit{tang-} and \textit{tep-}. \textit{Nak-} occurs with a range of fruits, vegetables and roots. \textit{Tang-}, which as a part-of-whole noun means `seed' (§\ref{sec:inal}), is for nuts, legumes and some other fruits. \textit{Tep-} is used for a range of fruits. Some fruits, like \textit{tamatil} `tomato', can be classified with two classifiers: \textit{nak-} and \textit{tang-}.

While classifiers are prefixes to numerals, there are three nouns in the corpus that take numerals as suffixes. These are \textit{wan} `time', \textit{pak} `month' and \textit{tanggon} `year'. Of these three, only \textit{wan} `time' is a bound root. Like classifiers, it cannot occur independently, unless followed by a number or by \textit{puraman} `how many'. \textit{Pak} and \textit{tanggon} are words. Unlike classifiers, these forms do not modify another noun.

\begin{exe}	
	\ex
	{\glll \textbf{Wanggaruok} ye \textbf{wanggansuor} masaret ma se kararak.\\
		wan-karuok ye wan-kansuor masat=et ma se kararak\\
		time-three or time-four dry={\glet} \textsc{3sg} {\glse} dry\\
		\glt `Dry [it] three or four times, it's already dry.' \jambox*{\href{http://hdl.handle.net/10050/00-0000-0000-0004-1BF1-6}{[narr12\_5:24]}}
	}
	\label{exe:wan}
	\ex
	{\glll Mungkin \textbf{paruok} ye \textbf{pansuor} ye, ah, mindi.\\
		mungkin pak-karuok ye pak-kansuor ye ah mindi\\
		maybe month-three or month-four or \textsc{int} like\_that\\
		\glt `Maybe three or four months, like that.' \jambox*{\href{http://hdl.handle.net/10050/00-0000-0000-0004-1BD0-8}{[narr13\_2:46]}}
	} 
	\label{exe:pak}	
	\ex
	{\glll \textbf{Tanggonggaruok} koyeret me se... \\
		tanggon-karuok koyet=et me se\\
		year-three finish={\glet} {\glme} {\glse}\\	
		\glt `After three years...' \jambox*{\href{http://hdl.handle.net/10050/00-0000-0000-0004-1BD8-4}{[narr1\_6:21]}}
	} 
	\label{exe:tanggongg}	
\end{exe}\is{classifier|)}

\subsection{Other structural properties of numeral quantifiers}
\label{sec:othernum}
Kalamang has a fraction-like operation marked with \textit{taikon}, which literally means `one side' but can be used to mean `half', as in~(\ref{exe:taikon}). Indigenous ways of doing arithmetic operations are so far unattested.\is{arithmetic operation!fraction}

\begin{exe}
	\ex
	{\gll koi mun \textbf{taikon}\\
		then lime half\\	
		\glt `Then half a lime...' \jambox*{\href{http://hdl.handle.net/10050/00-0000-0000-0004-1BD6-8}{[stim38\_10:12]}}
	} 
	\label{exe:taikon}	
\end{exe}


Numerals can be juxtaposed, except when counting, to make an estimation of the number of referents.

\begin{exe}
	\ex \gll tik-un jumat \textbf{kon} \textbf{eir} ki-mun an=at sanggara=in\\
	long-\textsc{nmlz} Friday one two \textsc{2pl-proh} \textsc{1sg=obj} search=\textsc{proh}\\
	\glt `For one or two Fridays, don't you search for me.' \jambox*{\href{http://hdl.handle.net/10050/00-0000-0000-0004-1C98-7}{[narr26\_7:32]}}
	\label{exe:koneir}
\end{exe}
%also kon eirtaero. not checked for other nums than 1-2.

Alternatively, estimations are expressed with \textit{ye} `or' in between the numerals, as exemplified in~(\ref{exe:wan}) and~(\ref{exe:pak}) above.\is{numeral|)}

\section{Non-numeral quantifiers}\is{quantifier!non-numeral|(}
\label{sec:quantall}
As introduced in §\ref{sec:wcquant}, Kalamang has six non-numeral quantifiers. They are listed in Table~\ref{tab:quant2}. 

\begin{table}
	\caption{Non-numeral quantifiers}
	\label{tab:quant2}
	
		\begin{tabular}{l l}
			\lsptoprule 
			\textit{bolon} & a little \\
			\textit{taukon} & some\\
			\textit{ikon}& some \\
			\textit{reidak} & much/many\\
			\textit{reingge} & not much/many\\
			\textit{tebonggan} & all \\\lspbottomrule		
		\end{tabular}
	
\end{table}

\textit{Bolon} occurs with non-count\is{count noun} referents. It is the only non-numeral quantifier that co-occurs with \textit{-tak} `only, just', as illustrated in~(\ref{exe:buoksarun}). Like with \textit{kon} `one' + \textit{-tak}, which becomes \textit{kodak}, the final nasal of the root is deleted and the plosive is voiced: \textit{bolon} + \textit{-tak} = \textit{bolodak}.

\begin{exe}
	\ex
	{\gll mu buoksarun=at paruo ba \textbf{bolodak} to\\
		\textsc{3pl} offering=\textsc{obj} make but little\_only right\\
		\glt `They are making the offering, but just a little, right.' \jambox*{\href{http://hdl.handle.net/10050/00-0000-0000-0004-1BB3-0}{[narr7\_0:49]}}}
	\label{exe:buoksarun}
\end{exe}

\textit{Taukon} `some' and \textit{ikon} `some' appear to have the same meaning, although the former only occurs seven times in the natural spoken corpus, whereas the latter has 28 occurrences. Both can be used with animate and inanimate referents. They are illustrated modifying an animate noun in~(\ref{exe:tumtau}) and~(\ref{exe:murkon}). It is likely that these quantifiers were originally morphologically complex (cf. words like \textit{kon} `one', \textit{tawir} `side'\footnote{maybe \textit{tau} `?' + \textit{eir} `two', although `two sides' is \textit{tawirir}} and \textit{taikon} `half; one side').

\begin{exe}
	\ex \gll o tumtum \textbf{taukon} me Bobi emun=a kona\\
	\textsc{emph} children some {\glme} Bobi mother=\textsc{foc} see\\
	\glt `O, some children, Bobi's mother saw them.' \jambox*{\href{http://hdl.handle.net/10050/00-0000-0000-0004-1BCE-D}{[conv4\_5:09]}}
	\label{exe:tumtau}
	\ex \gll emumur \textbf{ikon} toni a ma se me\\
	woman.\textsc{pl} some say \textsc{int} \textsc{3sg} {\glse} {\glme}\\
	\glt `Some women said: ``Ah, that's it.''' \jambox*{\href{http://hdl.handle.net/10050/00-0000-0000-0004-1BCC-E}{[conv2\_9:51]}}
	\label{exe:murkon}
\end{exe}	

\textit{Reidak} `much; many' and \textit{reingge} `not much; not many' each consist of two morphemes. The first, \textit{rei}, may be related to the numeral building block \textit{reit-} `hundred'. The second morpheme in \textit{reingge} is a prenasalised\is{prenasalisation} \textit{ge} `no' (see §\ref{sec:prenas} on remnants of prenasalisation and §\ref{sec:negnonv} on negation). The second morpheme in \textit{reidak} is reminiscent of \textit{-tak} `just, only'. The fact that \textit{-tak} (or \textit{-dak}) is found on numerals and on \textit{bolon} `a little' suggests that \textit{-dak} in \textit{reidak} is the same morpheme, unique to the quantifier word class.

\begin{exe}
	\ex 
	{\gll sontum \textbf{reidak} toni mu\\
		person many say \textsc{3pl}\\
		\glt `Many people say they [...].' \jambox*{\href{http://hdl.handle.net/10050/00-0000-0000-0004-1BA4-1}{[conv16\_1:04]}}
	}
	\label{exe:sontumreidak}
	\ex 
	{\gll sontum \textbf{reingge} opa me sinara=at paruon\\
		person not\_many {\glopa} {\glme} offering=\textsc{obj} do\\
		\glt `Those few people did the offering.' \jambox*{\href{http://hdl.handle.net/10050/00-0000-0000-0004-1BB3-0}{[narr7\_1:45]}}
	}
	\label{exe:sontumreingge}
\end{exe}

Although Kalamang has a quantifier \textit{tebonggan} `all', the construction Verb\textit{-i koyet} can be used to express the same meaning. This construction is also a completive (§\ref{sec:compl}), and refers in its quantifier use to a totality of referents being affected. In contrast to the use of the construction with completive events, it can be negated when it is used to express `all'. 

\begin{exe}
	\ex \gll wa me elak∼lak=ko=\textbf{i} \textbf{koyet} paden-un saerak\\
	\textsc{prox} {\glme} bottom∼\textsc{red=loc}={\gli} finish pole-\textsc{3poss} \textsc{neg\_exist}\\
	\glt `The [one] has everything at the bottom, there are no poles.' \jambox*{\href{http://hdl.handle.net/10050/00-0000-0000-0004-1BB3-0}{[stim40\_2:52]}}
	\ex \gll mu tok nan=\textbf{i} \textbf{koyet=nin} mu tok karuar keit=ko\\
	\textsc{3pl} yet consume={\gli} finish=\textsc{neg} \textsc{3pl} still drying\_rack top=\textsc{loc}\\
	\glt `They had not yet eaten everything, they still [had food] on the drying rack.' \jambox*{\href{http://hdl.handle.net/10050/00-0000-0000-0004-1BDB-C}{[narr28\_6:45]}}
	\label{exe:koyetnin}
\end{exe}

This construction may be used in combination with suffixes and words that express `all', such as the nominal suffix \textit{-mahap}, the pronominal suffix \textit{-naninggan} and the quantifier \textit{tebonggan} (illustrated in~\ref{exe:mmah}--\ref{exe:nggan}). Although this makes it possible to combine the meaning `all' with the completive aspect, there are no clear examples where this is the case. It is difficult to tease the two meanings apart: when a totality of referents is affected, a completive reading is often possible.

\begin{exe}
	\ex \gll sontum-\textbf{mahap} taluk=te kome=\textbf{i} \textbf{koyet}\\
	person-all come\_out={\glte} look={\gli} finish\\
	\glt `Everyone came out to look.' \jambox*{\href{http://hdl.handle.net/10050/00-0000-0000-0004-1BDE-7}{[narr25\_6:58]}}
	\label{exe:mmah}
	\ex \gll in-\textbf{naninggan} kiem-\textbf{i} kelek=ko \textbf{koyet} mu leng-un=ko kiem\\
	\textsc{1pl.excl}-all flee={\gli} mountain=\textsc{loc} finish \textsc{3pl} village-\textsc{3sg=loc} flee\\
	\glt `We all fled to the mountains, they from the village (also) fled.' \jambox*{\href{http://hdl.handle.net/10050/00-0000-0000-0004-1BBB-2}{[narr40\_2:15]}}
	\ex \gll \textbf{tebonggan} muin=bon=\textbf{i} \textbf{koyet} {\ob}...{\cb} tamandi=et muap\\
	all \textsc{3poss=com}={\gli} finish {} how={\glet} eat\\
	\glt `Everyone had theirs [... otherwise] how [could they] eat.' \jambox*{\href{http://hdl.handle.net/10050/00-0000-0000-0004-1BAD-2}{[narr29\_5:45]}}
	\label{exe:nggan}
\end{exe}

Kalamang also has two negative polarity items \textit{-barak} `any' and \textit{don kon∼kon} `any' which are described in §\ref{sec:negpol}.

\is{quantifier!non-numeral|)}

\section{Quantifier inflection}\is{quantifier!inflection}
\label{sec:quantinfl}
Quantifiers may be inflected in a number of ways except for its use with classifiers as described in §\ref{sec:clf}. Suffixes and clitics are only attested with lower numerals and \textit{bolon} `little'. Numeral quantifiers and pronouns inflected with a numeral may carry the suffix \textit{-gan} `all' as shown in~(\ref{exe:karugan}) and~(\ref{exe:rgan}). The enclitic \textit{=tak} `just; only' (example~\ref{exe:rtak}) is found on the numeral two, pronouns inflected with a numeral, and (fossilised) in \textit{bolodak} `just a little' and \textit{kodak} `just one'. \is{intensification}Intensification with \textit{=tun} is found with \textit{bolon} `little' (example~\ref{exe:sei}) and \textit{kodak} `just one' (example~\ref{exe:kodakk}). \textit{Tebonggan} `all' seems to contain the morpheme \textit{-gan} `all'. While \textit{tebon} cannot be used on its own, it is a root that can be reduplicated and intensified with \textit{=tun} as in~(\ref{exe:tebonteb}).

\begin{exe}
	\ex \gll gorun karuok-\textbf{gan} kodak-pis\\
	stalk three-all just\_one-side\\
	\glt `All three stalks are on one side.' \jambox*{\href{http://hdl.handle.net/10050/00-0000-0000-0004-1BD6-8}{[stim38\_11:02]}}
	\label{exe:karugan}
	\ex \gll inier-\textbf{gan} arekmang\\
	\textsc{1du.ex}-all be\_mad\\
	\glt `Both of us were mad.' \jambox*{\href{http://hdl.handle.net/10050/00-0000-0000-0004-1BA2-F}{[conv11\_5:40]}}
	\label{exe:rgan}
	\ex \gll an bara komet=ta me kies-eir=\textbf{tak}\\
	\textsc{1sg} descend look={\glta} {\glme} \textsc{clf\_long}-two=only\\
	\glt `I went down to look; [there were] only two pieces.' \jambox*{\href{http://hdl.handle.net/10050/00-0000-0000-0004-1BA3-3}{[conv10\_16:10]}}
	\label{exe:rtak}
	\ex \gll ma mat sei bolon∼bolon=\textbf{tun}\\
	\textsc{3sg} \textsc{3sg.obj} askew little∼\textsc{ints=ints}\\
	\glt `He is a tiny bit askew from it.' \jambox*{\href{http://hdl.handle.net/10050/00-0000-0000-0004-1BC8-8}{[stim26\_7:36]}}
	\label{exe:sei}
	\ex \gll ma-autak kodak∼dak=\textbf{tun}\\
	\textsc{3sg}-alone just\_one∼\textsc{ints=ints}\\
	\glt `He was all alone.' \jambox*{\href{http://hdl.handle.net/10050/00-0000-0000-0004-1BC3-B}{[conv7\_8:29]}}
	\label{exe:kodakk}
	\ex \gll esun=kin tebon∼tebon=\textbf{tun} mu don kon∼kon paning=nin\\
	father.\textsc{3poss}=\textsc{poss} all∼\textsc{ints}=\textsc{ints} \textsc{3pl} thing one∼\textsc{red} ask=\textsc{neg}\\
	\glt `From his father's side everyone didn't ask for anything.' \jambox*{\href{http://hdl.handle.net/10050/00-0000-0000-0004-1BCF-3}{[narr2\_6:43]}}
	\label{exe:tebonteb}
\end{exe}

Both numeral and non-numeral quantifiers may be reduplicated\is{reduplication!numerals}. The non-numeral quantifiers that are found reduplicated in the corpus are \textit{bolon} `little' and \textit{tebonggan} `all'. These have already been exemplified in combination with \textit{=tun} `very' in~(\ref{exe:tebonteb}) and~(\ref{exe:sei}). \textit{Bolon} `little' is illustrated in~(\ref{exe:bolbol}) without \textit{=tun} `very'.

\begin{exe}
	\ex \gll tok bolon∼bolon\\
	still little∼\textsc{ints}\\
	\glt `A little bit more.' \jambox*{\href{http://hdl.handle.net/10050/00-0000-0000-0004-1B93-C}{[conv14\_7:47]}}
	\label{exe:bolbol}
\end{exe}	

The reduplication of numeral quantifiers creates \is{distributive}distributives.

\begin{exe}
	\ex \gll kiel-un jien=i koyet kirakira neba {\ob}...{\cb} potma kies-\textbf{kan∼san∼suor} ye\\
	root-\textsc{3poss} get={\gli} finish approximately \textsc{ph} {} cut \textsc{clf\_long}-four∼\textsc{distr} or\\
	\glt `After getting its root, [you] eh, cut about four long pieces.' \jambox*{\href{http://hdl.handle.net/10050/00-0000-0000-0004-1BCA-4}{[conv20\_1:53]}}
	\ex \gll koi tanbes=kin=at bor=taet \textbf{purir-ba-ka∼ra∼ruok}\\
	then right\_side=\textsc{poss=obj} drill=again twenty-\textsc{num.lnk}-three∼\textsc{distr}\\
	\glt `Then [I] drilled the right side, twenty-three [holes] on each side.' \jambox*{\href{http://hdl.handle.net/10050/00-0000-0000-0004-1BDD-5}{[narr42\_11:31]}}
	\ex \gll kanien kanien o poup-un wa∼ra∼rip ukir-te sen \textbf{putkon∼kon}\\
	tie tie \textsc{emph} bundle-\textsc{3sg}  \textsc{prox.qlt}∼\textsc{distr} measure={\glte} cent ten∼\textsc{distr}\\
	\glt `Tying, bundles this big each, measure [for the price of] ten cents each.' \jambox*{\href{http://hdl.handle.net/10050/00-0000-0000-0004-1BC1-0}{[narr19\_1:16]}}
\end{exe}

In addition, reduplication of \textit{kon} `one' has indefinite-like meanings\is{indefinite pronoun}. The use of \textit{konkon} with a negated verb and combined with \textit{don} `thing' so that we get \textit{don konkon Verb=\textsc{neg}} results in the meaning `nothing', as exemplified in~(\ref{exe:dkonkon1}). \textit{Konkon=nin} can also be used predicatively, inflected with the negator \textit{=nin} itself, where it means `it doesn't matter'. See~(\ref{exe:dkonkonn2}). These constructions are well-established in the corpus.

\begin{exe}
	\ex
	{\gll lembaga nerun=ko an \textbf{don} \textbf{kon∼kon} konat=\textbf{nin}\\
		prison in=\textsc{loc} \textsc{1sg} thing one∼\textsc{red} see=\textsc{neg}\\
		\glt `He saw nothing / he didn't see a thing.' \jambox*{\href{http://hdl.handle.net/10050/00-0000-0000-0004-1BAA-C}{[stim7\_24:42]}}
	}
	\label{exe:dkonkon1}
	\ex
	{\gll kian ma sala-un \textbf{don} \textbf{kon∼kon=nin}\\
		wife.\textsc{1sg.poss} \textsc{3sg} mistake-\textsc{3poss} thing one∼\textsc{red=neg}\\	
		\glt \glt `My wife's mistake doesn't matter.' \jambox*{\href{http://hdl.handle.net/10050/00-0000-0000-0004-1BAA-C}{[stim7\_16:56]}}
	}
	\label{exe:dkonkonn2}		
\end{exe}

The corpus also contains two other indefinite-like examples of reduplicated \textit{kon} `one'. In~(\ref{exe:samuret}), the best translation of \textit{konkon} is `few' or 'some'. It is taken from a discussion about who was invited to a big funeral on Karas. The context of~(\ref{exe:obatkon}) gives fewer clues about the meaning of \textit{konkon}, but it seems to mean `other', or alternatively, `not any'.

\begin{exe}
	\ex \gll samur-et \textbf{kon∼kon}\\
	Mbaham-person one∼\textsc{red}\\
	\glt `A few Mbaham people.' \jambox*{\href{http://hdl.handle.net/10050/00-0000-0000-0004-1BC3-B}{[conv7\_8:56]}}
	\label{exe:samuret}
	\ex \gll sontum pasier=ka bot=nin {\ob}...{\cb} obat \textbf{kon∼kon} eranun pi neba=et me {\ob}...{\cb} pirawilak met koyak\\
	person beach=\textsc{lat} go=\textsc{neg} {} medicine one∼\textsc{red} cannot \textsc{1pl.incl} \textsc{ph}={\glet} {\glme} {} kind\_of\_tree \textsc{dist.obj} cut\\
	\glt `[When] people can't go to the toilet, [... if we] cannot use other medicine, we whatsit [...] cut that \textit{pirawilak}.' \jambox*{\href{http://hdl.handle.net/10050/00-0000-0000-0004-1BCA-4}{[conv20\_15:17]}}
	\label{exe:obatkon}
\end{exe} 	

Besides reduplication, there is another, less common strategy to create \is{distributive}distributive numerals: the suffix \textit{-te}. Consider the following two examples. In~(\ref{exe:nakkon}), this strategy is combined with reduplication.

\begin{exe}
	\ex \gll an se taruon ripi-ap-\textbf{te} karung kon\\
	\textsc{1sg} {\glse} say thousand-five-\textsc{distr} sack one\\
	\glt `I said five thousand per sack.' \jambox*{\href{http://hdl.handle.net/10050/00-0000-0000-0004-1B9F-F}{[conv9\_18:22]}}
	\ex \gll som-kon-\textbf{te} nak-kon∼kon\\
	person-one-\textsc{distr} fruit-one∼\textsc{distr}\\
	\glt `Each person one fruit.' \jambox*{\href{http://hdl.handle.net/10050/00-0000-0000-0004-1BD1-D}{[stim31\_2:48]}}
	\label{exe:nakkon}
\end{exe}	
	
Approximate quantities\is{numeral!approximate} are expressed by attaching \textit{-kon} (homonymous with \textit{kon} `one') to a numeral. This construction may be accompanied by the similative marker \textit{=kap}, and the Malay loans \textit{kirakira} `approximately' or \textit{mungkin} `maybe'.

\begin{exe}
	\ex \gll ikon-i an se parair mungkin et-purir-\textbf{kon}=\textbf{kap} ye\\
	some-\textsc{objqnt} \textsc{1sg} {\glse} split maybe \textsc{clf\_an}-twenty-approximately=\textsc{sim} or\\
	\glt `Some I already split, maybe twenty or so?' \jambox*{\href{http://hdl.handle.net/10050/00-0000-0000-0004-1BAE-4}{[narr44\_9:14]}}
	\label{exe:purirkon}
	\ex \gll luas-un me mungkin meter ap-\textbf{kon}\\
	wide-\textsc{nmlz} {\glme} maybe metre five-approximately\\
	\glt `The width is maybe five metres.' \jambox*{\href{http://hdl.handle.net/10050/00-0000-0000-0004-1C78-C}{[narr46\_2:52]}}
\end{exe}	

The last inflection attested on quantifiers is the quantifier object marker\is{quantifier object} \textit{-i} (§\ref{sec:numobj}), for quantifiers in object NPs. An example of this is \textit{ikon} `some' in~(\ref{exe:purirkon}) above.
\is{quantifier|)}
